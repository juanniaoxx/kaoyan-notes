\ifx\allfiles\undefined
\documentclass[12pt, a4paper, oneside, UTF8]{ctexbook}
\def\path{../../config}
\usepackage{amsthm}
\usepackage{amssymb}
\usepackage{array}
\usepackage{xcolor}
\usepackage{graphicx}
\usepackage{mathrsfs}
\usepackage{enumitem}
\usepackage{geometry}
\usepackage[colorlinks, linkcolor=black]{hyperref}
\usepackage{stackengine}
\usepackage{yhmath}
\usepackage{extarrows}
\usepackage{tikz}
\usepackage{forest}
\usetikzlibrary{decorations.pathreplacing, positioning}
% \usepackage{unicode-math}
\usepackage{esint}
\usepackage{pifont}
\usepackage{tcolorbox}
\tcbuselibrary{skins, breakable}

\usepackage{multicol} 
\usepackage{fontspec} % 使用字体

\setmainfont{Times New Roman}
\setCJKmainfont{LXGWWenKai-Light}[
    SlantedFont=*
]

\usepackage{listings} % 用于插入代码

% 定义代码高亮风格
\lstset{
    basicstyle=\ttfamily\small,        % 基本字体样式(等宽小字体)
    keywordstyle=\color{blue},         % 关键字颜色
    commentstyle=\color{green},        % 注释颜色
    stringstyle=\color{red},           % 字符串颜色
    numbers=none,
    breaklines=true,                   % 自动换行
    frame=single,                      % 代码框边框
    rulecolor=\color{black},           % 边框颜色
    captionpos=b,                      % 标题位置(底部)
    showspaces=false,                  % 不显示空格标记
    showstringspaces=false,            % 不显示字符串中的空格标记
    language=C                         % 设置语言为 C
}

\usepackage{fontawesome5}

\usepackage{amsmath}
\usepackage{booktabs, array}
\usepackage{makecell}
\usepackage{fancyhdr}
\usepackage[dvipsnames, svgnames]{xcolor}
\usepackage{listings}
\usepackage{tasks}[2020/01/11]

\everymath{\displaystyle}

\definecolor{mygreen}{rgb}{0,0.6,0}
\definecolor{mygray}{rgb}{0.5,0.5,0.5}
\definecolor{mymauve}{rgb}{0.58,0,0.82}
\definecolor{NavyBlue}{RGB}{0,0,128}
\definecolor{Rhodamine}{RGB}{255,0,255}
\definecolor{PineGreen}{RGB}{0,128,0}

\graphicspath{ {figures/},{../figures/}, {config/}, {../config/} }

\linespread{1.6}

\geometry{
    top=25.4mm, 
    bottom=25.4mm, 
    left=20mm, 
    right=20mm, 
    headheight=2.17cm, 
    headsep=4mm, 
    footskip=12mm
}

\setenumerate[1]{itemsep=5pt,partopsep=0pt,parsep=\parskip,topsep=5pt}
\setitemize[1]{itemsep=5pt,partopsep=0pt,parsep=\parskip,topsep=5pt}
\setdescription{itemsep=5pt,partopsep=0pt,parsep=\parskip,topsep=5pt}



% \begin{lstlisting}[language=TeX] ... \end{lstlisting}

% 定理环境设置
% ---------- 颜色 ----------
\definecolor{ExBlue}{HTML}{4F81BD}
\definecolor{SolGreen}{HTML}{77933C}
\definecolor{DefRed}{HTML}{C5504B}
\definecolor{ThmOrange}{HTML}{E97132}
\definecolor{RemGray}{HTML}{7F7F7F}
\definecolor{CorPurple}{HTML}{7030A0}
\definecolor{ForGray}{HTML}{595959}

% ---------- 通用“变色”模板 ----------
\tcbset{
    mybox/.style n args={1}{
        enhanced, breakable,
        arc=6pt,
        boxrule=0.6pt,
        left=8pt, right=8pt, top=6pt, bottom=6pt,
        drop shadow={black!25},
        fonttitle=\bfseries,
        coltitle=white,
        colbacktitle=#1!85,
        colback=#1!10,
        colframe=#1,
    }
}

% ---------- 各环境 ----------
% 例题
\newtcolorbox{example}[1][]{mybox={ExBlue}, title={\ifstrempty{#1}{Example}{#1}}}
% 解答
\newtcolorbox{solution}[1][]{mybox={SolGreen}, title={\ifstrempty{#1}{Solution}{#1}}}
% 定义
\newtcolorbox{definition}[1][]{mybox={DefRed}, title={\ifstrempty{#1}{Definition}{#1}}}
% 定理
\newtcolorbox{theorem}[1][]{mybox={ThmOrange}, title={\ifstrempty{#1}{Theorem}{#1}}}
% 标注
\newtcolorbox{remark}[1][]{mybox={RemGray}, title={\ifstrempty{#1}{Remark}{#1}}}
% 推论
\newtcolorbox{corollary}[1][]{mybox={CorPurple}, title={\ifstrempty{#1}{Corollary}{#1}}}
% 公式
\newtcolorbox{formula}[1][]{mybox={ForGray}, title={\ifstrempty{#1}{Formula}{#1}}}


\settasks{
    label-format = \bfseries,
    label        = \Alph*.,
    label-width  = 1.2em,
    label-offset = 0.3em,
    item-indent  = 1.9em,
    column-sep   = 0.5em
}

\newenvironment{choices}[1][4]   % 默认 4 栏
    {\begin{tasks}(#1)}
    {\end{tasks}}

% 自定义命令的文件

\def\d{\mathrm{d}}
\def\R{\mathbb{R}}
\def\P{\partial} 
\newcommand{\bs}[1]{\begin{solution}#1\end{solution}}
\newcommand{\bt}[1][1]{% 默认参数为1
    \ensuremath{% 确保数学模式
        \foreach \n in {1,...,#1} {\blacktriangle}% 循环输出 #1 个黑色三角形
    }%
}

\newcommand{\bl}[1][1]{% 默认参数为1
    \ensuremath{% 确保数学模式
        \foreach \n in {1,...,#1} {\blacklozenge}% 循环输出 #1 个黑色三角形
    }%
}
\newif\ifshowanswers
%\showanswerstrue % 注释掉这行就不显示答案

% 定义答案环境
\newcommand{\answer}[1]{%
    \ifshowanswers
        #1%
    \fi
}




% 修改参数改变封面样式,0 默认原始封面、内置其他1、2、3种封面样式
\def\myIndex{3}


\ifnum\myIndex>0
    \input{\path/cover_package_\myIndex} 
\fi

\def\myTitle{冲刺150笔记}
\def\myAuthor{Weary Bird}
\def\myDateCover{\today}
\def\myDateForeword{\today}
\def\myForeword{行香子}
\def\myForewordText{
树绕村庄,水满陂塘;倚东风、豪兴徜徉。小园几许,收尽春光。有桃花红,李花白,菜花黄。 \\
远远苔墙,隐隐茅堂;飏青旗、流水桥旁。偶然乘兴,步过东冈。正莺儿啼,燕儿舞,蝶儿忙。 \\
}
\def\mySubheading{知错能改善莫大焉}


\begin{document}
% \input{../config/cover}
\else
\fi

\chapter{数字特征}

\section{期望与方差的计算}

\begin{enumerate}[label=\arabic*.]
    \item 设随机变量$X$的概率密度为$f(x)=\frac{1}{\pi(1+x^2)}$,$-\infty<x<\infty$,则$E[\min\{|X|,1\}]=$?.
    
    \begin{solution}
    【详解】
    \end{solution}
    
    \item (2016,数三)设随机变量$X$与$Y$相互独立,$X\sim N(1,2)$,$Y\sim N(1,4)$,则$D(XY)=$
    \begin{align*}
        (A)\ 6 \quad (B)\ 8 \quad (C)\ 14 \quad (D)\ 15
    \end{align*}
    
    \begin{solution}
    【详解】
    \end{solution}
    
    \item 设随机变量$X$与$Y$同分布,则$E\left(\frac{X+Y}{2}\right)=$?.
    
    \begin{solution}
    【详解】
    \end{solution}
    
    \item 设随机变量$X$与$Y$相互独立,$X\sim P(\lambda_1)$,$Y\sim P(\lambda_2)$,且$P\{X+Y>0\}=1-e^{-1}$,则$E(X+Y)^2=$?.
    
    \begin{solution}
    【详解】
    \end{solution}
    
    \item 设随机变量$X$与$Y$相互独立,$X\sim E(\lambda)$,$Y\sim E\left(\frac{1}{6}\right)$,若$U=\max\{X,Y\}$,$V=\min\{X,Y\}$,则$EU=$?,$EV=$?.
    
    \begin{solution}
    【详解】
    \end{solution}
    
    \item (2017,数一)设随机变量$X$的分布函数为$F(x)=0.5\Phi(x)+0.5\Phi\left(\frac{x-4}{2}\right)$,其中$\Phi(x)$为标准正态分布函数,则$EX=$?.
    
    \begin{solution}
    【详解】
    \end{solution}
    
    \item 设随机变量$X\sim N(0,1)$,则$E|X|=$?,$D|X|=$?.
    
    \begin{solution}
    【详解】
    \end{solution}
    
    \item 设随机变量$X$与$Y$相互独立,均服从$N(\mu,\sigma^2)$,求$E[\max\{X,Y\}]$,$E[\min\{X,Y\}]$.
    
    \begin{solution}
    【详解】
    \end{solution}
    
    \item 设独立重复的射击每次命中的概率为$p$,$X$表示第$n$次命中时的射击次数,求$EX$,$DX$.
    
    \begin{solution}
    【详解】
    \end{solution}
    
    \item  (2015,数一、三)设随机变量$X$的概率密度为$f(x)=\begin{cases}2^{-x}\ln2, & x>0 \\ 0, & x\leq0\end{cases}$,对$X$进行独立的观测,直到第2个大于3的观测值出现时停止,记$Y$为观测次数。
    \begin{enumerate}
        \item 求$Y$的概率分布;
        \item 求$EY$.
    \end{enumerate}
    
    \begin{solution}
    【详解】
    \end{solution}
\end{enumerate}

\section{协方差的计算}

\begin{enumerate}[label=\arabic*.,start=11]
    \item  设$X_1,X_2,\cdots,X_n$为来自总体$X$的简单随机样本。若$DX=4$,正整数$s\leq n$,$t\leq n$,则
    \begin{align*}
        \text{Cov}\left(\frac{1}{s}\sum_{i=1}^s X_i,\frac{1}{t}\sum_{j=1}^t X_j\right)=
    \end{align*}
    \begin{align*}
        (A)\ 4\max\{s,t\} \quad (B)\ 4\min\{s,t\} \quad (C)\ \frac{4}{\max\{s,t\}} \quad (D)\ \frac{4}{\min\{s,t\}}
    \end{align*}
    
    \begin{solution}
    【详解】
    \end{solution}
    
    \item  (2005,数三)设$X_1,X_2,\cdots,X_n(n>2)$为来自总体$N(0,\sigma^2)$的简单随机样本,样本均值为$\bar{X}$。记$Y_i=X_i-\bar{X}$,$i=1,2,\cdots,n$。
    \begin{enumerate}
        \item 求$Y_i$的方差$DY_i$,$i=1,2,\cdots,n$;
        \item 若$c(Y_1+Y_n)^2$为$\sigma^2$的无偏估计量,求常数$c$.
    \end{enumerate}
    
    \begin{solution}
    【详解】
    \end{solution}
\end{enumerate}

\section{相关系数的计算}

\begin{enumerate}[label=\arabic*.,start=13]
    \item  (2016,数一)设试验有三个两两互不相容的结果$A_1,A_2,A_3$,且三个结果发生的概率均为$\frac{1}{3}$。将试验独立重复地做两次,$X$表示两次试验中$A_1$发生的次数,$Y$表示两次试验中$A_2$发生的次数,则$X$与$Y$的相关系数为
    \begin{align*}
        (A)\ -\frac{1}{2} \quad (B)\ -\frac{1}{3} \quad (C)\ \frac{1}{3} \quad (D)\ \frac{1}{2}
    \end{align*}
    
    \begin{solution}
    【详解】
    \end{solution}
    
    \item  设随机变量$X\sim B\left(1,\frac{3}{4}\right)$,$Y\sim B\left(1,\frac{1}{2}\right)$,且$\rho_{XY}=\frac{\sqrt{3}}{3}$。
    \begin{enumerate}
        \item 求$(X,Y)$的联合概率分布;
        \item 求$P\{Y=1|X=1\}$.
    \end{enumerate}
    
    \begin{solution}
    【详解】
    \end{solution}
\end{enumerate}

\section{相关与独立的判定}

\begin{enumerate}[label=\arabic*.,start=15]
    \item  设二维随机变量$(X,Y)$服从区域$D=\{(x,y)|x^2+y^2\leq a^2\}$上的均匀分布,则
    \begin{align*}
        (A)\ X与Y不相关,也不相互独立 \\
        (B)\ X与Y相互独立 \\
        (C)\ X与Y相关 \\
        (D)\ X与Y均服从U(-a,a)
    \end{align*}
    
    \begin{solution}
    【详解】
    \end{solution}
    
    \item  设随机变量$X$的概率密度为$f(x)=\frac{1}{2}e^{-|x|}$,$-\infty<x<+\infty$。
    \begin{enumerate}
        \item 求$X$的期望与方差;
        \item 求$X$与$|X|$的协方差,问$X$与$|X|$是否不相关?
        \item 问$X$与$|X|$是否相互独立?并说明理由.
    \end{enumerate}
    
    \begin{solution}
    【详解】
    \end{solution}
\end{enumerate}
\ifx\allfiles\undefined
\end{document}
\fi