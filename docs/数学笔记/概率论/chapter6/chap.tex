\ifx\allfiles\undefined
\documentclass[12pt, a4paper, oneside, UTF8]{ctexbook}
\def\path{../../config}
\usepackage{amsmath}
\usepackage{amsthm}
\usepackage{amssymb}
\usepackage{array}
\usepackage{xcolor}
\usepackage{graphicx}
\usepackage{mathrsfs}
\usepackage{enumitem}
\usepackage{geometry}
\usepackage[colorlinks, linkcolor=black]{hyperref}
\usepackage{stackengine}
\usepackage{yhmath}
\usepackage{extarrows}
\usepackage{tikz}
\usepackage{pgfplots}
\usepackage{asymptote}
\usepackage{float}
\usepackage{fontspec} % 使用字体

\setmainfont{Times New Roman}
\setCJKmainfont{LXGWWenKai-Light}[
    SlantedFont=*
]

\everymath{\displaystyle}

\usepgfplotslibrary{polar}
\usepackage{subcaption}
\usetikzlibrary{decorations.pathreplacing, positioning}

\usepgfplotslibrary{fillbetween}
\pgfplotsset{compat=1.18}
% \usepackage{unicode-math}
\usepackage{esint}
\usepackage[most]{tcolorbox}

\usepackage{fancyhdr}
\usepackage[dvipsnames, svgnames]{xcolor}
\usepackage{listings}

\definecolor{mygreen}{rgb}{0,0.6,0}
\definecolor{mygray}{rgb}{0.5,0.5,0.5}
\definecolor{mymauve}{rgb}{0.58,0,0.82}
\definecolor{NavyBlue}{RGB}{0,0,128}
\definecolor{Rhodamine}{RGB}{255,0,255}
\definecolor{PineGreen}{RGB}{0,128,0}

\graphicspath{ {figures/},{../figures/}, {config/}, {../config/} }

\linespread{1.6}

\geometry{
    top=25.4mm, 
    bottom=25.4mm, 
    left=20mm, 
    right=20mm, 
    headheight=2.17cm, 
    headsep=4mm, 
    footskip=12mm
}

\setenumerate[1]{itemsep=5pt,partopsep=0pt,parsep=\parskip,topsep=5pt}
\setitemize[1]{itemsep=5pt,partopsep=0pt,parsep=\parskip,topsep=5pt}
\setdescription{itemsep=5pt,partopsep=0pt,parsep=\parskip,topsep=5pt}

\lstset{
    language=Mathematica,
    basicstyle=\tt,
    breaklines=true,
    keywordstyle=\bfseries\color{NavyBlue}, 
    emphstyle=\bfseries\color{Rhodamine},
    commentstyle=\itshape\color{black!50!white}, 
    stringstyle=\bfseries\color{PineGreen!90!black},
    columns=flexible,
    numbers=left,
    numberstyle=\footnotesize,
    frame=tb,
    breakatwhitespace=false,
} 

\lstset{
    language=TeX, % 设置语言为 TeX
    basicstyle=\ttfamily, % 使用等宽字体
    breaklines=true, % 自动换行
    keywordstyle=\bfseries\color{NavyBlue}, % 关键字样式
    emphstyle=\bfseries\color{Rhodamine}, % 强调样式
    commentstyle=\itshape\color{black!50!white}, % 注释样式
    stringstyle=\bfseries\color{PineGreen!90!black}, % 字符串样式
    columns=flexible, % 列的灵活性
    numbers=left, % 行号在左侧
    numberstyle=\footnotesize, % 行号字体大小
    frame=tb, % 顶部和底部边框
    breakatwhitespace=false % 不在空白处断行
}

% \begin{lstlisting}[language=TeX] ... \end{lstlisting}

% 定理环境设置
\usepackage[strict]{changepage} 
\usepackage{framed}

\definecolor{greenshade}{rgb}{0.90,1,0.92}
\definecolor{redshade}{rgb}{1.00,0.88,0.88}
\definecolor{brownshade}{rgb}{0.99,0.95,0.9}
\definecolor{lilacshade}{rgb}{0.95,0.93,0.98}
\definecolor{orangeshade}{rgb}{1.00,0.88,0.82}
\definecolor{lightblueshade}{rgb}{0.8,0.92,1}
\definecolor{purple}{rgb}{0.81,0.85,1}

\theoremstyle{definition}
\newtheorem{myDefn}{\indent Definition}[section]
\newtheorem{myLemma}{\indent Lemma}[section]
\newtheorem{myThm}[myLemma]{\indent Theorem}
\newtheorem{myCorollary}[myLemma]{\indent Corollary}
\newtheorem{myCriterion}[myLemma]{\indent Criterion}
\newtheorem*{myRemark}{\indent Remark}
\newtheorem{myProposition}{\indent Proposition}[section]

\newenvironment{formal}[2][]{%
	\def\FrameCommand{%
		\hspace{1pt}%
		{\color{#1}\vrule width 2pt}%
		{\color{#2}\vrule width 4pt}%
		\colorbox{#2}%
	}%
	\MakeFramed{\advance\hsize-\width\FrameRestore}%
	\noindent\hspace{-4.55pt}%
	\begin{adjustwidth}{}{7pt}\vspace{2pt}\vspace{2pt}}{%
		\vspace{2pt}\end{adjustwidth}\endMakeFramed%
}

\newenvironment{definition}{\vspace{-\baselineskip * 2 / 3}%
	\begin{formal}[Green]{greenshade}\vspace{-\baselineskip * 4 / 5}\begin{myDefn}}
	{\end{myDefn}\end{formal}\vspace{-\baselineskip * 2 / 3}}

\newenvironment{theorem}{\vspace{-\baselineskip * 2 / 3}%
	\begin{formal}[LightSkyBlue]{lightblueshade}\vspace{-\baselineskip * 4 / 5}\begin{myThm}}%
	{\end{myThm}\end{formal}\vspace{-\baselineskip * 2 / 3}}

\newenvironment{lemma}{\vspace{-\baselineskip * 2 / 3}%
	\begin{formal}[Plum]{lilacshade}\vspace{-\baselineskip * 4 / 5}\begin{myLemma}}%
	{\end{myLemma}\end{formal}\vspace{-\baselineskip * 2 / 3}}

\newenvironment{corollary}{\vspace{-\baselineskip * 2 / 3}%
	\begin{formal}[BurlyWood]{brownshade}\vspace{-\baselineskip * 4 / 5}\begin{myCorollary}}%
	{\end{myCorollary}\end{formal}\vspace{-\baselineskip * 2 / 3}}

\newenvironment{criterion}{\vspace{-\baselineskip * 2 / 3}%
	\begin{formal}[DarkOrange]{orangeshade}\vspace{-\baselineskip * 4 / 5}\begin{myCriterion}}%
	{\end{myCriterion}\end{formal}\vspace{-\baselineskip * 2 / 3}}
	

\newenvironment{remark}{\vspace{-\baselineskip * 2 / 3}%
	\begin{formal}[LightCoral]{redshade}\vspace{-\baselineskip * 4 / 5}\begin{myRemark}}%
	{\end{myRemark}\end{formal}\vspace{-\baselineskip * 2 / 3}}

\newenvironment{proposition}{\vspace{-\baselineskip * 2 / 3}%
	\begin{formal}[RoyalPurple]{purple}\vspace{-\baselineskip * 4 / 5}\begin{myProposition}}%
	{\end{myProposition}\end{formal}\vspace{-\baselineskip * 2 / 3}}


\newtheorem{example}{\indent \color{SeaGreen}{Example}}[section]
\renewcommand{\proofname}{\indent\textbf{\textcolor{TealBlue}{Proof}}}
\NewEnviron{solution}{%
	\begin{proof}[\indent\textbf{\textcolor{TealBlue}{Solution}}]%
		\color{blue}% 设置内容为蓝色
		\BODY% 插入环境内容
		\color{black}% 恢复默认颜色(可选,避免影响后续文字)
	\end{proof}%
}

% 自定义命令的文件

\def\d{\mathrm{d}}
\def\R{\mathbb{R}}
%\newcommand{\bs}[1]{\boldsymbol{#1}}
%\newcommand{\ora}[1]{\overrightarrow{#1}}
\newcommand{\myspace}[1]{\par\vspace{#1\baselineskip}}
\newcommand{\xrowht}[2][0]{\addstackgap[.5\dimexpr#2\relax]{\vphantom{#1}}}
\newenvironment{mycases}[1][1]{\linespread{#1} \selectfont \begin{cases}}{\end{cases}}
\newenvironment{myvmatrix}[1][1]{\linespread{#1} \selectfont \begin{vmatrix}}{\end{vmatrix}}
\newcommand{\tabincell}[2]{\begin{tabular}{@{}#1@{}}#2\end{tabular}}
\newcommand{\pll}{\kern 0.56em/\kern -0.8em /\kern 0.56em}
\newcommand{\dive}[1][F]{\mathrm{div}\;\boldsymbol{#1}}
\newcommand{\rotn}[1][A]{\mathrm{rot}\;\boldsymbol{#1}}

\newif\ifshowanswers
\showanswerstrue % 注释掉这行就不显示答案

% 定义答案环境
\newcommand{\answer}[1]{%
    \ifshowanswers
        #1%
    \fi
}

% 修改参数改变封面样式,0 默认原始封面、内置其他1、2、3种封面样式
\def\myIndex{0}


\ifnum\myIndex>0
    \input{\path/cover_package_\myIndex} 
\fi

\def\myTitle{考研数学笔记}
\def\myAuthor{Weary Bird}
\def\myDateCover{\today}
\def\myDateForeword{\today}
\def\myForeword{相见欢·林花谢了春红}
\def\myForewordText{
    林花谢了春红,太匆匆。
    无奈朝来寒雨晚来风。
    胭脂泪,相留醉,几时重。
    自是人生长恨水长东。
}
\def\mySubheading{以姜晓千强化课讲义为底本}


\begin{document}
% \input{\path/cover_text_\myIndex.tex}

\newpage
\thispagestyle{empty}
\begin{center}
    \Huge\textbf{\myForeword}
\end{center}
\myForewordText
\begin{flushright}
    \begin{tabular}{c}
        \myDateForeword
    \end{tabular}
\end{flushright}

\newpage
\pagestyle{plain}
\setcounter{page}{1}
\pagenumbering{Roman}
\tableofcontents

\newpage
\pagenumbering{arabic}
% \setcounter{chapter}{-1}
\setcounter{page}{1}

\pagestyle{fancy}
\fancyfoot[C]{\thepage}
\renewcommand{\headrulewidth}{0.4pt}
\renewcommand{\footrulewidth}{0pt}








\else
\fi

\chapter{统计初步}

\section{求统计量的抽样分布}
\begin{remark}
    样本均值与方差
    \[
    \bar{X}=\frac{1}{n}\sum_{i=1}{n}X_i, S^2=\frac{1}{n-1}\sum_{i=1}^{n}(X_i-\bar{X})^2
    \]
    $E\bar{X}=\mu,D\bar{X}=\frac{\sigma^2}{n},ES^2=\sigma^2${\color{red}来自同一总体的样本均值与方差是独立的} \\
    有偏估计量$S_{n}^{2}=\frac{1}{n}\sum_{i=1}^{n}(X_i-\bar{X})^2$其$ES_{n}^{2}=\frac{n-1}{n}\sigma^2$ \\
    统计的三大分布 \\
    $\chi^2$分布的定义\\ 
    设随机变量$X_1,X_2,\ldots,X_n$相互独立,均服从$N(0,1)$称$\chi^2=X_1^2+X_2^2+\ldots+X_n^2$
    服从自由度为$n$的$\chi^2$分布,记$\chi^2\sim\chi^2(n)$,特别的若$X\sim N(0,1)$,则$\chi^2\sim\chi^2(1)$ \\
    $\chi^2$分布的性质
    \item[(1)] 参数可加性\qquad 设$\chi_1^2$与$\chi_2^2$相互独立,且$\chi_1^2\sim\chi^2(n),\chi_2^2\sim\chi^2(m)$则
    $\chi_1^2+\chi_2^2\sim\chi^2(n+m)$
    \item[(2)] 设$\chi^2\sim\chi^2(n)$则$E\chi^2=n,D\chi^2=2n$ \\
    $F$分布的定义 \\
    设随机变量$X$和$Y$相互独立,且$X\sim \chi^2(n_1),Y\sim\chi^2(n_2),$称$F=\frac{X/n_1}{Y/n_2}$服从自由度为
    $n_1,n_2$的$F$分布,记作$F\sim F(n_1,n_2)$\\
    $F$分布的性质 
    \item[(1)] 设$F\sim F(n_1,n_2),$则$\frac{1}{F}\sim F(n_2,n_1)$
    \item[(2)] $F_{1-\alpha}(n_2,n_1)=\frac{1}{F_{\alpha}(n_1,n_2)}$ \\
    $t$分布的定义\qquad 设随机变量$X$和$Y$相互独立,$X\sim N(0,1),Y\sim \chi^2(n)$,则称$T=\frac{X}{\sqrt{Y/n}}$服从自由度为
    $n$的$t$分布,记作$T\sim t(n)$ \\
    $t$分布的性质
    \item[(1)] 设$T\sim t(n)$, 则$T^2\sim F(1, n),\frac{1}{T^2}\sim F(n,1)$ 
    \item[(2)] $t_{1-\alpha}(n)=-t_{\alpha}(n)$ 
\end{remark}
\begin{remark}
    单正态总体与双正态总体 \\
    单正态总体 \\
    设$X_1,X_2,\ldots,X_n$为来自总体$X\sim N(\mu,\sigma^2)$的简单随机样本,$\bar{X}$与$S^2$分别为样本均值与样本方差,则
    \item [(1)] $\frac{\bar{X}-\mu}{\sigma/\sqrt{n}}\sim N(0,1)$,即$\bar{X}\sim N(\mu,\sigma^2/n)$
    \item [(2)] $\frac{(n-1)S^2}{\sigma^2}\sim\chi^2(n-1)$即$\frac{1}{\sigma^2}\sum_{i=1}^{n}(X_i-\bar{X})^2\sim\chi^2(n-1)$,
    且$\bar{X}$与$S^2$相互独立 
    \item[(3)] $\frac{\bar{X}-\mu}{S/\sqrt{n}}\sim t(n-1)$ \\
    双正态总体 \\
    设总体 $X \sim  N\left( {{\mu }_{1},{\sigma }_{1}^{2}}\right)$ ,总体 $Y \sim  N\left( {{\mu }_{2},{\sigma }_{2}^{2}}\right) ,{X}_{1},{X}_{2},\cdots ,{X}_{{n}_{1}}$ 与 ${Y}_{1},{Y}_{2},\cdots ,{Y}_{{n}_{2}}$ 分别为来
    自总体 $X$ 与 $Y$ 的简单随机样本且相互独立,样本均值分别为 $\bar{X},\bar{Y}$ ,样本方差分别为 ${S}_{1}^{2},{S}_{2}^{2}$ ,则
    \item[(4)] $\frac{\bar{X} - \bar{Y} - \left( {{\mu }_{1} - {\mu }_{2}}\right) }{\sqrt{\frac{{\sigma }_{1}^{2}}{{n}_{1}} + \frac{{\sigma }_{2}^{2}}{{n}_{2}}}} \sim  N\left( {0,1}\right)$;
    \item[(5)]$\frac{S_1^2/{\sigma }_{1}^{2}}{{S}_{2}^{2}/\sigma_2^2} \sim  F\left( {{n}_{1} - 1,{n}_{2} - 1}\right)$ ;
    \item[(6)]当 ${\sigma }_{1}^{2} = {\sigma }_{2}^{2}$ 时, $\frac{\bar{X} - \bar{Y} - \left( {{\mu }_{1} - {\mu }_{2}}\right) }{{S}_{\omega }\sqrt{\frac{1}{{n}_{1}} + \frac{1}{{n}_{2}}}} \sim  t\left( {{n}_{1} + {n}_{2} - 2}\right)$ ,其中 ${S}_{\omega } = \sqrt{\frac{\left( {{n}_{1} - 1}\right) {S}_{1}^{2} + \left( {{n}_{2} - 1}\right) {S}_{2}^{2}}{{n}_{1} + {n}_{2} - 2}}$ .
\end{remark}
\begin{enumerate}[label=\arabic*.]
    \item (2013,数一)设随机变量$X\sim t(n)$,$Y\sim F(1,n)$。给定$\alpha(0<\alpha<0.5)$,常数$c$满足$P\{X>c\}=\alpha$,则$P\{Y>c^2\}=$
    \begin{align*}
        (A)\ \alpha \quad (B)\ 1-\alpha \quad (C)\ 2\alpha \quad (D)\ 1-2\alpha
    \end{align*}
    
    \begin{solution}
    这道题考察的是$t$分布的对称性,由题有
    \[
    Y=\frac{\chi^2(1)}{\chi^2(n)}\qquad X=\frac{N(0,1)}{\sqrt{\chi^2(n)/n}}
    \]
    则有$X^2=Y$,所求概率就变成$P\{X^2>c^2\}$由t分布的对称性有$P\{X^2>c^2\}=2\alpha$
    \end{solution}
    \begin{tcolorbox}[title=总结]
    正态分布与t分布具有相似的概率密度图像,F分布与$\chi^2$分布也有类似的图像.
    \end{tcolorbox}
    \item 设$X_1,X_2,\cdots,X_9$为来自总体$N(\mu,\sigma^2)$的简单随机样本,$Y_1=\frac{1}{6}(X_1+X_2+\cdots+X_6)$,$Y_2=\frac{1}{3}(X_7+X_8+X_9)$,$S^2=\frac{1}{2}\sum_{i=7}^9(X_i-Y_2)^2$,求$\frac{\sqrt{2}(Y_1-Y_2)}{S}$的分布.
    
    \begin{solution}
    这种题就是一步一步反推,注意凑题目要求的结果即可 \\
    $Y_1=\frac{1}{6}\sum_{i=1}^{6}X_i\sim N(\mu, \frac{\sigma^2}{6})$同理$Y_2\sim N(\mu,\frac{\sigma^2}{3})$ \\
    由$Y_1,Y_2$独立,知道$Y_1-Y_2\sim N(0, \frac{\sigma^2}{2})\implies \frac{Y_1-Y_2}{\sigma/\sqrt{2}}\sim N(0,1)$  \\
    又有$\frac{2s^2}{\sigma^2}\sim\chi^2(2)$,故 
    \[
    \frac{Y_1-Y_2}{\sqrt{\sigma^2/2}\sqrt{\frac{2s^2}{\sigma^2}/2}} = \frac{\sqrt{2}(Y_1-Y_2)}{s}\sim t(2)
    \]
    \end{solution}
\end{enumerate}

\section{求统计量的数字特征}

\begin{enumerate}[label=\arabic*.,start=3]
    \item 设$X_1,X_2,\cdots,X_n$为来自总体$N(\mu,\sigma^2)$的简单随机样本,则
    \begin{align*}
        E\left[\sum_{i=1}^{n}X_i\cdot\sum_{j=1}^{n}\left(nX_j-\sum_{k=1}^{n}X_k\right)^2\right]=
    \end{align*}
    
    \begin{solution}
    这道题就是个凑系数化简,过程省去$\text{原式}=n^3(n-1)\mu\sigma^2$
    \end{solution}
    
    \item 设$X_1,X_2,\cdots,X_9$为来自总体$N(0,\sigma^2)$的简单随机样本,样本均值为$\bar{X}$,样本方差为$S^2$。
    \begin{enumerate}
        \item[(1)]求$\frac{9\bar{X}^2}{S^2}$的分布
        \item[(2)]求$E[(\bar{X}^2S^2)^2]$;
    \end{enumerate}
    
    \begin{solution}
    \item [(1)] 和例题3一致,过程省去 $\frac{9\bar{X}^2}{S^2}\sim F(1, 8)$ 
    \item [(2)] 对于这种高幂次的一般都需要考虑用$\chi^2$的结论 
    \begin{align*}
        E\left[(\bar{X}^2S^2)^2\right] &=E\bar{X}^4\cdot ES^4  \\
        &=\left[D\bar{X}^2+(E\bar{X}^2)^2\right]\left[DS^2+(ES^2)^2\right] \\
        &=\frac{5}{107}\sigma^8
    \end{align*}
    又$\frac{9\bar{X}^2}{\sigma^2}\sim\chi^2(1)\implies D\bar{X}^2=\frac{2\sigma^4}{81}$同理有$DS^2=\frac{\sigma^4}{4}$
    \end{solution}
\end{enumerate}
\ifx\allfiles\undefined
\end{document}
\fi