\ifx\allfiles\undefined
\documentclass[12pt, a4paper, oneside, UTF8]{ctexbook}
\def\path{../config}
\usepackage{amsmath}
\usepackage{amsthm}
\usepackage{amssymb}
\usepackage{array}
\usepackage{xcolor}
\usepackage{graphicx}
\usepackage{mathrsfs}
\usepackage{enumitem}
\usepackage{geometry}
\usepackage[colorlinks, linkcolor=black]{hyperref}
\usepackage{stackengine}
\usepackage{yhmath}
\usepackage{extarrows}
\usepackage{tikz}
\usepackage{pgfplots}
\usepackage{asymptote}
\usepackage{float}
\usepackage{fontspec} % 使用字体

\setmainfont{Times New Roman}
\setCJKmainfont{LXGWWenKai-Light}[
    SlantedFont=*
]

\everymath{\displaystyle}

\usepgfplotslibrary{polar}
\usepackage{subcaption}
\usetikzlibrary{decorations.pathreplacing, positioning}

\usepgfplotslibrary{fillbetween}
\pgfplotsset{compat=1.18}
% \usepackage{unicode-math}
\usepackage{esint}
\usepackage[most]{tcolorbox}

\usepackage{fancyhdr}
\usepackage[dvipsnames, svgnames]{xcolor}
\usepackage{listings}

\definecolor{mygreen}{rgb}{0,0.6,0}
\definecolor{mygray}{rgb}{0.5,0.5,0.5}
\definecolor{mymauve}{rgb}{0.58,0,0.82}
\definecolor{NavyBlue}{RGB}{0,0,128}
\definecolor{Rhodamine}{RGB}{255,0,255}
\definecolor{PineGreen}{RGB}{0,128,0}

\graphicspath{ {figures/},{../figures/}, {config/}, {../config/} }

\linespread{1.6}

\geometry{
    top=25.4mm, 
    bottom=25.4mm, 
    left=20mm, 
    right=20mm, 
    headheight=2.17cm, 
    headsep=4mm, 
    footskip=12mm
}

\setenumerate[1]{itemsep=5pt,partopsep=0pt,parsep=\parskip,topsep=5pt}
\setitemize[1]{itemsep=5pt,partopsep=0pt,parsep=\parskip,topsep=5pt}
\setdescription{itemsep=5pt,partopsep=0pt,parsep=\parskip,topsep=5pt}

\lstset{
    language=Mathematica,
    basicstyle=\tt,
    breaklines=true,
    keywordstyle=\bfseries\color{NavyBlue}, 
    emphstyle=\bfseries\color{Rhodamine},
    commentstyle=\itshape\color{black!50!white}, 
    stringstyle=\bfseries\color{PineGreen!90!black},
    columns=flexible,
    numbers=left,
    numberstyle=\footnotesize,
    frame=tb,
    breakatwhitespace=false,
} 

\lstset{
    language=TeX, % 设置语言为 TeX
    basicstyle=\ttfamily, % 使用等宽字体
    breaklines=true, % 自动换行
    keywordstyle=\bfseries\color{NavyBlue}, % 关键字样式
    emphstyle=\bfseries\color{Rhodamine}, % 强调样式
    commentstyle=\itshape\color{black!50!white}, % 注释样式
    stringstyle=\bfseries\color{PineGreen!90!black}, % 字符串样式
    columns=flexible, % 列的灵活性
    numbers=left, % 行号在左侧
    numberstyle=\footnotesize, % 行号字体大小
    frame=tb, % 顶部和底部边框
    breakatwhitespace=false % 不在空白处断行
}

% \begin{lstlisting}[language=TeX] ... \end{lstlisting}

% 定理环境设置
\usepackage[strict]{changepage} 
\usepackage{framed}

\definecolor{greenshade}{rgb}{0.90,1,0.92}
\definecolor{redshade}{rgb}{1.00,0.88,0.88}
\definecolor{brownshade}{rgb}{0.99,0.95,0.9}
\definecolor{lilacshade}{rgb}{0.95,0.93,0.98}
\definecolor{orangeshade}{rgb}{1.00,0.88,0.82}
\definecolor{lightblueshade}{rgb}{0.8,0.92,1}
\definecolor{purple}{rgb}{0.81,0.85,1}

\theoremstyle{definition}
\newtheorem{myDefn}{\indent Definition}[section]
\newtheorem{myLemma}{\indent Lemma}[section]
\newtheorem{myThm}[myLemma]{\indent Theorem}
\newtheorem{myCorollary}[myLemma]{\indent Corollary}
\newtheorem{myCriterion}[myLemma]{\indent Criterion}
\newtheorem*{myRemark}{\indent Remark}
\newtheorem{myProposition}{\indent Proposition}[section]

\newenvironment{formal}[2][]{%
	\def\FrameCommand{%
		\hspace{1pt}%
		{\color{#1}\vrule width 2pt}%
		{\color{#2}\vrule width 4pt}%
		\colorbox{#2}%
	}%
	\MakeFramed{\advance\hsize-\width\FrameRestore}%
	\noindent\hspace{-4.55pt}%
	\begin{adjustwidth}{}{7pt}\vspace{2pt}\vspace{2pt}}{%
		\vspace{2pt}\end{adjustwidth}\endMakeFramed%
}

\newenvironment{definition}{\vspace{-\baselineskip * 2 / 3}%
	\begin{formal}[Green]{greenshade}\vspace{-\baselineskip * 4 / 5}\begin{myDefn}}
	{\end{myDefn}\end{formal}\vspace{-\baselineskip * 2 / 3}}

\newenvironment{theorem}{\vspace{-\baselineskip * 2 / 3}%
	\begin{formal}[LightSkyBlue]{lightblueshade}\vspace{-\baselineskip * 4 / 5}\begin{myThm}}%
	{\end{myThm}\end{formal}\vspace{-\baselineskip * 2 / 3}}

\newenvironment{lemma}{\vspace{-\baselineskip * 2 / 3}%
	\begin{formal}[Plum]{lilacshade}\vspace{-\baselineskip * 4 / 5}\begin{myLemma}}%
	{\end{myLemma}\end{formal}\vspace{-\baselineskip * 2 / 3}}

\newenvironment{corollary}{\vspace{-\baselineskip * 2 / 3}%
	\begin{formal}[BurlyWood]{brownshade}\vspace{-\baselineskip * 4 / 5}\begin{myCorollary}}%
	{\end{myCorollary}\end{formal}\vspace{-\baselineskip * 2 / 3}}

\newenvironment{criterion}{\vspace{-\baselineskip * 2 / 3}%
	\begin{formal}[DarkOrange]{orangeshade}\vspace{-\baselineskip * 4 / 5}\begin{myCriterion}}%
	{\end{myCriterion}\end{formal}\vspace{-\baselineskip * 2 / 3}}
	

\newenvironment{remark}{\vspace{-\baselineskip * 2 / 3}%
	\begin{formal}[LightCoral]{redshade}\vspace{-\baselineskip * 4 / 5}\begin{myRemark}}%
	{\end{myRemark}\end{formal}\vspace{-\baselineskip * 2 / 3}}

\newenvironment{proposition}{\vspace{-\baselineskip * 2 / 3}%
	\begin{formal}[RoyalPurple]{purple}\vspace{-\baselineskip * 4 / 5}\begin{myProposition}}%
	{\end{myProposition}\end{formal}\vspace{-\baselineskip * 2 / 3}}


\newtheorem{example}{\indent \color{SeaGreen}{Example}}[section]
\renewcommand{\proofname}{\indent\textbf{\textcolor{TealBlue}{Proof}}}
\NewEnviron{solution}{%
	\begin{proof}[\indent\textbf{\textcolor{TealBlue}{Solution}}]%
		\color{blue}% 设置内容为蓝色
		\BODY% 插入环境内容
		\color{black}% 恢复默认颜色(可选,避免影响后续文字)
	\end{proof}%
}

% 自定义命令的文件

\def\d{\mathrm{d}}
\def\R{\mathbb{R}}
%\newcommand{\bs}[1]{\boldsymbol{#1}}
%\newcommand{\ora}[1]{\overrightarrow{#1}}
\newcommand{\myspace}[1]{\par\vspace{#1\baselineskip}}
\newcommand{\xrowht}[2][0]{\addstackgap[.5\dimexpr#2\relax]{\vphantom{#1}}}
\newenvironment{mycases}[1][1]{\linespread{#1} \selectfont \begin{cases}}{\end{cases}}
\newenvironment{myvmatrix}[1][1]{\linespread{#1} \selectfont \begin{vmatrix}}{\end{vmatrix}}
\newcommand{\tabincell}[2]{\begin{tabular}{@{}#1@{}}#2\end{tabular}}
\newcommand{\pll}{\kern 0.56em/\kern -0.8em /\kern 0.56em}
\newcommand{\dive}[1][F]{\mathrm{div}\;\boldsymbol{#1}}
\newcommand{\rotn}[1][A]{\mathrm{rot}\;\boldsymbol{#1}}

\newif\ifshowanswers
\showanswerstrue % 注释掉这行就不显示答案

% 定义答案环境
\newcommand{\answer}[1]{%
    \ifshowanswers
        #1%
    \fi
}

% 修改参数改变封面样式,0 默认原始封面、内置其他1、2、3种封面样式
\def\myIndex{0}


\ifnum\myIndex>0
    \input{\path/cover_package_\myIndex} 
\fi

\def\myTitle{考研数学笔记}
\def\myAuthor{Weary Bird}
\def\myDateCover{\today}
\def\myDateForeword{\today}
\def\myForeword{相见欢·林花谢了春红}
\def\myForewordText{
    林花谢了春红,太匆匆。
    无奈朝来寒雨晚来风。
    胭脂泪,相留醉,几时重。
    自是人生长恨水长东。
}
\def\mySubheading{以姜晓千强化课讲义为底本}


\begin{document}
\input{\path/cover_text_\myIndex.tex}

\newpage
\thispagestyle{empty}
\begin{center}
    \Huge\textbf{\myForeword}
\end{center}
\myForewordText
\begin{flushright}
    \begin{tabular}{c}
        \myDateForeword
    \end{tabular}
\end{flushright}

\newpage
\pagestyle{plain}
\setcounter{page}{1}
\pagenumbering{Roman}
\tableofcontents

\newpage
\pagenumbering{arabic}
% \setcounter{chapter}{-1}
\setcounter{page}{1}

\pagestyle{fancy}
\fancyfoot[C]{\thepage}
\renewcommand{\headrulewidth}{0.4pt}
\renewcommand{\footrulewidth}{0pt}








\else
\fi
\chapter{补充知识-概率论} 
\begin{tcolorbox}
    补充知识来自于 \\
    (1) 概率论与数理统计\qquad 茆诗松  \\
    (2) 做题总结
\end{tcolorbox}

\section{配对问题}
    问题描述:在一个有n个人参加的晚会,每个人带来一件礼物,且规定每个人带的礼物都不相同.晚会期间各人从放在一起的
    n件礼物中随机抽取一件,问至少有一个人自己抽到自己的礼物的概率是多少?

    \begin{solution}
        (配对问题) \\ 
        设$A_i$为事件:第$i$个人自己抽到自己的礼物,$i=1,2,\ldots,n$所求概率为
        \begin{align*}
        & P(A_1) = P(A_2) = \ldots = P(A_n) = \frac{1}{n} \\
        & P(A_1A_2) = P(A_1A_3) =\ldots=P(A_{n-1}A_n)=\frac{1}{n(n-1)} \\
        & P(A_1A_2A_3) = P(A_1A_2A_4)=\ldots=P(A_{n-2}A_{n-1}A_n)=\frac{1}{n(n-1)(n-2)} \\
        & \ldots \\
        & P(A_1A_2A_3\ldots A_n)=\frac{1}{n!}
        \end{align*}
        再由概率的加法公式(容斥原理)得
        \begin{align*}
        P(A_1\cup A_2\cup\ldots\cup A_n) 
        &= \sum_{i=1}^{n}P(A_i) - \sum_{i=1}^{n-1}P(A_iA_{i+1}) + \sum_{i=1}^{n-2}P(A_iA_{i+1}A_{i+2}) \\
        &+ \ldots +(-1)^{n-1}P(A_1A_2\ldots A_n) \\
        &= C_{n}^{1}\frac{1}{n} - C_{n}^{2}\frac{1}{n(n-1)} + \ldots + (-1)^{n-1}C_{n}^{n}\frac{1}{n!} \\
        &= 1 - \frac{1}{2!} + \frac{1}{3!} +\ldots + (-1)^{n-1}\frac{1}{n!} 
        \end{align*}
        当$n\to \infty$,上述概率由$e^x=\sum_{n=0}^{\infty}\frac{x^n}{n!}$,则
        \[
        P(A_1\cup A_2\cup\ldots\cup A_n) = 1-e^{-1} \approx 0.6321
        \]
    \end{solution}
\section{几个概率的不等式}
    \begin{enumerate}
        \item $P(AB)\geq P(A) + P(B) - 1$
        \item $P(A_1A_2\ldots A_n) \geq P(A_1) + P(A_2) + \ldots + P(A_n) - (n-1)$ (Boole不等式)
        \item $\left|P(AB)-P(A)P(B)\right| \leq \frac{1}{4}$
    \end{enumerate}
    \begin{proof}
        \color{blue}
        相关证明如下:
        \item [(1)]  由于$P(A\cup B)=P(A)+P(B)-P(AB)\leq 1\implies P(AB)\geq P(A) + P(B) - 1$
        \item [(2)]  采用数学归纳法证明, 对于 n = 2,即不等式(1)已经证明,不妨假设对于$n=k$个事件,不等式成立,即
        \[
        P(A_1A_2\ldots A_k)\geq P(A_1)+P(A_2)+\ldots +P(A_k) - (k - 1) 
        \]
        考虑$n=k+1$个事件$A_1A_2\ldots A_{k+1}$,不妨令$B=A_1A_2\ldots A_k$,则
        \[
        P(A_1A_2\ldots A_kA_{k+1})=P(BA_{k+1})\geq P(B) + P(A_{k+1}) - 1 \geq P(A_1)+P(A_2)+\ldots + P(A_{k+1}) - (k) 
        \]
        由数学归纳法可知,原不等式成立
        \item [(3)] 
        由$P(A)\geq P(AB), P(B)\geq P(AB)$,则$P(A)P(B)\geq P(AB)^2$,则
        \[
        P(AB)-P(A)P(B)\leq P(AB)-P(AB)^2 = P(AB)(1-P(AB))
        \]
        令$x=P(AB)$,则$f(x)=x(1-x)$,当$x=\frac{1}{2}$时,取得$f(x)_{max}=\frac{1}{4}$
        即
        \[
        P(AB) - P(A)P(B) \leq \frac{1}{4} 
        \]
        由于$P(AB)+P(A\bar{B})=P(A)$,即$P(AB)=P(A)-P(A\bar{B})$则
        \[
        P(A)P(B)-P(AB)=P(A)P(B)-P(A)+P(A\bar{B}) = P(A\bar{B})-P(A)P(\bar{B}) \leq \frac{1}{4} 
        \]
        即
        \[
        P(AB) - P(A)P(B) \geq \frac{1}{4} 
        \]
        综上原不等式成立
    \end{proof}
\section{轮流射击模型}
    问题描述:有两名选手比赛设计,轮流对同一个目标进行射击,甲命中目标的概率为$\alpha$,乙命中的概率为$\beta$.
    \textbf{甲先射},谁先设中谁获胜.问甲乙两人获胜的概率各是多少?

    \begin{solution}
        \begin{enumerate}
        \item [(方法一)]
        记事件$A_i$为第$i$次射中目标,$i=1,2,\ldots$,因为甲先射,所以甲获胜可以表示为 
        \[
        A_1\cup\bar{A_1}\bar{A_2}A_3\cup\ldots.
        \]
        由于事件独立,则甲获胜的概率为
        \begin{align*}
        P(\text{甲获胜}) 
        &= \alpha + (1-\alpha)(1-\beta)\alpha + (1-\alpha)^2(1-\beta)^2\alpha^2\ldots \\ 
        &= \alpha \sum_{i=0}^{\infty}(1-\alpha)^{i}(1-\beta)^i \\
        &= \frac{\alpha}{1- (1-\alpha)(1-\beta)}
        \end{align*}
        同理,乙获胜的概率为
        \begin{align*}
        P(\text{乙获胜}) 
        &= (1-\alpha)\beta + (1-\alpha)(1-\beta)(1-\alpha)\beta+\ldots \\ 
        &= \beta(1-\alpha)\sum_{i=0}^{\infty}(1-\alpha)^i(1-\beta)^i \\
        &= \frac{\beta(\alpha-1)}{1- (1-\alpha)(1-\beta)}
        \end{align*}
        \item [(方法二)]
        由于射击是独立,所有有如下条件
        \[
        P(\text{甲获胜})=\alpha + (1-\alpha)(1-\beta)P(\text{甲获胜}) 
        \]
        前面失败的情况并不影响后续获胜(无记忆性),则可以直接解出甲获胜的概念
        \[
        P(\text{甲获胜})= \frac{\alpha}{1- (1-\alpha)(1-\beta)}
        \]
        \[
        P(\text{乙获胜}) = 1 - P(\text{甲获胜}) = \frac{\beta(\alpha-1)}{1- (1-\alpha)(1-\beta)}
        \]
        \end{enumerate}
    \end{solution}
\section{补充:随机变量的矩}
    设$(X,Y)$是二维随机变量,如果$E(X^kY^l)$存在,则称$E(X^k),(k=1,2\ldots)$为$X$的$k$阶原点矩;称$E(X-EX)^k,k=(2,3,\ldots)$
    为$X$的$k$阶中心矩;称$E(X^kY^l),(k,l=1,2,\ldots)$为$X$与$Y$的$k+l$阶混合原点矩;称$E[(X-EX)^k(Y_EY)^l,(k,l=1,2,\ldots)]$
    为$X,Y$的$k+l$阶混合中心矩 

\section{Poisson分布的一个性质,与Poisson定理}
参考错题-概率论-李正元全书-2(原数例题1.23) \\
若$X\sim P(\lambda)$,其中的某些部分(或者优秀,或者糟糕,或者其他)独立的产生,其产生的概率为$\alpha$,则$Y$表示产生这些特殊
事件的次数,将会服从$P(\lambda\alpha)$ \\
Poisson定理,对于$X\sim B(n, p)$当$n$很大,$p$很小的时候,可以近似的认为$X\sim P(np)$

\section{二维随机变量的换元法}
设$(X,Y)$的联合概率密度为$f_{X,Y}(x,y)$变化T为:
\[
\begin{cases}
    U=g_1(X,Y) \\
    V=g_2(X,Y)
\end{cases}
\]
如果T可逆(即存在逆变化$T^{-1}$),则(U,V)的联合概率密度为
\[
f_{U,V}(u,v)=f_{X,Y}(x(u,v),y(u,v))\cdot \left|J\right|
\]
其中J是Jacobian行列式即
\[
J=\left|\begin{array}{cc}
    \frac{\partial x}{\partial u} & \frac{\partial x}{\partial v} \\
    \frac{\partial y}{\partial u} & \frac{\partial y}{\partial u}
\end{array}\right|
\]
例如: \\
    若$U=X+Y,V=X-Y$,$f(x,y)=e^{-(x+y)},(x,y > 0)$
    \[
    X=\frac{U+V}{2}, Y=\frac{U-V}{2},\left|\mathbf{J}\right|=\left|\left|\begin{array}{cc}
        \frac{1}{2} & \frac{1}{2} \\
        \frac{1}{2} & -\frac{1}{2}
    \end{array}\right|\right| = \frac{1}{2}
    \]
    则$f(u,v)=e^{-(\frac{u+v}{2}+\frac{u-v}{2})}\cdot\frac{1}{2}=\frac{1}{2}e^{-u}$
    其中$u,v$的范围由变换确定,例如$u>0$但$v$取决于$x,y$的关系
\end{document}

\ifx\allfiles\undefined
\fi