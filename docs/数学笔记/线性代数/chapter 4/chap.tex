\ifx\allfiles\undefined
\documentclass[12pt, a4paper, oneside, UTF8]{ctexbook}
\usepackage{multirow}
\def\path{../../config}
\usepackage{amsthm}
\usepackage{amssymb}
\usepackage{array}
\usepackage{xcolor}
\usepackage{graphicx}
\usepackage{mathrsfs}
\usepackage{enumitem}
\usepackage{geometry}
\usepackage[colorlinks, linkcolor=black]{hyperref}
\usepackage{stackengine}
\usepackage{yhmath}
\usepackage{extarrows}
\usepackage{tikz}
\usepackage{forest}
\usetikzlibrary{decorations.pathreplacing, positioning}
% \usepackage{unicode-math}
\usepackage{esint}
\usepackage{pifont}
\usepackage{tcolorbox}
\tcbuselibrary{skins, breakable}

\usepackage{multicol} 
\usepackage{fontspec} % 使用字体

\setmainfont{Times New Roman}
\setCJKmainfont{LXGWWenKai-Light}[
    SlantedFont=*
]

\usepackage{listings} % 用于插入代码

% 定义代码高亮风格
\lstset{
    basicstyle=\ttfamily\small,        % 基本字体样式(等宽小字体)
    keywordstyle=\color{blue},         % 关键字颜色
    commentstyle=\color{green},        % 注释颜色
    stringstyle=\color{red},           % 字符串颜色
    numbers=none,
    breaklines=true,                   % 自动换行
    frame=single,                      % 代码框边框
    rulecolor=\color{black},           % 边框颜色
    captionpos=b,                      % 标题位置(底部)
    showspaces=false,                  % 不显示空格标记
    showstringspaces=false,            % 不显示字符串中的空格标记
    language=C                         % 设置语言为 C
}

\usepackage{fontawesome5}

\usepackage{amsmath}
\usepackage{booktabs, array}
\usepackage{makecell}
\usepackage{fancyhdr}
\usepackage[dvipsnames, svgnames]{xcolor}
\usepackage{listings}
\usepackage{tasks}[2020/01/11]

\everymath{\displaystyle}

\definecolor{mygreen}{rgb}{0,0.6,0}
\definecolor{mygray}{rgb}{0.5,0.5,0.5}
\definecolor{mymauve}{rgb}{0.58,0,0.82}
\definecolor{NavyBlue}{RGB}{0,0,128}
\definecolor{Rhodamine}{RGB}{255,0,255}
\definecolor{PineGreen}{RGB}{0,128,0}

\graphicspath{ {figures/},{../figures/}, {config/}, {../config/} }

\linespread{1.6}

\geometry{
    top=25.4mm, 
    bottom=25.4mm, 
    left=20mm, 
    right=20mm, 
    headheight=2.17cm, 
    headsep=4mm, 
    footskip=12mm
}

\setenumerate[1]{itemsep=5pt,partopsep=0pt,parsep=\parskip,topsep=5pt}
\setitemize[1]{itemsep=5pt,partopsep=0pt,parsep=\parskip,topsep=5pt}
\setdescription{itemsep=5pt,partopsep=0pt,parsep=\parskip,topsep=5pt}



% \begin{lstlisting}[language=TeX] ... \end{lstlisting}

% 定理环境设置
% ---------- 颜色 ----------
\definecolor{ExBlue}{HTML}{4F81BD}
\definecolor{SolGreen}{HTML}{77933C}
\definecolor{DefRed}{HTML}{C5504B}
\definecolor{ThmOrange}{HTML}{E97132}
\definecolor{RemGray}{HTML}{7F7F7F}
\definecolor{CorPurple}{HTML}{7030A0}
\definecolor{ForGray}{HTML}{595959}

% ---------- 通用“变色”模板 ----------
\tcbset{
    mybox/.style n args={1}{
        enhanced, breakable,
        arc=6pt,
        boxrule=0.6pt,
        left=8pt, right=8pt, top=6pt, bottom=6pt,
        drop shadow={black!25},
        fonttitle=\bfseries,
        coltitle=white,
        colbacktitle=#1!85,
        colback=#1!10,
        colframe=#1,
    }
}

% ---------- 各环境 ----------
% 例题
\newtcolorbox{example}[1][]{mybox={ExBlue}, title={\ifstrempty{#1}{Example}{#1}}}
% 解答
\newtcolorbox{solution}[1][]{mybox={SolGreen}, title={\ifstrempty{#1}{Solution}{#1}}}
% 定义
\newtcolorbox{definition}[1][]{mybox={DefRed}, title={\ifstrempty{#1}{Definition}{#1}}}
% 定理
\newtcolorbox{theorem}[1][]{mybox={ThmOrange}, title={\ifstrempty{#1}{Theorem}{#1}}}
% 标注
\newtcolorbox{remark}[1][]{mybox={RemGray}, title={\ifstrempty{#1}{Remark}{#1}}}
% 推论
\newtcolorbox{corollary}[1][]{mybox={CorPurple}, title={\ifstrempty{#1}{Corollary}{#1}}}
% 公式
\newtcolorbox{formula}[1][]{mybox={ForGray}, title={\ifstrempty{#1}{Formula}{#1}}}


\settasks{
    label-format = \bfseries,
    label        = \Alph*.,
    label-width  = 1.2em,
    label-offset = 0.3em,
    item-indent  = 1.9em,
    column-sep   = 0.5em
}

\newenvironment{choices}[1][4]   % 默认 4 栏
    {\begin{tasks}(#1)}
    {\end{tasks}}

% 自定义命令的文件

\def\d{\mathrm{d}}
\def\R{\mathbb{R}}
\def\P{\partial} 
\newcommand{\bs}[1]{\begin{solution}#1\end{solution}}
\newcommand{\bt}[1][1]{% 默认参数为1
    \ensuremath{% 确保数学模式
        \foreach \n in {1,...,#1} {\blacktriangle}% 循环输出 #1 个黑色三角形
    }%
}

\newcommand{\bl}[1][1]{% 默认参数为1
    \ensuremath{% 确保数学模式
        \foreach \n in {1,...,#1} {\blacklozenge}% 循环输出 #1 个黑色三角形
    }%
}
\newif\ifshowanswers
%\showanswerstrue % 注释掉这行就不显示答案

% 定义答案环境
\newcommand{\answer}[1]{%
    \ifshowanswers
        #1%
    \fi
}




% 修改参数改变封面样式,0 默认原始封面、内置其他1、2、3种封面样式
\def\myIndex{3}


\ifnum\myIndex>0
    \input{\path/cover_package_\myIndex} 
\fi

\def\myTitle{冲刺150笔记}
\def\myAuthor{Weary Bird}
\def\myDateCover{\today}
\def\myDateForeword{\today}
\def\myForeword{行香子}
\def\myForewordText{
树绕村庄,水满陂塘;倚东风、豪兴徜徉。小园几许,收尽春光。有桃花红,李花白,菜花黄。 \\
远远苔墙,隐隐茅堂;飏青旗、流水桥旁。偶然乘兴,步过东冈。正莺儿啼,燕儿舞,蝶儿忙。 \\
}
\def\mySubheading{知错能改善莫大焉}


\begin{document}
% \input{../config/cover}
\else
\fi
\chapter{线性方程组}

\section{解的判定}

\begin{enumerate}[label=\arabic*.]
    \item (2001,数三) 设 $A$ 为 $n$ 阶矩阵, $\alpha$ 为 $n$ 维列向量, 且 $\begin{pmatrix} A & \alpha \\ \alpha^T & 0 \end{pmatrix} = r(A)$,则线性方程组
    \begin{enumerate}
        \item (A) $Ax = \alpha$ 有无穷多解
        \item (B) $Ax = \alpha$ 有唯一解
        \item (C) $\begin{pmatrix} A & \alpha \\ \alpha^T & 0 \end{pmatrix} \begin{pmatrix} x \\ y \end{pmatrix} = 0$ 只有零解
        \item (D) $\begin{pmatrix} A & \alpha \\ \alpha^T & 0 \end{pmatrix} \begin{pmatrix} x \\ y \end{pmatrix} = 0$ 有非零解
    \end{enumerate}
    
    \begin{solution}
    【详解】
    \end{solution}
    
    \item 设 $A$ 为 $m \times n$ 阶矩阵, 且 $r(A) = m < n$,则下列结论不正确的是
    \begin{enumerate}
        \item (A) 线性方程组 $A^T x = 0$ 只有零解
        \item (B) 线性方程组 $A^T A x = 0$ 有非零解
        \item (C) $\forall b$,线性方程组 $A^T x = b$ 有唯一解
        \item (D) $\forall b$,线性方程组 $A x = b$ 有无穷多解
    \end{enumerate}
    
    \begin{solution}
    【详解】
    \end{solution}
\end{enumerate}

\section{求齐次线性方程组的基础解系与通解}

\begin{enumerate}[label=\arabic*.,start=2]
    \item (2011, 数一,二) 设 $A = (\alpha_1, \alpha_2, \alpha_3, \alpha_4)$ 为 4 阶矩阵, $(1,0,1,0)^T$ 为线性方程组 $Ax = 0$ 的基础解系,则 $A^* x = 0$ 的基础解系可为
    \begin{enumerate}
        \item (A) $\alpha_1, \alpha_2$
        \item (B) $\alpha_1, \alpha_3$
        \item (C) $\alpha_1, \alpha_2, \alpha_3$
        \item (D) $\alpha_2, \alpha_3, \alpha_4$
    \end{enumerate}
    
    \begin{solution}
    【详解】
    \end{solution}
    
    \item (2005, 数一、二) 设 3 阶矩阵 $A$ 的第 1 行为 $(a, b, c)$, $a, b, c$ 不全为零, $B = \begin{pmatrix} 2 & 4 & 6 \\ 3 & 6 & k \end{pmatrix}$ 满足 $AB = O$,求线性方程组 $Ax = 0$ 的通解。
    
    \begin{solution}
    【详解】
    \end{solution}
    
    \item (2002, 数三) 设线性方程组
    \begin{align*}
    a x_1 + b x_2 + b x_3 + \cdots + b x_n &= 0 \\
    b x_1 + a x_2 + b x_3 + \cdots + b x_n &= 0 \\
    \vdots \\
    b x_1 + b x_2 + b x_3 + \cdots + a x_n &= 0
    \end{align*}
    其中 $a \neq 0, b \neq 0, n \geq 2$。当 $a, b$ 为何值时,方程组只有零解、有非零解,当方程组有非零解时,求其通解。
    
    \begin{solution}
    【详解】
    \end{solution}
\end{enumerate}

\section{求非齐次线性方程组的通解}

\begin{enumerate}[label=\arabic*.,start=5]
    \item 设 $A$ 为 4 阶矩阵, $k$ 为任意常数, $\eta_1, \eta_2, \eta_3$ 为非齐次线性方程组 $Ax = b$ 的三个解, 满足
    \begin{align*}
    \eta_1 + \eta_2 &= \begin{pmatrix} 1 \\ 2 \\ 3 \\ 4 \end{pmatrix}, \quad \eta_2 + 2\eta_3 = \begin{pmatrix} 2 \\ 3 \\ 4 \\ 5 \end{pmatrix}.
    \end{align*}
    
    \begin{solution}
    【详解】
    \end{solution}
    
    \item (2017, 数一、三、三) 设 3 阶矩阵 $A = (\alpha_1', \alpha_2', \alpha_3')$ 有三个不同的特征值, 其中 $\alpha_3 = \alpha_1 + 2\alpha_2$。
    \begin{enumerate}
        \item (I) 证明 $r(A) = 2$;
        \item (II) 若 $\beta = \alpha_1 + \alpha_2 + \alpha_3$,求线性方程组 $Ax = \beta$ 的通解。
    \end{enumerate}
    
    \begin{solution}
    【详解】
    \end{solution}
    
    \item (I) 求 $\lambda, a$ 的值;
    \item (II) 求方程组 $Ax = b$ 的通解。
    
    \begin{solution}
    【详解】
    \end{solution}
    
    \item (I) $\eta$ 为非齐次线性方程组 $Ax = b$ 的特解, 证明:
    \begin{enumerate}
        \item (II) $\eta, \eta + \xi_1, \eta + \xi_2, \cdots, \eta + \xi_{n-r}$ 线性无关;
        \item (III) $\eta, \eta + \xi_1, \eta + \xi_2, \cdots, \eta + \xi_{n-r}$ 为 $Ax = b$ 所有解的极大线性无关组。
    \end{enumerate}
    
    \begin{solution}
    【详解】
    \end{solution}
\end{enumerate}

\section{解矩阵方程}

\begin{enumerate}[label=\arabic*.,start=9]
    \item 矩阵方程解的判定
    \begin{align*}
    AX = B \text{ 无解 } \Leftrightarrow r(A) < r(A|B) \\
    AX = B \text{ 有唯一解 } \Leftrightarrow r(A) = r(A|B) = n \\
    AX = B \text{ 有无穷多解 } \Leftrightarrow r(A) = r(A|B) < n
    \end{align*}
    
    \item 矩阵方程的求法
    对 $(A|B)$ 作初等行变换,化为行最简形矩阵,得矩阵 $X$。
    
    \item (例 4.10) 设 
    \begin{align*}
    A = \begin{pmatrix}
    1 & -2 & 3 & -4 \\
    0 & 1 & -1 & 1 \\
    1 & 2 & 0 & -3
    \end{pmatrix}
    \end{align*}
    矩阵 $X$ 满足 $AX + E = A^{2022} + 2X$,求矩阵 $X$。
    
    \begin{solution}
    【详解】
    \end{solution}
    
    \item (例 4.11) (2014, 数一、二、三) 设 
    \begin{align*}
    A = \begin{pmatrix}
    1 & -2 & 3 & -4 \\
    0 & 1 & -1 & 1 \\
    1 & 2 & 0 & -3
    \end{pmatrix}
    \end{align*}
    \begin{enumerate}
        \item (I) 求线性方程组 $Ax = 0$ 的一个基础解系;
        \item (II) 求满足 $AB = E$ 的所有矩阵 $B$。
    \end{enumerate}
    
    \begin{solution}
    【详解】
    \end{solution}
\end{enumerate}

\section{公共解的判定与计算}

\begin{enumerate}[label=\arabic*.,start=12]
    \item (2007, 数三) 设线性方程组
    \begin{align*}
    (I) \begin{cases}
    x_1 + x_2 + x_3 = 0 \\
    x_1 + 2x_2 + a x_3 = 0 \\
    x_1 + 4x_2 + a^2 x_3 = 0
    \end{cases}
    \end{align*}
    与方程
    \begin{align*}
    (II) x_1 + 2x_2 + x_3 = a - 1
    \end{align*}
    有公共解,求 $a$ 的值及所有公共解。
    
    \begin{solution}
    【详解】
    \end{solution}
    
    \item 设齐次线性方程组
    \begin{align*}
    (I) \begin{cases}
    2x_1 + 3x_2 - x_3 = 0 \\
    x_1 + 2x_2 + x_3 - x_4 = 0
    \end{cases}
    \end{align*}
    齐次线性方程组 (II) 的一个基础解系为 $\alpha_1 = (2, -1, a+2, 1)^T$, $\alpha_2 = (-1, 2, 4, a+8)^T$ 
    \begin{enumerate}
        \item (1) 求方程组 (I) 的一个基础解系;
        \item (2) 当 $a$ 为何值时,方程组 (I) 与 (II) 有非零公共解,并求所有非零公共解。
    \end{enumerate}
    
    \begin{solution}
    【详解】
    \end{solution}
    
    \item (2005,数三) 设线性方程组
    \begin{align*}
    (I) \begin{cases}
    x_1 + 2x_2 + 3x_3 = 0 \\
    2x_1 + 3x_2 + 5x_3 = 0 \\
    x_1 + x_2 + a x_3 = 0
    \end{cases}
    \end{align*}
    与 (II) 
    \begin{align*}
    \begin{cases}
    x_1 + b x_2 + c x_3 = 0 \\
    2x_1 + b^2 x_2 + (c+1) x_3 = 0
    \end{cases}
    \end{align*}
    同解,求 $a, b, c$ 的值。
    
    \begin{solution}
    【详解】
    \end{solution}
\end{enumerate}


\ifx\allfiles\undefined
\end{document}
\fi