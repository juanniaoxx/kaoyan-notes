\ifx\allfiles\undefined
\documentclass[12pt, a4paper, oneside, UTF8]{ctexbook}
\usepackage{multirow}

\def\path{../../config}
\usepackage{amsmath}
\usepackage{amsthm}
\usepackage{amssymb}
\usepackage{array}
\usepackage{xcolor}
\usepackage{graphicx}
\usepackage{mathrsfs}
\usepackage{enumitem}
\usepackage{geometry}
\usepackage[colorlinks, linkcolor=black]{hyperref}
\usepackage{stackengine}
\usepackage{yhmath}
\usepackage{extarrows}
\usepackage{tikz}
\usepackage{pgfplots}
\usepackage{asymptote}
\usepackage{float}
\usepackage{fontspec} % 使用字体

\setmainfont{Times New Roman}
\setCJKmainfont{LXGWWenKai-Light}[
    SlantedFont=*
]

\everymath{\displaystyle}

\usepgfplotslibrary{polar}
\usepackage{subcaption}
\usetikzlibrary{decorations.pathreplacing, positioning}

\usepgfplotslibrary{fillbetween}
\pgfplotsset{compat=1.18}
% \usepackage{unicode-math}
\usepackage{esint}
\usepackage[most]{tcolorbox}

\usepackage{fancyhdr}
\usepackage[dvipsnames, svgnames]{xcolor}
\usepackage{listings}

\definecolor{mygreen}{rgb}{0,0.6,0}
\definecolor{mygray}{rgb}{0.5,0.5,0.5}
\definecolor{mymauve}{rgb}{0.58,0,0.82}
\definecolor{NavyBlue}{RGB}{0,0,128}
\definecolor{Rhodamine}{RGB}{255,0,255}
\definecolor{PineGreen}{RGB}{0,128,0}

\graphicspath{ {figures/},{../figures/}, {config/}, {../config/} }

\linespread{1.6}

\geometry{
    top=25.4mm, 
    bottom=25.4mm, 
    left=20mm, 
    right=20mm, 
    headheight=2.17cm, 
    headsep=4mm, 
    footskip=12mm
}

\setenumerate[1]{itemsep=5pt,partopsep=0pt,parsep=\parskip,topsep=5pt}
\setitemize[1]{itemsep=5pt,partopsep=0pt,parsep=\parskip,topsep=5pt}
\setdescription{itemsep=5pt,partopsep=0pt,parsep=\parskip,topsep=5pt}

\lstset{
    language=Mathematica,
    basicstyle=\tt,
    breaklines=true,
    keywordstyle=\bfseries\color{NavyBlue}, 
    emphstyle=\bfseries\color{Rhodamine},
    commentstyle=\itshape\color{black!50!white}, 
    stringstyle=\bfseries\color{PineGreen!90!black},
    columns=flexible,
    numbers=left,
    numberstyle=\footnotesize,
    frame=tb,
    breakatwhitespace=false,
} 

\lstset{
    language=TeX, % 设置语言为 TeX
    basicstyle=\ttfamily, % 使用等宽字体
    breaklines=true, % 自动换行
    keywordstyle=\bfseries\color{NavyBlue}, % 关键字样式
    emphstyle=\bfseries\color{Rhodamine}, % 强调样式
    commentstyle=\itshape\color{black!50!white}, % 注释样式
    stringstyle=\bfseries\color{PineGreen!90!black}, % 字符串样式
    columns=flexible, % 列的灵活性
    numbers=left, % 行号在左侧
    numberstyle=\footnotesize, % 行号字体大小
    frame=tb, % 顶部和底部边框
    breakatwhitespace=false % 不在空白处断行
}

% \begin{lstlisting}[language=TeX] ... \end{lstlisting}

% 定理环境设置
\usepackage[strict]{changepage} 
\usepackage{framed}

\definecolor{greenshade}{rgb}{0.90,1,0.92}
\definecolor{redshade}{rgb}{1.00,0.88,0.88}
\definecolor{brownshade}{rgb}{0.99,0.95,0.9}
\definecolor{lilacshade}{rgb}{0.95,0.93,0.98}
\definecolor{orangeshade}{rgb}{1.00,0.88,0.82}
\definecolor{lightblueshade}{rgb}{0.8,0.92,1}
\definecolor{purple}{rgb}{0.81,0.85,1}

\theoremstyle{definition}
\newtheorem{myDefn}{\indent Definition}[section]
\newtheorem{myLemma}{\indent Lemma}[section]
\newtheorem{myThm}[myLemma]{\indent Theorem}
\newtheorem{myCorollary}[myLemma]{\indent Corollary}
\newtheorem{myCriterion}[myLemma]{\indent Criterion}
\newtheorem*{myRemark}{\indent Remark}
\newtheorem{myProposition}{\indent Proposition}[section]

\newenvironment{formal}[2][]{%
	\def\FrameCommand{%
		\hspace{1pt}%
		{\color{#1}\vrule width 2pt}%
		{\color{#2}\vrule width 4pt}%
		\colorbox{#2}%
	}%
	\MakeFramed{\advance\hsize-\width\FrameRestore}%
	\noindent\hspace{-4.55pt}%
	\begin{adjustwidth}{}{7pt}\vspace{2pt}\vspace{2pt}}{%
		\vspace{2pt}\end{adjustwidth}\endMakeFramed%
}

\newenvironment{definition}{\vspace{-\baselineskip * 2 / 3}%
	\begin{formal}[Green]{greenshade}\vspace{-\baselineskip * 4 / 5}\begin{myDefn}}
	{\end{myDefn}\end{formal}\vspace{-\baselineskip * 2 / 3}}

\newenvironment{theorem}{\vspace{-\baselineskip * 2 / 3}%
	\begin{formal}[LightSkyBlue]{lightblueshade}\vspace{-\baselineskip * 4 / 5}\begin{myThm}}%
	{\end{myThm}\end{formal}\vspace{-\baselineskip * 2 / 3}}

\newenvironment{lemma}{\vspace{-\baselineskip * 2 / 3}%
	\begin{formal}[Plum]{lilacshade}\vspace{-\baselineskip * 4 / 5}\begin{myLemma}}%
	{\end{myLemma}\end{formal}\vspace{-\baselineskip * 2 / 3}}

\newenvironment{corollary}{\vspace{-\baselineskip * 2 / 3}%
	\begin{formal}[BurlyWood]{brownshade}\vspace{-\baselineskip * 4 / 5}\begin{myCorollary}}%
	{\end{myCorollary}\end{formal}\vspace{-\baselineskip * 2 / 3}}

\newenvironment{criterion}{\vspace{-\baselineskip * 2 / 3}%
	\begin{formal}[DarkOrange]{orangeshade}\vspace{-\baselineskip * 4 / 5}\begin{myCriterion}}%
	{\end{myCriterion}\end{formal}\vspace{-\baselineskip * 2 / 3}}
	

\newenvironment{remark}{\vspace{-\baselineskip * 2 / 3}%
	\begin{formal}[LightCoral]{redshade}\vspace{-\baselineskip * 4 / 5}\begin{myRemark}}%
	{\end{myRemark}\end{formal}\vspace{-\baselineskip * 2 / 3}}

\newenvironment{proposition}{\vspace{-\baselineskip * 2 / 3}%
	\begin{formal}[RoyalPurple]{purple}\vspace{-\baselineskip * 4 / 5}\begin{myProposition}}%
	{\end{myProposition}\end{formal}\vspace{-\baselineskip * 2 / 3}}


\newtheorem{example}{\indent \color{SeaGreen}{Example}}[section]
\renewcommand{\proofname}{\indent\textbf{\textcolor{TealBlue}{Proof}}}
\NewEnviron{solution}{%
	\begin{proof}[\indent\textbf{\textcolor{TealBlue}{Solution}}]%
		\color{blue}% 设置内容为蓝色
		\BODY% 插入环境内容
		\color{black}% 恢复默认颜色(可选,避免影响后续文字)
	\end{proof}%
}

% 自定义命令的文件

\def\d{\mathrm{d}}
\def\R{\mathbb{R}}
%\newcommand{\bs}[1]{\boldsymbol{#1}}
%\newcommand{\ora}[1]{\overrightarrow{#1}}
\newcommand{\myspace}[1]{\par\vspace{#1\baselineskip}}
\newcommand{\xrowht}[2][0]{\addstackgap[.5\dimexpr#2\relax]{\vphantom{#1}}}
\newenvironment{mycases}[1][1]{\linespread{#1} \selectfont \begin{cases}}{\end{cases}}
\newenvironment{myvmatrix}[1][1]{\linespread{#1} \selectfont \begin{vmatrix}}{\end{vmatrix}}
\newcommand{\tabincell}[2]{\begin{tabular}{@{}#1@{}}#2\end{tabular}}
\newcommand{\pll}{\kern 0.56em/\kern -0.8em /\kern 0.56em}
\newcommand{\dive}[1][F]{\mathrm{div}\;\boldsymbol{#1}}
\newcommand{\rotn}[1][A]{\mathrm{rot}\;\boldsymbol{#1}}

\newif\ifshowanswers
\showanswerstrue % 注释掉这行就不显示答案

% 定义答案环境
\newcommand{\answer}[1]{%
    \ifshowanswers
        #1%
    \fi
}

% 修改参数改变封面样式,0 默认原始封面、内置其他1、2、3种封面样式
\def\myIndex{0}


\ifnum\myIndex>0
    \input{\path/cover_package_\myIndex} 
\fi

\def\myTitle{考研数学笔记}
\def\myAuthor{Weary Bird}
\def\myDateCover{\today}
\def\myDateForeword{\today}
\def\myForeword{相见欢·林花谢了春红}
\def\myForewordText{
    林花谢了春红,太匆匆。
    无奈朝来寒雨晚来风。
    胭脂泪,相留醉,几时重。
    自是人生长恨水长东。
}
\def\mySubheading{以姜晓千强化课讲义为底本}


\begin{document}
% \input{\path/cover_text_\myIndex.tex}

\newpage
\thispagestyle{empty}
\begin{center}
    \Huge\textbf{\myForeword}
\end{center}
\myForewordText
\begin{flushright}
    \begin{tabular}{c}
        \myDateForeword
    \end{tabular}
\end{flushright}

\newpage
\pagestyle{plain}
\setcounter{page}{1}
\pagenumbering{Roman}
\tableofcontents

\newpage
\pagenumbering{arabic}
% \setcounter{chapter}{-1}
\setcounter{page}{1}

\pagestyle{fancy}
\fancyfoot[C]{\thepage}
\renewcommand{\headrulewidth}{0.4pt}
\renewcommand{\footrulewidth}{0pt}








\else
\fi
\chapter{线性方程组}
\[
\fbox{
    \begin{tabular}{c}
        线 \\ 性 \\ 方 \\ 程 \\ 组
    \end{tabular}
} \begin{cases}
    \fbox{解的性质与判定} \begin{cases}
        \fbox{性质} \\
        \fbox{判定} \begin{cases}
            \fbox{$Ax=0$} \begin{cases}
                Ax=0\text{只有零解}\iff r(A)=n \\
                Ax=0\text{有非零解}\iff r(A)<n 
            \end{cases} \\
            \fbox{$Ax=b$} \begin{cases}
                Ax=b\text{无解}\iff r(A)<r(\bar{A}) \iff r(A)=r(\bar{A})-1 \\
                Ax=b\text{有唯一解}\iff r(A)=r(\bar{A})=n \\
                Ax=b\text{有无穷多解}\iff r(A)=r(\bar{A})<n
            \end{cases}
        \end{cases}
    \end{cases}\\
    \fbox{解的结构} \begin{cases}
        \fbox{$Ax=0$的通解} k_1\xi_1+k_2\xi_2+\ldots+k_{n-r}\xi_{n-r}\\
        \fbox{$Ax=b$的通解} k_1\xi_1+k_2\xi_2+\ldots+k_{n-r}\xi_{n-r}+\eta \\
    \end{cases} \\
    \fbox{矩阵方程} \begin{cases}
        \fbox{定义} AX=B \\
        \fbox{判定} \begin{cases}
            AX=B\text{无解}\iff r(A)<r(A\mid B) \\
            AX=B\text{有唯一解}\iff r(A)=r(A\mid B) = n \\
            AX+B\text{有无穷多解}\iff r(A)=r(A\mid B) < n
        \end{cases} \\
        \fbox{求法} (A\mid B)\xrightarrow{\text{初等行变换}}\text{行最简形矩阵}
    \end{cases}
\end{cases}
\]

$$
\fbox{
    \begin{tabular}{c}
        线 \\ 性 \\ 方 \\ 程 \\ 组
    \end{tabular}
}  \begin{cases}
        \fbox{公共解与同解} \begin{cases}
        \fbox{公共解} \begin{cases}
            \fbox{定义} \\
            \fbox{求法}
        \end{cases}\\
        \fbox{同解} \begin{cases}
            \fbox{定义} \\
            \fbox{充要条件} \begin{cases}
                A,B\text{的行向量组等价} \\
                r(A)=r\left(\begin{array}{c}
                    A \\
                    B
                \end{array}\right) = r(B)
            \end{cases}
        \end{cases}
    \end{cases}
\end{cases}
$$
\section{解的判定}
\begin{enumerate}
    \item (2001,数三) 设 $A$ 为 $n$ 阶矩阵, $\alpha$ 为 $n$ 维列向量, 且 $\begin{pmatrix} A & \alpha \\ \alpha^T & 0 \end{pmatrix} = r(A)$,则线性方程组
    \begin{enumerate}
        \item [(A)] $Ax = \alpha$ 有无穷多解
        \item [(B)] $Ax = \alpha$ 有唯一解
        \item [(C)] $\begin{pmatrix} A & \alpha \\ \alpha^T & 0 \end{pmatrix} \begin{pmatrix} x \\ y \end{pmatrix} = 0$ 只有零解
        \item [(D)] $\begin{pmatrix} A & \alpha \\ \alpha^T & 0 \end{pmatrix} \begin{pmatrix} x \\ y \end{pmatrix} = 0$ 有非零解
    \end{enumerate}
    
    \begin{solution}
    对于A,B选项有 
    $$
    r(A)\leq r(A,\alpha) \leq r\begin{pmatrix}
        A &\alpha \\
        \alpha^T & 0 
    \end{pmatrix} = r(A)
    $$ 只能得到$r(A)=r(A,\alpha)$但与$n$的关系无法得出,故$Ax=\alpha$有解,但无法确定是无穷解还是
    唯一解. (C,D)选项比较较大,有题设可以直接知道D正确. 
    \end{solution}
    
    \item 设 $A$ 为 $m \times n$ 阶矩阵, 且 $r(A) = m < n$,则下列结论不正确的是
    \begin{enumerate}
        \item [(A)] 线性方程组 $A^T x = 0$ 只有零解
        \item [(B)] 线性方程组 $A^T A x = 0$ 有非零解
        \item [(C)] $\forall b$,线性方程组 $A^T x = b$ 有唯一解
        \item [(D)] $\forall b$,线性方程组 $A x = b$ 有无穷多解
    \end{enumerate}
    
    \begin{solution}
    (A) $r(A^T)=r(A)=m \implies $只有零解 \\
    (B) $r(A^TA)=r(A)=m<n \implies$ 有非零解 \\
    (D) $m=r(A)\leq r(A,b)\leq \min\{m,m+1\} = m \implies r(A)=r(A,b)=m<n$有无穷多解 
    \end{solution}

    \begin{corollary}[行/列满秩总结]
        \text{行满秩}$A_{m\times n}, r(A)=m$
        \begin{enumerate}
            \item [(1)] 右乘行满秩满足消去律 
            \item [(2)] 右乘行满秩秩不变 $r(BA)=r(B)$ 
            \item [(3)] $A$的行向量组线性无关 
            \item [(4)] 非齐次方程组$Ax=b\implies r(A)=r(A,b)\implies$有解
        \end{enumerate}
        \text{列满秩}$A_{m\times n}, r(A)=n$
        \begin{enumerate}
            \item [(1)] 左乘列满秩满足消去律 
            \item [(2)] 左乘列满秩秩不变 $r(AB)=r(B)$ 
            \item [(3)] $A$的列向量组线性无关 
            \item [(4)] $Ax=O$只有零解 
            \item [(5)] $ABx=O$与$Bx=O$同解
        \end{enumerate}
    \end{corollary}
\end{enumerate}

\section{求齐次线性方程组的基础解系与通解}

\begin{enumerate}
    \item (2011, 数一,二) 设 $A = (\alpha_1, \alpha_2, \alpha_3, \alpha_4)$ 为 4 阶矩阵, $(1,0,1,0)^T$ 为线性方程组 $Ax = 0$ 的基础解系,则 $A^* x = 0$ 的基础解系可为
    \begin{enumerate}
        \item [(A)] $\alpha_1, \alpha_2$
        \item [(B)] $\alpha_1, \alpha_3$
        \item [(C)] $\alpha_1, \alpha_2, \alpha_3$
        \item [(D)] $\alpha_2, \alpha_3, \alpha_4$
    \end{enumerate}
    
    \begin{solution}
    由题设可知$n-r(A)=1\implies r(A)=3$ 且$\alpha_1+\alpha_3=O\implies \alpha_1,\alpha_3$线性相关,而$r(A)=3$其列向量
    的极大无关组个数为3,从而其一个极大无关组可以是$(\alpha_1,\alpha_2,\alpha_4)$ \\
    由$r(A)=3=n-1\implies r(A^*)=1$从而$A^*x=O$的基础解系中线性无关解的个数为$n-r(A^*)=3$个,由$A^*A=\left|A\right|E=O$ 
    从而有
    $$
    A^*(\alpha_1,\alpha_2,\alpha_3,\alpha_4)=O
    $$
    从而$A^*x=O$的基础解系可以是$(\alpha_1(\alpha_3),\alpha_2,\alpha_4)$
    \end{solution}
    
    \item (2005, 数一、二) 设 3 阶矩阵 $A$ 的第 1 行为 $(a, b, c)$, $a, b, c$ 不全为零, 
    $B = \begin{pmatrix} 1 & 2 &3 \\ 2 & 4 & 6 \\ 3 & 6 & k \end{pmatrix}$ 满足 $AB = O$,求线性方程组 $Ax = 0$ 的通解。
    
    \begin{solution}
    由于$AB=O$且$r(A)+r(B)\leq 3$ \\
    当$k\neq 9,r(B)=2$, $r(A)=1$,从而$Ax=0$的基础解系中线性无关解的个数为$3-r(A)=2$个,此时通解为 
    $k_1(1,2,3)^T+k_2(3,6,k)^T$ 其中$k_1,k_2$为任意常数 \\
    当$k=9$时候$r(B)=1$此时$r(A)=1\text{或者}r(A)=2$.  
    \begin{enumerate}
        \item [(1)] 当$r(A)=2<3$时,$3-r(A)=1$,此时基础解析中只有一个线性无关的解$\beta=(1,2,3)^T$通解为$k\beta$,k为任意常数 
        \item [(2)] 当$r(A)=1<3$时候,$A=\alpha^T\beta \rightarrow \begin{pmatrix}
            a & b & c \\
            0 & 0 & 0 \\
            0 & 0 & 0
        \end{pmatrix}$不妨设$a\neq 0$ 此时基础解系可以是$\xi_1=(-\frac{b}{a},1,0),\xi_2=(-\frac{c}{a},0,1)$ 从而通解为
        $k_1\xi_1+k_2\xi_2$其中$k_1,k_2$为任意常数 
    \end{enumerate}
    \end{solution}
    
    \item (2002, 数三) 设线性方程组
    \[
    \begin{cases}
    a x_1 + b x_2 + b x_3 + \cdots + b x_n &= 0 \\
    b x_1 + a x_2 + b x_3 + \cdots + b x_n &= 0 \\
    b x_1 + b x_2 + a x_3 + \cdots + b x_n &= 0 \\
    \vdots \\
    b x_1 + b x_2 + b x_3 + \cdots + a x_n &= 0
    \end{cases}
    \]
    其中 $a \neq 0, b \neq 0, n \geq 2$。当 $a, b$ 为何值时,方程组只有零解、有非零解,当方程组有非零解时,求其通解。
    
    \begin{solution}
    记系数矩阵为$A=\begin{pmatrix}
        a & b & \ldots & b \\
        b & a & \ldots & b \\
        \vdots & \vdots & \ldots & \vdots \\
        b & b & \ldots & a
    \end{pmatrix}$ 其行列式为$\left|A\right|=\begin{vmatrix}
        a & b & \ldots & b \\
        b & a & \ldots & b \\
        \vdots & \vdots & \ldots & \vdots \\
        b & b & \ldots & a
    \end{vmatrix}=[a+(n-1)b](a-b)^{n-1}$ 有
    $$
    \begin{cases}
        a\neq b \text{且} a+(n-1)b\neq 0 \implies \left|A\right|\neq 0 \text{此时齐次方程只有零解} \\
        a=b\neq 0, A\rightarrow \begin{pmatrix}
            1 & 1 & \ldots & 1 \\
            0 & 0 & \ldots & 0 \\
            \vdots & \vdots & \ldots & \vdots 
        \end{pmatrix} \\
        \text{此时基础解系为} \xi_1=(-1,0,\ldots,0)^T,\ldots,\xi_{n-1}=(-1,0,\ldots,1)^T \\
        a+(n-1)b=0, A\rightarrow \begin{pmatrix}
            1 & 0 & \ldots & -1 \\
            0 & 1 & \ldots & -1 \\
            \vdots & \vdots & \ldots & \vdots  \\
            0 & 0 & \ldots & 0
        \end{pmatrix} \text{此时基础解系为} \xi=(1,1,\ldots,1)^T
    \end{cases}
    $$
    \end{solution}
\end{enumerate}

\section{求非齐次线性方程组的通解}
\begin{remark}[求特解的方法]
    \begin{enumerate}
        \item 对于抽象矩阵,用定义和性质凑一个特解$\sum k_i\mu_i(\sum k_i = 1)$
        \item 对于数字矩阵,$\bar{A}\rightarrow \text{行最简型}$让自由变量取0 
    \end{enumerate}
\end{remark}
\begin{enumerate}
    \item 设 $A$ 为 4 阶矩阵, $k$ 为任意常数, $\eta_1, \eta_2, \eta_3$ 为非齐次线性方程组 $Ax = b$ 的三个解, 满足
    \begin{align*}
    \eta_1 + \eta_2 &= \begin{pmatrix} 1 \\ 2 \\ 3 \\ 4 \end{pmatrix}, \quad \eta_2 + 2\eta_3 = \begin{pmatrix} 2 \\ 3 \\ 4 \\ 5 \end{pmatrix}.
    \end{align*}
    若$r(A)=3$则$Ax=b$的通解为() \\
    $(A)\begin{pmatrix}
        1 \\
        2 \\
        3 \\
        4 
    \end{pmatrix} + k \begin{pmatrix}
        -1 \\
        0 \\
        1 \\
        2
    \end{pmatrix} 
    (B) \begin{pmatrix}
        2 \\
        3 \\
        4 \\
        5
    \end{pmatrix} + k\begin{pmatrix}
        1 \\
        2 \\
        0 \\
        1
    \end{pmatrix}
    (C) \begin{pmatrix}
        0 \\
        1 \\
        2 \\
        3
    \end{pmatrix} + k\begin{pmatrix}
        -1 \\
        0 \\
        1 \\
        2
    \end{pmatrix} 
    (D) \begin{pmatrix}
        1 \\
        1 \\
        1 \\
        1
    \end{pmatrix} + k\begin{pmatrix}
        1 \\
        2 \\
        0 \\
        1
    \end{pmatrix}
    $
    \begin{solution}
    由题设可知$r(A)=3$,可知$Ax=0$基础解系里面有$n-r(A)=4-3=1$个线性无关的向量.根据解的形式可知要凑一个$\sum k_i = 0$ 
    $$
    3(\mu_1+\mu_2)-2(\mu_2+2\mu_3) = \begin{pmatrix}
        -1 \\
        0 \\
        1 \\
        2
    \end{pmatrix}
    $$ 为基础解系, 凑一个$\sum k_i = 1$为特解,考虑选项可知 
    $$
    2(\mu_1+\mu_2)-(\mu_2+2\mu_3) = \begin{pmatrix}
        0 \\
        1 \\
        2 \\
        3
    \end{pmatrix}
    $$为特解,故其通解为 $$\begin{pmatrix}
        0 \\
        1 \\
        2 \\
        3 
    \end{pmatrix} + k\begin{pmatrix}
        -1 \\
        0 \\
        1\\
        2
    \end{pmatrix}$$
    \end{solution}
    
    \item (2017, 数一、三、三) 设 3 阶矩阵 $A = (\alpha_1, \alpha_2, \alpha_3)$ 有三个不同的特征值, 其中 $\alpha_3 = \alpha_1 + 2\alpha_2$。
    \begin{enumerate}
        \item [(I)] 证明 $r(A) = 2$;
        \item [(II)] 若 $\beta = \alpha_1 + \alpha_2 + \alpha_3$,求线性方程组 $Ax = \beta$ 的通解。
    \end{enumerate}
    
    \begin{solution}
    (1) 由于$A$具有三个不同的特征值可知$A$可相似对角化即$P^{-1}AP=\Lambda$ 故$r(A)=r(\Lambda)\geq 2$ 又因为
    $\alpha_3=\alpha_1+2\alpha_2$ 可知A的列向量组的极大无关组至多为2,故$r(A)\leq 2$ 综上$r(A)=2$  \\
    (2) 由于$r(A)=2,Ax=0$的基础解系里有$n-r(A)=3-2=1$个线性无关的向量, 又因为
    $$
    \alpha_1 + 2\alpha_2 - \alpha_3 = A\begin{pmatrix}
        1 \\
        2 \\
        -1 
    \end{pmatrix} = 0
    $$
    故基础解系为$\xi\begin{pmatrix}
        1 \\
        2 \\
        -1
    \end{pmatrix}$ 又因为 $\beta = A\begin{pmatrix}
        1 \\
        1 \\
        1 
    \end{pmatrix} = A\mu$ 故通解为 $\mu+k\xi$,其中k为任意常数
    \end{solution}
    
    \item 设$A=\begin{pmatrix}
        \lambda & 1 & 1 \\
        0 & \lambda - 1 & 0 \\
        1 & 1 & \lambda 
    \end{pmatrix}, b = \begin{pmatrix}
        a \\
        1 \\
        1
    \end{pmatrix}$, 线性方程组$Ax=b$有两个不同的解.

    \begin{enumerate}
        \item [(I)] 求 $\lambda, a$ 的值;
        \item [(II)] 求方程组 $Ax = b$ 的通解。
    \end{enumerate}
    
    \begin{solution}
    (1) 有题设可知$Ax=b$有无穷多解,即$r(A)=r{\bar{(A)}}<3$ 对增广矩阵做初等行变换有 
    $$
    \bar{A}\rightarrow \begin{pmatrix}
        1 & 1 & \lambda & 1 \\
        0 & \lambda -1 & 0 & 1 \\
        0 & 0 & 1-\lambda^2 & a+1-\lambda
    \end{pmatrix} \implies \begin{cases}
        \lambda = -1\\
        a = -2
    \end{cases}
    $$
    (2) 将$\bar{A}$经过初等行变换转换为行最简型有 
    $$
    \bar{A}\rightarrow\begin{pmatrix}
        1 & 0 & -1 & \frac{3}{2} \\
        0 & 1 & 0 & -\frac{1}{2} \\
        0 & 0 & 0 & 0
    \end{pmatrix}
    $$
    可知其基础解系和特解分别为 
    $$
    \xi=\begin{pmatrix}
        1 \\
        0 \\
        1
    \end{pmatrix}, \eta = \begin{pmatrix}
        \frac{3}{2} \\
        -\frac{1}{2} \\
        0
    \end{pmatrix}
    $$ 故该方程组的通解为$\eta+k\xi$,其中k为任意常数 
    \end{solution}
    
    \item 设$A$为$m\times n$阶矩阵,且 $r(A)=r$,若$\xi_1\xi_2\ldots\xi_{n-r}$为齐次方程组$Ax=0$的基础解系,$\eta$为非其次
    线性方程组$Ax=b$的特解,证明:
    \begin{enumerate}
        \item [(I)] $\eta,\xi_1,\xi_2,\ldots,\xi_{n-r}$线性无关
        \item [(II)] $\eta, \eta + \xi_1, \eta + \xi_2, \cdots, \eta + \xi_{n-r}$ 线性无关;
        \item [(III)] $\eta, \eta + \xi_1, \eta + \xi_2, \cdots, \eta + \xi_{n-r}$ 为 $Ax = b$ 所有解的极大线性无关组。
    \end{enumerate}
    
    \begin{solution}
    (1)用定义证明,设$\exists k_1,\ldots,k_{n-r}$使得 
    \begin{align*}
        k_0\eta+k_1\xi_1+\ldots+k_{n-r}\xi_{n-r}=0 \tag{*}
    \end{align*}
    *式左乘A,可知$k_0 b = 0$又$b\neq 0$ 故$k_0=0$将其值带回*式可知
    $$
    k_1\xi_1+\ldots+k_{n-r}\xi_{n-r} = 0
    $$
    又因为$\xi_i$之间线性无关,可知$k_1=\ldots=k_{n-r}=0$ 故由线性无关的定义可知$\eta,\xi_1,\ldots,\xi_{n-r}$线性无关.  \\
    (2) 方法一:用定义 \\
    设$\exists l_0,\ldots,l_{n-r}$使得 
    $$
    l_0\eta + l_1(\eta+\xi_1) + \ldots + l_{n-r}(\eta+\xi_{n-t}) = 0
    $$
    即
    $$
    (l_0+\ldots+l_{n-r})\eta + l_1\xi_1 + \ldots + l_{n-r}\xi_{n-r} = 0
    $$
    由以可知上面的系数都为0,即$l_i = 0$ 从而原命题成立 \\
    方法二:用秩证明
    \begin{align*}
        &(\eta, \eta+\xi_1,\ldots,\eta+\xi_{n-r}) \\
        &=(\eta,\xi_1,\ldots,\xi_{n-r})\begin{pmatrix}
            1 & 1 & \ldots & 1 \\
            0 & 1 & \ldots & 0 \\
            \vdots & \vdots & \ldots & \vdots \\
            0 & 0 & \ldots & 1 
        \end{pmatrix} \\
        & = (\eta,\xi_1,\ldots,\xi_{n-r}) A_{(n-r+1)\times(n-r+1)}
    \end{align*}
    有(1)可知$(\eta,\xi_1,\ldots,\xi_{n-r})$线性无关,即列满秩,故有
    $$
    r(\eta, \eta+\xi_1,\ldots,\eta+\xi_{n-r}) = r(A) = n - r + 1
    $$
    由线性无关的充要条件可知,该向量组线性无关.  \\
    (3) 由(2)可知$(\eta, \eta+\xi_1,\ldots,\eta+\xi_{n-r})$为方程$Ax=b$线性无关的解,且$\eta,\xi_1,\ldots,\xi_{n-r}$可由
    其线性表示,并且$\eta,\xi_1,\ldots,\xi_{n-r}$可表示所有解.从而可知$(\eta, \eta+\xi_1,\ldots,\eta+\xi_{n-r})$亦可以表示
    所有解,故而其为所有解的极大线性无关组. 
    \end{solution}
    \begin{definition}[(非)齐次方程解的个数]
        齐次方程组$Ax=0$的基础解系(解的极大无关组)中解的个数为$n-r$ \\
        有上题的(3)可知,方程$Ax=b$解的极大无关组中解的个数为$n-r+1$  
    \end{definition}

    \item 设3阶非零矩阵$A$满足$A^2=O$,非齐次线性方程$Ax=b$有解,则$Ax=b$的线性无关解向量的个数为
    \_\_\_\_ 

    \begin{solution}
        由$A^2 = A\cdot A = O \implies r(A) + r(A) \leq 3 \implies r(A)\leq 1$又因为$A\neq O$可知 
        $r(A) = 1$,由上述结论可知$Ax=b$的线性无关解的个数为$n-r(A)+1 = 3 - 1 + 1 = 3$个. 
    \end{solution} 

    \item 设n阶矩阵$A$的伴随矩阵$A^*\neq O,\xi_1,\xi_2,\xi_3,\xi_4$为非齐次线性方程组$Ax=b$的互不相等
    的解,则$Ax=b$的线性无关解向量的个数为\_\_\_\_  

    \begin{solution}
        由$A^*\neq O \implies r(A^*)\geq 1 \implies r(A) = n-1\text{或} n$,有题设可知$Ax=b$有无穷多解,故$r(A)=r(\bar{A})<n$
        从而$r(A)=n-1$,由结论可知$Ax=b$的线性无关解的个数为$n-r(A)+1 = n - n + 1 + 1 = 2$个
    \end{solution}
\end{enumerate}

\section{解矩阵方程}
\begin{remark}[解$Ax=B$三种方法]
    (方法一) 若$A$可逆,此时$X=A^{-1}B$ 
    \begin{enumerate}
        \item [(i)] 先求$A^{-1}$,再做$A^{-1}B$ 一般不用 
        \item [(ii)] 联立做初等行变换 $(A\mid B)\xrightarrow{\text{初等行变换}}(E\mid A^{-1}B)$
    \end{enumerate}
    (方法二) 若$A$不可逆,且是二阶的时候直接待定系数 
    $$
    X=\begin{pmatrix}
        x_1 & x_2 \\
        x_3 & x_4 
    \end{pmatrix}
    $$
    (方法三) 若$A$不可逆且,大于二阶.用分块(按列)矩阵乘法  
    $$
    A(x_1,x_2\ldots,x_n)=(\beta_1,\beta_2,\ldots,\beta_n) \implies \begin{cases}
        Ax_1 = \beta_1 \\
        Ax_2 = \beta_2 \\
        \vdots \\
        Ax_n = \beta_n 
    \end{cases}
    $$
    转换为求解非齐次方程组,此时联立 
    $$
    (A\mid \beta_1,\beta_2,\ldots,\beta_n) \xrightarrow{\text{初等行变换}}\text{行最简型}
    $$
\end{remark}
\underline{变种}若矩阵方程为$XA=B$则转换为$A^TX^T=B^T$

\begin{enumerate}
    \item 设$A=\begin{pmatrix}
        -1 & 0 & 1\\
        1 & 0 & -1 \\
        -2 & 0 & 2
    \end{pmatrix}$矩阵 $X$ 满足 $AX + E = A^{2022} + 2X$,求矩阵 $X$。
    
    \begin{solution}
    有题$r(A)=1$可知$A^{2022}=[tr(A)]^{2021}A=A$此时原矩阵方程可以转换为
    $$
    (A-2E)X=A-E
    $$
    此时联立,做初等行变换 
    $$
    \begin{pmatrix}
        -3 & 0 & 1 & -2 & 0 & 1 \\
        1 & -2 & -1 & 1 & -1 & -1 \\
        -2 & 0 & 0 & -2 & 0 & 1
    \end{pmatrix} \rightarrow \begin{pmatrix}
        E & \begin{pmatrix}
            1 & 0 & -\frac{1}{2} \\
            -\frac{1}{2}  & \frac{1}{2} & \frac{1}{2} \\
            1 & 0 & -\frac{1}{2}
        \end{pmatrix}
    \end{pmatrix}
    $$
    \end{solution}
    
    \item (2014, 数一、二、三) 设 
    $
    A = \begin{pmatrix}
    1 & -2 & 3 & -4 \\
    0 & 1 & -1 & 1 \\
    1 & 2 & 0 & -3
    \end{pmatrix}
    $
    \begin{enumerate}
        \item [(I)] 求线性方程组 $Ax = 0$ 的一个基础解系;
        \item [(II)] 求满足 $AB = E$ 的所有矩阵 $B$。
    \end{enumerate}
    
    \begin{solution}
    直接联立$A,E$做初等行变换,可以一次把两道题一起做了. 
    $$
    (A\mid E) \rightarrow \left(
        \begin{array}{@{}c|c@{}}
        \begin{matrix}
            1 & 0 & 0 & 1 \\
            0 & 1 & 0 & -2 \\
            0 & 0 & 1 & -3
        \end{matrix} &
        \begin{matrix}
            2 & 6 & -1 \\
            -1 & -3 & 1 \\
            -1 & -4 & 1
        \end{matrix}
        \end{array}
        \right)
    $$
    通过左边的矩阵可以解出基础解系为$\xi=\begin{pmatrix}
        -1 \\
        2 \\
        3 \\
        1
    \end{pmatrix}$ 通过右边的矩阵,可以解出B,此时结果为 
    $$
    B = \begin{pmatrix}
        2-k_1 & 6 - k_2 & -2-k_3 \\
        2k_1-1 & 2k_2-3 & 2k_3+1 \\
        3k_1-1 & 3k_2-4 & 3k_3+1 \\
        k_1 & k_2 & k_3 
    \end{pmatrix}
    $$
    其中$k_1,k_2,k_3$为任意常数
    \end{solution}
    \begin{remark}[分块矩阵解矩阵方程的注意点]
        解非齐次方程时候,自由变量取$k$,解其余变量.
    \end{remark}
\end{enumerate}

\section{公共解的判定与计算}
\begin{remark}[公共解的三种情况]
    (情况一)已知两个方程组(直接联立) \\
    (情况二)已知一个方程组与另一个方程组的通解,将该通解带入方程组 \\
    (情况三)已知两个方程组的通解(令通解相等) 
\end{remark}

\begin{enumerate}[label=\arabic*.,start=12]
    \item (2007, 数三) 设线性方程组
    \begin{align*}
    (I) \begin{cases}
    x_1 + x_2 + x_3 = 0 \\
    x_1 + 2x_2 + a x_3 = 0 \\
    x_1 + 4x_2 + a^2 x_3 = 0
    \end{cases}
    \end{align*}
    与方程
    \begin{align*}
    (II) x_1 + 2x_2 + x_3 = a - 1
    \end{align*}
    有公共解,求 $a$ 的值及所有公共解。
    
    \begin{solution}
    直接联立$I,II$有
    $$
    \bar{A} = \begin{pmatrix}
        1 & 1 & 1 & 0 \\
        1 & 2 & a & 0 \\
        1 & 4 & a^2 & 0 \\
        1 & 2 & 1 & a - 1
    \end{pmatrix} \rightarrow \begin{pmatrix}
        1 & 1 & 1 & 0 \\
        0 & 1 & a - 1 & 0 \\
        0 & 0 & (a-1)(a-2)& 0 \\
        0 & 0 & 1-a & a-1
    \end{pmatrix}
    $$
    此时讨论参数a的值 \\
    (当$a\neq 1$且$a\neq 2$) 此时$r(A)<r(\bar{A})$ 无公共解 \\
    (当$a=1$时)公共解为$k(-1,0,1)^{T}$其中k为任意常数  \\
    (当$a=2$)只有唯一解$(0,1,-1)^{T}$
    \end{solution}
    
    \item 设齐次线性方程组
    \begin{align*}
    (I) \begin{cases}
    2x_1 + 3x_2 - x_3 = 0 \\
    x_1 + 2x_2 + x_3 - x_4 = 0
    \end{cases}
    \end{align*}
    齐次线性方程组 (II) 的一个基础解系为 $\alpha_1 = (2, -1, a+2, 1)^T$, $\alpha_2 = (-1, 2, 4, a+8)^T$ 
    \begin{enumerate}
        \item [(1)] 求方程组 (I) 的一个基础解系;
        \item [(2)] 当 $a$ 为何值时,方程组 (I) 与 (II) 有非零公共解,并求所有非零公共解。
    \end{enumerate}
    
    \begin{solution}
    (I)比较简单答案是$k_1(5,-3,1,0)^{T}+k_2(-3,2,0,1)^T$其中$k_1,k_2$为任意常数 \\
    (II,方法一) 令$k_1\xi_1+k_2\xi_2=k_3\alpha_1+k_4\alpha_4$ 则有 
    $$
    k_1\xi_1+k_2\xi_2-k_3\alpha_1-k_4\alpha_4 = 0 
    $$
    可以转换为求解齐次方程组$(\xi_1,\xi_2,-\alpha_1,-\alpha_2)k=0$的解 \\
    (II,方法二)将$$
    m_1\alpha_1+m_2\alpha_2 = \begin{pmatrix}
        2m_1-m_2 \\
        2m_2-m_1 \\
        (a+2)m_1 + 4m_2 \\
        m_1 + (a+8)m_2
    \end{pmatrix}\begin{array}{c}
        x_1 \\
        x_2 \\
        x_3 \\
        x_4
    \end{array}
    $$
    代入方程组(I)有$\begin{cases}
        (a+1)m_1 = 0 \\
        (a+1)m_2 = 0
    \end{cases}$
    当$a\neq -1$时候,$m_1=m_2=0$此时只有零解不合题意舍去 \\
    当$a=-1$时候,非零公共解为$m_1\alpha_1+m_2\alpha_2$其中$m_1,m_2$为任意常数
    \end{solution}
\end{enumerate}

\section{方程组同解}
\begin{remark}[同解问题的求法]
    (1)方程组同解的定义 \\
    (2)秩(三秩相等) $r(A)=r\begin{pmatrix}
        A \\
        B
    \end{pmatrix}=r(B)$ 即行向量组等价
\end{remark}
\begin{enumerate}
    \item (2005,数三) 设线性方程组
    \begin{align*}
    (I) \begin{cases}
    x_1 + 2x_2 + 3x_3 = 0 \\
    2x_1 + 3x_2 + 5x_3 = 0 \\
    x_1 + x_2 + a x_3 = 0
    \end{cases}
    \end{align*}
    与 (II) 
    \begin{align*}
    \begin{cases}
    x_1 + b x_2 + c x_3 = 0 \\
    2x_1 + b^2 x_2 + (c+1) x_3 = 0
    \end{cases}
    \end{align*}
    同解,求 $a, b, c$ 的值,并求出同解. 
    
    \begin{solution}
    联立A,B有 
    $$
    \begin{pmatrix}
        A \\
        B
    \end{pmatrix} \rightarrow \begin{pmatrix}
        1 & 2 & 3 \\
        0 & 1 & 1 \\
        0 & 0 & a-2 \\
        \hline 
        0 & 0 & c-b-1 \\
        0 & 0 & c-b^2-1  
    \end{pmatrix}
    $$
    {\color{red}不要忘记单独讨论B的秩},由方程组同解可知$r(A)=r\begin{pmatrix}
        A \\
        B 
    \end{pmatrix} = r(B)$且显然由$r(A)\geq 2, r(B)\leq 2$ 秩应该为2,此时可以解出 
    $$
    \begin{cases}
        a = 2 \\
        b = 0 \\
        c = 1
    \end{cases} \text{或} \begin{cases}
        a = 2 \\
        b = 1 \\
        c = 2
    \end{cases}
    $$ 注意当$\begin{cases}
        a = 2 \\
        b = 0 \\
        c = 1
    \end{cases}$时$r(B)=1$不满足条件,应该舍去. 由于它们都同解,随便解一个方程即可. 不妨解$Ax=0$,即
    $$
    A\rightarrow \begin{pmatrix}
        1 & 0 & 1 \\
        0 & 1 & 1 \\
        0 & 0 & 0
    \end{pmatrix}
    $$ 可知基础解系为$\xi=(1,1,-1)^T$故两个方程的同解为$k\xi$,$k$为任意常数.
    \end{solution}
\end{enumerate}

\ifx\allfiles\undefined
\end{document}
\fi