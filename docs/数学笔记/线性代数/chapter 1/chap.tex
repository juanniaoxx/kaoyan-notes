\ifx\allfiles\undefined
\documentclass[12pt, a4paper, oneside, UTF8]{ctexbook}
\usepackage{multirow}
\def\path{../../config}
\usepackage{amsmath}
\usepackage{amsthm}
\usepackage{amssymb}
\usepackage{array}
\usepackage{xcolor}
\usepackage{graphicx}
\usepackage{mathrsfs}
\usepackage{enumitem}
\usepackage{geometry}
\usepackage[colorlinks, linkcolor=black]{hyperref}
\usepackage{stackengine}
\usepackage{yhmath}
\usepackage{extarrows}
\usepackage{tikz}
\usepackage{pgfplots}
\usepackage{asymptote}
\usepackage{float}
\usepackage{fontspec} % 使用字体

\setmainfont{Times New Roman}
\setCJKmainfont{LXGWWenKai-Light}[
    SlantedFont=*
]

\everymath{\displaystyle}

\usepgfplotslibrary{polar}
\usepackage{subcaption}
\usetikzlibrary{decorations.pathreplacing, positioning}

\usepgfplotslibrary{fillbetween}
\pgfplotsset{compat=1.18}
% \usepackage{unicode-math}
\usepackage{esint}
\usepackage[most]{tcolorbox}

\usepackage{fancyhdr}
\usepackage[dvipsnames, svgnames]{xcolor}
\usepackage{listings}

\definecolor{mygreen}{rgb}{0,0.6,0}
\definecolor{mygray}{rgb}{0.5,0.5,0.5}
\definecolor{mymauve}{rgb}{0.58,0,0.82}
\definecolor{NavyBlue}{RGB}{0,0,128}
\definecolor{Rhodamine}{RGB}{255,0,255}
\definecolor{PineGreen}{RGB}{0,128,0}

\graphicspath{ {figures/},{../figures/}, {config/}, {../config/} }

\linespread{1.6}

\geometry{
    top=25.4mm, 
    bottom=25.4mm, 
    left=20mm, 
    right=20mm, 
    headheight=2.17cm, 
    headsep=4mm, 
    footskip=12mm
}

\setenumerate[1]{itemsep=5pt,partopsep=0pt,parsep=\parskip,topsep=5pt}
\setitemize[1]{itemsep=5pt,partopsep=0pt,parsep=\parskip,topsep=5pt}
\setdescription{itemsep=5pt,partopsep=0pt,parsep=\parskip,topsep=5pt}

\lstset{
    language=Mathematica,
    basicstyle=\tt,
    breaklines=true,
    keywordstyle=\bfseries\color{NavyBlue}, 
    emphstyle=\bfseries\color{Rhodamine},
    commentstyle=\itshape\color{black!50!white}, 
    stringstyle=\bfseries\color{PineGreen!90!black},
    columns=flexible,
    numbers=left,
    numberstyle=\footnotesize,
    frame=tb,
    breakatwhitespace=false,
} 

\lstset{
    language=TeX, % 设置语言为 TeX
    basicstyle=\ttfamily, % 使用等宽字体
    breaklines=true, % 自动换行
    keywordstyle=\bfseries\color{NavyBlue}, % 关键字样式
    emphstyle=\bfseries\color{Rhodamine}, % 强调样式
    commentstyle=\itshape\color{black!50!white}, % 注释样式
    stringstyle=\bfseries\color{PineGreen!90!black}, % 字符串样式
    columns=flexible, % 列的灵活性
    numbers=left, % 行号在左侧
    numberstyle=\footnotesize, % 行号字体大小
    frame=tb, % 顶部和底部边框
    breakatwhitespace=false % 不在空白处断行
}

% \begin{lstlisting}[language=TeX] ... \end{lstlisting}

% 定理环境设置
\usepackage[strict]{changepage} 
\usepackage{framed}

\definecolor{greenshade}{rgb}{0.90,1,0.92}
\definecolor{redshade}{rgb}{1.00,0.88,0.88}
\definecolor{brownshade}{rgb}{0.99,0.95,0.9}
\definecolor{lilacshade}{rgb}{0.95,0.93,0.98}
\definecolor{orangeshade}{rgb}{1.00,0.88,0.82}
\definecolor{lightblueshade}{rgb}{0.8,0.92,1}
\definecolor{purple}{rgb}{0.81,0.85,1}

\theoremstyle{definition}
\newtheorem{myDefn}{\indent Definition}[section]
\newtheorem{myLemma}{\indent Lemma}[section]
\newtheorem{myThm}[myLemma]{\indent Theorem}
\newtheorem{myCorollary}[myLemma]{\indent Corollary}
\newtheorem{myCriterion}[myLemma]{\indent Criterion}
\newtheorem*{myRemark}{\indent Remark}
\newtheorem{myProposition}{\indent Proposition}[section]

\newenvironment{formal}[2][]{%
	\def\FrameCommand{%
		\hspace{1pt}%
		{\color{#1}\vrule width 2pt}%
		{\color{#2}\vrule width 4pt}%
		\colorbox{#2}%
	}%
	\MakeFramed{\advance\hsize-\width\FrameRestore}%
	\noindent\hspace{-4.55pt}%
	\begin{adjustwidth}{}{7pt}\vspace{2pt}\vspace{2pt}}{%
		\vspace{2pt}\end{adjustwidth}\endMakeFramed%
}

\newenvironment{definition}{\vspace{-\baselineskip * 2 / 3}%
	\begin{formal}[Green]{greenshade}\vspace{-\baselineskip * 4 / 5}\begin{myDefn}}
	{\end{myDefn}\end{formal}\vspace{-\baselineskip * 2 / 3}}

\newenvironment{theorem}{\vspace{-\baselineskip * 2 / 3}%
	\begin{formal}[LightSkyBlue]{lightblueshade}\vspace{-\baselineskip * 4 / 5}\begin{myThm}}%
	{\end{myThm}\end{formal}\vspace{-\baselineskip * 2 / 3}}

\newenvironment{lemma}{\vspace{-\baselineskip * 2 / 3}%
	\begin{formal}[Plum]{lilacshade}\vspace{-\baselineskip * 4 / 5}\begin{myLemma}}%
	{\end{myLemma}\end{formal}\vspace{-\baselineskip * 2 / 3}}

\newenvironment{corollary}{\vspace{-\baselineskip * 2 / 3}%
	\begin{formal}[BurlyWood]{brownshade}\vspace{-\baselineskip * 4 / 5}\begin{myCorollary}}%
	{\end{myCorollary}\end{formal}\vspace{-\baselineskip * 2 / 3}}

\newenvironment{criterion}{\vspace{-\baselineskip * 2 / 3}%
	\begin{formal}[DarkOrange]{orangeshade}\vspace{-\baselineskip * 4 / 5}\begin{myCriterion}}%
	{\end{myCriterion}\end{formal}\vspace{-\baselineskip * 2 / 3}}
	

\newenvironment{remark}{\vspace{-\baselineskip * 2 / 3}%
	\begin{formal}[LightCoral]{redshade}\vspace{-\baselineskip * 4 / 5}\begin{myRemark}}%
	{\end{myRemark}\end{formal}\vspace{-\baselineskip * 2 / 3}}

\newenvironment{proposition}{\vspace{-\baselineskip * 2 / 3}%
	\begin{formal}[RoyalPurple]{purple}\vspace{-\baselineskip * 4 / 5}\begin{myProposition}}%
	{\end{myProposition}\end{formal}\vspace{-\baselineskip * 2 / 3}}


\newtheorem{example}{\indent \color{SeaGreen}{Example}}[section]
\renewcommand{\proofname}{\indent\textbf{\textcolor{TealBlue}{Proof}}}
\NewEnviron{solution}{%
	\begin{proof}[\indent\textbf{\textcolor{TealBlue}{Solution}}]%
		\color{blue}% 设置内容为蓝色
		\BODY% 插入环境内容
		\color{black}% 恢复默认颜色(可选,避免影响后续文字)
	\end{proof}%
}

% 自定义命令的文件

\def\d{\mathrm{d}}
\def\R{\mathbb{R}}
%\newcommand{\bs}[1]{\boldsymbol{#1}}
%\newcommand{\ora}[1]{\overrightarrow{#1}}
\newcommand{\myspace}[1]{\par\vspace{#1\baselineskip}}
\newcommand{\xrowht}[2][0]{\addstackgap[.5\dimexpr#2\relax]{\vphantom{#1}}}
\newenvironment{mycases}[1][1]{\linespread{#1} \selectfont \begin{cases}}{\end{cases}}
\newenvironment{myvmatrix}[1][1]{\linespread{#1} \selectfont \begin{vmatrix}}{\end{vmatrix}}
\newcommand{\tabincell}[2]{\begin{tabular}{@{}#1@{}}#2\end{tabular}}
\newcommand{\pll}{\kern 0.56em/\kern -0.8em /\kern 0.56em}
\newcommand{\dive}[1][F]{\mathrm{div}\;\boldsymbol{#1}}
\newcommand{\rotn}[1][A]{\mathrm{rot}\;\boldsymbol{#1}}

\newif\ifshowanswers
\showanswerstrue % 注释掉这行就不显示答案

% 定义答案环境
\newcommand{\answer}[1]{%
    \ifshowanswers
        #1%
    \fi
}

% 修改参数改变封面样式,0 默认原始封面、内置其他1、2、3种封面样式
\def\myIndex{0}


\ifnum\myIndex>0
    \input{\path/cover_package_\myIndex} 
\fi

\def\myTitle{考研数学笔记}
\def\myAuthor{Weary Bird}
\def\myDateCover{\today}
\def\myDateForeword{\today}
\def\myForeword{相见欢·林花谢了春红}
\def\myForewordText{
    林花谢了春红,太匆匆。
    无奈朝来寒雨晚来风。
    胭脂泪,相留醉,几时重。
    自是人生长恨水长东。
}
\def\mySubheading{以姜晓千强化课讲义为底本}


\begin{document}
% \input{\path/cover_text_\myIndex.tex}

\newpage
\thispagestyle{empty}
\begin{center}
    \Huge\textbf{\myForeword}
\end{center}
\myForewordText
\begin{flushright}
    \begin{tabular}{c}
        \myDateForeword
    \end{tabular}
\end{flushright}

\newpage
\pagestyle{plain}
\setcounter{page}{1}
\pagenumbering{Roman}
\tableofcontents

\newpage
\pagenumbering{arabic}
% \setcounter{chapter}{-1}
\setcounter{page}{1}

\pagestyle{fancy}
\fancyfoot[C]{\thepage}
\renewcommand{\headrulewidth}{0.4pt}
\renewcommand{\footrulewidth}{0pt}








\else
\fi
\chapter{行列式}

$$
\text{行列式的主要内容}\left\{\begin{matrix}
    \text{行列式的概念}&\left\{\begin{matrix}
    \text{定义}& n!\text{项不同行不同列元素乘积的代数和} \\
    \text{性质}&
    \end{matrix}\right. \\ 
    \text{重要行列式}&\left\{\begin{matrix}
    \text{上(或下)三角,主对角矩阵}\\
    \text{副对角行列式}\\
    \text{ab型行列式}\\
    \text{拉普拉斯展开式}\\
    \text{范德蒙行列式}\\

\end{matrix}\right. \\
    \text{展开定理}& \left\{\begin{matrix}
    a_{i1}A_{j1}+a_{i2}A_{j2}+\ldots+a_{in}A_{jn} = & \left\{\begin{matrix}
    \left | A \right |, &i=j \\
    0,& i = j
\end{matrix}\right. \\
    a_{1i}A_{1j}+a_{2i}A_{2j}+\ldots+a_{ni}A_{nj} = & \left\{\begin{matrix}
    \left | A \right |, &i=j \\
    0,& i = j
\end{matrix}\right. \\
\end{matrix}\right.\\
    \text{行列式的公式}& \left\{\begin{matrix}
    \left | KA \right | = K^n\left | A \right | & \\
    \left | AB \right | = \left | A \right | \left | B \right | &\\
    \left | A^T \right |= \left | A \right |& \\
    \left | A^{-1} \right | = \left | A \right | ^{-1}& \\
    \left | A^{*} \right |=\left | A \right |^{n-1}& \\
    \text{设A的特征值为}\lambda_1,\lambda _2,\ldots,\lambda_n,&\text{则}\left | A \right | =\prod_{i=1}^{n}\lambda_i  \\
    \text{若A与B相似},&\text{则}\left | A \right | =\left | B \right | 
\end{matrix}\right.\\
    \text{Cramer法则}& x_1=\frac{D_1}{D},x_2=\frac{D_2}{D},\ldots,x_n\frac{D_n}{D}
\end{matrix}\right.
$$
\newpage

拉普拉斯展开式(上,下三角分块行列式的结论)
$$
D = 
\begin{vmatrix}
A & C \\
\mathbf{0}& D  
\end{vmatrix} 
= \begin{vmatrix}
A & \mathbf{0} \\
C & D
\end{vmatrix}
= \det{(A)}\det{(D)} \\
$$
\text{对于一般分块矩阵}
$$
A = \begin{pmatrix}
B & C \\
D & E
\end{pmatrix} 
$$
若B可逆,则有如下结论
$$
\det(A) = \det(B) \cdot \det(E - D B^{-1} C)
$$

\newpage

\section{数字行列式的计算}
\begin{remark}
    基本方法
    \item [(1)]利用行列式的性质(5条)来化简
    \item [(2)] 要么出现重要行列式(5组) 
    \item [(3)] 要么展开定理(0比较多的时候)
\end{remark}
\begin{enumerate}[label=\arabic*.]
    % 例题1.1
    \item 设
    \begin{align*}
    f(x)=\left|\begin{array}{llll}
    x-2 & x-1 & x-2 & x-3 \\
    2 x-2 & 2 x-1 & 2 x-2 & 2 x-3 \\
    3 x-3 & 3 x-2 & 4 x-5 & 3 x-5 \\
    4 x & 4 x-3 & 5 x-7 & 4 x-3
    \end{array}\right|
    \end{align*}
    则方程 $f(x)=0$ 根的个数为\_\_\_\_ 
    \begin{solution}
    \color{blue}
    第一列乘$-1$加到其他列
    \begin{align*}
    f(x) &\xlongequal{\text{第一列乘-1加到其他列上面去}}{}\left|\begin{array}{llll}
    x-2 & 1 & 0 & -1 \\
    2 x-2 & 1 & 0 & -1 \\
    3 x-3 & 1 &  x-2 &  -2 \\
    4 x & 4 -3 & x-7 & -3
    \end{array}\right| \\
    &\xlongequal{\text{第二列加到第四列}}{}\left|\begin{array}{rr|rr}
    x-2 & 1 & 0 & 0 \\
    2 x-2 & 1 & 0 & 0 \\
    \hline
    3 x-3 & 1 &  x-2 &  -1 \\
    4 x & -3 & x-7 & -6
    \end{array}\right| \\
    &\xlongequal{\text{拉普拉斯型}}{}\left|\begin{array}{ll}
        x-2 & 1 \\
        2x-2 & 1
    \end{array} \right|
    \left|\begin{array}{ll}
        x-2 & -1 \\
        x-7 & -6
    \end{array} \right| \\
    & = [(x-2)-(2x-2)][-6(x-2)+(x-7)] = 15x(x-1) 
    \end{align*}
    则$x=0$或$x=1$
    \end{solution}
    
    % 例题1.2
    \item 利用范德蒙行列式计算 
    \begin{align*}
    \text{范德蒙行列式}
    V(x_1, x_2, \ldots, x_n) = \begin{vmatrix}
    1 & x_1 & x_1^2 & \cdots & x_1^{n-1} \\
    1 & x_2 & x_2^2 & \cdots & x_2^{n-1} \\
    \vdots & \vdots & \vdots & \ddots & \vdots \\
    1 & x_n & x_n^2 & \cdots & x_n^{n-1}
    \end{vmatrix}=\prod_{1 \leq i < j \leq n} (x_j - x_i)
    \end{align*}

    \begin{align*}
    \left|\begin{array}{lll}
    a & a^2 & bc \\
    b & b^2 & a c \\
    c & c^2 & a b 
    \end{array}\right|=\_\_\_\_
    \end{align*}
    
    \begin{solution}
    \color{blue}
    \begin{align*}
    \text{原式} &\xlongequal{\text{第一列乘以(a+b+c)加到第三列}}\left|\begin{array}{lll}
    a & a^2 & a^2 + ac + ab + bc \\
    b & b^2 & a^2 + ac + ab + bc \\
    c & c^2 & a^2 + ac + ab + bc 
    \end{array}\right| \\
    &\xlongequal{\text{第二列乘-1加到最后一列,提取公因式,并交换}} (ab+ac+bc)\left|\begin{array}{lll}
    1 & a & a^2 \\
    1 & b & b^2 \\
    1 & c & c^2 
    \end{array}\right| \\
    &= (ac+bc+ab)(b-a)(c-a)(c-b)
    \end{align*}
    \end{solution}
    
    \newpage

    % 例题1.3
    \item 设 \( {x}_{1}{x}_{2}{x}_{3}{x}_{4} \neq  0 \) ,则 
    \( \left| \begin{matrix} 
        {x}_{1}+{a}_{1}^{2}&{a}_{1}{a}_{2}&{a}_{1}{a}_{3} & {a}_{1}{a}_{4} \\  
        {a}_{2}{a}_{1} & {x}_{2} + {a}_{2}^{2} & {a}_{2}{a}_{3} & {a}_{2}{a}_{4} \\  
        {a}_{3}{a}_{1} & {a}_{3}{a}_{2} & {x}_{3} + {a}_{3}^{2} & {a}_{3}{a}_{4} \\  
        {a}_{4}{a}_{1} & {a}_{4}{a}_{2} & {a}_{4}{a}_{3} & {x}_{4} + {a}_{4}^{2} 
    \end{matrix}\right|  = \) \_\_\_\_.
    
    \begin{solution}
    \color{blue}
    考虑加边法,为该行列式增加一行一列,变成如下行列式 
    \begin{align*}
    \text{原行列式} 
    &= 
    \left| \begin{matrix} 
        1 & 0 & 0 & 0 & 0 \\
        a_1 &{x}_{1}+{a}_{1}^{2}&{a}_{1}{a}_{2}&{a}_{1}{a}_{3} & {a}_{1}{a}_{4} \\  
        a_2 &{a}_{2}{a}_{1} & {x}_{2} + {a}_{2}^{2} & {a}_{2}{a}_{3} & {a}_{2}{a}_{4} \\  
        a_3 &{a}_{3}{a}_{1} & {a}_{3}{a}_{2} & {x}_{3} + {a}_{3}^{2} & {a}_{3}{a}_{4} \\  
        a_4 &{a}_{4}{a}_{1} & {a}_{4}{a}_{2} & {a}_{4}{a}_{3} & {x}_{4} + {a}_{4}^{2} 
    \end{matrix}\right| \\
    &\xlongequal{\text{将第一行分别乘以}-a_1,-a_2\ldots,\text{分别加到第}2,3,\ldots\text{列}}{}
        \left| \begin{matrix} 
        1   & -a_1 & -a_2 & -a_3 & -a_4 \\
        a_1 &x_1   &0     & 0    & 0\\  
        a_2 &0     & x_2  & 0    & 0\\  
        a_3 &0     & 0    & x_3  & 0\\  
        a_4 &0     & 0    & 0    & x_4
    \end{matrix}\right| \\
    &\xlongequal{\text{从下往上消,分别乘以}\frac{a_i}{x_i},\text{加到第一行}}{}
        \left| \begin{matrix} 
        1 + \sum_{i=1}^4 \frac{a_i^2}{x_i}   & 0 & 0 & 0 & 0 \\
        a_1 &x_1   &0     & 0    & 0\\  
        a_2 &0     & x_2  & 0    & 0\\  
        a_3 &0     & 0    & x_3  & 0\\  
        a_4 &0     & 0    & 0    & x_4
    \end{matrix}\right| \\
    & = (x_1x_2x_3x_4)(1 + \sum_{i=1}^4 \frac{a_i^2}{x_i})
    \end{align*}
    \end{solution}
    
    \begin{tcolorbox}[title=爪型行列式]
        关键点在于\textbf{化简掉一条爪子}
        $$
        \begin{vmatrix}
            a_{11} & a_{12} & a_{13} & \cdots & a_{1n} \\
            a_{21} & a_{22} & 0      & \cdots & 0      \\
            a_{31} & 0      & a_{33} & \cdots & 0      \\
            \vdots & \vdots & \vdots & \ddots & \vdots \\
            a_{n1} & 0      & 0      & \cdots & a_{nn}
        \end{vmatrix}
        $$
    \end{tcolorbox}
    % 例题1.4
    \item 计算三对角线行列式
    \begin{align*}
    D_{n}= 
    \begin{vmatrix}
    \alpha+\beta & \alpha & 0 & \cdots & 0 & 0 \\
    \beta & \alpha+\beta & \alpha & \cdots & 0 & 0 \\
    0 & \beta & \alpha+\beta & \cdots & 0 & 0 \\
    \vdots & \vdots & \vdots & \ddots & \vdots & \vdots \\
    0 & 0 & 0 & \cdots & \alpha+\beta & \alpha \\
    0 & 0 & 0 & \cdots & \beta & \alpha+\beta
    \end{vmatrix}
    \end{align*}
    
    \begin{solution}
    \color{blue}
    \item [(方法一) 递推法] 
    \begin{align*}
        D_1&=\alpha+\beta \\
        D_2&=\alpha^2+\alpha\beta+\beta^2 \\
        &\ldots \\
        D_n &= (\alpha+\beta)D_{n-1} - \alpha\beta D_{n-2} \\
        D_n-\alpha D_{n-1} &=\beta (D_{n-1}-\alpha D_{n-2}) \\
        &=\beta^2(D_{n-2}-\alpha D_{n-3}) \\
        &\ldots \\
        &=\beta^{n-1}(D_2-D_1) = \beta^{n} \\
        D_n &= \beta^n+\alpha D_{n-1} = \beta^n + \alpha (\beta^{n-1} + \alpha D_{n-2})\\
        &\ldots \\
        &= \beta^n + \alpha\beta^{n-1}+\ldots + \alpha^n
    \end{align*}
    \item [(方法二) 数学归纳法]
    \begin{align*}
        & if\ \alpha=\beta, D_1 = 2\alpha, D_2=3\alpha^2,assume,D_{n-1}=n\alpha^{n-1} \\
        & then D_n=D_n = (\alpha+\beta)D_{n-1} - \alpha\beta D_{n-2} = (n+1)\alpha^n \\
        & when\ \alpha\neq\beta, D_1=\frac{\alpha^2 - \beta ^ 2}{\alpha - \beta}, D_2=\frac{\alpha^3-\beta^3}{\alpha - \beta}, \\
        & Assume,D_{n-1}=\frac{\alpha^n-\beta^n}{\alpha-\beta},then, \\
        & D_n=\ldots= \frac{\alpha^{n+1}-\beta^{n+1}}{\alpha-\beta}
    \end{align*}
    \item [(方法三) 二阶差分方程]
    \begin{align*}
        & D_n - (\alpha+\beta)D_{n-1} + \alpha\beta D_{n-2} = 0 \\
        & D_{n+2}-(\alpha+\beta)D_{n+1} + \alpha\beta D_{n} = 0 
    \end{align*}
    \text{类似于二阶微分方程解特征方程}
    \begin{align*}
        & r^2 - (\alpha+\beta) r + \alpha\beta = 0 \\
        & r_1 = \alpha\qquad r_2 = \beta
    \end{align*}
    \text{差分方程的关键}$r^n$\text{代换}$e^{rx}$ \\
    \text{如果}\ $\alpha=\beta$
    \begin{align*}
        & D_n=(C_1+C_2n)\alpha^n, D_1 = 2\alpha, D_2 = 3\alpha^2 \\
        & \text{得到}C_1=C_2=1,{故}D_n=(n+1)\alpha^n 
    \end{align*}
    \text{如果} $\alpha\neq\beta$ 
    \begin{align*}
        &D_n=C_1\alpha^n+C_2\beta^n, \text{由}D_1 = 2\alpha, D_2 = 3\alpha^2 \\
        &C_1 = \frac{\alpha}{\alpha - \beta}, C_2=\frac{-\beta}{\alpha-\beta} \\
        &D_n= \frac{\alpha^{n+1}-\beta^{n+1}}{\alpha-\beta}
    \end{align*}
    \end{solution}
\end{enumerate}

\begin{corollary}
    如下行列式有和例题4完全相等的性质
    $$ 
    D_{n} = \left| 
    \begin{array}{cccccc}
    \alpha+\beta & \alpha\beta & 0 & \cdots & 0 & 0 \\
    1 & \alpha+\beta & \alpha\beta & \cdots & 0 & 0 \\
    0 & 1 & \alpha+\beta & \cdots & 0 & 0 \\
    \vdots & \vdots & \vdots & \ddots & \vdots & \vdots \\
    0 & 0 & 0 & \cdots & \alpha+\beta & \alpha\beta \\
    0 & 0 & 0 & \cdots & 1 & \alpha+\beta \\
    \end{array} 
    \right| 
    $$

    \( {D}_{n} = 
    \left\{  
        {\begin{matrix} \left( {n + 1}\right) {\alpha }^{n}, 
            & \alpha  = \beta \\  
            \frac{{\alpha }^{n + 1} - {\beta }^{n + 1}}
                {\alpha  - \beta }, & 
                \alpha  \neq  \beta  \end{matrix}.}
    \right. \)
\end{corollary}
\section{代数余子式求和}

\begin{remark}
    代数余子式求和的基本办法
    \item [(1)] 代数余子式的定义(求一个的时候使用)
    \item [(2)] 展开定理(求一行或者一列的时候使用)
    \item [(3)] 利用伴随矩阵的定义(求全部代数余子式的时候使用)
\end{remark}

\begin{enumerate}[label=\arabic*.,start=5]
    % 例题1.5
    \item 已知 
    \begin{align*}
    |A|=\left|\begin{array}{lllll}
    1 & 2 & 3 & 4 & 5 \\
    2 & 2 & 2 & 1 & 1 \\
    3 & 1 & 2 & 4 & 5 \\
    1 & 1 & 1 & 2 & 2 \\
    4 & 3 & 1 & 5 & 0
    \end{array}\right|=27
    \end{align*}
    则 $A_{41}+A_{42}+A_{43}=$ \underline{\hspace{3cm}}, $A_{44}+A_{45}=$ \underline{\hspace{3cm}}
    
    \begin{solution}
    \item {(方法一)}
    利用展开定理构建新的矩阵来计算
        \begin{align*}
    A_{41}+A_{42}+A_{43} &=\left|\begin{array}{lllll}
    1 & 2 & 3 & 4 & 5 \\
    2 & 2 & 2 & 1 & 1 \\
    3 & 1 & 2 & 4 & 5 \\
    1 & 1 & 1 & 0 & 0 \\
    4 & 3 & 1 & 5 & 0
    \end{array}\right| \\
    A_{44}+A_{45}&=\left|\begin{array}{lllll}
    1 & 2 & 3 & 4 & 5 \\
    2 & 2 & 2 & 1 & 1 \\
    3 & 1 & 2 & 4 & 5 \\
    0 & 0 & 0 & 1 & 1 \\
    4 & 3 & 1 & 5 & 0
    \end{array}\right|
    \end{align*}
    但这样$\left|A\right|=27$的条件就没用到
    \item {(方法二)}
    
    \text{直接对第四行使用展开定理,则}
    $$\left|A\right| = A_{41} + A_{42} + A_{43} + 2A_{44} + 2A_{45} = 27$$
    \text{直接对第二行使用展开定理,则}
    $$\left|A\right| = 2A_{41} + 2A_{42} + 2A_{43} + A_{44} + A_{45} = 0$$
    相当于解$A+2B=27, 2A+B = 0$,容易计算$A_{41} + A_{42} + A_{43} = -9, A_{44} + A_{45} = 18$
    \end{solution}
    
    % 例题1.6
    \item 设 
    \begin{align*}
    A=\begin{pmatrix}
    0 & 1 & 0 & \cdots & 0\\
    0 & 0 & 2 &\cdots & 0 \\
    \vdots & \vdots & \vdots & \ddots &\vdots  \\
    0 & 0 & 0&\cdots & n-1\\
    n & 0 & 0&\cdots & 0 
    \end{pmatrix}
    \end{align*}
    则 $|A|$ 的所有代数余子式的和为\underline{\hspace{3cm}}
    
    \begin{solution}
    对于求所有代数余子,基本都是考察$A^{*}$的定义,即
    \[
    A^* = \begin{pmatrix}
    C_{11} & C_{21} & \cdots & C_{n1} \\
    C_{12} & C_{22} & \cdots & C_{n2} \\
    \vdots & \vdots & \ddots & \vdots \\
    C_{1n} & C_{2n} & \cdots & C_{nn}
    \end{pmatrix},
    \]
    又由于$A^{*}=\left|A\right|A^{-1}$,对于这道题
    \[
    \left|A\right|=(-1)^{(n+1)}n!
    \]
    $A^{-1}$可以通过分块矩阵来求
    \begin{align*}
    |A|A^{-1} 
    &=|A|\left(
    \begin{array}{r|rrrr}
    0 & 1 & 0 & \cdots & 0\\
    0 & 0 & 2 &\cdots & 0 \\
    \vdots & \vdots & \vdots & \ddots &\vdots  \\
    0 & 0 & 0&\cdots & n-1\\
    \hline
    n & 0 & 0&\cdots & 0 
    \end{array}\right)^{-1} \\
    &=|A|\left(
    \begin{array}{l|l}
    0 & \frac{1}{n} \\
    \hline
    diag(1,\frac{1}{2},\ldots,\frac{1}{n-1}) & 0
    \end{array}\right) \\
    &= \left(
    \begin{array}{l|l}
    0 & \frac{1}{n} |A| \\
    \hline
    diag(|A|,\frac{|A|}{2},\ldots,\frac{|A|}{n-1}) & 0
    \end{array}\right)
    \end{align*}
    则所有代数余子式之和为$$(-1)^{(n+1)}n!\sum_{i=1}^{n}{\frac{1}{i}}$$
    \end{solution}
\end{enumerate}

\section{抽象行列式的计算}
\begin{remark}
    抽象行列式的计算方法
    \item [(1)] 通过行列式的性质
    \item [(2)] 行列式的公式(7个)
\end{remark}
\begin{enumerate}[label=\arabic*.,start=7]
    % 例题1.7
    \item (2005,数一、二) 设 $\alpha_1,\alpha_2,\alpha_3$ 均为 3维列向量, $A=(\alpha_1,\alpha_2,\alpha_3)$,
    $B=(\alpha_1+\alpha_2+\alpha_3,\alpha_1+2\alpha_2+4\alpha_3,\alpha_1+3\alpha_2+9\alpha_3)$.若 $|A|=1$,则 $|B|=$ \underline{\hspace{3cm}}
    
    \begin{solution}
    \item [(方法一 利用性质)]
    \begin{align*}
        B &=(\alpha_1+\alpha_2+\alpha_3,\alpha_1+2\alpha_2+4\alpha_3,\alpha_1+3\alpha_2+9\alpha_3) \\
        &= (\alpha_1+\alpha_2+\alpha_3, \alpha_2+3\alpha_3, \alpha_2+5\alpha_3) \\
        &= 2(\alpha_1+\alpha_2+\alpha_3, \alpha_2+3\alpha_3, \alpha_3) \\
        &= 2(\alpha_1,\alpha_2,\alpha_3) \\
        \left|B\right| &= 2\left|A\right| = 2
    \end{align*}
    \item [(方法二 分块矩阵)]
    \begin{align*}
        B &= (\alpha_1,\alpha_2,\alpha_3)\left(\begin{array}{rrr}
            1 &1 &1 \\
            1 &2 &3 \\
            1 &3 &9
        \end{array}\right) \\
        |B| &= |A|\left|\begin{array}{rrr}
            1 &1 &1 \\
            1 &2 &3 \\
            1 &3 &9
        \end{array}
        \right| = |A|(2-1)(3-1)(3-2) = 2
    \end{align*}
    \end{solution}
    
    % 例题1.8
    \item 设 A为 n阶矩阵, $\alpha,\beta$ 为 n维列向量.若 $|A|=a$,
    $\left|\begin{array}{ll}A & \alpha \\ \beta^{T} & b\end{array}\right|=0$,则
    $\left|\begin{array}{ll}A & \alpha \\ \beta^{T} & c\end{array}\right|=$ \underline{\hspace{3cm}}
    
    \begin{solution}
    这道题的关键在于巧妙构建行列式的和
    \begin{align*}
        \left|\begin{array}{rr}
            A & \alpha \\
            \beta^{T} & c\\
        \end{array} 
        \right| 
        &= \left|\begin{array}{rr}
            A & \alpha + 0 \\
            \beta^{T} & b + c -b\\
        \end{array}\right|  \\
        &=\left|\begin{array}{rr}
            A & \alpha \\
            \beta^{T} & b\\
        \end{array}\right| + 
        \left|\begin{array}{rr}
            A & 0 \\
            \beta^{T} & c-b\\
        \end{array}\right| \\
        &=\left|A\right|(c-b) = a(c-b)
    \end{align*}
    \end{solution}
    
    \newpage
    % 例题1.9
    \item 设 A为 2阶矩阵, 
    $B=2\left(
    \begin{array}{ll}
        (2A)^{-1}-(2A)^* & 0 \\
        0 & A
    \end{array}\right)
    $
    若 $|A|=-1$,则 $|B|=$ \underline{\hspace{3cm}}
    
    \begin{solution}
    这道题比较纯粹就是行列式公式的应用
    \begin{align*}
        \left|B\right| &= 2^4\left|A\right|\cdot\left|(2A)^{-1}-(2A)^{*}\right| \\
        &= 2^4\left|A\right|\cdot\left|\frac{1}{2}A^{-1}-2A^{*}\right| \\
        &= 2^4\left|\frac{1}{2}E-2|A|\right| = 100
    \end{align*}
    \end{solution}
    % 例题1.10

    \item  设 n阶矩阵 A满足 $A^2=A$, $A\neq E$,证明 $|A|=0$
    \begin{tcolorbox}[title=易错点]
        由$|A|^2=|A|\implies |A|=1\text{或}=0$,又$A\neq E\implies |A|\neq 1$,故$|A| = 0$
        注意矩阵不等关系是无法推出行列式的不等关系的,矩阵式数表只要顺序不同就不一样,但不一样的矩阵其行列式完全有可能相等.\\
        等于1的矩阵并非只能是$E$
    \end{tcolorbox}
    \begin{solution}
    \item [(方法一,反证法)]
    若$|A|\neq 0$,则$A$可逆,对于等式$A^2=A$两边同乘$A^{-1}$,则$A=E$与题设矛盾,故$|A|\neq 0$
    \item [(方法二,秩)] 
    由于$A(A-E)=0\implies r(A)+r(A-E)\leq n,$又$A\neq E, r(A-E)\geq 1,$故$r(A)\leq n$,故$|A|=0$
    \item [(方法三,方程组)]
    由于$A(A-E)=0$,且$A\neq E$可知方程$AX=0$有非零解即$(A-E)$中的非零列,故$r(A)<n,|A|=0$
    \item [(方法四,特征值与特征向量)]
    由于$A(A-E)=0,A\neq E$,取$A-E$的非零列向量$\beta\neq 0,A\beta=0$故由特征值与特征值向量的定义,$A$由特征值$0$,而
    $|A|=\prod_{i=1}^{n}\lambda_i=0$
    \end{solution}
    \begin{tcolorbox}[title=总结]
    若$AB=0$有如下结论 \\
    (1) $r(A)+r(B)\leq n$ \\
    (2)$B$的列向量均为方程$AX=0$的解 \\
    (3)若$A_{n\times n}$,则$B$的非零列向量均为$A$的特征值为$0$的特征向量
    \end{tcolorbox}
\end{enumerate}

\ifx\allfiles\undefined
\end{document}
\fi