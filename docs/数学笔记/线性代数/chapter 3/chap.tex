\ifx\allfiles\undefined
\documentclass[12pt, a4paper, oneside, UTF8]{ctexbook}
\usepackage{multirow}
\def\path{../../config}
\usepackage{amsmath}
\usepackage{amsthm}
\usepackage{amssymb}
\usepackage{array}
\usepackage{xcolor}
\usepackage{graphicx}
\usepackage{mathrsfs}
\usepackage{enumitem}
\usepackage{geometry}
\usepackage[colorlinks, linkcolor=black]{hyperref}
\usepackage{stackengine}
\usepackage{yhmath}
\usepackage{extarrows}
\usepackage{tikz}
\usepackage{pgfplots}
\usepackage{asymptote}
\usepackage{float}
\usepackage{fontspec} % 使用字体

\setmainfont{Times New Roman}
\setCJKmainfont{LXGWWenKai-Light}[
    SlantedFont=*
]

\everymath{\displaystyle}

\usepgfplotslibrary{polar}
\usepackage{subcaption}
\usetikzlibrary{decorations.pathreplacing, positioning}

\usepgfplotslibrary{fillbetween}
\pgfplotsset{compat=1.18}
% \usepackage{unicode-math}
\usepackage{esint}
\usepackage[most]{tcolorbox}

\usepackage{fancyhdr}
\usepackage[dvipsnames, svgnames]{xcolor}
\usepackage{listings}

\definecolor{mygreen}{rgb}{0,0.6,0}
\definecolor{mygray}{rgb}{0.5,0.5,0.5}
\definecolor{mymauve}{rgb}{0.58,0,0.82}
\definecolor{NavyBlue}{RGB}{0,0,128}
\definecolor{Rhodamine}{RGB}{255,0,255}
\definecolor{PineGreen}{RGB}{0,128,0}

\graphicspath{ {figures/},{../figures/}, {config/}, {../config/} }

\linespread{1.6}

\geometry{
    top=25.4mm, 
    bottom=25.4mm, 
    left=20mm, 
    right=20mm, 
    headheight=2.17cm, 
    headsep=4mm, 
    footskip=12mm
}

\setenumerate[1]{itemsep=5pt,partopsep=0pt,parsep=\parskip,topsep=5pt}
\setitemize[1]{itemsep=5pt,partopsep=0pt,parsep=\parskip,topsep=5pt}
\setdescription{itemsep=5pt,partopsep=0pt,parsep=\parskip,topsep=5pt}

\lstset{
    language=Mathematica,
    basicstyle=\tt,
    breaklines=true,
    keywordstyle=\bfseries\color{NavyBlue}, 
    emphstyle=\bfseries\color{Rhodamine},
    commentstyle=\itshape\color{black!50!white}, 
    stringstyle=\bfseries\color{PineGreen!90!black},
    columns=flexible,
    numbers=left,
    numberstyle=\footnotesize,
    frame=tb,
    breakatwhitespace=false,
} 

\lstset{
    language=TeX, % 设置语言为 TeX
    basicstyle=\ttfamily, % 使用等宽字体
    breaklines=true, % 自动换行
    keywordstyle=\bfseries\color{NavyBlue}, % 关键字样式
    emphstyle=\bfseries\color{Rhodamine}, % 强调样式
    commentstyle=\itshape\color{black!50!white}, % 注释样式
    stringstyle=\bfseries\color{PineGreen!90!black}, % 字符串样式
    columns=flexible, % 列的灵活性
    numbers=left, % 行号在左侧
    numberstyle=\footnotesize, % 行号字体大小
    frame=tb, % 顶部和底部边框
    breakatwhitespace=false % 不在空白处断行
}

% \begin{lstlisting}[language=TeX] ... \end{lstlisting}

% 定理环境设置
\usepackage[strict]{changepage} 
\usepackage{framed}

\definecolor{greenshade}{rgb}{0.90,1,0.92}
\definecolor{redshade}{rgb}{1.00,0.88,0.88}
\definecolor{brownshade}{rgb}{0.99,0.95,0.9}
\definecolor{lilacshade}{rgb}{0.95,0.93,0.98}
\definecolor{orangeshade}{rgb}{1.00,0.88,0.82}
\definecolor{lightblueshade}{rgb}{0.8,0.92,1}
\definecolor{purple}{rgb}{0.81,0.85,1}

\theoremstyle{definition}
\newtheorem{myDefn}{\indent Definition}[section]
\newtheorem{myLemma}{\indent Lemma}[section]
\newtheorem{myThm}[myLemma]{\indent Theorem}
\newtheorem{myCorollary}[myLemma]{\indent Corollary}
\newtheorem{myCriterion}[myLemma]{\indent Criterion}
\newtheorem*{myRemark}{\indent Remark}
\newtheorem{myProposition}{\indent Proposition}[section]

\newenvironment{formal}[2][]{%
	\def\FrameCommand{%
		\hspace{1pt}%
		{\color{#1}\vrule width 2pt}%
		{\color{#2}\vrule width 4pt}%
		\colorbox{#2}%
	}%
	\MakeFramed{\advance\hsize-\width\FrameRestore}%
	\noindent\hspace{-4.55pt}%
	\begin{adjustwidth}{}{7pt}\vspace{2pt}\vspace{2pt}}{%
		\vspace{2pt}\end{adjustwidth}\endMakeFramed%
}

\newenvironment{definition}{\vspace{-\baselineskip * 2 / 3}%
	\begin{formal}[Green]{greenshade}\vspace{-\baselineskip * 4 / 5}\begin{myDefn}}
	{\end{myDefn}\end{formal}\vspace{-\baselineskip * 2 / 3}}

\newenvironment{theorem}{\vspace{-\baselineskip * 2 / 3}%
	\begin{formal}[LightSkyBlue]{lightblueshade}\vspace{-\baselineskip * 4 / 5}\begin{myThm}}%
	{\end{myThm}\end{formal}\vspace{-\baselineskip * 2 / 3}}

\newenvironment{lemma}{\vspace{-\baselineskip * 2 / 3}%
	\begin{formal}[Plum]{lilacshade}\vspace{-\baselineskip * 4 / 5}\begin{myLemma}}%
	{\end{myLemma}\end{formal}\vspace{-\baselineskip * 2 / 3}}

\newenvironment{corollary}{\vspace{-\baselineskip * 2 / 3}%
	\begin{formal}[BurlyWood]{brownshade}\vspace{-\baselineskip * 4 / 5}\begin{myCorollary}}%
	{\end{myCorollary}\end{formal}\vspace{-\baselineskip * 2 / 3}}

\newenvironment{criterion}{\vspace{-\baselineskip * 2 / 3}%
	\begin{formal}[DarkOrange]{orangeshade}\vspace{-\baselineskip * 4 / 5}\begin{myCriterion}}%
	{\end{myCriterion}\end{formal}\vspace{-\baselineskip * 2 / 3}}
	

\newenvironment{remark}{\vspace{-\baselineskip * 2 / 3}%
	\begin{formal}[LightCoral]{redshade}\vspace{-\baselineskip * 4 / 5}\begin{myRemark}}%
	{\end{myRemark}\end{formal}\vspace{-\baselineskip * 2 / 3}}

\newenvironment{proposition}{\vspace{-\baselineskip * 2 / 3}%
	\begin{formal}[RoyalPurple]{purple}\vspace{-\baselineskip * 4 / 5}\begin{myProposition}}%
	{\end{myProposition}\end{formal}\vspace{-\baselineskip * 2 / 3}}


\newtheorem{example}{\indent \color{SeaGreen}{Example}}[section]
\renewcommand{\proofname}{\indent\textbf{\textcolor{TealBlue}{Proof}}}
\NewEnviron{solution}{%
	\begin{proof}[\indent\textbf{\textcolor{TealBlue}{Solution}}]%
		\color{blue}% 设置内容为蓝色
		\BODY% 插入环境内容
		\color{black}% 恢复默认颜色(可选,避免影响后续文字)
	\end{proof}%
}

% 自定义命令的文件

\def\d{\mathrm{d}}
\def\R{\mathbb{R}}
%\newcommand{\bs}[1]{\boldsymbol{#1}}
%\newcommand{\ora}[1]{\overrightarrow{#1}}
\newcommand{\myspace}[1]{\par\vspace{#1\baselineskip}}
\newcommand{\xrowht}[2][0]{\addstackgap[.5\dimexpr#2\relax]{\vphantom{#1}}}
\newenvironment{mycases}[1][1]{\linespread{#1} \selectfont \begin{cases}}{\end{cases}}
\newenvironment{myvmatrix}[1][1]{\linespread{#1} \selectfont \begin{vmatrix}}{\end{vmatrix}}
\newcommand{\tabincell}[2]{\begin{tabular}{@{}#1@{}}#2\end{tabular}}
\newcommand{\pll}{\kern 0.56em/\kern -0.8em /\kern 0.56em}
\newcommand{\dive}[1][F]{\mathrm{div}\;\boldsymbol{#1}}
\newcommand{\rotn}[1][A]{\mathrm{rot}\;\boldsymbol{#1}}

\newif\ifshowanswers
\showanswerstrue % 注释掉这行就不显示答案

% 定义答案环境
\newcommand{\answer}[1]{%
    \ifshowanswers
        #1%
    \fi
}

% 修改参数改变封面样式,0 默认原始封面、内置其他1、2、3种封面样式
\def\myIndex{0}


\ifnum\myIndex>0
    \input{\path/cover_package_\myIndex} 
\fi

\def\myTitle{考研数学笔记}
\def\myAuthor{Weary Bird}
\def\myDateCover{\today}
\def\myDateForeword{\today}
\def\myForeword{相见欢·林花谢了春红}
\def\myForewordText{
    林花谢了春红,太匆匆。
    无奈朝来寒雨晚来风。
    胭脂泪,相留醉,几时重。
    自是人生长恨水长东。
}
\def\mySubheading{以姜晓千强化课讲义为底本}


\begin{document}
% \input{\path/cover_text_\myIndex.tex}

\newpage
\thispagestyle{empty}
\begin{center}
    \Huge\textbf{\myForeword}
\end{center}
\myForewordText
\begin{flushright}
    \begin{tabular}{c}
        \myDateForeword
    \end{tabular}
\end{flushright}

\newpage
\pagestyle{plain}
\setcounter{page}{1}
\pagenumbering{Roman}
\tableofcontents

\newpage
\pagenumbering{arabic}
% \setcounter{chapter}{-1}
\setcounter{page}{1}

\pagestyle{fancy}
\fancyfoot[C]{\thepage}
\renewcommand{\headrulewidth}{0.4pt}
\renewcommand{\footrulewidth}{0pt}








\else
\fi

\chapter{向量}
\section{知识体系}
$$
\fbox{\text{向量}}
\begin{cases}
    \fbox{\text{基本运算}} \begin{cases}
        \alpha + \beta \\
        k\alpha \\
        [\alpha, \beta]
    \end{cases} \\
    \fbox{\text{线性表示}} \begin{cases}
        \fbox{\text{定义}}\quad\beta=k_1\alpha_1+k_2\alpha_2+\ldots+k_s\alpha_s \\
        \fbox{\text{充要条件}} \begin{cases}
            \text{非其次线性方程组}(\alpha_1,\alpha_2,\ldots,\alpha_s)(x_1,x_2,\ldots,x_s)^{T}=\beta\text{有解} \\
            r(\alpha_1,\alpha_2,\ldots,alpha_s)=r(\alpha_1,\alpha_2,\ldots,\alpha_s\mid \beta)
        \end{cases} \\
        \fbox{\text{求法}} \left(\alpha_1,\alpha_2,\ldots,\alpha_s\mid\beta\right) \xrightarrow{\text{初等行变换}}\text{行最简形矩阵}
    \end{cases} \\
    \fbox{\text{线性相关}} \begin{cases}
        \fbox{\text{定义}} \\
        \fbox{\text{充要条件}} \begin{cases}
            \text{至少有一个向量可以由其余向量线性表示} \\
            \text{齐次线性方程组} (\alpha_1,\alpha_2,\ldots,\alpha_s)(x_1,x_2,\ldots,x_s)^{T}=0 \text{有非零解} \\
            r(\alpha_1,\alpha_2,\ldots,\alpha_s) < s
        \end{cases}\\
        \fbox{\text{充分条件}}
    \end{cases}
\end{cases}
$$

$$
\fbox{向量}\begin{cases}
        \fbox{\text{线性无关}} \begin{cases}
        \fbox{\text{定义}} \\
        \fbox{\text{充要条件}}  \begin{cases}
            \text{任意向量都不可以由其余向量线性表示} \\
            \text{齐次线性方程组} (\alpha_1,\alpha_2,\ldots,\alpha_s)(x_1,x_2,\ldots,x_s)^{T}=0 \text{只有零解} \\
            r(\alpha_1,\alpha_2,\ldots,\alpha_s) = s
        \end{cases}\\
        \fbox{\text{充分条件}} \\
    \end{cases} \\
    \fbox{\text{极大无关组与向量组的秩}} \begin{cases}
        \fbox{\text{定义}} \\
        \fbox{\text{求法}} (\alpha_1,\alpha_2,\ldots,\alpha_s)\xrightarrow{\text{初等行变换}}\text{行阶梯形}
    \end{cases}
\end{cases}
$$


\section{线性表示的判定与计算}
\begin{remark}[线性表示的判定与计算]
    (题型一\ 判断) 
    \begin{enumerate}
        \item [(I)]\underline{线性表示的定义} $\beta=k_1\alpha_1+k_2\alpha_2+\ldots+k_s\alpha_s$
        \item [(II)]\underline{秩} $r(\alpha_1,\ldots,\alpha_s)=r(\alpha_1,\ldots,\alpha_s \mid \beta)$
    \end{enumerate}
    (题型二\ 计算) 
    $$
    \left(\alpha_1,\ldots,\alpha_s,\mid \beta\right)\xrightarrow{\text{初等行变换}}\text{行最简型} 
    $$
    (题型三\ 向量组等价) 
    \begin{enumerate}
        \item [(I)] \underline{向量组等价的定义} 向量组$I,II$可以相互线性表示
        \item [(II)] \underline{三秩相等} $r(I)=r(I,II)=r(II)$
    \end{enumerate}
\end{remark}

\begin{enumerate}[label=\arabic*.]
    \item 设向量组 $\alpha, \beta, \gamma$ 与数 $k, l, m$ 满足 $k\alpha + l\beta + m\gamma = 0$ ($km \neq 0$),则
    \begin{enumerate}
        \item [(A)] $\alpha, \beta$ 与 $\alpha, \gamma$ 等价
        \item [(B)] $\alpha, \beta$ 与 $\beta, \gamma$ 等价
        \item [(C)] $\alpha, \gamma$ 与 $\beta, \gamma$ 等价
        \item [(D)] $\alpha$ 与 $\gamma$ 等价
    \end{enumerate}
    
    \begin{solution}
    由于$km\neq 0$则有$\begin{cases}
        \alpha = -\frac{1}{k}\left(l\beta+m\gamma\right) \\
        \gamma = -\frac{1}{k}\left(l\beta+k\alpha\right) 
    \end{cases} \implies \begin{cases}
        \beta,\gamma \rightarrow \alpha \\
        \beta,\alpha \rightarrow \gamma 
    \end{cases}$
    又因为$(\beta,\gamma) \rightarrow \beta,(\beta,\alpha) \rightarrow \beta $是显然的,故
    $(\alpha,\beta)\rightarrow(\beta,\gamma)$
    \end{solution}
    
    \item (2004, 数三) 设 $\alpha_1 = (1,2,0)^T$, $\alpha_2 = (1, a+2, -3a)^T$, $\alpha_3 = (-1, -b-2, a+2b)^T$,
    $\beta = (1,3,-3)^T$。当 $a, b$ 为何值时,
    \begin{enumerate}
        \item [(I)] $\beta$不能由$\alpha_1, \alpha_2, \alpha_3$线性表示
        \item [(II)] $\beta$ 可由 $\alpha_1, \alpha_2, \alpha_3$ 唯一地线性表示,并求出表示式;
        \item [(III)] $\beta$ 可由 $\alpha_1, \alpha_2, \alpha_3$ 线性表示,但表示式不唯一,并求出表示式。
    \end{enumerate}
    
    \begin{solution}
    数字矩阵多半带参数,关键就是讨论这个参数的范围.记$A=\left(\alpha_1,\alpha_2,\alpha_3\right)$ 联立有
    $$
    (A\mid\beta) \rightarrow \begin{pmatrix}
        1 & 1 & -1 & 1\\
        0 & a & -b & 1 \\
        0 & 0 & a-b & 0
    \end{pmatrix}
    $$
    \begin{enumerate}
        \item [(1)] 当$a\neq 0$的时候
        $$
        (A\mid\beta) = \begin{pmatrix}
            1 & 1 & -1 & 1 \\
            0 & 0 & 0 &  1 \\
            0 & 0 & 0 & 0
        \end{pmatrix}
        $$
        此时$r(A)<r(A\mid\beta)$ 即$\beta$不可以有$\alpha_i$表示
        \item [(2)] 当$a\neq 0$且$a\neq b$时有
        $$
        (A\mid\beta) = \begin{pmatrix}
            E  & \begin{matrix}
                1 - \frac{1}{a} \\
                \frac{1}{a} \\
                0
            \end{matrix}
        \end{pmatrix}
        $$
        此时$r(A)=r(A\mid\beta)$故$\beta$可由$\alpha_i$唯一表示即
        $$
        \beta = (1-\frac{1}{a})\alpha_1+\frac{1}{a}\alpha_2
        $$
        \item [(3)] 当$a\neq 0, a\neq b$时有
        $$
        (A\mid\beta) = \begin{pmatrix}
            1 & 0 & 0 & 1-\frac{1}{a} \\
            0 & 1 & -1 & \frac{1}{a} \\
            0 & 0 & 0 & 0
        \end{pmatrix}
        $$
        此时$\beta$可由$\alpha_i$无穷多表示,即
        $$
        \beta = (1-\frac{1}{a})\alpha_1 + (k+\frac{1}{a})\alpha_2 + k\alpha_3, k\in\R 
        $$
    \end{enumerate}
    \end{solution}
    
    \item (2019, 数二、三) 设向量组 (I) $\alpha_1 = (1,1,4)^T$, $\alpha_2 = (1,0,4)^T$, $\alpha_3 = (1,2, a^2+3)^T$;
    向量组 (II) $\beta_1 = (1,1, a+3)^T$, $\beta_2 = (0,2,1-a)^T$, $\beta_3 = (1,3, a^2+3)^T$。若向量组 (I) 与 (II) 等价,求 $a$ 的值,
    并将 $\beta_3$ 由 $\alpha_1, \alpha_2, \alpha_3$ 线性表示。
    
    \begin{solution}
    数字矩阵直接用三秩相等即可$r(I)=r(I,II)=r(II)$要分两部分令$A=(\alpha_1,\alpha_2,\alpha_3),B=(\beta_1,\beta_2,\beta_3)$
    $$
    (A\mid B)\rightarrow \left(\begin{array}{c|c}
        \begin{matrix}
            1 & 0 & -2 \\
            0 & 1 & -1 \\
            0 & 0 & a^2 -1 
        \end{matrix} & \begin{matrix}
            1 & 2 & 3 \\
            0 & -2 & -2 \\
            a-1 & 1-a & a^2 -1 
        \end{matrix}
    \end{array}\right)
    B \rightarrow \begin{pmatrix}
        1 & 0 & 1 \\
        0 & 1 & 1 \\
        0 & 0 & a^2-1
    \end{pmatrix}
    $$
    当$a=1$的时候$r(I)=r(I,II)=r(II)=2$此时线性组等价
    $$
    (A\mid\beta_3) \rightarrow \begin{pmatrix}
        1 & 0 & 2 & 3 \\
        0 & 1 & -1 & -2 \\
        0 & 0 & 0 & 0
    \end{pmatrix}
    $$
    即$\beta_3 = (3-2k)\alpha_1+(k-2)\alpha_2+k\alpha_3$ \\
    当$a^2\neq 1$的时候$r(I)=r(I,II)=r(II)=3$此时线性组等价
    $$
    (A\mid\beta_3)\rightarrow\begin{pmatrix}
        E & \begin{matrix}
            1 \\
            -1 \\
            1
        \end{matrix}
    \end{pmatrix}
    $$
    此时$\beta_3=\alpha_1-\alpha_2+\alpha_3$
    \end{solution}
\end{enumerate}

\section{线性相关与线性无关的判定}
\begin{remark}[相关/无关的判定]
    (方法一\ 用定义) \\
    (方法二\ 用秩)
\end{remark}
\begin{enumerate}[label=\arabic*.,start=4]
    \item (2014, 数一、二、三) 设 $\alpha_1, \alpha_2, \alpha_3$ 均为 3 维列向量,则对任意常数 $k, l$,$\alpha_1 + k\alpha_3$, $\alpha_2 + l\alpha_3$ 线性无关是 $\alpha_1, \alpha_2, \alpha_3$ 线性无关的
    \begin{enumerate}
        \item [(A)] 必要非充分条件
        \item [(B)] 充分非必要条件
        \item [(C)] 充分必要条件
        \item [(D)] 既非充分又非必要条件
    \end{enumerate}
    
    \begin{solution}
    证明充分性,取$\alpha_1=(1,0,0)^T,\alpha_2=(0,1,0)^T,\alpha_3=O$显然证明不了$\alpha_i$无关 \\
    证明必要性 \\
    (方法一\ 用定义证明) 
    由线性无关的定义,只需证明$\forall k,l,\exists k_1,k_2$
    $$
    k_1(\alpha_1+k\alpha_3)+k_2(\alpha_2+l\alpha_3) = 0
    $$
    即
    $$
    k_1\alpha_1 + k_2\alpha_2 + (k_1k+l)\alpha_3 = 0
    $$
    由$\alpha_i$线性无关有$\begin{cases}
        k_1 = 0 \\
        k_2 = 0 \\
        k_1k+l = 0
    \end{cases}$ \\
    (方法二\ 用秩)
    $$
    (\alpha_1+k\alpha_3,\alpha_2+l\alpha_3) = (\alpha_1,\alpha_2,\alpha_3)\begin{pmatrix}
        1 & 0 \\
        0 & 1 \\
        k & l 
    \end{pmatrix}
    $$
    记$C=\begin{pmatrix}
        1 & 0 \\
        0 & 1 \\
        k & l 
    \end{pmatrix}$
    又$(\alpha_1,\alpha_2,\alpha_3)$线性无关,故$r(\alpha_1+k\alpha_3,\alpha_2+l\alpha_3)=r(C)=2$
    \end{solution}
    
    \item 设 $A$ 为 $n$ 阶矩阵,$\alpha_1, \alpha_2, \alpha_3$ 均为 $n$ 维列向量,满足 $A^2\alpha_1 = A\alpha_1 \neq 0$, $A^2\alpha_2 = \alpha_1 + A\alpha_2$,
    $A^2\alpha_3 = \alpha_2 + A\alpha_3$,证明 $\alpha_1, \alpha_2, \alpha_3$ 线性无关。
    
    \begin{solution}
    \newpage
    \end{solution}
    
    \item 设 4 维列向量 $\alpha_1, \alpha_2, \alpha_3$ 线性无关,与 4 维列向量 $\beta_1, \beta_2$ 两两正交,证明 $\beta_1, \beta_2$ 线性相关。
    
    \begin{solution}
    \newpage
    \end{solution}
\end{enumerate}

\section{极大线性无关组的判定与计算}

\begin{enumerate}[label=\arabic*.,start=7]
    \item 设 $\alpha_1 = (1,1,1,3)^T$, $\alpha_2 = (-1,-3,5,1)^T$, $\alpha_3 = (3,2,-1, a+2)^T$, $\alpha_4 = (-2,-6,10, a)^T$。
    \begin{enumerate}
        \item [(I)] 当 $a$ 为何值时,该向量组线性相关,并求其一个极大线性无关组;
        \item [(II)] 当 $a$ 为何值时,该向量组线性无关,并将 $\alpha = (4,1,6,10)^T$ 由其线性表示。
    \end{enumerate}
    
    \begin{solution}
    \newpage
    \end{solution}
    
    \item 证明:
    \begin{enumerate}
        \item [(I)] 设 $A, B$ 为 $m \times n$ 矩阵,则 $r(A+B) \leq r(A) + r(B)$;
        \item [(II)] 设 $A$ 为 $m \times n$ 矩阵,$B$ 为 $n \times s$ 矩阵,则 $r(AB) \leq \min\{r(A), r(B)\}$。
    \end{enumerate}
    
    \begin{solution}
    \newpage
    \end{solution}
\end{enumerate}

\section{向量空间(数一专题)}
\begin{remark}
    向量空间 \\
    \textbf{过度矩阵} \\
    由基$\alpha_1,\alpha_2,\ldots,\alpha_n$到基$\beta_1,\beta_2,\ldots,\beta_n$的过渡矩阵为$(\beta_1,\beta_2,\ldots,\beta_n)
    =(\alpha_1,\alpha_2,\ldots,\alpha_n)C$即$C=(\alpha_1,\alpha_2,\ldots,\alpha_n)^{-1}(\beta_1,\beta_2,\ldots,\beta_n)$ \\
    \textbf{坐标转换公式} \\
    设向量$\gamma$在基$\alpha_1,\alpha_2,\ldots,\alpha_n$中的坐标为$x=(x_1,x_2,\ldots,x_n)^{T}$,在基$\beta_1,\beta_2,\ldots,\beta_n$
    中的坐标为$y=(y_1,y_2,\ldots,y_n)^{T}$则坐标转换公式为$x=Cy$
\end{remark}
\begin{enumerate}[label=\arabic*.,start=8]
    \item (2015, 数一) 设向量组 $\alpha_1, \alpha_2, \alpha_3$ 为 $R^3$ 的一个基,$\beta_1 = 2\alpha_1 + 2k\alpha_3$, $\beta_2 = 2\alpha_2$,
    $\beta_3 = \alpha_1 + (k+1)\alpha_3$。
    \begin{enumerate}
        \item (I) 证明向量组 $\beta_1, \beta_2, \beta_3$ 为 $R^3$ 的一个基:
        \item (II) 当 $k$ 为何值时,存在非零向量 $\xi$ 在基 $\alpha_1, \alpha_2, \alpha_3$ 与基 $\beta_1, \beta_2, \beta_3$ 下的坐标相同,并求所有的 $\xi$。
    \end{enumerate}
    
    \begin{solution}
    \newpage
    \end{solution}
\end{enumerate}

\ifx\allfiles\undefined
\end{document}
\fi