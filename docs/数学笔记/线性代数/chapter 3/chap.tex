\ifx\allfiles\undefined
\documentclass[12pt, a4paper, oneside, UTF8]{ctexbook}
\usepackage{multirow}
\def\path{../../config}
\usepackage{amsthm}
\usepackage{amssymb}
\usepackage{array}
\usepackage{xcolor}
\usepackage{graphicx}
\usepackage{mathrsfs}
\usepackage{enumitem}
\usepackage{geometry}
\usepackage[colorlinks, linkcolor=black]{hyperref}
\usepackage{stackengine}
\usepackage{yhmath}
\usepackage{extarrows}
\usepackage{tikz}
\usepackage{forest}
\usetikzlibrary{decorations.pathreplacing, positioning}
% \usepackage{unicode-math}
\usepackage{esint}
\usepackage{pifont}
\usepackage{tcolorbox}
\tcbuselibrary{skins, breakable}

\usepackage{multicol} 
\usepackage{fontspec} % 使用字体

\setmainfont{Times New Roman}
\setCJKmainfont{LXGWWenKai-Light}[
    SlantedFont=*
]

\usepackage{listings} % 用于插入代码

% 定义代码高亮风格
\lstset{
    basicstyle=\ttfamily\small,        % 基本字体样式(等宽小字体)
    keywordstyle=\color{blue},         % 关键字颜色
    commentstyle=\color{green},        % 注释颜色
    stringstyle=\color{red},           % 字符串颜色
    numbers=none,
    breaklines=true,                   % 自动换行
    frame=single,                      % 代码框边框
    rulecolor=\color{black},           % 边框颜色
    captionpos=b,                      % 标题位置(底部)
    showspaces=false,                  % 不显示空格标记
    showstringspaces=false,            % 不显示字符串中的空格标记
    language=C                         % 设置语言为 C
}

\usepackage{fontawesome5}

\usepackage{amsmath}
\usepackage{booktabs, array}
\usepackage{makecell}
\usepackage{fancyhdr}
\usepackage[dvipsnames, svgnames]{xcolor}
\usepackage{listings}
\usepackage{tasks}[2020/01/11]

\everymath{\displaystyle}

\definecolor{mygreen}{rgb}{0,0.6,0}
\definecolor{mygray}{rgb}{0.5,0.5,0.5}
\definecolor{mymauve}{rgb}{0.58,0,0.82}
\definecolor{NavyBlue}{RGB}{0,0,128}
\definecolor{Rhodamine}{RGB}{255,0,255}
\definecolor{PineGreen}{RGB}{0,128,0}

\graphicspath{ {figures/},{../figures/}, {config/}, {../config/} }

\linespread{1.6}

\geometry{
    top=25.4mm, 
    bottom=25.4mm, 
    left=20mm, 
    right=20mm, 
    headheight=2.17cm, 
    headsep=4mm, 
    footskip=12mm
}

\setenumerate[1]{itemsep=5pt,partopsep=0pt,parsep=\parskip,topsep=5pt}
\setitemize[1]{itemsep=5pt,partopsep=0pt,parsep=\parskip,topsep=5pt}
\setdescription{itemsep=5pt,partopsep=0pt,parsep=\parskip,topsep=5pt}



% \begin{lstlisting}[language=TeX] ... \end{lstlisting}

% 定理环境设置
% ---------- 颜色 ----------
\definecolor{ExBlue}{HTML}{4F81BD}
\definecolor{SolGreen}{HTML}{77933C}
\definecolor{DefRed}{HTML}{C5504B}
\definecolor{ThmOrange}{HTML}{E97132}
\definecolor{RemGray}{HTML}{7F7F7F}
\definecolor{CorPurple}{HTML}{7030A0}
\definecolor{ForGray}{HTML}{595959}

% ---------- 通用“变色”模板 ----------
\tcbset{
    mybox/.style n args={1}{
        enhanced, breakable,
        arc=6pt,
        boxrule=0.6pt,
        left=8pt, right=8pt, top=6pt, bottom=6pt,
        drop shadow={black!25},
        fonttitle=\bfseries,
        coltitle=white,
        colbacktitle=#1!85,
        colback=#1!10,
        colframe=#1,
    }
}

% ---------- 各环境 ----------
% 例题
\newtcolorbox{example}[1][]{mybox={ExBlue}, title={\ifstrempty{#1}{Example}{#1}}}
% 解答
\newtcolorbox{solution}[1][]{mybox={SolGreen}, title={\ifstrempty{#1}{Solution}{#1}}}
% 定义
\newtcolorbox{definition}[1][]{mybox={DefRed}, title={\ifstrempty{#1}{Definition}{#1}}}
% 定理
\newtcolorbox{theorem}[1][]{mybox={ThmOrange}, title={\ifstrempty{#1}{Theorem}{#1}}}
% 标注
\newtcolorbox{remark}[1][]{mybox={RemGray}, title={\ifstrempty{#1}{Remark}{#1}}}
% 推论
\newtcolorbox{corollary}[1][]{mybox={CorPurple}, title={\ifstrempty{#1}{Corollary}{#1}}}
% 公式
\newtcolorbox{formula}[1][]{mybox={ForGray}, title={\ifstrempty{#1}{Formula}{#1}}}


\settasks{
    label-format = \bfseries,
    label        = \Alph*.,
    label-width  = 1.2em,
    label-offset = 0.3em,
    item-indent  = 1.9em,
    column-sep   = 0.5em
}

\newenvironment{choices}[1][4]   % 默认 4 栏
    {\begin{tasks}(#1)}
    {\end{tasks}}

% 自定义命令的文件

\def\d{\mathrm{d}}
\def\R{\mathbb{R}}
\def\P{\partial} 
\newcommand{\bs}[1]{\begin{solution}#1\end{solution}}
\newcommand{\bt}[1][1]{% 默认参数为1
    \ensuremath{% 确保数学模式
        \foreach \n in {1,...,#1} {\blacktriangle}% 循环输出 #1 个黑色三角形
    }%
}

\newcommand{\bl}[1][1]{% 默认参数为1
    \ensuremath{% 确保数学模式
        \foreach \n in {1,...,#1} {\blacklozenge}% 循环输出 #1 个黑色三角形
    }%
}
\newif\ifshowanswers
%\showanswerstrue % 注释掉这行就不显示答案

% 定义答案环境
\newcommand{\answer}[1]{%
    \ifshowanswers
        #1%
    \fi
}




% 修改参数改变封面样式,0 默认原始封面、内置其他1、2、3种封面样式
\def\myIndex{3}


\ifnum\myIndex>0
    \input{\path/cover_package_\myIndex} 
\fi

\def\myTitle{冲刺150笔记}
\def\myAuthor{Weary Bird}
\def\myDateCover{\today}
\def\myDateForeword{\today}
\def\myForeword{行香子}
\def\myForewordText{
树绕村庄,水满陂塘;倚东风、豪兴徜徉。小园几许,收尽春光。有桃花红,李花白,菜花黄。 \\
远远苔墙,隐隐茅堂;飏青旗、流水桥旁。偶然乘兴,步过东冈。正莺儿啼,燕儿舞,蝶儿忙。 \\
}
\def\mySubheading{知错能改善莫大焉}


\begin{document}
% \input{../config/cover}
\else
\fi

\chapter{向量}
\section{知识体系}
$$
\fbox{\text{向量}}
\begin{cases}
    \fbox{\text{基本运算}} \begin{cases}
        \alpha + \beta \\
        k\alpha \\
        [\alpha, \beta]
    \end{cases} \\
    \fbox{\text{线性表示}} \begin{cases}
        \fbox{\text{定义}}\quad\beta=k_1\alpha_1+k_2\alpha_2+\ldots+k_s\alpha_s \\
        \fbox{\text{充要条件}} \begin{cases}
            \text{非其次线性方程组}(\alpha_1,\alpha_2,\ldots,\alpha_s)(x_1,x_2,\ldots,x_s)^{T}=\beta\text{有解} \\
            r(\alpha_1,\alpha_2,\ldots,alpha_s)=r(\alpha_1,\alpha_2,\ldots,\alpha_s\mid \beta)
        \end{cases} \\
        \fbox{\text{求法}} \left(\alpha_1,\alpha_2,\ldots,\alpha_s\mid\beta\right) \xrightarrow{\text{初等行变换}}\text{行最简形矩阵}
    \end{cases} \\
    \fbox{\text{线性相关}} \begin{cases}
        \fbox{\text{定义}} \\
        \fbox{\text{充要条件}} \begin{cases}
            \text{至少有一个向量可以由其余向量线性表示} \\
            \text{齐次线性方程组} (\alpha_1,\alpha_2,\ldots,\alpha_s)(x_1,x_2,\ldots,x_s)^{T}=0 \text{有非零解} \\
            r(\alpha_1,\alpha_2,\ldots,\alpha_s) < s
        \end{cases}\\
        \fbox{\text{充分条件}}
    \end{cases}
\end{cases}
$$

$$
\fbox{向量}\begin{cases}
        \fbox{\text{线性无关}} \begin{cases}
        \fbox{\text{定义}} \\
        \fbox{\text{充要条件}}  \begin{cases}
            \text{任意向量都不可以由其余向量线性表示} \\
            \text{齐次线性方程组} (\alpha_1,\alpha_2,\ldots,\alpha_s)(x_1,x_2,\ldots,x_s)^{T}=0 \text{只有零解} \\
            r(\alpha_1,\alpha_2,\ldots,\alpha_s) = s
        \end{cases}\\
        \fbox{\text{充分条件}} \\
    \end{cases} \\
    \fbox{\text{极大无关组与向量组的秩}} \begin{cases}
        \fbox{\text{定义}} \\
        \fbox{\text{求法}} (\alpha_1,\alpha_2,\ldots,\alpha_s)\xrightarrow{\text{初等行变换}}\text{行阶梯形}
    \end{cases}
\end{cases}
$$


\section{线性表示的判定与计算}

\begin{enumerate}[label=\arabic*.]
    \item 设向量组 $\alpha, \beta, \gamma$ 与数 $k, l, m$ 满足 $k\alpha + l\beta + m\gamma = 0$ ($km \neq 0$),则
    \begin{enumerate}
        \item [(A)] $\alpha, \beta$ 与 $\alpha, \gamma$ 等价
        \item [(B)] $\alpha, \beta$ 与 $\beta, \gamma$ 等价
        \item [(C)] $\alpha, \gamma$ 与 $\beta, \gamma$ 等价
        \item [(D)] $\alpha$ 与 $\gamma$ 等价
    \end{enumerate}
    
    \begin{solution}
    \newpage
    \end{solution}
    
    \item (2004, 数三) 设 $\alpha_1 = (1,2,0)^T$, $\alpha_2 = (1, a+2, -3a)^T$, $\alpha_3 = (-1, -b-2, a+2b)^T$,
    $\beta = (1,3,-3)^T$。当 $a, b$ 为何值时,
    \begin{enumerate}
        \item [(I)] $\beta$不能由$\alpha_1, \alpha_2, \alpha_3$线性表示
        \item [(II)] $\beta$ 可由 $\alpha_1, \alpha_2, \alpha_3$ 唯一地线性表示,并求出表示式;
        \item [(III)] $\beta$ 可由 $\alpha_1, \alpha_2, \alpha_3$ 线性表示,但表示式不唯一,并求出表示式。
    \end{enumerate}
    
    \begin{solution}
    \newpage
    \end{solution}
    
    \item (2019, 数二、三) 设向量组 (I) $\alpha_1 = (1,1,4)^T$, $\alpha_2 = (1,0,4)^T$, $\alpha_3 = (1,2, a^2+3)^T$;
    向量组 (II) $\beta_1 = (1,1, a+3)^T$, $\beta_2 = (0,2,1-a)^T$, $\beta_3 = (1,3, a^2+3)^T$。若向量组 (I) 与 (II) 等价,求 $a$ 的值,
    并将 $\beta_3$ 由 $\alpha_1, \alpha_2, \alpha_3$ 线性表示。
    
    \begin{solution}
    \newpage
    \end{solution}
\end{enumerate}

\section{线性相关与线性无关的判定}

\begin{enumerate}[label=\arabic*.,start=4]
    \item (2014, 数一、二、三) 设 $\alpha_1, \alpha_2, \alpha_3$ 均为 3 维列向量,则对任意常数 $k, l$,$\alpha_1 + k\alpha_3$, $\alpha_2 + l\alpha_3$ 线性无关是 $\alpha_1, \alpha_2, \alpha_3$ 线性无关的
    \begin{enumerate}
        \item [(A)] 必要非充分条件
        \item [(B)] 充分非必要条件
        \item [(C)] 充分必要条件
        \item [(D)] 既非充分又非必要条件
    \end{enumerate}
    
    \begin{solution}
    \newpage
    \end{solution}
    
    \item 设 $A$ 为 $n$ 阶矩阵,$\alpha_1, \alpha_2, \alpha_3$ 均为 $n$ 维列向量,满足 $A^2\alpha_1 = A\alpha_1 \neq 0$, $A^2\alpha_2 = \alpha_1 + A\alpha_2$,
    $A^2\alpha_3 = \alpha_2 + A\alpha_3$,证明 $\alpha_1, \alpha_2, \alpha_3$ 线性无关。
    
    \begin{solution}
    \newpage
    \end{solution}
    
    \item 设 4 维列向量 $\alpha_1, \alpha_2, \alpha_3$ 线性无关,与 4 维列向量 $\beta_1, \beta_2$ 两两正交,证明 $\beta_1, \beta_2$ 线性相关。
    
    \begin{solution}
    \newpage
    \end{solution}
\end{enumerate}

\section{极大线性无关组的判定与计算}

\begin{enumerate}[label=\arabic*.,start=7]
    \item 设 $\alpha_1 = (1,1,1,3)^T$, $\alpha_2 = (-1,-3,5,1)^T$, $\alpha_3 = (3,2,-1, a+2)^T$, $\alpha_4 = (-2,-6,10, a)^T$。
    \begin{enumerate}
        \item [(I)] 当 $a$ 为何值时,该向量组线性相关,并求其一个极大线性无关组;
        \item [(II)] 当 $a$ 为何值时,该向量组线性无关,并将 $\alpha = (4,1,6,10)^T$ 由其线性表示。
    \end{enumerate}
    
    \begin{solution}
    \newpage
    \end{solution}
    
    \item 证明:
    \begin{enumerate}
        \item [(I)] 设 $A, B$ 为 $m \times n$ 矩阵,则 $r(A+B) \leq r(A) + r(B)$;
        \item [(II)] 设 $A$ 为 $m \times n$ 矩阵,$B$ 为 $n \times s$ 矩阵,则 $r(AB) \leq \min\{r(A), r(B)\}$。
    \end{enumerate}
    
    \begin{solution}
    \newpage
    \end{solution}
\end{enumerate}

\section{向量空间(数一专题)}
\begin{remark}
    向量空间 \\
    \textbf{过度矩阵} \\
    由基$\alpha_1,\alpha_2,\ldots,\alpha_n$到基$\beta_1,\beta_2,\ldots,\beta_n$的过渡矩阵为$(\beta_1,\beta_2,\ldots,\beta_n)
    =(\alpha_1,\alpha_2,\ldots,\alpha_n)C$即$C=(\alpha_1,\alpha_2,\ldots,\alpha_n)^{-1}(\beta_1,\beta_2,\ldots,\beta_n)$ \\
    \textbf{坐标转换公式} \\
    设向量$\gamma$在基$\alpha_1,\alpha_2,\ldots,\alpha_n$中的坐标为$x=(x_1,x_2,\ldots,x_n)^{T}$,在基$\beta_1,\beta_2,\ldots,\beta_n$
    中的坐标为$y=(y_1,y_2,\ldots,y_n)^{T}$则坐标转换公式为$x=Cy$
\end{remark}
\begin{enumerate}[label=\arabic*.,start=8]
    \item (2015, 数一) 设向量组 $\alpha_1, \alpha_2, \alpha_3$ 为 $R^3$ 的一个基,$\beta_1 = 2\alpha_1 + 2k\alpha_3$, $\beta_2 = 2\alpha_2$,
    $\beta_3 = \alpha_1 + (k+1)\alpha_3$。
    \begin{enumerate}
        \item (I) 证明向量组 $\beta_1, \beta_2, \beta_3$ 为 $R^3$ 的一个基:
        \item (II) 当 $k$ 为何值时,存在非零向量 $\xi$ 在基 $\alpha_1, \alpha_2, \alpha_3$ 与基 $\beta_1, \beta_2, \beta_3$ 下的坐标相同,并求所有的 $\xi$。
    \end{enumerate}
    
    \begin{solution}
    \newpage
    \end{solution}
\end{enumerate}

\ifx\allfiles\undefined
\end{document}
\fi