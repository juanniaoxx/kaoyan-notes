\ifx\allfiles\undefined
\documentclass[12pt, a4paper, oneside, UTF8]{ctexbook}
\usepackage{multirow}
\def\path{../../config}
\usepackage{amsmath}
\usepackage{amsthm}
\usepackage{amssymb}
\usepackage{array}
\usepackage{xcolor}
\usepackage{graphicx}
\usepackage{mathrsfs}
\usepackage{enumitem}
\usepackage{geometry}
\usepackage[colorlinks, linkcolor=black]{hyperref}
\usepackage{stackengine}
\usepackage{yhmath}
\usepackage{extarrows}
\usepackage{tikz}
\usepackage{pgfplots}
\usepackage{asymptote}
\usepackage{float}
\usepackage{fontspec} % 使用字体

\setmainfont{Times New Roman}
\setCJKmainfont{LXGWWenKai-Light}[
    SlantedFont=*
]

\everymath{\displaystyle}

\usepgfplotslibrary{polar}
\usepackage{subcaption}
\usetikzlibrary{decorations.pathreplacing, positioning}

\usepgfplotslibrary{fillbetween}
\pgfplotsset{compat=1.18}
% \usepackage{unicode-math}
\usepackage{esint}
\usepackage[most]{tcolorbox}

\usepackage{fancyhdr}
\usepackage[dvipsnames, svgnames]{xcolor}
\usepackage{listings}

\definecolor{mygreen}{rgb}{0,0.6,0}
\definecolor{mygray}{rgb}{0.5,0.5,0.5}
\definecolor{mymauve}{rgb}{0.58,0,0.82}
\definecolor{NavyBlue}{RGB}{0,0,128}
\definecolor{Rhodamine}{RGB}{255,0,255}
\definecolor{PineGreen}{RGB}{0,128,0}

\graphicspath{ {figures/},{../figures/}, {config/}, {../config/} }

\linespread{1.6}

\geometry{
    top=25.4mm, 
    bottom=25.4mm, 
    left=20mm, 
    right=20mm, 
    headheight=2.17cm, 
    headsep=4mm, 
    footskip=12mm
}

\setenumerate[1]{itemsep=5pt,partopsep=0pt,parsep=\parskip,topsep=5pt}
\setitemize[1]{itemsep=5pt,partopsep=0pt,parsep=\parskip,topsep=5pt}
\setdescription{itemsep=5pt,partopsep=0pt,parsep=\parskip,topsep=5pt}

\lstset{
    language=Mathematica,
    basicstyle=\tt,
    breaklines=true,
    keywordstyle=\bfseries\color{NavyBlue}, 
    emphstyle=\bfseries\color{Rhodamine},
    commentstyle=\itshape\color{black!50!white}, 
    stringstyle=\bfseries\color{PineGreen!90!black},
    columns=flexible,
    numbers=left,
    numberstyle=\footnotesize,
    frame=tb,
    breakatwhitespace=false,
} 

\lstset{
    language=TeX, % 设置语言为 TeX
    basicstyle=\ttfamily, % 使用等宽字体
    breaklines=true, % 自动换行
    keywordstyle=\bfseries\color{NavyBlue}, % 关键字样式
    emphstyle=\bfseries\color{Rhodamine}, % 强调样式
    commentstyle=\itshape\color{black!50!white}, % 注释样式
    stringstyle=\bfseries\color{PineGreen!90!black}, % 字符串样式
    columns=flexible, % 列的灵活性
    numbers=left, % 行号在左侧
    numberstyle=\footnotesize, % 行号字体大小
    frame=tb, % 顶部和底部边框
    breakatwhitespace=false % 不在空白处断行
}

% \begin{lstlisting}[language=TeX] ... \end{lstlisting}

% 定理环境设置
\usepackage[strict]{changepage} 
\usepackage{framed}

\definecolor{greenshade}{rgb}{0.90,1,0.92}
\definecolor{redshade}{rgb}{1.00,0.88,0.88}
\definecolor{brownshade}{rgb}{0.99,0.95,0.9}
\definecolor{lilacshade}{rgb}{0.95,0.93,0.98}
\definecolor{orangeshade}{rgb}{1.00,0.88,0.82}
\definecolor{lightblueshade}{rgb}{0.8,0.92,1}
\definecolor{purple}{rgb}{0.81,0.85,1}

\theoremstyle{definition}
\newtheorem{myDefn}{\indent Definition}[section]
\newtheorem{myLemma}{\indent Lemma}[section]
\newtheorem{myThm}[myLemma]{\indent Theorem}
\newtheorem{myCorollary}[myLemma]{\indent Corollary}
\newtheorem{myCriterion}[myLemma]{\indent Criterion}
\newtheorem*{myRemark}{\indent Remark}
\newtheorem{myProposition}{\indent Proposition}[section]

\newenvironment{formal}[2][]{%
	\def\FrameCommand{%
		\hspace{1pt}%
		{\color{#1}\vrule width 2pt}%
		{\color{#2}\vrule width 4pt}%
		\colorbox{#2}%
	}%
	\MakeFramed{\advance\hsize-\width\FrameRestore}%
	\noindent\hspace{-4.55pt}%
	\begin{adjustwidth}{}{7pt}\vspace{2pt}\vspace{2pt}}{%
		\vspace{2pt}\end{adjustwidth}\endMakeFramed%
}

\newenvironment{definition}{\vspace{-\baselineskip * 2 / 3}%
	\begin{formal}[Green]{greenshade}\vspace{-\baselineskip * 4 / 5}\begin{myDefn}}
	{\end{myDefn}\end{formal}\vspace{-\baselineskip * 2 / 3}}

\newenvironment{theorem}{\vspace{-\baselineskip * 2 / 3}%
	\begin{formal}[LightSkyBlue]{lightblueshade}\vspace{-\baselineskip * 4 / 5}\begin{myThm}}%
	{\end{myThm}\end{formal}\vspace{-\baselineskip * 2 / 3}}

\newenvironment{lemma}{\vspace{-\baselineskip * 2 / 3}%
	\begin{formal}[Plum]{lilacshade}\vspace{-\baselineskip * 4 / 5}\begin{myLemma}}%
	{\end{myLemma}\end{formal}\vspace{-\baselineskip * 2 / 3}}

\newenvironment{corollary}{\vspace{-\baselineskip * 2 / 3}%
	\begin{formal}[BurlyWood]{brownshade}\vspace{-\baselineskip * 4 / 5}\begin{myCorollary}}%
	{\end{myCorollary}\end{formal}\vspace{-\baselineskip * 2 / 3}}

\newenvironment{criterion}{\vspace{-\baselineskip * 2 / 3}%
	\begin{formal}[DarkOrange]{orangeshade}\vspace{-\baselineskip * 4 / 5}\begin{myCriterion}}%
	{\end{myCriterion}\end{formal}\vspace{-\baselineskip * 2 / 3}}
	

\newenvironment{remark}{\vspace{-\baselineskip * 2 / 3}%
	\begin{formal}[LightCoral]{redshade}\vspace{-\baselineskip * 4 / 5}\begin{myRemark}}%
	{\end{myRemark}\end{formal}\vspace{-\baselineskip * 2 / 3}}

\newenvironment{proposition}{\vspace{-\baselineskip * 2 / 3}%
	\begin{formal}[RoyalPurple]{purple}\vspace{-\baselineskip * 4 / 5}\begin{myProposition}}%
	{\end{myProposition}\end{formal}\vspace{-\baselineskip * 2 / 3}}


\newtheorem{example}{\indent \color{SeaGreen}{Example}}[section]
\renewcommand{\proofname}{\indent\textbf{\textcolor{TealBlue}{Proof}}}
\NewEnviron{solution}{%
	\begin{proof}[\indent\textbf{\textcolor{TealBlue}{Solution}}]%
		\color{blue}% 设置内容为蓝色
		\BODY% 插入环境内容
		\color{black}% 恢复默认颜色(可选,避免影响后续文字)
	\end{proof}%
}

% 自定义命令的文件

\def\d{\mathrm{d}}
\def\R{\mathbb{R}}
%\newcommand{\bs}[1]{\boldsymbol{#1}}
%\newcommand{\ora}[1]{\overrightarrow{#1}}
\newcommand{\myspace}[1]{\par\vspace{#1\baselineskip}}
\newcommand{\xrowht}[2][0]{\addstackgap[.5\dimexpr#2\relax]{\vphantom{#1}}}
\newenvironment{mycases}[1][1]{\linespread{#1} \selectfont \begin{cases}}{\end{cases}}
\newenvironment{myvmatrix}[1][1]{\linespread{#1} \selectfont \begin{vmatrix}}{\end{vmatrix}}
\newcommand{\tabincell}[2]{\begin{tabular}{@{}#1@{}}#2\end{tabular}}
\newcommand{\pll}{\kern 0.56em/\kern -0.8em /\kern 0.56em}
\newcommand{\dive}[1][F]{\mathrm{div}\;\boldsymbol{#1}}
\newcommand{\rotn}[1][A]{\mathrm{rot}\;\boldsymbol{#1}}

\newif\ifshowanswers
\showanswerstrue % 注释掉这行就不显示答案

% 定义答案环境
\newcommand{\answer}[1]{%
    \ifshowanswers
        #1%
    \fi
}

% 修改参数改变封面样式,0 默认原始封面、内置其他1、2、3种封面样式
\def\myIndex{0}


\ifnum\myIndex>0
    \input{\path/cover_package_\myIndex} 
\fi

\def\myTitle{考研数学笔记}
\def\myAuthor{Weary Bird}
\def\myDateCover{\today}
\def\myDateForeword{\today}
\def\myForeword{相见欢·林花谢了春红}
\def\myForewordText{
    林花谢了春红,太匆匆。
    无奈朝来寒雨晚来风。
    胭脂泪,相留醉,几时重。
    自是人生长恨水长东。
}
\def\mySubheading{以姜晓千强化课讲义为底本}


\begin{document}
% \input{\path/cover_text_\myIndex.tex}

\newpage
\thispagestyle{empty}
\begin{center}
    \Huge\textbf{\myForeword}
\end{center}
\myForewordText
\begin{flushright}
    \begin{tabular}{c}
        \myDateForeword
    \end{tabular}
\end{flushright}

\newpage
\pagestyle{plain}
\setcounter{page}{1}
\pagenumbering{Roman}
\tableofcontents

\newpage
\pagenumbering{arabic}
% \setcounter{chapter}{-1}
\setcounter{page}{1}

\pagestyle{fancy}
\fancyfoot[C]{\thepage}
\renewcommand{\headrulewidth}{0.4pt}
\renewcommand{\footrulewidth}{0pt}








\else
\fi

\chapter{特征值与特征向量}
$$
\fbox{
    \begin{tabular}{c}
        特 \\征 \\值 \\与 \\特 \\征 \\向 \\量 
    \end{tabular}
}\begin{cases}
    \fbox{特征值与特征向量}\begin{cases}
        \fbox{定义} & A\alpha = \lambda \alpha (\alpha\neq 0) \\
        \fbox{性质} & \\
        \fbox{特征方程法}, &\begin{cases}
            \left|A-\lambda E\right| \\
            \left(A-\lambda E\right)x = 0
        \end{cases}
    \end{cases} \\
    \fbox{相似矩阵} \begin{cases}
        \fbox{定义}, & B = P^{-1}AP \\
        \fbox{性质}
    \end{cases} \\
    \fbox{相似对角化}\begin{cases}
        \fbox{定义} & P^{-1}AP = \Lambda \\
        \fbox{充要条件} &\begin{cases}
            \iff A\text{有}n\text{个线性无关的特征向量} \\
            \iff k\text{重特征值有}k\text{个线性无关的特征向量}
        \end{cases} \\
        \fbox{充分条件} &\begin{cases}
            A\text{有}n\text{个不同的特征值} \\
            A\text{为是实对称矩阵}
        \end{cases}
    \end{cases} \\
    \fbox{实对称矩阵}\begin{cases}
        \text{特征值均为实数} \\
        \text{不同特征值的特征向量正交} \\
        k\text{重特征值有}k\text{个线性无关的特征向量} \\
        A\text{可正交相似对角化,即存在正交矩阵}Q,\text{使得}Q^{-1}AQ=A^{T}AQ=\Lambda
    \end{cases}
\end{cases}
$$
\section{特征值与特征向量的计算}
\begin{remark}
    特征值与特征值向量的性质
    \begin{enumerate}
        \item [(1)] 不同特征值的特征向量线性无关 
        \item [(2)] 不同特征值的特征向量之和不是特征向量
        \item [(3)] $k\text{重特征值有}k\text{个线性无关的特征向量}$ 
        \item [(4)] 设$A$的特征值为$\lambda_1,\lambda_2\ldots,\lambda_n$则$\displaystyle \sum_{i=1}^{n}\lambda_i=tr(A),\Pi_{i=1}^{n}\lambda_i=\left|A\right|$ 
        \item [(5)] 若$r(A)=1$则$A=\alpha\beta^{T}$,其中$\alpha,\beta$是$n$维非零列向量,则$A$的特征值为 
        $$
        \lambda_1=tr(A)=\alpha^T\beta=\beta^T\alpha, \lambda_2=\ldots=\lambda_n = 0
        $$
        \item [(6)] 设$\alpha$为矩阵$A$属于特征值$\lambda$的特征值向量则,有 
        \begin{center}
            \begin{tabular}{ |c|c|c|c|c|c| }
                \hline
                $A$ & $f(A)$ & $A^{-1}$ & $A^*$ & $A^T$ & $P^{-1}AP$  \\
                \hline
                $\lambda$ & $f(\lambda)$ & $\frac{1}{\lambda}$ & $\frac{\left|A\right|}{\lambda}$ & $\lambda$ & $\lambda$ \\
                \hline 
                $\alpha$ & $\alpha$ & $\alpha$ & $\alpha$ & ??? & $P^{-1}\alpha$ \\
                \hline
            \end{tabular}
        \end{center}
    \end{enumerate}
\end{remark}
\begin{enumerate}[label=\arabic*.]
    \item 设
    \begin{align*}
    A = \begin{pmatrix}
    1 & 1 & 1 & 1 \\
    1 & 1 & -1 & -1\\
    1 & -1 & 1 & -1 \\
    1 & -1 & -1 & 1
    \end{pmatrix}
    \end{align*}
    求 $A$ 的特征值与特征向量。
    
    \begin{solution}
    \newpage
    \end{solution}
    
    \item (2003, 数一) 设
    $A = \begin{pmatrix}
    3 & 2 & 2 \\
    2 & 3 & 2 \\
    2 & 2 & 3
    \end{pmatrix},P=\begin{pmatrix}
        0 & 1 & 0 \\
        1 & 0 & 1 \\
        0 & 0 & 1 
    \end{pmatrix},B = P^{-1} A^* P$求 $B + 2E$ 的特征值与特征向量。
    
    \begin{solution}
    \newpage
    \end{solution}
    
    \item 设$
    A = \begin{pmatrix}
    1 & 2 & 2 \\
    -1 & 4 & -2 \\
    1 & -2 & a
    \end{pmatrix}$
    的特征方程有一个二重根,求 $A$ 的特征值与特征向量。
    
    \begin{solution}
    \newpage
    \end{solution}
    
    \item 设 3 阶非零矩阵 $A$ 满足 $A^2 = O$,则 $A$ 的线性无关的特征向量的个数是 \\
    A.0\qquad B.1\qquad C.2\qquad D.3
    
    \begin{solution}
    \newpage
    \end{solution}
    
    \item 设 $A = \alpha \beta^T + \beta \alpha^T$,其中 $\alpha, \beta$ 为 3 维单位列向量,且 $\alpha^T \beta = \frac{1}{3}$,证明:
    \begin{enumerate}
        \item [(I)] 0 为 $A$ 的特征值;
        \item [(II)] $\alpha + \beta, \alpha - \beta$ 为 $A$ 的特征向量;
        \item [(III)] $A$ 可相似对角化。
    \end{enumerate}
    
    \begin{solution}
    \newpage
    \end{solution}
\end{enumerate}

\section{相似的判定与计算}
\begin{remark}
    相似的性质
    \begin{enumerate}
        \item [(1)] 若$A\sim B$,则$A,B$具有相同的行列式,秩,特征方程,特征值与迹
        \item [(2)] 若$A\sim B$,则$f(A)\sim f(B),A^{-1}\sim B^{-1}, AB\sim BA(\left|A\neq 0\right|),A^{T}\sim B^{T},A^{*}\sim B^{*}$ 
        \item [(3)] 若$A\sim B, B\sim C$ 则 $A\sim C$
    \end{enumerate}
\end{remark}
\begin{enumerate}[label=\arabic*.,start=6]
    \item 设$A=\begin{pmatrix}
        1 & 0 & 0 & 0 \\
        0 & 3 & 0 & 0 \\
        0 & 0 & 1 & 1 \\
        0 & 0 & 2 & 2
    \end{pmatrix}$矩阵$B,A$相似,则$r(B-A)+r(B-3E)=\_\_\_\_$
    
    \begin{solution}
        \newpage
    \end{solution}

    \item 设$n$阶矩阵$A,B$相似,满足$A^2=2E$,则$\left|AB+A-B-E\right|=\_\_\_$
    
    \begin{solution}
        \newpage
    \end{solution}

    \item (2019, 数一、二、三) 设
    $
    A = \begin{pmatrix}
    -2 & -2 & 1 \\
    2 & x & -2 \\
    0 & 0 & -2
    \end{pmatrix},
    B = \begin{pmatrix}
    2 & 1 & 0 \\
    0 & -1 & 0 \\
    0 & 0 & y
    \end{pmatrix}
    $相似.\\
    (I) 求 $x, y$ 的值; \\
    (II) 求可逆矩阵 $P$,使得 $P^{-1}AP = B$。
    
    \begin{solution}
    \newpage
    \end{solution}
\end{enumerate}

\section{相似对角化的判定与计算}

\begin{enumerate}[label=\arabic*.,start=9]
    \item (2005, 数一、二) 设 3 阶矩阵 $A$ 的特征值为$1, 3, -2$,对应的特征向量分别为 $\alpha_1, \alpha_2, \alpha_3$。若
    $P = (\alpha_1, 2\alpha_3, -\alpha_2)$
    则 $P^{-1}AP = $ \underline{\hspace{3cm}}。
    
    \begin{solution}
    \newpage
    \end{solution}
    
    \item 设 $n$ 阶方阵 $A$ 满足 $A^2 - 3A + 2E = O$,证明 $A$ 可相似对角化。
    
    \begin{solution}
    \newpage
    \end{solution}
    
    \item (2020, 数一、二、三) 设 $A$ 为 2 阶矩阵,$P = (\alpha, A\alpha)$,其中 $\alpha$ 为非零向量且不是 $A$ 的特征向量。
    \begin{enumerate}
        \item [(I)] 证明 $P$ 为可逆矩阵;
        \item [(II)] 若 $A^2\alpha + 6A\alpha - 10\alpha = 0$,求 $P^{-1}AP$,并判断 $A$ 是否相似于对角矩阵。
    \end{enumerate}
    
    \begin{solution}
    \newpage
    \end{solution}
\end{enumerate}

\section{实对称矩阵的计算}

\begin{enumerate}[label=\arabic*.,start=12]
    \item 设$n$阶实对称矩阵$A$满足$A^2+A=O,n$阶矩阵$B$满足$B^2+B=E$且$r(AB)=2$则$\left|A+2E\right|=\_\_\_$
    
    \begin{solution}
        \newpage
    \end{solution}

    \item (2010, 数二、三) 设
    $
    A = \begin{pmatrix}
    0 & -1 & 4 \\
    -1 & 3 & a  \\
    4 & a & 0  \\
    \end{pmatrix}
    $正交矩阵 $Q$ 使得 $Q^T A Q$ 为对角矩阵。若 $Q$ 的第 1 列为 
    $\frac{1}{\sqrt{6}}(1,2,1)^T$,求 $a, Q$。
    
    \begin{solution}
    \newpage
    \end{solution}
    
    \item 设 3 阶实对称矩阵 $A$ 满足 $A^2 = E$,$A+E$ 的各行元素之和均为零,且 $r(A+E) = 2$。
    \begin{enumerate}
        \item [(I)] 求 $A$ 的特征值与特征向量;
        \item [(II)] 求矩阵 $A$。
    \end{enumerate}
    
    \begin{solution}
    \newpage
    \end{solution}
\end{enumerate}

\ifx\allfiles\undefined
\end{document}
\fi