\ifx\allfiles\undefined
\documentclass[12pt, a4paper, oneside, UTF8]{ctexbook}
\def\path{../../config}
\usepackage{amsthm}
\usepackage{amssymb}
\usepackage{array}
\usepackage{xcolor}
\usepackage{graphicx}
\usepackage{mathrsfs}
\usepackage{enumitem}
\usepackage{geometry}
\usepackage[colorlinks, linkcolor=black]{hyperref}
\usepackage{stackengine}
\usepackage{yhmath}
\usepackage{extarrows}
\usepackage{tikz}
\usepackage{forest}
\usetikzlibrary{decorations.pathreplacing, positioning}
% \usepackage{unicode-math}
\usepackage{esint}
\usepackage{pifont}
\usepackage{tcolorbox}
\tcbuselibrary{skins, breakable}

\usepackage{multicol} 
\usepackage{fontspec} % 使用字体

\setmainfont{Times New Roman}
\setCJKmainfont{LXGWWenKai-Light}[
    SlantedFont=*
]

\usepackage{listings} % 用于插入代码

% 定义代码高亮风格
\lstset{
    basicstyle=\ttfamily\small,        % 基本字体样式(等宽小字体)
    keywordstyle=\color{blue},         % 关键字颜色
    commentstyle=\color{green},        % 注释颜色
    stringstyle=\color{red},           % 字符串颜色
    numbers=none,
    breaklines=true,                   % 自动换行
    frame=single,                      % 代码框边框
    rulecolor=\color{black},           % 边框颜色
    captionpos=b,                      % 标题位置(底部)
    showspaces=false,                  % 不显示空格标记
    showstringspaces=false,            % 不显示字符串中的空格标记
    language=C                         % 设置语言为 C
}

\usepackage{fontawesome5}

\usepackage{amsmath}
\usepackage{booktabs, array}
\usepackage{makecell}
\usepackage{fancyhdr}
\usepackage[dvipsnames, svgnames]{xcolor}
\usepackage{listings}
\usepackage{tasks}[2020/01/11]

\everymath{\displaystyle}

\definecolor{mygreen}{rgb}{0,0.6,0}
\definecolor{mygray}{rgb}{0.5,0.5,0.5}
\definecolor{mymauve}{rgb}{0.58,0,0.82}
\definecolor{NavyBlue}{RGB}{0,0,128}
\definecolor{Rhodamine}{RGB}{255,0,255}
\definecolor{PineGreen}{RGB}{0,128,0}

\graphicspath{ {figures/},{../figures/}, {config/}, {../config/} }

\linespread{1.6}

\geometry{
    top=25.4mm, 
    bottom=25.4mm, 
    left=20mm, 
    right=20mm, 
    headheight=2.17cm, 
    headsep=4mm, 
    footskip=12mm
}

\setenumerate[1]{itemsep=5pt,partopsep=0pt,parsep=\parskip,topsep=5pt}
\setitemize[1]{itemsep=5pt,partopsep=0pt,parsep=\parskip,topsep=5pt}
\setdescription{itemsep=5pt,partopsep=0pt,parsep=\parskip,topsep=5pt}



% \begin{lstlisting}[language=TeX] ... \end{lstlisting}

% 定理环境设置
% ---------- 颜色 ----------
\definecolor{ExBlue}{HTML}{4F81BD}
\definecolor{SolGreen}{HTML}{77933C}
\definecolor{DefRed}{HTML}{C5504B}
\definecolor{ThmOrange}{HTML}{E97132}
\definecolor{RemGray}{HTML}{7F7F7F}
\definecolor{CorPurple}{HTML}{7030A0}
\definecolor{ForGray}{HTML}{595959}

% ---------- 通用“变色”模板 ----------
\tcbset{
    mybox/.style n args={1}{
        enhanced, breakable,
        arc=6pt,
        boxrule=0.6pt,
        left=8pt, right=8pt, top=6pt, bottom=6pt,
        drop shadow={black!25},
        fonttitle=\bfseries,
        coltitle=white,
        colbacktitle=#1!85,
        colback=#1!10,
        colframe=#1,
    }
}

% ---------- 各环境 ----------
% 例题
\newtcolorbox{example}[1][]{mybox={ExBlue}, title={\ifstrempty{#1}{Example}{#1}}}
% 解答
\newtcolorbox{solution}[1][]{mybox={SolGreen}, title={\ifstrempty{#1}{Solution}{#1}}}
% 定义
\newtcolorbox{definition}[1][]{mybox={DefRed}, title={\ifstrempty{#1}{Definition}{#1}}}
% 定理
\newtcolorbox{theorem}[1][]{mybox={ThmOrange}, title={\ifstrempty{#1}{Theorem}{#1}}}
% 标注
\newtcolorbox{remark}[1][]{mybox={RemGray}, title={\ifstrempty{#1}{Remark}{#1}}}
% 推论
\newtcolorbox{corollary}[1][]{mybox={CorPurple}, title={\ifstrempty{#1}{Corollary}{#1}}}
% 公式
\newtcolorbox{formula}[1][]{mybox={ForGray}, title={\ifstrempty{#1}{Formula}{#1}}}


\settasks{
    label-format = \bfseries,
    label        = \Alph*.,
    label-width  = 1.2em,
    label-offset = 0.3em,
    item-indent  = 1.9em,
    column-sep   = 0.5em
}

\newenvironment{choices}[1][4]   % 默认 4 栏
    {\begin{tasks}(#1)}
    {\end{tasks}}

% 自定义命令的文件

\def\d{\mathrm{d}}
\def\R{\mathbb{R}}
\def\P{\partial} 
\newcommand{\bs}[1]{\begin{solution}#1\end{solution}}
\newcommand{\bt}[1][1]{% 默认参数为1
    \ensuremath{% 确保数学模式
        \foreach \n in {1,...,#1} {\blacktriangle}% 循环输出 #1 个黑色三角形
    }%
}

\newcommand{\bl}[1][1]{% 默认参数为1
    \ensuremath{% 确保数学模式
        \foreach \n in {1,...,#1} {\blacklozenge}% 循环输出 #1 个黑色三角形
    }%
}
\newif\ifshowanswers
%\showanswerstrue % 注释掉这行就不显示答案

% 定义答案环境
\newcommand{\answer}[1]{%
    \ifshowanswers
        #1%
    \fi
}




% 修改参数改变封面样式,0 默认原始封面、内置其他1、2、3种封面样式
\def\myIndex{3}


\ifnum\myIndex>0
    \input{\path/cover_package_\myIndex} 
\fi

\def\myTitle{冲刺150笔记}
\def\myAuthor{Weary Bird}
\def\myDateCover{\today}
\def\myDateForeword{\today}
\def\myForeword{行香子}
\def\myForewordText{
树绕村庄,水满陂塘;倚东风、豪兴徜徉。小园几许,收尽春光。有桃花红,李花白,菜花黄。 \\
远远苔墙,隐隐茅堂;飏青旗、流水桥旁。偶然乘兴,步过东冈。正莺儿啼,燕儿舞,蝶儿忙。 \\
}
\def\mySubheading{知错能改善莫大焉}


\begin{document}
\input{../../config/cover}
\else
\fi

\chapter{二重积分}
\section{二重积分的概念}

\begin{enumerate}[label=\arabic*.]
    \item (2010,数一、数二) 
    $\displaystyle\lim_{n\rightarrow\infty}\sum_{i=1}^n\sum_{j=1}^n\frac{n}{(n+i)(n^2+j^2)}=$ \\
    $\displaystyle(A)\int_0^1 dx\int_0^x\frac{1}{(1+x)(1+y^2)}dy \quad (B)\int_0^1 dx\int_0^x\frac{1}{(1+x)(1+y)}dy$ \\
    $\displaystyle(C)\int_0^1 dx\int_0^1\frac{1}{(1+x)(1+y)}dy \quad (D)\int_0^1 dx\int_0^1\frac{1}{(1+x)(1+y^2)}dy$
    
    \begin{solution}
    \newpage
    \end{solution}
    
    \item (2016,数三)设$\displaystyle J_i=\iint_{D_i}\sqrt[3]{x-y}\d x\d y(i=1,2,3)$,其中 \\
        $D_1=\{(x,y)|0\leq x\leq 1,0\leq y\leq 1\}$, \\
        $D_2=\{(x,y)|0\leq x\leq 1,0\leq y\leq \sqrt{x}\}$, \\
        $D_3=\{(x,y)|0\leq x\leq 1,x^2\leq y\leq 1\},$
    则 \\
    $(A)\ J_1<J_2<J_3 \qquad (B)\ J_3<J_1<J_2$ \\
    $(C)\ J_2<J_3<J_1 \qquad (D)\ J_2<J_1<J_3$
    
    \begin{solution}
    \newpage
    \end{solution}
\end{enumerate}

\section{交换积分次序}

\begin{enumerate}[label=\arabic*.,start=3]
    \item (2001,数一)交换二次积分的积分次序:
    $\displaystyle\int_{-1}^0 dy\int_2^{1-y} f(x,y)dx$=\_\_\_\_
    
    \begin{solution}
    \newpage
    \end{solution}
    
    \item 二次积分$\displaystyle\int_{0}^{1}\d y\int_{y}^{1}\left(\frac{e^{x^2}}{x}-e^{y^2}\right)\d x$=\_\_\_\_
    
    \begin{solution}
        \newpage
    \end{solution}
    \item 交换$\displaystyle I=\int_{-\frac{\pi}{4}}^{\frac{\pi}{2}}d\theta\int_0^{a\cos\theta}f(r,\theta)dr$的积分次序。
    
    \begin{solution}
    \newpage
    \end{solution}
\end{enumerate}

\section{二重积分的计算}

\begin{enumerate}[label=\arabic*.,start=6]
    \item (2011,数一、数二)已知函数$f(x,y)$具有二阶连续偏导数,且$f(1,y)=0$,$f(x,1)=0$,$\iint_D f(x,y)dxdy=a$,其中$D=\{(x,y)|0\leq x\leq 1,0\leq y\leq 1\}$,计算二重积分
    \begin{align*}
        I=\iint_D xyf_{xy}''(x,y)\d x\d y.
    \end{align*}
    
    \begin{solution}
    \newpage
    \end{solution}
    
    \item 计算$\iint_D\sqrt{|y-x^2|}dxdy$,其中$D=\{(x,y)|-1\leq x\leq 1,0\leq y\leq 2\}$。
    
    \begin{solution}
    \newpage
    \end{solution}
    
    \item (2018,数二)设平面区域$D$由曲线$\begin{cases}x=t-\sin t \\ y=1-\cos t\end{cases}(0\leq t\leq 2\pi)$与$x$轴围成,计算二重积分$\iint_D(x+2y)dxdy$。
    
    \begin{solution}
    \newpage
    \end{solution}
    
    \item (2007,数二、数三)设二元函数
    \begin{align*}
        f(x,y)=\begin{cases}
            x^2, & |x|+|y|\leq 1 \\
            \frac{1}{\sqrt{x^2+y^2}}, & 1<|x|+|y|\leq 2
        \end{cases}
    \end{align*}
    计算二重积分$\iint_D f(x,y)dxdy$,其中$D=\{(x,y)||x|+|y|\leq 2\}$。
    
    \begin{solution}
    \newpage
    \end{solution}
    
    \item (2014,数二、数三)设平面区域$D=\{(x,y)|1\leq x^2+y^2\leq 4,x\geq 0,y\geq 0\}$,计算
    \begin{align*}
        \iint_D\frac{x\sin(\pi\sqrt{x^2+y^2})}{x+y}\d x\d y.
    \end{align*}
    
    \begin{solution}
    \newpage
    \end{solution}
    
    \item (2019,数二)已知平面区域$D=\{(x,y)||x|\leq y,(x^2+y^2)^3\leq y^4\}$,计算二重积分
    \begin{align*}
        \iint_D\frac{x+y}{\sqrt{x^2+y^2}}\d x\d y.
    \end{align*}
    
    \begin{solution}
    \newpage
    \end{solution}
\end{enumerate}

\section{其他题型}

\begin{enumerate}[label=\arabic*.,start=12]
    \item (2010,数二)计算二重积分$\displaystyle I=\iint_D r^2\sin\theta\sqrt{1-r^2\cos 2\theta} drd\theta$ \\
    其中$\displaystyle D=\left\{(r,\theta)\mid 0\leq r\leq\sec{\theta},0\leq\theta\leq\frac{\pi}{4}\right\}$
    
    \begin{solution}
    \newpage
    \end{solution}
    
    \item (2009,数二、数三)计算二重积分$\iint_D(x-y)dxdy$\\
    其中$\displaystyle D=\{(x,y)|(x-1)^2+(y-1)^2\leq 2,y\geq x\}$.
    
    \begin{solution}
    \newpage
    \end{solution}
\end{enumerate}

\ifx\allfiles\undefined
\end{document}
\fi