\ifx\allfiles\undefined
\documentclass[12pt, a4paper, oneside, UTF8]{ctexbook}
\def\path{../../config}
\usepackage{amsthm}
\usepackage{amssymb}
\usepackage{array}
\usepackage{xcolor}
\usepackage{graphicx}
\usepackage{mathrsfs}
\usepackage{enumitem}
\usepackage{geometry}
\usepackage[colorlinks, linkcolor=black]{hyperref}
\usepackage{stackengine}
\usepackage{yhmath}
\usepackage{extarrows}
\usepackage{tikz}
\usepackage{forest}
\usetikzlibrary{decorations.pathreplacing, positioning}
% \usepackage{unicode-math}
\usepackage{esint}
\usepackage{pifont}
\usepackage{tcolorbox}
\tcbuselibrary{skins, breakable}

\usepackage{multicol} 
\usepackage{fontspec} % 使用字体

\setmainfont{Times New Roman}
\setCJKmainfont{LXGWWenKai-Light}[
    SlantedFont=*
]

\usepackage{listings} % 用于插入代码

% 定义代码高亮风格
\lstset{
    basicstyle=\ttfamily\small,        % 基本字体样式(等宽小字体)
    keywordstyle=\color{blue},         % 关键字颜色
    commentstyle=\color{green},        % 注释颜色
    stringstyle=\color{red},           % 字符串颜色
    numbers=none,
    breaklines=true,                   % 自动换行
    frame=single,                      % 代码框边框
    rulecolor=\color{black},           % 边框颜色
    captionpos=b,                      % 标题位置(底部)
    showspaces=false,                  % 不显示空格标记
    showstringspaces=false,            % 不显示字符串中的空格标记
    language=C                         % 设置语言为 C
}

\usepackage{fontawesome5}

\usepackage{amsmath}
\usepackage{booktabs, array}
\usepackage{makecell}
\usepackage{fancyhdr}
\usepackage[dvipsnames, svgnames]{xcolor}
\usepackage{listings}
\usepackage{tasks}[2020/01/11]

\everymath{\displaystyle}

\definecolor{mygreen}{rgb}{0,0.6,0}
\definecolor{mygray}{rgb}{0.5,0.5,0.5}
\definecolor{mymauve}{rgb}{0.58,0,0.82}
\definecolor{NavyBlue}{RGB}{0,0,128}
\definecolor{Rhodamine}{RGB}{255,0,255}
\definecolor{PineGreen}{RGB}{0,128,0}

\graphicspath{ {figures/},{../figures/}, {config/}, {../config/} }

\linespread{1.6}

\geometry{
    top=25.4mm, 
    bottom=25.4mm, 
    left=20mm, 
    right=20mm, 
    headheight=2.17cm, 
    headsep=4mm, 
    footskip=12mm
}

\setenumerate[1]{itemsep=5pt,partopsep=0pt,parsep=\parskip,topsep=5pt}
\setitemize[1]{itemsep=5pt,partopsep=0pt,parsep=\parskip,topsep=5pt}
\setdescription{itemsep=5pt,partopsep=0pt,parsep=\parskip,topsep=5pt}



% \begin{lstlisting}[language=TeX] ... \end{lstlisting}

% 定理环境设置
% ---------- 颜色 ----------
\definecolor{ExBlue}{HTML}{4F81BD}
\definecolor{SolGreen}{HTML}{77933C}
\definecolor{DefRed}{HTML}{C5504B}
\definecolor{ThmOrange}{HTML}{E97132}
\definecolor{RemGray}{HTML}{7F7F7F}
\definecolor{CorPurple}{HTML}{7030A0}
\definecolor{ForGray}{HTML}{595959}

% ---------- 通用“变色”模板 ----------
\tcbset{
    mybox/.style n args={1}{
        enhanced, breakable,
        arc=6pt,
        boxrule=0.6pt,
        left=8pt, right=8pt, top=6pt, bottom=6pt,
        drop shadow={black!25},
        fonttitle=\bfseries,
        coltitle=white,
        colbacktitle=#1!85,
        colback=#1!10,
        colframe=#1,
    }
}

% ---------- 各环境 ----------
% 例题
\newtcolorbox{example}[1][]{mybox={ExBlue}, title={\ifstrempty{#1}{Example}{#1}}}
% 解答
\newtcolorbox{solution}[1][]{mybox={SolGreen}, title={\ifstrempty{#1}{Solution}{#1}}}
% 定义
\newtcolorbox{definition}[1][]{mybox={DefRed}, title={\ifstrempty{#1}{Definition}{#1}}}
% 定理
\newtcolorbox{theorem}[1][]{mybox={ThmOrange}, title={\ifstrempty{#1}{Theorem}{#1}}}
% 标注
\newtcolorbox{remark}[1][]{mybox={RemGray}, title={\ifstrempty{#1}{Remark}{#1}}}
% 推论
\newtcolorbox{corollary}[1][]{mybox={CorPurple}, title={\ifstrempty{#1}{Corollary}{#1}}}
% 公式
\newtcolorbox{formula}[1][]{mybox={ForGray}, title={\ifstrempty{#1}{Formula}{#1}}}


\settasks{
    label-format = \bfseries,
    label        = \Alph*.,
    label-width  = 1.2em,
    label-offset = 0.3em,
    item-indent  = 1.9em,
    column-sep   = 0.5em
}

\newenvironment{choices}[1][4]   % 默认 4 栏
    {\begin{tasks}(#1)}
    {\end{tasks}}

% 自定义命令的文件

\def\d{\mathrm{d}}
\def\R{\mathbb{R}}
\def\P{\partial} 
\newcommand{\bs}[1]{\begin{solution}#1\end{solution}}
\newcommand{\bt}[1][1]{% 默认参数为1
    \ensuremath{% 确保数学模式
        \foreach \n in {1,...,#1} {\blacktriangle}% 循环输出 #1 个黑色三角形
    }%
}

\newcommand{\bl}[1][1]{% 默认参数为1
    \ensuremath{% 确保数学模式
        \foreach \n in {1,...,#1} {\blacklozenge}% 循环输出 #1 个黑色三角形
    }%
}
\newif\ifshowanswers
%\showanswerstrue % 注释掉这行就不显示答案

% 定义答案环境
\newcommand{\answer}[1]{%
    \ifshowanswers
        #1%
    \fi
}




% 修改参数改变封面样式,0 默认原始封面、内置其他1、2、3种封面样式
\def\myIndex{3}


\ifnum\myIndex>0
    \input{\path/cover_package_\myIndex} 
\fi

\def\myTitle{冲刺150笔记}
\def\myAuthor{Weary Bird}
\def\myDateCover{\today}
\def\myDateForeword{\today}
\def\myForeword{行香子}
\def\myForewordText{
树绕村庄,水满陂塘;倚东风、豪兴徜徉。小园几许,收尽春光。有桃花红,李花白,菜花黄。 \\
远远苔墙,隐隐茅堂;飏青旗、流水桥旁。偶然乘兴,步过东冈。正莺儿啼,燕儿舞,蝶儿忙。 \\
}
\def\mySubheading{知错能改善莫大焉}


\begin{document}
\input{../../config/cover}
\else
\fi

\chapter{多元函数微分学}
\section{多元函数的概念}

\begin{enumerate}[label=\arabic*.]
    \item 例1 求下列重极限:
    \begin{align*}
        (1)\ \lim_{\substack{x\to 0\\ y\to 0}}\frac{x^\alpha y^\beta}{x^2+y^2}\quad (\alpha\geq0,\beta\geq0); \\
        (2)\ \lim_{\substack{x\to 0\\ y\to 0}}\frac{xy(x^{2}-y^{2})}{x^{2}+y^{2}};
    \end{align*}
    
    \begin{solution}
    【详解】
    \end{solution}
    
    \item 例2 (2012,数一)如果函数$f(x,y)$在点$(0,0)$处连续,那么下列命题正确的是
    \begin{align*}
        (A)\ \text{若极限}\lim_{\substack{x\to 0\\ y\to 0}}\frac{f(x,y)}{|x|+|y|}\text{存在,则}f(x,y)\text{在点}(0,0)\text{处可微} \\
        (B)\ \text{若极限}\lim_{\substack{x\to 0\\ y\to 0}}\frac{f(x,y)}{x^{2}+y^{2}}\text{存在,则}f(x,y)\text{在点}(0,0)\text{处可微} \\
        (C)\ \text{若}f(x,y)\text{在点}(0,0)\text{处可微,则极限}\lim_{\substack{x\to 0\\ y\to 0}}\frac{f(x,y)}{|x|+|y|}\text{存在} \\
        (D)\ \text{若}f(x,y)\text{在点}(0,0)\text{处可微,则极限}\lim_{\substack{x\to 0\\ y\to 0}}\frac{f(x,y)}{x^{2}+y^{2}}\text{存在}
    \end{align*}
    
    \begin{solution}
    【详解】
    \end{solution}
    
    \item 例3 (2012,数三)设连续函数$z=f(x,y)$满足
    \begin{align*}
        \lim_{\substack{x\to 0\\ y\to 1}}\frac{f(x,y)-2x+y-2}{\sqrt{x^2+(y-1)^2}}=0
    \end{align*}
    则$\left.dz\right|_{(0,1)}=$
    
    \begin{solution}
    【详解】
    \end{solution}
\end{enumerate}

\section{多元复合函数求偏导数与全微分}

\begin{enumerate}[label=\arabic*.,start=4]
    \item 例4 (2021,数一、数二、数三)设函数$f(x,y)$可微,且
    \begin{align*}
        f(x+1,e^x)=x(x+1)^2, \\
        f(x,x^2)=2x^2\ln x
    \end{align*}
    则$df(1,1)=$
    \begin{align*}
        (A)\ dx+dy \quad (B)\ dx-dy \quad (C)\ dy
    \end{align*}
    
    \begin{solution}
    【详解】
    \end{solution}
    
    \item 例5 (2011,数一、数二)设$z=f(xy,yg(x))$,其中函数$f$具有二阶连续偏导数,函数$g(x)$可导,且在$x=1$处取得极值$g(1)=1$,求$\left.\frac{\partial^2 z}{\partial x\partial y}\right|_{x=1,y=1}$。
    
    \begin{solution}
    【详解】
    \end{solution}
\end{enumerate}

\section{多元隐函数求偏导数与全微分}

\begin{enumerate}[label=\arabic*.,start=6]
    \item 例6 (2005,数一)设有三元方程$xy-z\ln y+e^{xz}=1$,根据隐函数存在定理,存在点$(0,1,1)$的一个邻域,在此邻域内该方程
    \begin{align*}
        (A)\ \text{只能确定一个具有连续偏导数的隐函数}z=z(x,y) \\
        (B)\ \text{可确定两个具有连续偏导数的隐函数}x=x(y,z)\text{和}z=z(x,y) \\
        (C)\ \text{可确定两个具有连续偏导数的隐函数}y=y(x,z)\text{和}z=z(x,y) \\
        (D)\ \text{可确定两个具有连续偏导数的隐函数}x=x(y,z)\text{和}y=y(x,z)
    \end{align*}
    
    \begin{solution}
    【详解】
    \end{solution}
    
    \item 例7 (1999,数一)设$y=y(x),z=z(x)$是由方程$z=xf(x+y)$和$F(x,y,z)=0$所确定的函数,其中$f$和$F$分别具有一阶连续导数和一阶连续偏导数,求$\frac{dz}{dx}$。
    
    \begin{solution}
    【详解】
    \end{solution}
\end{enumerate}

\section{变量代换化简偏微分方程}

\begin{enumerate}[label=\arabic*.,start=8]
    \item 例8 (2010,数二)设函数$u=f(x,y)$具有二阶连续偏导数,且满足等式
    \begin{align*}
        4\frac{\partial^2 u}{\partial x^2}+12\frac{\partial^2 u}{\partial x\partial y}+5\frac{\partial^2 u}{\partial y^2}=0
    \end{align*}
    确定$a,b$的值,使等式在变换$\xi=x+ay,\eta=x+by$下简化为$\frac{\partial^2 u}{\partial \xi\partial \eta}=0$。
    
    \begin{solution}
    【详解】
    \end{solution}
\end{enumerate}

\section{求无条件极值}

\begin{enumerate}[label=\arabic*.,start=9]
    \item 例9 (2003,数一)已知函数$f(x,y)$在点$(0,0)$的某个邻域内连续,且
    \begin{align*}
        \lim_{\substack{x\to 0\\ y\to 0}}\frac{f(x,y)-xy}{(x^2+y^2)^2}=1
    \end{align*}
    则
    \begin{align*}
        (A)\ \text{点}(0,0)\text{不是}f(x,y)\text{的极值点} \\
        (B)\ \text{点}(0,0)\text{是}f(x,y)\text{的极大值点} \\
        (C)\ \text{点}(0,0)\text{是}f(x,y)\text{的极小值点} \\
        (D)\ \text{根据所给条件无法判别点}(0,0)\text{是否为}f(x,y)\text{的极值点}
    \end{align*}
    
    \begin{solution}
    【详解】
    \end{solution}
    
    \item 例10 (2004,数一)设$z=z(x,y)$是由$x^2-6xy+10y^2-2yz-z^2+18=0$确定的函数,求$z=z(x,y)$的极值点和极值。
    
    \begin{solution}
    【详解】
    \end{solution}
\end{enumerate}

\section{求条件极值(边界最值)}

\begin{enumerate}[label=\arabic*.,start=11]
    \item 例11 (2006,数一、数二、数三)设$f(x,y)$与$\varphi(x,y)$均为可微函数,且$\varphi_y'(x,y)\neq 0$。已知$(x_0,y_0)$是$f(x,y)$在约束条件$\varphi(x,y)=0$下的一个极值点,下列选项正确的是
    \begin{align*}
        (A)\ \text{若}f_x'(x_0,y_0)=0\text{,则}f_y'(x_0,y_0)=0 \\
        (B)\ \text{若}f_x'(x_0,y_0)=0\text{,则}f_y'(x_0,y_0)\neq 0 \\
        (C)\ \text{若}f_x'(x_0,y_0)\neq 0\text{,则}f_y'(x_0,y_0)=0 \\
        (D)\ \text{若}f_x'(x_0,y_0)\neq 0\text{,则}f_y'(x_0,y_0)\neq 0
    \end{align*}
    
    \begin{solution}
    【详解】
    \end{solution}
    
    \item 例12 (2013,数二)求曲线$x^3-xy+y^3=1(x\geq 0,y\geq 0)$上的点到坐标原点的最长距离与最短距离。
    
    \begin{solution}
    【详解】
    \end{solution}
    
    \item 例13 (2014,数二)设函数$u(x,y)$在有界闭区域$D$上连续,在$D$的内部具有二阶连续偏导数,且满足$\frac{\partial^2 u}{\partial x\partial y}\neq 0$及$\frac{\partial^2 u}{\partial x^2}+\frac{\partial^2 u}{\partial y^2}=0$,则
    \begin{align*}
        (A)\ u(x,y)\text{的最大值和最小值都在}D\text{的边界上取得} \\
        (B)\ u(x,y)\text{的最大值和最小值都在}D\text{的内部取得} \\
        (C)\ u(x,y)\text{的最大值在}D\text{的内部取得,最小值在}D\text{的边界上取得} \\
        (D)\ u(x,y)\text{的最小值在}D\text{的内部取得,最大值在}D\text{的边界上取得}
    \end{align*}
    
    \begin{solution}
    【详解】
    \end{solution}
    
    \item 例14 (2005,数二)已知函数$z=f(x,y)$的全微分$dz=2xdx-2ydy$,且$f(1,1)=2$,求$f(x,y)$在椭圆域$D=\{(x,y)|x^2+\frac{y^2}{4}\leq 1\}$上的最大值和最小值。
    
    \begin{solution}
    【详解】
    \end{solution}
\end{enumerate}

\ifx\allfiles\undefined
\end{document}
\fi