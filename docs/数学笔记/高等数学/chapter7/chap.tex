\ifx\allfiles\undefined
\documentclass[12pt, a4paper, oneside, UTF8]{ctexbook}
\def\path{../../config}
\usepackage{amsthm}
\usepackage{amssymb}
\usepackage{array}
\usepackage{xcolor}
\usepackage{graphicx}
\usepackage{mathrsfs}
\usepackage{enumitem}
\usepackage{geometry}
\usepackage[colorlinks, linkcolor=black]{hyperref}
\usepackage{stackengine}
\usepackage{yhmath}
\usepackage{extarrows}
\usepackage{tikz}
\usepackage{forest}
\usetikzlibrary{decorations.pathreplacing, positioning}
% \usepackage{unicode-math}
\usepackage{esint}
\usepackage{pifont}
\usepackage{tcolorbox}
\tcbuselibrary{skins, breakable}

\usepackage{multicol} 
\usepackage{fontspec} % 使用字体

\setmainfont{Times New Roman}
\setCJKmainfont{LXGWWenKai-Light}[
    SlantedFont=*
]

\usepackage{listings} % 用于插入代码

% 定义代码高亮风格
\lstset{
    basicstyle=\ttfamily\small,        % 基本字体样式(等宽小字体)
    keywordstyle=\color{blue},         % 关键字颜色
    commentstyle=\color{green},        % 注释颜色
    stringstyle=\color{red},           % 字符串颜色
    numbers=none,
    breaklines=true,                   % 自动换行
    frame=single,                      % 代码框边框
    rulecolor=\color{black},           % 边框颜色
    captionpos=b,                      % 标题位置(底部)
    showspaces=false,                  % 不显示空格标记
    showstringspaces=false,            % 不显示字符串中的空格标记
    language=C                         % 设置语言为 C
}

\usepackage{fontawesome5}

\usepackage{amsmath}
\usepackage{booktabs, array}
\usepackage{makecell}
\usepackage{fancyhdr}
\usepackage[dvipsnames, svgnames]{xcolor}
\usepackage{listings}
\usepackage{tasks}[2020/01/11]

\everymath{\displaystyle}

\definecolor{mygreen}{rgb}{0,0.6,0}
\definecolor{mygray}{rgb}{0.5,0.5,0.5}
\definecolor{mymauve}{rgb}{0.58,0,0.82}
\definecolor{NavyBlue}{RGB}{0,0,128}
\definecolor{Rhodamine}{RGB}{255,0,255}
\definecolor{PineGreen}{RGB}{0,128,0}

\graphicspath{ {figures/},{../figures/}, {config/}, {../config/} }

\linespread{1.6}

\geometry{
    top=25.4mm, 
    bottom=25.4mm, 
    left=20mm, 
    right=20mm, 
    headheight=2.17cm, 
    headsep=4mm, 
    footskip=12mm
}

\setenumerate[1]{itemsep=5pt,partopsep=0pt,parsep=\parskip,topsep=5pt}
\setitemize[1]{itemsep=5pt,partopsep=0pt,parsep=\parskip,topsep=5pt}
\setdescription{itemsep=5pt,partopsep=0pt,parsep=\parskip,topsep=5pt}



% \begin{lstlisting}[language=TeX] ... \end{lstlisting}

% 定理环境设置
% ---------- 颜色 ----------
\definecolor{ExBlue}{HTML}{4F81BD}
\definecolor{SolGreen}{HTML}{77933C}
\definecolor{DefRed}{HTML}{C5504B}
\definecolor{ThmOrange}{HTML}{E97132}
\definecolor{RemGray}{HTML}{7F7F7F}
\definecolor{CorPurple}{HTML}{7030A0}
\definecolor{ForGray}{HTML}{595959}

% ---------- 通用“变色”模板 ----------
\tcbset{
    mybox/.style n args={1}{
        enhanced, breakable,
        arc=6pt,
        boxrule=0.6pt,
        left=8pt, right=8pt, top=6pt, bottom=6pt,
        drop shadow={black!25},
        fonttitle=\bfseries,
        coltitle=white,
        colbacktitle=#1!85,
        colback=#1!10,
        colframe=#1,
    }
}

% ---------- 各环境 ----------
% 例题
\newtcolorbox{example}[1][]{mybox={ExBlue}, title={\ifstrempty{#1}{Example}{#1}}}
% 解答
\newtcolorbox{solution}[1][]{mybox={SolGreen}, title={\ifstrempty{#1}{Solution}{#1}}}
% 定义
\newtcolorbox{definition}[1][]{mybox={DefRed}, title={\ifstrempty{#1}{Definition}{#1}}}
% 定理
\newtcolorbox{theorem}[1][]{mybox={ThmOrange}, title={\ifstrempty{#1}{Theorem}{#1}}}
% 标注
\newtcolorbox{remark}[1][]{mybox={RemGray}, title={\ifstrempty{#1}{Remark}{#1}}}
% 推论
\newtcolorbox{corollary}[1][]{mybox={CorPurple}, title={\ifstrempty{#1}{Corollary}{#1}}}
% 公式
\newtcolorbox{formula}[1][]{mybox={ForGray}, title={\ifstrempty{#1}{Formula}{#1}}}


\settasks{
    label-format = \bfseries,
    label        = \Alph*.,
    label-width  = 1.2em,
    label-offset = 0.3em,
    item-indent  = 1.9em,
    column-sep   = 0.5em
}

\newenvironment{choices}[1][4]   % 默认 4 栏
    {\begin{tasks}(#1)}
    {\end{tasks}}

% 自定义命令的文件

\def\d{\mathrm{d}}
\def\R{\mathbb{R}}
\def\P{\partial} 
\newcommand{\bs}[1]{\begin{solution}#1\end{solution}}
\newcommand{\bt}[1][1]{% 默认参数为1
    \ensuremath{% 确保数学模式
        \foreach \n in {1,...,#1} {\blacktriangle}% 循环输出 #1 个黑色三角形
    }%
}

\newcommand{\bl}[1][1]{% 默认参数为1
    \ensuremath{% 确保数学模式
        \foreach \n in {1,...,#1} {\blacklozenge}% 循环输出 #1 个黑色三角形
    }%
}
\newif\ifshowanswers
%\showanswerstrue % 注释掉这行就不显示答案

% 定义答案环境
\newcommand{\answer}[1]{%
    \ifshowanswers
        #1%
    \fi
}




% 修改参数改变封面样式,0 默认原始封面、内置其他1、2、3种封面样式
\def\myIndex{3}


\ifnum\myIndex>0
    \input{\path/cover_package_\myIndex} 
\fi

\def\myTitle{冲刺150笔记}
\def\myAuthor{Weary Bird}
\def\myDateCover{\today}
\def\myDateForeword{\today}
\def\myForeword{行香子}
\def\myForewordText{
树绕村庄,水满陂塘;倚东风、豪兴徜徉。小园几许,收尽春光。有桃花红,李花白,菜花黄。 \\
远远苔墙,隐隐茅堂;飏青旗、流水桥旁。偶然乘兴,步过东冈。正莺儿啼,燕儿舞,蝶儿忙。 \\
}
\def\mySubheading{知错能改善莫大焉}


\begin{document}
\input{../../config/cover}
\else
\fi

\chapter{无穷级数}
\section{数项级数敛散性的判定}

\begin{enumerate}[label=\arabic*.]
    \item 例1 (2015,数三)下列级数中发散的是
    \begin{align*}
        (A)\sum_{n=1}^{\infty}\frac{n}{3^n} \quad (C)\sum_{n=2}^{\infty}\frac{(-1)^n+1}{\ln n} \quad (D)\sum_{n=1}^{\infty}\frac{n!}{n^n}
    \end{align*}
    
    \begin{solution}
    【详解】
    \end{solution}
    
    \item 例2 (2017,数三)若级数$\sum_{n=1}^{\infty}\left[\sin\frac{1}{n}-k\ln\left(1-\frac{1}{n}\right)\right]$收敛,则$k=$
    \begin{align*}
        (A)\ 1 \quad (B)\ 2 \quad (C)\ -1 \quad (D)\ -2
    \end{align*}
    
    \begin{solution}
    【详解】
    \end{solution}
\end{enumerate}

\section{交错级数}

\begin{enumerate}[label=\arabic*.,start=3]
    \item 例3 判定下列级数的敛散性:
    \begin{align*}
        (1)\sum_{n=1}^{\infty}\frac{(-1)^{n-1}}{n-\ln n} \quad (2)\sum_{n=2}^{\infty}\frac{(-1)^n}{\sqrt{n}+(-1)^n}.
    \end{align*}
    
    \begin{solution}
    【详解】
    \end{solution}
\end{enumerate}

\section{任意项级数}

\begin{enumerate}[label=\arabic*.,start=4]
    \item 例4 (2002,数一)设$u_n\neq 0(n=1,2,3,\cdots)$,且$\lim_{n\rightarrow\infty}\frac{n}{u_n}=1$,则级数$\sum_{n=1}^{\infty}(-1)^{n+1}\left(\frac{1}{u_n}+\frac{1}{u_{n+1}}\right)$
    \begin{align*}
        (A)\ 发散 \quad (B)\ 绝对收敛 \quad (C)\ 条件收敛 \quad (D)\ 敛散性根据所给条件不能判定
    \end{align*}
    
    \begin{solution}
    【详解】
    \end{solution}
    
    \item 例5 (2019,数三)若$\sum_{n=1}^{\infty}\frac{v_n}{n}$条件收敛,则
    \begin{align*}
        (A)\sum_{n=1}^{\infty} u_n v_n\text{条件收敛} \quad (B)\sum_{n=1}^{\infty} u_n v_n\text{绝对收敛} \\
        (C)\sum_{n=1}^{\infty}\left(u_n+v_n\right)\text{收敛} \quad (D)\sum_{n=1}^{\infty}\left(u_n+v_n\right)\text{发散}
    \end{align*}
    
    \begin{solution}
    【详解】
    \end{solution}
\end{enumerate}

\section{幂级数求收敛半径与收敛域}

\begin{enumerate}[label=\arabic*.,start=6]
    \item 例6 (2015,数一)若级数$\sum_{n=1}^{\infty} a_n$条件收敛,则$x=\sqrt{3}$与$x=3$依次为幂级数$\sum_{n=1}^{\infty} n a_n(x-1)^n$的
    \begin{align*}
        (A)\ 收敛点,收敛点 \quad (B)\ 收敛点,发散点 \\
        (C)\ 发散点,收敛点 \quad (D)\ 发散点,发散点
    \end{align*}
    
    \begin{solution}
    【详解】
    \end{solution}
    
    \item 例7 求幂级数$\sum_{n=1}^{\infty}\frac{3n}{2n+1}x^n$的收敛域.
    
    \begin{solution}
    【详解】
    \end{solution}
\end{enumerate}

\section{幂级数求和}

\begin{enumerate}[label=\arabic*.,start=8]
    \item 例8 (2005,数一)求幂级数$\sum_{n=1}^{\infty}(-1)^{n-1}\left[1+\frac{1}{n(2n-1)}\right] x^{2n}$的收敛区间与和函数$f(x)$.
    
    \begin{solution}
    【详解】
    \end{solution}
    
    \item 例9 (2012,数一)求幂级数$\sum_{n=0}^{\infty}\frac{4n^2+4n+3}{2n+1} x^{2n}$的收敛域及和函数.
    
    \begin{solution}
    【详解】
    \end{solution}
    
    \item 例10 (2004,数三)设级数$\frac{x^4}{2\cdot 4}+\frac{x^6}{2\cdot 4\cdot 6}+\frac{x^8}{2\cdot 4\cdot 6\cdot 8}+\cdots\quad(-\infty<x<+\infty)$的和函数为$S(x)$。求:
    \begin{enumerate}[label=(\roman*)]
        \item $S(x)$所满足的一阶微分方程;
        \item $S(x)$的表达式.
    \end{enumerate}
    
    \begin{solution}
    【详解】
    \end{solution}
\end{enumerate}

\section{幂级数展开}

\begin{enumerate}[label=\arabic*.,start=11]
    \item 例11 (2007,数三)将函数$f(x)=\frac{1}{x^2-3x-4}$展开成$x-1$的幂级数,并指出其收敛区间.
    
    \begin{solution}
    【详解】
    \end{solution}
    
    \item 例12 将函数$f(x)=\ln\frac{x}{x+1}$在$x=1$处展开成幂级数.
    
    \begin{solution}
    【详解】
    \end{solution}
\end{enumerate}

\section{无穷级数证明题}

\begin{enumerate}[label=\arabic*.,start=13]
    \item 例13 (2016,数一)已知函数$f(x)$可导,且$f(0)=1$,$0<f'(x)<\frac{1}{2}$。设数列$\{x_n\}$满足$x_{n+1}=f(x_n)(n=1,2,\cdots)$。证明:
    \begin{enumerate}[label=(\roman*)]
        \item 级数$\sum_{n=1}^{\infty}(x_{n+1}-x_n)$绝对收敛;
        \item $\lim_{n\rightarrow\infty} x_n$存在,且$0<\lim_{n\rightarrow\infty} x_n<2$.
    \end{enumerate}
    
    \begin{solution}
    【详解】
    \end{solution}
    
    \item 例14 (2014,数一)设数列$\{a_n\}$,$\{b_n\}$满足$0<a_n<\frac{\pi}{2}$,$0<b_n<\frac{\pi}{2}$,$\cos a_n-a_n=\cos b_n$,且级数$\sum_{n=1}^{\infty} b_n$收敛。
    \begin{enumerate}[label=(\roman*)]
        \item 证明$\lim_{n\rightarrow\infty} a_n=0$;
        \item 证明级数$\sum_{n=1}^{\infty}\frac{a_n}{b_n}$收敛.
    \end{enumerate}
    
    \begin{solution}
    【详解】
    \end{solution}
\end{enumerate}

\section{傅里叶级数}

\begin{enumerate}[label=\arabic*.,start=15]
    \item 例15 设函数
    \begin{align*}
    f(x)=\begin{cases}
    e^x, & -\pi\leq x<0 \\
    1, & 0\leq x<\pi
    \end{cases}
    \end{align*}
    则其以$2\pi$为周期的傅里叶级数在$x=\pi$收敛于?,在$x=2\pi$收敛于?.
    \begin{solution}
    【详解】
    由狄利克雷收敛定理知,$f(x)$以$2\pi$为周期的傅里叶级数在$x=\pi$收敛于
    \begin{align*}
    S(\pi)=\frac{f(\pi-0)+f(-\pi+0)}{2}=\frac{1+e^{-\pi}}{2}
    \end{align*}
    在$x=2\pi$收敛于
    \begin{align*}
    S(2\pi)=S(0)=\frac{f(0-0)+f(0+0)}{2}=\frac{1+1}{2}=1
    \end{align*}
    \end{solution}
    
    \item 例16 将$f(x)=1-x^2,0\leq x\leq\pi$,展开成余弦级数,并求级数$\sum_{n=1}^{\infty}\frac{(-1)^{n-1}}{n^2}$的和.
    
    \begin{solution}
    【详解】
    对$f(x)=1-x^2$进行偶延拓,由$f(x)=1-x^2$为偶函数,知$b_n=0$。
    \begin{align*}
    a_0&=\frac{2}{\pi}\int_0^\pi(1-x^2)dx=2\left(1-\frac{\pi^2}{3}\right) \\
    a_n&=\frac{2}{\pi}\int_0^\pi(1-x^2)\cos nx dx=\frac{4(-1)^{n+1}}{n^2} \quad (n=1,2,\cdots)
    \end{align*}
    \begin{align*}
    f(x)=1-x^2=\frac{a_0}{2}+\sum_{n=1}^{\infty}a_n\cos nx=1-\frac{\pi^2}{3}+\sum_{n=1}^{\infty}\frac{4(-1)^{n+1}}{n^2}\cos nx
    \end{align*}
    令$x=0$,代入上式,得
    \begin{align*}
    \sum_{n=1}^{\infty}\frac{(-1)^{n-1}}{n^2}=\frac{\pi^2}{12}
    \end{align*}
    \end{solution}
\end{enumerate}

\ifx\allfiles\undefined
\end{document}
\fi