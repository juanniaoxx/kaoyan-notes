\ifx\allfiles\undefined
\documentclass[12pt, a4paper, oneside, UTF8]{ctexbook}
\def\path{../../config}
\usepackage{amsthm}
\usepackage{amssymb}
\usepackage{array}
\usepackage{xcolor}
\usepackage{graphicx}
\usepackage{mathrsfs}
\usepackage{enumitem}
\usepackage{geometry}
\usepackage[colorlinks, linkcolor=black]{hyperref}
\usepackage{stackengine}
\usepackage{yhmath}
\usepackage{extarrows}
\usepackage{tikz}
\usepackage{forest}
\usetikzlibrary{decorations.pathreplacing, positioning}
% \usepackage{unicode-math}
\usepackage{esint}
\usepackage{pifont}
\usepackage{tcolorbox}
\tcbuselibrary{skins, breakable}

\usepackage{multicol} 
\usepackage{fontspec} % 使用字体

\setmainfont{Times New Roman}
\setCJKmainfont{LXGWWenKai-Light}[
    SlantedFont=*
]

\usepackage{listings} % 用于插入代码

% 定义代码高亮风格
\lstset{
    basicstyle=\ttfamily\small,        % 基本字体样式(等宽小字体)
    keywordstyle=\color{blue},         % 关键字颜色
    commentstyle=\color{green},        % 注释颜色
    stringstyle=\color{red},           % 字符串颜色
    numbers=none,
    breaklines=true,                   % 自动换行
    frame=single,                      % 代码框边框
    rulecolor=\color{black},           % 边框颜色
    captionpos=b,                      % 标题位置(底部)
    showspaces=false,                  % 不显示空格标记
    showstringspaces=false,            % 不显示字符串中的空格标记
    language=C                         % 设置语言为 C
}

\usepackage{fontawesome5}

\usepackage{amsmath}
\usepackage{booktabs, array}
\usepackage{makecell}
\usepackage{fancyhdr}
\usepackage[dvipsnames, svgnames]{xcolor}
\usepackage{listings}
\usepackage{tasks}[2020/01/11]

\everymath{\displaystyle}

\definecolor{mygreen}{rgb}{0,0.6,0}
\definecolor{mygray}{rgb}{0.5,0.5,0.5}
\definecolor{mymauve}{rgb}{0.58,0,0.82}
\definecolor{NavyBlue}{RGB}{0,0,128}
\definecolor{Rhodamine}{RGB}{255,0,255}
\definecolor{PineGreen}{RGB}{0,128,0}

\graphicspath{ {figures/},{../figures/}, {config/}, {../config/} }

\linespread{1.6}

\geometry{
    top=25.4mm, 
    bottom=25.4mm, 
    left=20mm, 
    right=20mm, 
    headheight=2.17cm, 
    headsep=4mm, 
    footskip=12mm
}

\setenumerate[1]{itemsep=5pt,partopsep=0pt,parsep=\parskip,topsep=5pt}
\setitemize[1]{itemsep=5pt,partopsep=0pt,parsep=\parskip,topsep=5pt}
\setdescription{itemsep=5pt,partopsep=0pt,parsep=\parskip,topsep=5pt}



% \begin{lstlisting}[language=TeX] ... \end{lstlisting}

% 定理环境设置
% ---------- 颜色 ----------
\definecolor{ExBlue}{HTML}{4F81BD}
\definecolor{SolGreen}{HTML}{77933C}
\definecolor{DefRed}{HTML}{C5504B}
\definecolor{ThmOrange}{HTML}{E97132}
\definecolor{RemGray}{HTML}{7F7F7F}
\definecolor{CorPurple}{HTML}{7030A0}
\definecolor{ForGray}{HTML}{595959}

% ---------- 通用“变色”模板 ----------
\tcbset{
    mybox/.style n args={1}{
        enhanced, breakable,
        arc=6pt,
        boxrule=0.6pt,
        left=8pt, right=8pt, top=6pt, bottom=6pt,
        drop shadow={black!25},
        fonttitle=\bfseries,
        coltitle=white,
        colbacktitle=#1!85,
        colback=#1!10,
        colframe=#1,
    }
}

% ---------- 各环境 ----------
% 例题
\newtcolorbox{example}[1][]{mybox={ExBlue}, title={\ifstrempty{#1}{Example}{#1}}}
% 解答
\newtcolorbox{solution}[1][]{mybox={SolGreen}, title={\ifstrempty{#1}{Solution}{#1}}}
% 定义
\newtcolorbox{definition}[1][]{mybox={DefRed}, title={\ifstrempty{#1}{Definition}{#1}}}
% 定理
\newtcolorbox{theorem}[1][]{mybox={ThmOrange}, title={\ifstrempty{#1}{Theorem}{#1}}}
% 标注
\newtcolorbox{remark}[1][]{mybox={RemGray}, title={\ifstrempty{#1}{Remark}{#1}}}
% 推论
\newtcolorbox{corollary}[1][]{mybox={CorPurple}, title={\ifstrempty{#1}{Corollary}{#1}}}
% 公式
\newtcolorbox{formula}[1][]{mybox={ForGray}, title={\ifstrempty{#1}{Formula}{#1}}}


\settasks{
    label-format = \bfseries,
    label        = \Alph*.,
    label-width  = 1.2em,
    label-offset = 0.3em,
    item-indent  = 1.9em,
    column-sep   = 0.5em
}

\newenvironment{choices}[1][4]   % 默认 4 栏
    {\begin{tasks}(#1)}
    {\end{tasks}}

% 自定义命令的文件

\def\d{\mathrm{d}}
\def\R{\mathbb{R}}
\def\P{\partial} 
\newcommand{\bs}[1]{\begin{solution}#1\end{solution}}
\newcommand{\bt}[1][1]{% 默认参数为1
    \ensuremath{% 确保数学模式
        \foreach \n in {1,...,#1} {\blacktriangle}% 循环输出 #1 个黑色三角形
    }%
}

\newcommand{\bl}[1][1]{% 默认参数为1
    \ensuremath{% 确保数学模式
        \foreach \n in {1,...,#1} {\blacklozenge}% 循环输出 #1 个黑色三角形
    }%
}
\newif\ifshowanswers
%\showanswerstrue % 注释掉这行就不显示答案

% 定义答案环境
\newcommand{\answer}[1]{%
    \ifshowanswers
        #1%
    \fi
}




% 修改参数改变封面样式,0 默认原始封面、内置其他1、2、3种封面样式
\def\myIndex{3}


\ifnum\myIndex>0
    \input{\path/cover_package_\myIndex} 
\fi

\def\myTitle{冲刺150笔记}
\def\myAuthor{Weary Bird}
\def\myDateCover{\today}
\def\myDateForeword{\today}
\def\myForeword{行香子}
\def\myForewordText{
树绕村庄,水满陂塘;倚东风、豪兴徜徉。小园几许,收尽春光。有桃花红,李花白,菜花黄。 \\
远远苔墙,隐隐茅堂;飏青旗、流水桥旁。偶然乘兴,步过东冈。正莺儿啼,燕儿舞,蝶儿忙。 \\
}
\def\mySubheading{知错能改善莫大焉}


\begin{document}
\input{../../config/cover}
\else
\fi

\chapter{常微分方程}
\begin{enumerate}[label=\arabic*.]
    \item 例1 (1998,数一、数二)已知函数$y=y(x)$在任意点$x$处的增量$\Delta y=\frac{y\Delta x}{1+x^2}+\alpha$,其中$\alpha$是$\Delta x$的高阶无穷小,$y(0)=\pi$,则$y(1)$等于
    \begin{align*}
        (A)\ 2\pi \quad (B)\ \pi \quad (C)\ e^{\frac{\pi}{4}} \quad (D)\ \pi e^{\frac{\pi}{4}}
    \end{align*}
    
    \begin{solution}
    【详解】
    \end{solution}
    
    \item 例2 (2002,数二)已知函数$f(x)$在$(0,+\infty)$内可导,$f(x)>0$,$\lim_{x\to+\infty}f(x)=1$,且满足
    \begin{align*}
        \lim_{h\to0}\left(\frac{f(x+hx)}{f(x)}\right)^{\frac{1}{h}}=e^{\frac{1}{x}}
    \end{align*}
    求$f(x)$。
    
    \begin{solution}
    【详解】
    \end{solution}
\end{enumerate}

\section{ 一阶微分方程的解法}

\begin{remark}[类型一 可分离变量]
\end{remark}

\begin{enumerate}[label=\arabic*.,start=3]
    \item 例3 (1999,数二)求初值问题
    \begin{align*}
        \begin{cases}
            (y+\sqrt{x^2+y^2})dx-xdy=0 & (x>0) \\
            y|_{x=1}=0
        \end{cases}
    \end{align*}
    
    \begin{solution}
    【详解】
    \end{solution}
\end{enumerate}

\begin{remark}[类型二 一阶齐次]
\end{remark}

\begin{enumerate}[label=\arabic*.,start=4]
    \item 例4 (2010,数二、数三)设$y_1,y_2$是一阶线性非齐次微分方程$y'+p(x)y=q(x)$的两个特解。若常数$\lambda,\mu$使$\lambda y_1+\mu y_2$是该方程的解,$\lambda y_1-\mu y_2$是该方程对应的齐次方程的解,则
    \begin{align*}
        (A)\ \lambda=\frac{1}{2},\ \mu=\frac{1}{2} \quad (C)\ \lambda=\frac{2}{3},\ \mu=\frac{1}{3}
    \end{align*}
    
    \begin{solution}
    【详解】
    \end{solution}
\end{enumerate}

\begin{remark}[类型三 一阶线性]
\end{remark}

\begin{enumerate}[label=\arabic*.,start=5]
    \item 例5 (2018,数一)已知微分方程$y'+y=f(x)$,其中$f(x)$是$\mathbb{R}$上的连续函数。
    \begin{enumerate}[label=(\roman*)]
        \item 若$f(x)=x$,求方程的通解;
        \item 若$f(x)$是周期为$T$的函数,证明:方程存在唯一的以$T$为周期的解。
    \end{enumerate}
    
    \begin{solution}
    【详解】
    \end{solution}
\end{enumerate}

\begin{remark}[类型四 伯努利方程(数一掌握)]
\end{remark}

\begin{enumerate}[label=\arabic*.,start=6]
    \item 例6 求解微分方程$y'=\frac{y}{x}+\sqrt{\frac{y^2}{x^2}-1}$。
    
    \begin{solution}
    【详解】
    \end{solution}
\end{enumerate}

\begin{remark}[类型五 全微分方程(数一掌握)]
\end{remark}

\begin{enumerate}[label=\arabic*.,start=7]
    \item 例7 求解下列微分方程:
    \begin{align*}
        (1)\ &(2xe^y+3x^2-1)dx+(x^2e^y-2y)dy=0; \\
        (2)\ &\frac{2x}{y^3}dx+\frac{y^2-3x^2}{y^4}dy=0.
    \end{align*}
    
    \begin{solution}
    【详解】
    \end{solution}
\end{enumerate}

\section{二阶常系数线性微分方程}

\begin{enumerate}[label=\arabic*.,start=8]
    \item 例8 (2017,数二)微分方程$y''-4y'+8y=e^{2x}(1+\cos2x)$的特解可设为$y^*=$
    \begin{align*}
        (A)\ Ae^{2x}+e^{2x}(B\cos2x+C\sin2x) \\
        (B)\ Axe^{2x}+e^{2x}(B\cos2x+C\sin2x) \\
        (C)\ Ae^{2x}+xe^{2x}(B\cos2x+C\sin2x) \\
        (D)\ Axe^{2x}+xe^{2x}(B\cos2x+C\sin2x)
    \end{align*}
    
    \begin{solution}
    【详解】
    \end{solution}
    
    \item 例9 (2015,数一)设$y=\frac{1}{2}e^{2x}+(x-\frac{1}{3})e^x$是二阶常系数非齐次线性微分方程$y''+ay'+by=ce^x$的一个特解,则
    \begin{align*}
        (A)\ a=-3,b=2,c=-1 \\
        (B)\ a=3,b=2,c=-1 \\
        (C)\ a=-3,b=2,c=1 \\
        (D)\ a=3,b=2,c=1
    \end{align*}
    
    \begin{solution}
    【详解】
    \end{solution}
    
    \item 例10 (2016,数二)已知$y_1(x)=e^x$,$y_2(x)=u(x)e^x$是二阶微分方程$(2x-1)y''-(2x+1)y'+2y=0$的两个解。若$u(-1)=e$,$u(0)=-1$,求$u(x)$,并写出该微分方程的通解。
    
    \begin{solution}
    【详解】
    \end{solution}
    
    \item 例11 (2016,数一)设函数$y(x)$满足方程$y''+2y'+ky=0$,其中$0<k<1$。
    \begin{enumerate}[label=(\roman*)]
        \item 证明反常积分$\int_0^{+\infty}y(x)dx$收敛;
        \item 若$y(0)=1$,$y'(0)=1$,求$\int_0^{+\infty}y(x)dx$的值。
    \end{enumerate}
    
    \begin{solution}
    【详解】
    \end{solution}
\end{enumerate}

\section{ 高阶常系数线性齐次微分方程}

\begin{enumerate}[label=\arabic*.,start=12]
    \item 例12 求解微分方程$y^{(4)}-3y''-4y=0$。
    
    \begin{solution}
    【详解】
    \end{solution}
\end{enumerate}

\section{ 二阶可降阶微分方程}

\begin{remark}[方法 数一、数二掌握 数三大纲不要求]
\end{remark}

\begin{enumerate}[label=\arabic*.,start=13]
    \item 例13 求微分方程$y''(x+y'^2)=y'$满足初始条件$y(1)=y'(1)=1$的特解。
    
    \begin{solution}
    【详解】
    \end{solution}
\end{enumerate}

\section{ 欧拉方程}

\begin{remark}[方法 数一掌握 数二、数三大纲不要求]
\end{remark}

\begin{enumerate}[label=\arabic*.,start=14]
    \item 例14 求解微分方程$x^2y''+xy'+y=2\sin\ln x$。
    
    \begin{solution}
    【详解】
    \end{solution}
\end{enumerate}

\section{ 变量代换求解二阶变系数线性微分方程}

\begin{enumerate}[label=\arabic*.,start=17]
    \item 例17 (2005,数二)用变量代换$x=\cos t(0<t<\pi)$化简微分方程$(1-x^2)y''-xy'+y=0$,并求其满足$y|_{x=0}=1$,$y'|_{x=0}=2$的特解。
    
    \begin{solution}
    【详解】
    \end{solution}
\end{enumerate}

\section{ 微分方程综合题}

\begin{remark}[类型一 综合导数应用]
\end{remark}

\begin{enumerate}[label=\arabic*.,start=18]
    \item 例18 (2001,数二)设$L$是一条平面曲线,其上任意一点$P(x,y)(x>0)$到坐标原点的距离,恒等于该点处的切线在$y$轴上的截距,且$L$经过点$(\frac{1}{2},0)$。求曲线$L$的方程。
    
    \begin{solution}
    【详解】
    \end{solution}
\end{enumerate}

\begin{remark}[类型二 综合定积分应用]
\end{remark}

\begin{enumerate}[label=\arabic*.,start=19]
    \item 例19 (2009,数三)设曲线$y=f(x)$,其中$f(x)$是可导函数,且$f(x)>0$。已知曲线$y=f(x)$与直线$y=0$,$x=1$及$x=t(t>1)$所围成的曲边梯形绕$x$轴旋转一周所得的立体体积值是该曲边梯形面积值的$\pi t$倍,求该曲线的方程。
    
    \begin{solution}
    【详解】
    \end{solution}
\end{enumerate}

\begin{remark}[类型三 综合变限积分]
\end{remark}

\begin{enumerate}[label=\arabic*.,start=20]
    \item 例20 (2016,数三)设函数$f(x)$连续,且满足$\int_0^x f(x-t)dt=\int_0^x(x-t)f(t)dt+e^{-x}-1$,求$f(x)$。
    
    \begin{solution}
    【详解】
    \end{solution}
\end{enumerate}

\begin{remark}[类型四 综合多元复合函数]
\end{remark}

\begin{enumerate}[label=\arabic*.,start=21]
    \item 例21 (2014,数一、数二、数三)设函数$f(u)$具有二阶连续导数,$z=f(e^x\cos y)$满足
    \begin{align*}
        \frac{\partial^2 z}{\partial x^2}+\frac{\partial^2 z}{\partial y^2}=(4z+e^x\cos y)e^{2x}
    \end{align*}
    若$f(0)=0$,$f'(0)=0$,求$f(u)$的表达式。
    
    \begin{solution}
    【详解】
    \end{solution}
\end{enumerate}

\begin{remark}[类型五 综合重积分]
\end{remark}

\begin{enumerate}[label=\arabic*.,start=22]
    \item 例22 (2011,数三)设函数$f(x)$在区间$[0,1]$上具有连续导数,$f(0)=1$,且满足
    \begin{align*}
        \iint_{D_t} f'(x+y)dxdy=\iint_{D_t} f(t)dxdy
    \end{align*}
    其中$D_t=\{(x,y)|0\leq y\leq t-x,0\leq x\leq t\}(0<t\leq1)$,求$f(x)$的表达式。
    
    \begin{solution}
    【详解】
    \end{solution}
\end{enumerate}

\ifx\allfiles\undefined
\end{document}
\fi