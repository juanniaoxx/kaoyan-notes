\ifx\allfiles\undefined
\documentclass[12pt, a4paper, oneside, UTF8]{ctexbook}
\def\path{../../config}
\usepackage{amsmath}
\usepackage{amsthm}
\usepackage{amssymb}
\usepackage{array}
\usepackage{xcolor}
\usepackage{graphicx}
\usepackage{mathrsfs}
\usepackage{enumitem}
\usepackage{geometry}
\usepackage[colorlinks, linkcolor=black]{hyperref}
\usepackage{stackengine}
\usepackage{yhmath}
\usepackage{extarrows}
\usepackage{tikz}
\usepackage{pgfplots}
\usepackage{asymptote}
\usepackage{float}
\usepackage{fontspec} % 使用字体

\setmainfont{Times New Roman}
\setCJKmainfont{LXGWWenKai-Light}[
    SlantedFont=*
]

\everymath{\displaystyle}

\usepgfplotslibrary{polar}
\usepackage{subcaption}
\usetikzlibrary{decorations.pathreplacing, positioning}

\usepgfplotslibrary{fillbetween}
\pgfplotsset{compat=1.18}
% \usepackage{unicode-math}
\usepackage{esint}
\usepackage[most]{tcolorbox}

\usepackage{fancyhdr}
\usepackage[dvipsnames, svgnames]{xcolor}
\usepackage{listings}

\definecolor{mygreen}{rgb}{0,0.6,0}
\definecolor{mygray}{rgb}{0.5,0.5,0.5}
\definecolor{mymauve}{rgb}{0.58,0,0.82}
\definecolor{NavyBlue}{RGB}{0,0,128}
\definecolor{Rhodamine}{RGB}{255,0,255}
\definecolor{PineGreen}{RGB}{0,128,0}

\graphicspath{ {figures/},{../figures/}, {config/}, {../config/} }

\linespread{1.6}

\geometry{
    top=25.4mm, 
    bottom=25.4mm, 
    left=20mm, 
    right=20mm, 
    headheight=2.17cm, 
    headsep=4mm, 
    footskip=12mm
}

\setenumerate[1]{itemsep=5pt,partopsep=0pt,parsep=\parskip,topsep=5pt}
\setitemize[1]{itemsep=5pt,partopsep=0pt,parsep=\parskip,topsep=5pt}
\setdescription{itemsep=5pt,partopsep=0pt,parsep=\parskip,topsep=5pt}

\lstset{
    language=Mathematica,
    basicstyle=\tt,
    breaklines=true,
    keywordstyle=\bfseries\color{NavyBlue}, 
    emphstyle=\bfseries\color{Rhodamine},
    commentstyle=\itshape\color{black!50!white}, 
    stringstyle=\bfseries\color{PineGreen!90!black},
    columns=flexible,
    numbers=left,
    numberstyle=\footnotesize,
    frame=tb,
    breakatwhitespace=false,
} 

\lstset{
    language=TeX, % 设置语言为 TeX
    basicstyle=\ttfamily, % 使用等宽字体
    breaklines=true, % 自动换行
    keywordstyle=\bfseries\color{NavyBlue}, % 关键字样式
    emphstyle=\bfseries\color{Rhodamine}, % 强调样式
    commentstyle=\itshape\color{black!50!white}, % 注释样式
    stringstyle=\bfseries\color{PineGreen!90!black}, % 字符串样式
    columns=flexible, % 列的灵活性
    numbers=left, % 行号在左侧
    numberstyle=\footnotesize, % 行号字体大小
    frame=tb, % 顶部和底部边框
    breakatwhitespace=false % 不在空白处断行
}

% \begin{lstlisting}[language=TeX] ... \end{lstlisting}

% 定理环境设置
\usepackage[strict]{changepage} 
\usepackage{framed}

\definecolor{greenshade}{rgb}{0.90,1,0.92}
\definecolor{redshade}{rgb}{1.00,0.88,0.88}
\definecolor{brownshade}{rgb}{0.99,0.95,0.9}
\definecolor{lilacshade}{rgb}{0.95,0.93,0.98}
\definecolor{orangeshade}{rgb}{1.00,0.88,0.82}
\definecolor{lightblueshade}{rgb}{0.8,0.92,1}
\definecolor{purple}{rgb}{0.81,0.85,1}

\theoremstyle{definition}
\newtheorem{myDefn}{\indent Definition}[section]
\newtheorem{myLemma}{\indent Lemma}[section]
\newtheorem{myThm}[myLemma]{\indent Theorem}
\newtheorem{myCorollary}[myLemma]{\indent Corollary}
\newtheorem{myCriterion}[myLemma]{\indent Criterion}
\newtheorem*{myRemark}{\indent Remark}
\newtheorem{myProposition}{\indent Proposition}[section]

\newenvironment{formal}[2][]{%
	\def\FrameCommand{%
		\hspace{1pt}%
		{\color{#1}\vrule width 2pt}%
		{\color{#2}\vrule width 4pt}%
		\colorbox{#2}%
	}%
	\MakeFramed{\advance\hsize-\width\FrameRestore}%
	\noindent\hspace{-4.55pt}%
	\begin{adjustwidth}{}{7pt}\vspace{2pt}\vspace{2pt}}{%
		\vspace{2pt}\end{adjustwidth}\endMakeFramed%
}

\newenvironment{definition}{\vspace{-\baselineskip * 2 / 3}%
	\begin{formal}[Green]{greenshade}\vspace{-\baselineskip * 4 / 5}\begin{myDefn}}
	{\end{myDefn}\end{formal}\vspace{-\baselineskip * 2 / 3}}

\newenvironment{theorem}{\vspace{-\baselineskip * 2 / 3}%
	\begin{formal}[LightSkyBlue]{lightblueshade}\vspace{-\baselineskip * 4 / 5}\begin{myThm}}%
	{\end{myThm}\end{formal}\vspace{-\baselineskip * 2 / 3}}

\newenvironment{lemma}{\vspace{-\baselineskip * 2 / 3}%
	\begin{formal}[Plum]{lilacshade}\vspace{-\baselineskip * 4 / 5}\begin{myLemma}}%
	{\end{myLemma}\end{formal}\vspace{-\baselineskip * 2 / 3}}

\newenvironment{corollary}{\vspace{-\baselineskip * 2 / 3}%
	\begin{formal}[BurlyWood]{brownshade}\vspace{-\baselineskip * 4 / 5}\begin{myCorollary}}%
	{\end{myCorollary}\end{formal}\vspace{-\baselineskip * 2 / 3}}

\newenvironment{criterion}{\vspace{-\baselineskip * 2 / 3}%
	\begin{formal}[DarkOrange]{orangeshade}\vspace{-\baselineskip * 4 / 5}\begin{myCriterion}}%
	{\end{myCriterion}\end{formal}\vspace{-\baselineskip * 2 / 3}}
	

\newenvironment{remark}{\vspace{-\baselineskip * 2 / 3}%
	\begin{formal}[LightCoral]{redshade}\vspace{-\baselineskip * 4 / 5}\begin{myRemark}}%
	{\end{myRemark}\end{formal}\vspace{-\baselineskip * 2 / 3}}

\newenvironment{proposition}{\vspace{-\baselineskip * 2 / 3}%
	\begin{formal}[RoyalPurple]{purple}\vspace{-\baselineskip * 4 / 5}\begin{myProposition}}%
	{\end{myProposition}\end{formal}\vspace{-\baselineskip * 2 / 3}}


\newtheorem{example}{\indent \color{SeaGreen}{Example}}[section]
\renewcommand{\proofname}{\indent\textbf{\textcolor{TealBlue}{Proof}}}
\NewEnviron{solution}{%
	\begin{proof}[\indent\textbf{\textcolor{TealBlue}{Solution}}]%
		\color{blue}% 设置内容为蓝色
		\BODY% 插入环境内容
		\color{black}% 恢复默认颜色(可选,避免影响后续文字)
	\end{proof}%
}

% 自定义命令的文件

\def\d{\mathrm{d}}
\def\R{\mathbb{R}}
%\newcommand{\bs}[1]{\boldsymbol{#1}}
%\newcommand{\ora}[1]{\overrightarrow{#1}}
\newcommand{\myspace}[1]{\par\vspace{#1\baselineskip}}
\newcommand{\xrowht}[2][0]{\addstackgap[.5\dimexpr#2\relax]{\vphantom{#1}}}
\newenvironment{mycases}[1][1]{\linespread{#1} \selectfont \begin{cases}}{\end{cases}}
\newenvironment{myvmatrix}[1][1]{\linespread{#1} \selectfont \begin{vmatrix}}{\end{vmatrix}}
\newcommand{\tabincell}[2]{\begin{tabular}{@{}#1@{}}#2\end{tabular}}
\newcommand{\pll}{\kern 0.56em/\kern -0.8em /\kern 0.56em}
\newcommand{\dive}[1][F]{\mathrm{div}\;\boldsymbol{#1}}
\newcommand{\rotn}[1][A]{\mathrm{rot}\;\boldsymbol{#1}}

\newif\ifshowanswers
\showanswerstrue % 注释掉这行就不显示答案

% 定义答案环境
\newcommand{\answer}[1]{%
    \ifshowanswers
        #1%
    \fi
}

% 修改参数改变封面样式,0 默认原始封面、内置其他1、2、3种封面样式
\def\myIndex{0}


\ifnum\myIndex>0
    \input{\path/cover_package_\myIndex} 
\fi

\def\myTitle{考研数学笔记}
\def\myAuthor{Weary Bird}
\def\myDateCover{\today}
\def\myDateForeword{\today}
\def\myForeword{相见欢·林花谢了春红}
\def\myForewordText{
    林花谢了春红,太匆匆。
    无奈朝来寒雨晚来风。
    胭脂泪,相留醉,几时重。
    自是人生长恨水长东。
}
\def\mySubheading{以姜晓千强化课讲义为底本}


\begin{document}
\input{\path/cover_text_\myIndex.tex}

\newpage
\thispagestyle{empty}
\begin{center}
    \Huge\textbf{\myForeword}
\end{center}
\myForewordText
\begin{flushright}
    \begin{tabular}{c}
        \myDateForeword
    \end{tabular}
\end{flushright}

\newpage
\pagestyle{plain}
\setcounter{page}{1}
\pagenumbering{Roman}
\tableofcontents

\newpage
\pagenumbering{arabic}
% \setcounter{chapter}{-1}
\setcounter{page}{1}

\pagestyle{fancy}
\fancyfoot[C]{\thepage}
\renewcommand{\headrulewidth}{0.4pt}
\renewcommand{\footrulewidth}{0pt}








\else
\fi

\chapter{常微分方程}
\section{一阶微分方程}

\begin{enumerate}[label=\arabic*.]
    \item (1998,数一、数二)已知函数$y=y(x)$在任意点$x$处的增量$\Delta y=\frac{y\Delta x}{1+x^2}+\alpha$,其中$\alpha$是$\Delta x$的高阶无穷小,$y(0)=\pi$,则$y(1)$等于 \\
    $(A)\ 2\pi \quad (B)\ \pi \quad (C)\ e^{\frac{\pi}{4}} \quad (D)\ \pi e^{\frac{\pi}{4}}$

    \begin{solution}
    \newpage
    \end{solution}
    
    \item (2002,数二)已知函数$f(x)$在$(0,+\infty)$内可导,$f(x)>0$,$\lim_{x\to+\infty}f(x)=1$,且满足
    \begin{align*}
        \lim_{h\to0}\left(\frac{f(x+hx)}{f(x)}\right)^{\frac{1}{h}}=e^{\frac{1}{x}}
    \end{align*}
    求$f(x)$。
    
    \begin{solution}
    \newpage
    \end{solution}
\end{enumerate}

\begin{enumerate}[label=\arabic*.,start=3]
    \item (1999,数二)求初值问题
    \begin{align*}
        \begin{cases}
            (y+\sqrt{x^2+y^2})dx-xdy=0 & (x>0) \\
            y|_{x=1}=0
        \end{cases}
    \end{align*}
    
    \begin{solution}
    \newpage
    \end{solution}
\end{enumerate}



\begin{enumerate}[label=\arabic*.,start=4]
    \item (2010,数二、数三)设$y_1,y_2$是一阶线性非齐次微分方程$y'+p(x)y=q(x)$的两个特解。若常数$\lambda,\mu$使$\lambda y_1+\mu y_2$是该方程的解,$\lambda y_1-\mu y_2$是该方程对应的齐次方程的解,则 \\
    $(A)\ \lambda=\frac{1}{2},\ \mu=\frac{1}{2} \qquad (B)\ \lambda=-\frac{1}{2},\mu=-\frac{1}{2}$ \\
    $(C)\ \lambda=\frac{2}{3},\ \mu=\frac{1}{3}\qquad (D)\ \lambda=\frac{2}{3},\mu=\frac{2}{3}$ 

    
    \begin{solution}
    \newpage
    \end{solution}
\end{enumerate}

\begin{enumerate}[label=\arabic*.,start=5]
    \item (2018,数一)已知微分方程$y'+y=f(x)$,其中$f(x)$是$\mathbb{R}$上的连续函数。
    \begin{enumerate}[label=(\roman*)]
        \item[(1)] 若$f(x)=x$,求方程的通解;
        \item[(2)] 若$f(x)$是周期为$T$的函数,证明:方程存在唯一的以$T$为周期的解。
    \end{enumerate}
    
    \begin{solution}
    \newpage
    \end{solution}
\end{enumerate}


\begin{enumerate}[label=\arabic*.,start=6]
    \item 求解微分方程$y'-\frac{4}{x}y=x^2\sqrt{y}$.
    
    \begin{solution}
    \newpage
    \end{solution}
\end{enumerate}

\begin{enumerate}[label=\arabic*.,start=7]
    \item 求解下列微分方程:\\
        $(1)\ (2xe^y+3x^2-1)dx+(x^2e^y-2y)dy=0;$ \\
        $(2)\ \frac{2x}{y^3}dx+\frac{y^2-3x^2}{y^4}dy=0$.
    
    \begin{solution}
    \newpage
    \end{solution}
\end{enumerate}

\section{二阶常系数线性微分方程}

\begin{enumerate}[label=\arabic*.,start=8]
    \item (2017,数二)微分方程$y''-4y'+8y=e^{2x}(1+\cos2x)$的特解可设为$y^*=$ \\
        $(A)\ Ae^{2x}+e^{2x}(B\cos2x+C\sin2x)$ \\
        $(B)\ Axe^{2x}+e^{2x}(B\cos2x+C\sin2x)$ \\
        $(C)\ Ae^{2x}+xe^{2x}(B\cos2x+C\sin2x)$ \\
        $(D)\ Axe^{2x}+xe^{2x}(B\cos2x+C\sin2x)$
    
    \begin{solution}
    \newpage
    \end{solution}
    
    \item (2015,数一)设$y=\frac{1}{2}e^{2x}+(x-\frac{1}{3})e^x$是二阶常系数非齐次线性微分方程$y''+ay'+by=ce^x$的一个特解,则 \\
    $(A)\ a=-3,b=2,c=-1$ \qquad
    $(B)\ a=3,b=2,c=-1$ \\
    $(C)\ a=-3,b=2,c=1$ \qquad
    $(D)\ a=3,b=2,c=1$ 
    
    \begin{solution}
    \newpage
    \end{solution}
    
    \item (2016,数二)已知$y_1(x)=e^x$,$y_2(x)=u(x)e^x$是二阶微分方程$(2x-1)y''-(2x+1)y'+2y=0$的两个解。若$u(-1)=e$,$u(0)=-1$,求$u(x)$,并写出该微分方程的通解。
    
    \begin{solution}
    \newpage
    \end{solution}
    
    \item (2016,数一)设函数$y(x)$满足方程$y''+2y'+ky=0$,其中$0<k<1$。
    \begin{enumerate}[label=(\roman*)]
        \item[(1)] 证明反常积分$\int_0^{+\infty}y(x)dx$收敛;
        \item[(2)] 若$y(0)=1$,$y'(0)=1$,求$\int_0^{+\infty}y(x)dx$的值。
    \end{enumerate}
    
    \begin{solution}
    \newpage
    \end{solution}
\end{enumerate}

\section{ 高阶常系数线性齐次微分方程}

\begin{enumerate}[label=\arabic*.,start=12]
    \item 求解微分方程$y^{(4)}-3y''-4y=0$。
    
    \begin{solution}
    \newpage
    \end{solution}
\end{enumerate}

\section{ 二阶可降阶微分方程}
\begin{enumerate}[label=\arabic*.,start=13]
    \item 求微分方程$y''(x+y'^2)=y'$满足初始条件$y(1)=y'(1)=1$的特解。
    
    \begin{solution}
    
    \end{solution}
\end{enumerate}

\section{ 欧拉方程}

\begin{enumerate}[label=\arabic*.,start=14]
    \item 求解微分方程$x^2y''+xy'+y=2\sin\ln x$。
    
    \begin{solution}
    \newpage
    \end{solution}
\end{enumerate}

\section{ 变量代换求解二阶变系数线性微分方程}

\begin{enumerate}[label=\arabic*.,start=17]
    \item (2005,数二)用变量代换$x=\cos t(0<t<\pi)$化简微分方程$(1-x^2)y''-xy'+y=0$,并求其满足$y|_{x=0}=1$,$y'|_{x=0}=2$的特解。
    
    \begin{solution}
    \newpage
    \end{solution}
\end{enumerate}

\section{ 微分方程综合题}

\begin{enumerate}[label=\arabic*.,start=18]
    \item (2001,数二)设$L$是一条平面曲线,其上任意一点$P(x,y)(x>0)$到坐标原点的距离,恒等于该点处的切线在$y$轴上的截距,且$L$经过点$(\frac{1}{2},0)$。求曲线$L$的方程。
    
    \begin{solution}
    \newpage
    \end{solution}
\end{enumerate}

\begin{enumerate}[label=\arabic*.,start=19]
    \item (2009,数三)设曲线$y=f(x)$,其中$f(x)$是可导函数,且$f(x)>0$。已知曲线$y=f(x)$与直线$y=0$,$x=1$及$x=t(t>1)$所围成的曲边梯形绕$x$轴旋转一周所得的立体体积值是该曲边梯形面积值的$\pi t$倍,求该曲线的方程。
    
    \begin{solution}
    \newpage
    \end{solution}
\end{enumerate}

\begin{enumerate}[label=\arabic*.,start=20]
    \item (2016,数三)设函数$f(x)$连续,且满足$\int_0^x f(x-t)dt=\int_0^x(x-t)f(t)dt+e^{-x}-1$,求$f(x)$。
    
    \begin{solution}
    \newpage
    \end{solution}
\end{enumerate}

\begin{enumerate}[label=\arabic*.,start=21]
    \item (2014,数一、数二、数三)设函数$f(u)$具有二阶连续导数,$z=f(e^x\cos y)$满足
    \begin{align*}
        \frac{\partial^2 z}{\partial x^2}+\frac{\partial^2 z}{\partial y^2}=(4z+e^x\cos y)e^{2x}
    \end{align*}
    若$f(0)=0$,$f'(0)=0$,求$f(u)$的表达式。
    
    \begin{solution}
    \newpage
    \end{solution}
\end{enumerate}

\begin{enumerate}[label=\arabic*.,start=22]
    \item (2011,数三)设函数$f(x)$在区间$[0,1]$上具有连续导数,$f(0)=1$,且满足
    \begin{align*}
        \iint_{D_t} f'(x+y)\d x\d y=\iint_{D_t} f(t)\d x\d y
    \end{align*}
    其中$D_t=\{(x,y)|0\leq y\leq t-x,0\leq x\leq t\}(0<t\leq1)$,求$f(x)$的表达式。
    
    \begin{solution}
    \newpage
    \end{solution}
\end{enumerate}

\ifx\allfiles\undefined
\end{document}
\fi