\ifx\allfiles\undefined
\documentclass[12pt, a4paper, oneside, UTF8]{ctexbook}
\def\path{../../config}
\usepackage{amsmath}
\usepackage{amsthm}
\usepackage{amssymb}
\usepackage{array}
\usepackage{xcolor}
\usepackage{graphicx}
\usepackage{mathrsfs}
\usepackage{enumitem}
\usepackage{geometry}
\usepackage[colorlinks, linkcolor=black]{hyperref}
\usepackage{stackengine}
\usepackage{yhmath}
\usepackage{extarrows}
\usepackage{tikz}
\usepackage{pgfplots}
\usepackage{asymptote}
\usepackage{float}
\usepackage{fontspec} % 使用字体

\setmainfont{Times New Roman}
\setCJKmainfont{LXGWWenKai-Light}[
    SlantedFont=*
]

\everymath{\displaystyle}

\usepgfplotslibrary{polar}
\usepackage{subcaption}
\usetikzlibrary{decorations.pathreplacing, positioning}

\usepgfplotslibrary{fillbetween}
\pgfplotsset{compat=1.18}
% \usepackage{unicode-math}
\usepackage{esint}
\usepackage[most]{tcolorbox}

\usepackage{fancyhdr}
\usepackage[dvipsnames, svgnames]{xcolor}
\usepackage{listings}

\definecolor{mygreen}{rgb}{0,0.6,0}
\definecolor{mygray}{rgb}{0.5,0.5,0.5}
\definecolor{mymauve}{rgb}{0.58,0,0.82}
\definecolor{NavyBlue}{RGB}{0,0,128}
\definecolor{Rhodamine}{RGB}{255,0,255}
\definecolor{PineGreen}{RGB}{0,128,0}

\graphicspath{ {figures/},{../figures/}, {config/}, {../config/} }

\linespread{1.6}

\geometry{
    top=25.4mm, 
    bottom=25.4mm, 
    left=20mm, 
    right=20mm, 
    headheight=2.17cm, 
    headsep=4mm, 
    footskip=12mm
}

\setenumerate[1]{itemsep=5pt,partopsep=0pt,parsep=\parskip,topsep=5pt}
\setitemize[1]{itemsep=5pt,partopsep=0pt,parsep=\parskip,topsep=5pt}
\setdescription{itemsep=5pt,partopsep=0pt,parsep=\parskip,topsep=5pt}

\lstset{
    language=Mathematica,
    basicstyle=\tt,
    breaklines=true,
    keywordstyle=\bfseries\color{NavyBlue}, 
    emphstyle=\bfseries\color{Rhodamine},
    commentstyle=\itshape\color{black!50!white}, 
    stringstyle=\bfseries\color{PineGreen!90!black},
    columns=flexible,
    numbers=left,
    numberstyle=\footnotesize,
    frame=tb,
    breakatwhitespace=false,
} 

\lstset{
    language=TeX, % 设置语言为 TeX
    basicstyle=\ttfamily, % 使用等宽字体
    breaklines=true, % 自动换行
    keywordstyle=\bfseries\color{NavyBlue}, % 关键字样式
    emphstyle=\bfseries\color{Rhodamine}, % 强调样式
    commentstyle=\itshape\color{black!50!white}, % 注释样式
    stringstyle=\bfseries\color{PineGreen!90!black}, % 字符串样式
    columns=flexible, % 列的灵活性
    numbers=left, % 行号在左侧
    numberstyle=\footnotesize, % 行号字体大小
    frame=tb, % 顶部和底部边框
    breakatwhitespace=false % 不在空白处断行
}

% \begin{lstlisting}[language=TeX] ... \end{lstlisting}

% 定理环境设置
\usepackage[strict]{changepage} 
\usepackage{framed}

\definecolor{greenshade}{rgb}{0.90,1,0.92}
\definecolor{redshade}{rgb}{1.00,0.88,0.88}
\definecolor{brownshade}{rgb}{0.99,0.95,0.9}
\definecolor{lilacshade}{rgb}{0.95,0.93,0.98}
\definecolor{orangeshade}{rgb}{1.00,0.88,0.82}
\definecolor{lightblueshade}{rgb}{0.8,0.92,1}
\definecolor{purple}{rgb}{0.81,0.85,1}

\theoremstyle{definition}
\newtheorem{myDefn}{\indent Definition}[section]
\newtheorem{myLemma}{\indent Lemma}[section]
\newtheorem{myThm}[myLemma]{\indent Theorem}
\newtheorem{myCorollary}[myLemma]{\indent Corollary}
\newtheorem{myCriterion}[myLemma]{\indent Criterion}
\newtheorem*{myRemark}{\indent Remark}
\newtheorem{myProposition}{\indent Proposition}[section]

\newenvironment{formal}[2][]{%
	\def\FrameCommand{%
		\hspace{1pt}%
		{\color{#1}\vrule width 2pt}%
		{\color{#2}\vrule width 4pt}%
		\colorbox{#2}%
	}%
	\MakeFramed{\advance\hsize-\width\FrameRestore}%
	\noindent\hspace{-4.55pt}%
	\begin{adjustwidth}{}{7pt}\vspace{2pt}\vspace{2pt}}{%
		\vspace{2pt}\end{adjustwidth}\endMakeFramed%
}

\newenvironment{definition}{\vspace{-\baselineskip * 2 / 3}%
	\begin{formal}[Green]{greenshade}\vspace{-\baselineskip * 4 / 5}\begin{myDefn}}
	{\end{myDefn}\end{formal}\vspace{-\baselineskip * 2 / 3}}

\newenvironment{theorem}{\vspace{-\baselineskip * 2 / 3}%
	\begin{formal}[LightSkyBlue]{lightblueshade}\vspace{-\baselineskip * 4 / 5}\begin{myThm}}%
	{\end{myThm}\end{formal}\vspace{-\baselineskip * 2 / 3}}

\newenvironment{lemma}{\vspace{-\baselineskip * 2 / 3}%
	\begin{formal}[Plum]{lilacshade}\vspace{-\baselineskip * 4 / 5}\begin{myLemma}}%
	{\end{myLemma}\end{formal}\vspace{-\baselineskip * 2 / 3}}

\newenvironment{corollary}{\vspace{-\baselineskip * 2 / 3}%
	\begin{formal}[BurlyWood]{brownshade}\vspace{-\baselineskip * 4 / 5}\begin{myCorollary}}%
	{\end{myCorollary}\end{formal}\vspace{-\baselineskip * 2 / 3}}

\newenvironment{criterion}{\vspace{-\baselineskip * 2 / 3}%
	\begin{formal}[DarkOrange]{orangeshade}\vspace{-\baselineskip * 4 / 5}\begin{myCriterion}}%
	{\end{myCriterion}\end{formal}\vspace{-\baselineskip * 2 / 3}}
	

\newenvironment{remark}{\vspace{-\baselineskip * 2 / 3}%
	\begin{formal}[LightCoral]{redshade}\vspace{-\baselineskip * 4 / 5}\begin{myRemark}}%
	{\end{myRemark}\end{formal}\vspace{-\baselineskip * 2 / 3}}

\newenvironment{proposition}{\vspace{-\baselineskip * 2 / 3}%
	\begin{formal}[RoyalPurple]{purple}\vspace{-\baselineskip * 4 / 5}\begin{myProposition}}%
	{\end{myProposition}\end{formal}\vspace{-\baselineskip * 2 / 3}}


\newtheorem{example}{\indent \color{SeaGreen}{Example}}[section]
\renewcommand{\proofname}{\indent\textbf{\textcolor{TealBlue}{Proof}}}
\NewEnviron{solution}{%
	\begin{proof}[\indent\textbf{\textcolor{TealBlue}{Solution}}]%
		\color{blue}% 设置内容为蓝色
		\BODY% 插入环境内容
		\color{black}% 恢复默认颜色(可选,避免影响后续文字)
	\end{proof}%
}

% 自定义命令的文件

\def\d{\mathrm{d}}
\def\R{\mathbb{R}}
%\newcommand{\bs}[1]{\boldsymbol{#1}}
%\newcommand{\ora}[1]{\overrightarrow{#1}}
\newcommand{\myspace}[1]{\par\vspace{#1\baselineskip}}
\newcommand{\xrowht}[2][0]{\addstackgap[.5\dimexpr#2\relax]{\vphantom{#1}}}
\newenvironment{mycases}[1][1]{\linespread{#1} \selectfont \begin{cases}}{\end{cases}}
\newenvironment{myvmatrix}[1][1]{\linespread{#1} \selectfont \begin{vmatrix}}{\end{vmatrix}}
\newcommand{\tabincell}[2]{\begin{tabular}{@{}#1@{}}#2\end{tabular}}
\newcommand{\pll}{\kern 0.56em/\kern -0.8em /\kern 0.56em}
\newcommand{\dive}[1][F]{\mathrm{div}\;\boldsymbol{#1}}
\newcommand{\rotn}[1][A]{\mathrm{rot}\;\boldsymbol{#1}}

\newif\ifshowanswers
\showanswerstrue % 注释掉这行就不显示答案

% 定义答案环境
\newcommand{\answer}[1]{%
    \ifshowanswers
        #1%
    \fi
}

% 修改参数改变封面样式,0 默认原始封面、内置其他1、2、3种封面样式
\def\myIndex{0}


\ifnum\myIndex>0
    \input{\path/cover_package_\myIndex} 
\fi

\def\myTitle{考研数学笔记}
\def\myAuthor{Weary Bird}
\def\myDateCover{\today}
\def\myDateForeword{\today}
\def\myForeword{相见欢·林花谢了春红}
\def\myForewordText{
    林花谢了春红,太匆匆。
    无奈朝来寒雨晚来风。
    胭脂泪,相留醉,几时重。
    自是人生长恨水长东。
}
\def\mySubheading{以姜晓千强化课讲义为底本}


\begin{document}
\else
\fi

\chapter{函数 极限 连续}
\section{函数的性态}
\begin{remark}[有界性的判定]
    \begin{enumerate}
    \item[(1)] 连续函数在闭区间$[a,b]$上必然有界 

    \item[(2)] 连续函数在开区间$(a,b)$上只需要判断端点处的左右极限,若$\lim_{x\to a^{+}}\neq \infty$ 且
    $\lim_{x\to b^{-}}\neq \infty$,则连续函数在该区间内有界.

    \item[(3)] $f'(x)$在{\color{red} 有限}区间(a,b)内有界.
    \end{enumerate}

    \underline{Proof:}
    $\forall x\in (a,b),\text{由拉格朗日中值定理},\exists\xi$ 
    \begin{align*}
        f(x)-f(\frac{a+b}{2}) &=f'(\xi)(x-\frac{a+b}{2}) \\
        \left|f(x)\right| &\leq \left|f'(\xi)\right|\left|x-\frac{a+b}{2}\right|+\left|f(\frac{a+b}{2})\right| \\
        \left|f(x)\right| &\leq \frac{b-a}{2}\left|f'(\xi)\right|+\left|f(\frac{a+b}{2})\right| \leq M
    \end{align*}
\end{remark}

\begin{enumerate}[label=\arabic*.]
    \item  下列函数无界的是 \\
    A\quad $f(x)=\frac{1}{x}\sin x, x\in(0,+\infty)$\qquad
    B\quad $f(x)=x\sin\frac{1}{x}, x\in(0,+\infty)$ \\
    C\quad $f(x)=\frac{1}{x}\sin\frac{1}{x}, x\in(0,+\infty)$ \qquad
    D\quad $f(x)=\int_0^x\frac{\sin t}{t} dt, x\in(0,2022)$
    
    \begin{solution}
        \begin{enumerate}
        \item[(A)] $\lim_{x\to 0^{+}}f(x)=1$, $\lim_{x\to +\infty}=0$均为有限值,故A在区间$(0, +\infty)$有界
        \item[(B)] $\lim_{x\to 0^{+}}f(x)=0$, $\lim_{x\to +\infty}=1$均为有限值,故B在区间$(0, +\infty)$有界
        \item[(C)] $\lim_{x\to 0^{+}}f(x)=+\infty$, $\lim_{x\to +\infty}=0$在0点的极限不为有限值,故C在区间$(0, +\infty)$无界
        \item[(D)] $\lim_{x\to 0^{+}}f(x)=\lim_{x\to 0^{+}}\int_{0}^{x}1dt=0$, 
        $\lim_{x\to 2022^{-}}f(x)=\int_{0}^{2022}\frac{\sin{t}}{t}dt=\text{有限值}$ 故D在区间$(0, 2022)$有界
        \end{enumerate}
    \end{solution}
\end{enumerate}

\begin{corollary}[无穷VS无界]
    \underline{无界} 只有有一个子列趋于无穷即可 \\
    \underline{无穷} 任意子列均趋于无穷.  \\
    例如A选项,当 $x_n=\frac{1}{2n\pi+\pi/2},f(x_n)=2n\pi+\pi/2$,$n\to\infty,f(x_n)\to\infty$;当
    $x_n=\frac{1}{2n\pi},f(x_n)=0$, $n\to\infty,f(x_n)\to 0$不为无穷大,仅仅是无界.
\end{corollary}

\begin{remark}[导函数与原函数的奇偶性与周期性]
    \begin{enumerate}
    \item 连续奇函数的所有原函数$\int_{0}^{x}f(t)dt+C$都是偶函数 

    \item 连续偶函数仅有一个原函数$\int_{0}^{x}f(t)\d t$为奇函数
    
    \item 连续周期函数的原函数为周期函数 $\iff\int_{0}^{T}f(x)\d x=0$
    \end{enumerate}
\end{remark}

\begin{enumerate}[label=\arabic*.,start=2]
    \item  (2002,数二)设函数$f(x)$连续,则下列函数中,必为偶函数的是 \\
    A\quad $\int_0^x f(t^2) dt$ \qquad\qquad
    B\quad $\int_0^x f^2(t) dt$ \\
    C\quad $\int_0^x t[f(t)-f(-t)] dt$ \qquad
    D\quad $\int_0^x t[f(t)+f(-t)] dt$

    
    \begin{solution}
    这种题可以采用奇偶性的定义直接去做,如下面选项A,B的解法,也可以按照上述的函数奇偶性的性质判断
    \begin{enumerate}
    \item [(A)] 令$F(x) = \int_0^x f(t^2) dt$
    \[
    F(-x)=\int_0^{-x} f(t^2) dt=-\int_0^x f(t^2) dt=-F(x)
    \]
    则A选项是奇函数
    \item [(B)] 
    \[
    F(-x)=\int_0^{-x} f^2(t) dt = -\int_0^x f^2(-t) dt
    \]
    推导不出B的奇偶性
    \item [(C)] $t[f(t)-f(-t)]$是一个偶函数,故C选项是一个奇函数
    \item [(D)] $t[f(t)+f(-t)]$是一个奇函数,故D选项是一个偶函数
    \end{enumerate}
    \end{solution}
\end{enumerate}

\section{极限的概念}
\begin{definition}[函数极限的定义]
    设函数 $f(x)$ 在点 $x_0$ 的某去心邻域内有定义。若存在常数 $A$,使得对于任意给定的正数 $\epsilon$,总存在正数 $\delta$,使得当 $x$ 满足
    \[
    0 < |x - x_0| < \delta
    \]
    时,必有
    \[
    |f(x) - A| < \epsilon
    \]
    则称 $A$ 为函数 $f(x)$ 当 $x$ 趋近于 $x_0$ 时的极限,记作
    \[
    \lim_{x \to x_0} f(x) = A
    \]
    或
    \[
    f(x) \to A \quad (x \to x_0).
    \]
\end{definition}
\begin{enumerate}[label=\arabic*.,start=3]
    \item  (2014,数三)设$\lim_{n\to\infty} a_n = a$,且$a \neq 0$,则当$n$充分大时有
    \begin{align*}
        (\text{A}) |a_n| > \frac{|a|}{2} \qquad
        (\text{B}) |a_n| < \frac{|a|}{2} \qquad
        (\text{C}) a_n > a - \frac{1}{n} \qquad
        (\text{D}) a_n < a + \frac{1}{n}
    \end{align*}
    
    \begin{solution}
    令$\epsilon = \left|a\right|/2$, 则$\left|a_n-a\right|< |a|/2 \geq \left| |a_n| - |a|\right|$ 
    即
    $$
        |a|/2 < |a_n| < \frac{ 3|a| }{2}
    $$
    对于CD考虑当 \\
    $a_n=a-\frac{2}{n}$ 和 $a_n=a+\frac{2}{n}$ 简单来说$\forall\epsilon$这里面的$\epsilon$与$n$是无关的.
    \end{solution}
\end{enumerate}

\section{函数极限的计算}
\begin{remark}
    这一个题型基本上是计算能力的考察,对于常见未定式其实也没必要区分,目标都是往最简单$\frac{0}{0}$或者
$\frac{\cdot}{\infty}$模型上面靠,辅助以Taylor公式,拉格朗日中值定理结合夹逼准则来做就可以.
\end{remark}

\begin{enumerate}[label=\arabic*.,start=4]
    \item  (2000,数二)若$\displaystyle \lim_{x\to0}\frac{\sin6x+xf(x)}{x^3}=0$,则$\displaystyle \lim_{x\to0}\frac{6+f(x)}{x^2}$为 \\
    $(\text{A}) \ 0 \qquad (\text{B})\ 6 \qquad (\text{C})\ 36 \qquad (\text{D})\ \infty$
    
    \begin{solution}
    这个题第一次见可能想不到,但做多了就一个套路用Taylor就是了.

    $\sin{6x}=6x-36x^2+o(x^3)$,带入题目极限有
    \[
    \lim_{x\to 0}\frac{6x+xf(x)+o(x^3)}{x^3} = \lim_{x\to 0}\frac{6x+xf(x)}{x^3} = 36
    \]
    \end{solution}
    
    \item  (2002,数二)设$y=y(x)$是二阶常系数微分方程$y''+py'+qy=e^{3x}$
    满足初始条件$y(0)=y'(0)=0$的特解,
    则当$x\to0$时,函数$\frac{\ln(1+x^2)}{y(x)}$的极限
    \newline
    \text{(A)}不等于\qquad \text{(B)}等于1\qquad \text{(C)}等于2\qquad \text{(D)}等于3
    
    \begin{solution}
    由微分方程和$y(0)=y'(0)=0$可知$y''(0)=1$,则$y(x)=\frac{1}{2}x^2+o(x^2)$,则
    \[
    \lim_{x\to 0}\frac{\ln(1+x^2)}{y(x)} = \lim_{x\to 0}\frac{x^2}{\frac{1}{2}x^2} = 2
    \]
    \end{solution}
\end{enumerate}

\begin{enumerate}[label=\arabic*.,start=6]
    \item  (2014,数一、数二、数三)求极限
    $
    \lim_{x\to \infty}\frac{\int_{1}^{x}\left[t^2(e^{\frac{1}{t}} - 1) - t\right]dt}
    {x^2\ln{(1+\frac{1}{x})}}
    $
    \begin{solution}
    \begin{align*}
        \lim_{x\to \infty}\frac{\int_{1}^{x}\left[t^2(e^{\frac{1}{t}} - 1) - t\right]dt}
        {x} &= \lim_{x\to \infty} x^2(e^{\frac{1}{x}}-1)-x \\
        & =\lim_{t\to 0}\frac{e^t-1-x}{x^2} \\
        & = \frac{1}{2}
    \end{align*}
    \end{solution}
\end{enumerate}

\begin{enumerate}[label=\arabic*.,start=7]
    \item  求极限$\lim_{x\to0^+}\ln(1+x)\ln\left(1+e^{1/x}\right)$
    
    \begin{solution}
    
    \end{solution}
\end{enumerate}

\begin{enumerate}[label=\arabic*.,start=8]
    \item  求极限$\lim_{x\to\infty}\left(x^3\ln\frac{x+1}{x-1}-2x^2\right)$
    
    \begin{solution}
    \newpage
    \end{solution}
\end{enumerate}

\begin{enumerate}[label=\arabic*.,start=9]
    \item  (2010,数三)求极限$\lim_{x\to+\infty}\left(x^{1/x}-1\right)^{1/\ln x}$
    
    \begin{solution}
    \newpage
    \end{solution}
\end{enumerate}

\begin{enumerate}[label=\arabic*.,start=10]
    \item  求极限$\lim_{x\to0}\left(\frac{a^x+a^{2x}+\cdots+a^{nx}}{n}\right)^{1/x}\ (a>0,n\in\mathbb{N})$
    
    \begin{solution}
    
    \end{solution}
\end{enumerate}

\section{已知极限反求参数}

\begin{enumerate}[label=\arabic*.,start=11]
    \item  (1998,数二)确定常数$a,b,c$的值,使$\lim_{x\to0}\frac{ax-\sin x}{\int_b^x\frac{\ln(1+t^3)}{t}dt}=c\ (c\neq0)$
    
    \begin{solution}
    \newpage
    \end{solution}
\end{enumerate}

\section{无穷小阶的比较}

\begin{enumerate}[label=\arabic*.,start=12]
    \item  (2002,数二)设函数$f(x)$在$x=0$的某邻域内具有二阶连续导数,且$f(0)\neq0$,$f'(0)\neq0$,$f''(0)\neq0$。证明:存在唯一的一组实数$\lambda_1,\lambda_2,\lambda_3$,使得当$h\to0$时,$\lambda_1f(h)+\lambda_2f(2h)+\lambda_3f(3h)-f(0)$是比$h^2$高阶的无穷小。
    
    \begin{solution}
    \newpage
    \end{solution}
    
    \item  (2006,数二)试确定$A,B,C$的值,使得$e^x(1+Bx+Cx^2)=1+Ax+o(x^3)$,其中$o(x^3)$是当$x\to0$时比$x^3$高阶的无穷小量。
    
    \begin{solution}
    \newpage
    \end{solution}
    
    \item  (2013,数二、数三)当$x\to0$时,$1-\cos x\cdot\cos2x\cdot\cos3x$与$ax^n$为等价无穷小,求$n$与$a$的值。
    
    \begin{solution}
    \newpage
    \end{solution}
\end{enumerate}

\section{数列极限的计算}
\begin{remark}[方法]
    \begin{enumerate}
        \item [(1)] 单调有界准则 (三步走,先确定单调性,在确定有界性,最后解方程求极限) \\
        确定单调性,可以考虑作差/做商/求导
        \item [(2)] 压缩映射原理
        \begin{enumerate}
            \item [$\circ 1$] 先猜出极限$x^*$
            \item [$\circ 2$] 证明递推式$\left|x^*_{n+1}-x^*_{n}\right|\leq k\left|x_{n}^*\right|$ 其中$k<1$
        \end{enumerate}
        \item [(3)] 夹逼准则
        \item [(4)] 定积分的定义 (n项和/n项积)
    \end{enumerate}
\end{remark}
\begin{enumerate}[label=\arabic*.,start=15]
    \item  (2011,数一、数二)
    \begin{enumerate}[label=(\roman*)]
        \item 证明:对任意正整数$n$,都有$\frac{1}{n+1}<\ln\left(1+\frac{1}{n}\right)<\frac{1}{n}$
        \item 设$a_n=1+\frac{1}{2}+\cdots+\frac{1}{n}-\ln n\ (n=1,2,\cdots)$,证明数列$\{a_n\}$收敛。
    \end{enumerate}
    
    \begin{solution}
    \begin{enumerate}
        \item [(1)] 是基本不等式的证明,考虑拉格朗日中值即可
        \item [(2)] 考研大题,特别是分成几个小问的题目,都需要合理利用前面的结论 \\
        考虑$a_{n+1}-a_{n}$有 
        \begin{align*}
            a_{n+1}-a_{n}=\frac{1}{n+1}-\ln(n+1)+\ln(n) = \frac{1}{n+1}-\ln(1+n/1) < 0
        \end{align*}
        即$\{a_n\}$单调递减,考虑其有界性
        \begin{align*}
            a_n &= 1 + 1/2 + 1/3 +\ldots + 1/n - \ln(n) \\
            & < \ln(1+1) + \ln(1+1/2) + \ldots + \ln(1+n/1) - \ln(n) \\
            & =  \ln (n+1) - \ln(n) > 0 
        \end{align*}
        即$\{a_n\}$有上界,故由单调有界定理知数列$\{a_n\}$收敛.
    \end{enumerate}
    \end{solution}
    
    \item  (2018,数一、数二、数三)设数列$\{x_n\}$满足:$x_1>0$,$x_ne^{x_{n+1}}=e^{x_n}-1\ (n=1,2,\cdots)$。证明$\{x_n\}$收敛,并求$\lim_{n\to\infty}x_n$。
    
    \begin{solution}
    这道题的难度在于如何处理条件.考虑\underline{1}的妙用. 有
    \begin{align*}
        e^{x_{n+1}}=\frac{e^{x_n}-1}{x} &=\frac{e^{x_n}-e^{0}}{1} \\
        &= e^{\xi}, \xi\in(0,x_n)
    \end{align*}
    而由于$e^x$是单调递增的函数则必然有$\xi=x_{n+1}$即 $0<x_{n+1}<x_n$从而单调递减有下界.此时$\{x_n\}$极限存在. \\
    不妨设$\lim_{n\to\infty}x_n=a$问题转换为求方程 $ae^a=e^a-1$的解的问题.显然$a=0$是其一个根.考虑函数$f(x)=e^{x}(1-x)-1$其
    导数为$-xe^x$在$(0,\infty)$上单调递减故$x=a$是$f(x)$唯一零点,即$a=0$是唯一解.故
    $$
    \fbox{$\lim_{n\to\infty}x_n=0$}
    $$
    \end{solution}
    
    \begin{corollary}[常见的等价代换有]
        \underline{1}: $e^{0},\sin(\pi/2),\cos(0),\ln(e)$ 具体情况还得看题目,题目有啥用啥替换 \\
        \underline{0}: $\sin(0),\cos(pi/2),\ln(1)$ 
    \end{corollary}

    \item  (2019,数一、数三)设$a_n=\int_0^1 x^n\sqrt{1-x^2}dx\ (n=0,1,2,\cdots)$。
    \begin{enumerate}[label=(\roman*)]
        \item[(1)] 证明数列$\{a_n\}$单调减少,且$a_n=\frac{n-1}{n+2}a_{n-2}\ (n=2,3,\cdots)$
        \item[(2)] 求$\lim_{n\to\infty}\frac{a_n}{a_{n-1}}$
    \end{enumerate}
    
    \begin{solution}
        这道题第一问比较重要,第二问比较简单 \\
        (1)\ 方法一: \\ 
        可以直接求出$a_n$的值,令$x=\sin(t)$
        \begin{align*}
            a_n &=\int_{0}^{\pi/2}\sin^n(t)\cos^2(t)\d t \\
            &=\int_{0}^{\pi/2}\sin^{n}(t)-\int_{0}^{\pi/2}\sin^{n+2}(t)\d t \\
            &\xlongequal{\text{华里士公式}} \frac{1}{n+2}\frac{n-1}{n}\ldots\frac{1}{2}\frac{\pi}{2}, \text{当n时偶数的时候} \\
            a_{n-2} &=\frac{1}{n-1}\frac{n-1}{n}\ldots\frac{1}{2}\frac{\pi/2}{2} \\
            a_n &= \frac{n-1}{n+2}a_{n-2} \\
        \end{align*}
        当n为奇数的时候同理可得 \\
        (1)\ 方法二: \\
        也可以考虑分部积分法 
        \begin{align*}
            a_n &=\int_{0}^{1}x^n(1-x^2)^{1/2}\d x \\
            &= -\frac{1}{3}\left[x^{n-1}(1-x^2)^{3/2}\mid^{1}_0-\int_{0}^{1}(1-x^2)^\frac{3}{2}\d x^{n-1}\right] \\
            &= \frac{n-1}{3}\int_{0}^{1}\sqrt{1-x^2}(1-x^2)x^{n-2}\d x \\
            &= \frac{n-1}{3}a_{n-2}-\frac{n-1}{3}a_n \\
            &\implies a_n=\frac{n-1}{n+2}a_{n-2}
        \end{align*}
        (2) \\
        由(1)可知
        $$
        \frac{n-1}{n+2}<\frac{a_n}{a_{n-1}}=\frac{n-1}{n-2}\frac{a_{n-2}}{a_{n-1}}<1
        $$
        当$n\to\infty$由夹逼准则可知$\lim_{n\to\infty}\frac{a_n}{a_{n-1}}=1$
    \end{solution}
    
    \item  (2017,数一、数二、数三)求$\lim_{n\to\infty}\sum_{k=1}^n\frac{k}{n^2}\ln\left(1+\frac{k}{n}\right)$
    
    \begin{solution}
    这是最普通的定积分的定义的应用
    \begin{align*}
        \text{原式} &= \lim_{n\to\infty}\frac{1}{n}\sum_{k=1}^{n}\frac{k}{n}\ln(1+\frac{k}{n}) \\
        &\xlongequal{\text{定积分定义}}\int_{0}^{1}x\ln(1+x)\d x \\
        &=\frac{1}{2}\int_{0}^{1}\ln(1+x)\d x^2 \\
        &= \frac{1}{4}
    \end{align*}
    \end{solution}
\end{enumerate}

\section{间断点的判定}

\begin{enumerate}[label=\arabic*.,start=19]
    \item  (2000,数二)设函数$f(x)=\frac{x}{a+e^{bx}}$在$(-\infty,+\infty)$内连续,且$\lim_{x\to-\infty}f(x)=0$,
    则常数$a,b$满足 \\
        A\quad $a<0,b<0$ \qquad B\quad $a>0,b>0$ \qquad
        C\quad $a\leq0,b>0$ \qquad D\quad $a\geq0,b<0$
    
    \begin{solution}
    \newpage
    \end{solution}
\end{enumerate}



\ifx\allfiles\undefined
\end{document}
\fi