\ifx\allfiles\undefined
\documentclass[12pt, a4paper, oneside, UTF8]{ctexbook}
\def\path{../../config}
\usepackage{amsthm}
\usepackage{amssymb}
\usepackage{array}
\usepackage{xcolor}
\usepackage{graphicx}
\usepackage{mathrsfs}
\usepackage{enumitem}
\usepackage{geometry}
\usepackage[colorlinks, linkcolor=black]{hyperref}
\usepackage{stackengine}
\usepackage{yhmath}
\usepackage{extarrows}
\usepackage{tikz}
\usepackage{forest}
\usetikzlibrary{decorations.pathreplacing, positioning}
% \usepackage{unicode-math}
\usepackage{esint}
\usepackage{pifont}
\usepackage{tcolorbox}
\tcbuselibrary{skins, breakable}

\usepackage{multicol} 
\usepackage{fontspec} % 使用字体

\setmainfont{Times New Roman}
\setCJKmainfont{LXGWWenKai-Light}[
    SlantedFont=*
]

\usepackage{listings} % 用于插入代码

% 定义代码高亮风格
\lstset{
    basicstyle=\ttfamily\small,        % 基本字体样式(等宽小字体)
    keywordstyle=\color{blue},         % 关键字颜色
    commentstyle=\color{green},        % 注释颜色
    stringstyle=\color{red},           % 字符串颜色
    numbers=none,
    breaklines=true,                   % 自动换行
    frame=single,                      % 代码框边框
    rulecolor=\color{black},           % 边框颜色
    captionpos=b,                      % 标题位置(底部)
    showspaces=false,                  % 不显示空格标记
    showstringspaces=false,            % 不显示字符串中的空格标记
    language=C                         % 设置语言为 C
}

\usepackage{fontawesome5}

\usepackage{amsmath}
\usepackage{booktabs, array}
\usepackage{makecell}
\usepackage{fancyhdr}
\usepackage[dvipsnames, svgnames]{xcolor}
\usepackage{listings}
\usepackage{tasks}[2020/01/11]

\everymath{\displaystyle}

\definecolor{mygreen}{rgb}{0,0.6,0}
\definecolor{mygray}{rgb}{0.5,0.5,0.5}
\definecolor{mymauve}{rgb}{0.58,0,0.82}
\definecolor{NavyBlue}{RGB}{0,0,128}
\definecolor{Rhodamine}{RGB}{255,0,255}
\definecolor{PineGreen}{RGB}{0,128,0}

\graphicspath{ {figures/},{../figures/}, {config/}, {../config/} }

\linespread{1.6}

\geometry{
    top=25.4mm, 
    bottom=25.4mm, 
    left=20mm, 
    right=20mm, 
    headheight=2.17cm, 
    headsep=4mm, 
    footskip=12mm
}

\setenumerate[1]{itemsep=5pt,partopsep=0pt,parsep=\parskip,topsep=5pt}
\setitemize[1]{itemsep=5pt,partopsep=0pt,parsep=\parskip,topsep=5pt}
\setdescription{itemsep=5pt,partopsep=0pt,parsep=\parskip,topsep=5pt}



% \begin{lstlisting}[language=TeX] ... \end{lstlisting}

% 定理环境设置
% ---------- 颜色 ----------
\definecolor{ExBlue}{HTML}{4F81BD}
\definecolor{SolGreen}{HTML}{77933C}
\definecolor{DefRed}{HTML}{C5504B}
\definecolor{ThmOrange}{HTML}{E97132}
\definecolor{RemGray}{HTML}{7F7F7F}
\definecolor{CorPurple}{HTML}{7030A0}
\definecolor{ForGray}{HTML}{595959}

% ---------- 通用“变色”模板 ----------
\tcbset{
    mybox/.style n args={1}{
        enhanced, breakable,
        arc=6pt,
        boxrule=0.6pt,
        left=8pt, right=8pt, top=6pt, bottom=6pt,
        drop shadow={black!25},
        fonttitle=\bfseries,
        coltitle=white,
        colbacktitle=#1!85,
        colback=#1!10,
        colframe=#1,
    }
}

% ---------- 各环境 ----------
% 例题
\newtcolorbox{example}[1][]{mybox={ExBlue}, title={\ifstrempty{#1}{Example}{#1}}}
% 解答
\newtcolorbox{solution}[1][]{mybox={SolGreen}, title={\ifstrempty{#1}{Solution}{#1}}}
% 定义
\newtcolorbox{definition}[1][]{mybox={DefRed}, title={\ifstrempty{#1}{Definition}{#1}}}
% 定理
\newtcolorbox{theorem}[1][]{mybox={ThmOrange}, title={\ifstrempty{#1}{Theorem}{#1}}}
% 标注
\newtcolorbox{remark}[1][]{mybox={RemGray}, title={\ifstrempty{#1}{Remark}{#1}}}
% 推论
\newtcolorbox{corollary}[1][]{mybox={CorPurple}, title={\ifstrempty{#1}{Corollary}{#1}}}
% 公式
\newtcolorbox{formula}[1][]{mybox={ForGray}, title={\ifstrempty{#1}{Formula}{#1}}}


\settasks{
    label-format = \bfseries,
    label        = \Alph*.,
    label-width  = 1.2em,
    label-offset = 0.3em,
    item-indent  = 1.9em,
    column-sep   = 0.5em
}

\newenvironment{choices}[1][4]   % 默认 4 栏
    {\begin{tasks}(#1)}
    {\end{tasks}}

% 自定义命令的文件

\def\d{\mathrm{d}}
\def\R{\mathbb{R}}
\def\P{\partial} 
\newcommand{\bs}[1]{\begin{solution}#1\end{solution}}
\newcommand{\bt}[1][1]{% 默认参数为1
    \ensuremath{% 确保数学模式
        \foreach \n in {1,...,#1} {\blacktriangle}% 循环输出 #1 个黑色三角形
    }%
}

\newcommand{\bl}[1][1]{% 默认参数为1
    \ensuremath{% 确保数学模式
        \foreach \n in {1,...,#1} {\blacklozenge}% 循环输出 #1 个黑色三角形
    }%
}
\newif\ifshowanswers
%\showanswerstrue % 注释掉这行就不显示答案

% 定义答案环境
\newcommand{\answer}[1]{%
    \ifshowanswers
        #1%
    \fi
}




% 修改参数改变封面样式,0 默认原始封面、内置其他1、2、3种封面样式
\def\myIndex{3}


\ifnum\myIndex>0
    \input{\path/cover_package_\myIndex} 
\fi

\def\myTitle{冲刺150笔记}
\def\myAuthor{Weary Bird}
\def\myDateCover{\today}
\def\myDateForeword{\today}
\def\myForeword{行香子}
\def\myForewordText{
树绕村庄,水满陂塘;倚东风、豪兴徜徉。小园几许,收尽春光。有桃花红,李花白,菜花黄。 \\
远远苔墙,隐隐茅堂;飏青旗、流水桥旁。偶然乘兴,步过东冈。正莺儿啼,燕儿舞,蝶儿忙。 \\
}
\def\mySubheading{知错能改善莫大焉}


\begin{document}
\input{../../config/cover}
\else
\fi

\chapter{函数 极限 连续}
\section{函数的性态}
\begin{remark}(有界性的判定)
    \item 连续函数在闭区间$[a,b]$上必然有界 

    \item 连续函数在开区间$(a,b)$上只需要判断端点处的左右极限,若$\lim_{x\to a^{+}}\neq \infty$ 且
    $\lim_{x\to b^{-}}\neq \infty$,则连续函数在该区间内有界.
\end{remark}

\begin{enumerate}[label=\arabic*.]
    \item  下列函数无界的是
    \begin{align*}
        (\text{A})&\quad f(x)=\frac{1}{x}\sin x, x\in(0,+\infty) \\
        (\text{B})&\quad f(x)=x\sin\frac{1}{x}, x\in(0,+\infty) \\
        (\text{C})&\quad f(x)=\frac{1}{x}\sin\frac{1}{x}, x\in(0,+\infty) \\
        (\text{D})&\quad f(x)=\int_0^x\frac{\sin t}{t} dt, x\in(0,2022)
    \end{align*}
    
    \begin{solution}
    \item[(A)] $\lim_{x\to 0^{+}}f(x)=1$, $\lim_{x\to +\infty}=0$均为有限值,故A在区间$(0, +\infty)$有界
    \item[(B)] $\lim_{x\to 0^{+}}f(x)=0$, $\lim_{x\to +\infty}=1$均为有限值,故B在区间$(0, +\infty)$有界
    \item[(C)] $\lim_{x\to 0^{+}}f(x)=+\infty$, $\lim_{x\to +\infty}=0$在0点的极限不为有限值,故C在区间$(0, +\infty)$无界
    \item[(D)] $\lim_{x\to 0^{+}}f(x)=\lim_{x\to 0^{+}}\int_{0}^{x}1dt=0$, 
    $\lim_{x\to 2022^{-}}f(x)=\int_{0}^{2022}\frac{\sin{t}}{t}dt=\text{有限值}$ 故D在区间$(0, 2022)$有界
    \end{solution}
\end{enumerate}

\begin{remark}(导函数与原函数的奇偶性与周期性)
    \item 连续奇函数的所有原函数$\int_{0}^{x}f(t)dt+C$都是偶函数 

    \item 连续偶函数仅有一个原函数$\int_{0}^{x}f(t)\d t$为奇函数
\end{remark}

\begin{enumerate}[label=\arabic*.,start=2]
    \item  (2002,数二)设函数$f(x)$连续,则下列函数中,必为偶函数的是
    \begin{align*}
        (\text{A})&\quad \int_0^x f(t^2) dt \\
        (\text{B})&\quad \int_0^x f^2(t) dt \\
        (\text{C})&\quad \int_0^x t[f(t)-f(-t)] dt \\
        (\text{D})&\quad \int_0^x t[f(t)+f(-t)] dt
    \end{align*}
    
    \begin{solution}
    这种题可以采用奇偶性的定义直接去做,如下面选项A,B的解法,也可以按照上述的函数奇偶性的性质判断
    \item [(A)] 令$F(x) = \int_0^x f(t^2) dt$
    \[
    F(-x)=\int_0^{-x} f(t^2) dt=-\int_0^x f(t^2) dt=-F(x)
    \]
    则A选项是奇函数
    \item [(B)] 
    \[
    F(-x)=\int_0^{-x} f^2(t) dt = -\int_0^x f^2(-t) dt
    \]
    推导不出B的奇偶性
    \item [(C)] $t[f(t)-f(-t)]$是一个偶函数,故C选项是一个奇函数
    \item [(D)] $t[f(t)+f(-t)]$是一个奇函数,故D选项是一个偶函数
    \end{solution}
\end{enumerate}

\section{极限的概念}
\begin{definition}[函数极限的定义]
    设函数 $f(x)$ 在点 $x_0$ 的某去心邻域内有定义。若存在常数 $A$,使得对于任意给定的正数 $\epsilon$,总存在正数 $\delta$,使得当 $x$ 满足
    \[
    0 < |x - x_0| < \delta
    \]
    时,必有
    \[
    |f(x) - A| < \epsilon
    \]
    则称 $A$ 为函数 $f(x)$ 当 $x$ 趋近于 $x_0$ 时的极限,记作
    \[
    \lim_{x \to x_0} f(x) = A
    \]
    或
    \[
    f(x) \to A \quad (x \to x_0).
    \]
\end{definition}
\begin{enumerate}[label=\arabic*.,start=3]
    \item  (2014,数三)设$\lim_{n\to\infty} a_n = a$,且$a \neq 0$,则当$n$充分大时有
    \begin{align*}
        (\text{A}) |a_n| > \frac{|a|}{2} \qquad
        (\text{B}) |a_n| < \frac{|a|}{2} \qquad
        (\text{C}) a_n > a - \frac{1}{n} \qquad
        (\text{D}) a_n < a + \frac{1}{n}
    \end{align*}
    
    \begin{solution}
    由数列极限的定义可知当n充分大的时候有$\left|a_n-a\right|<\epsilon$
    \newline 考虑选项C,D,令$\epsilon=\frac{1}{n}$
    则$\left|a_n-a\right|<\frac{1}{n} \implies a-\frac{1}{n}<a_n<a+\frac{1}{n}$
    \end{solution}
\end{enumerate}

\section{函数极限的计算}
这一个题型基本上是计算能力的考察,对于常见未定式其实也没必要区分的那么明显,目标都是往最简单$\frac{0}{0}$或者
$\frac{\cdot}{\infty}$模型上面靠,辅助以Taylor公式,拉格朗日中值定理结合夹逼准则来做就可以.
\begin{remark}(类型一 $\frac{0}{0}$型)
\end{remark}

\begin{enumerate}[label=\arabic*.,start=4]
    \item  (2000,数二)若$\lim_{x\to0}\frac{\sin6x+xf(x)}{x^3}=0$,则$\lim_{x\to0}\frac{6+f(x)}{x^2}$为
    \begin{align*}
        (\text{A})&\ 0 \qquad (\text{B})\ 6 \qquad (\text{C})\ 36 \qquad (\text{D})\ \infty
    \end{align*}
    
    \begin{solution}
    这个题第一次见可能想不到,但做多了就一个套路用Taylor就是了.

    $\sin{6x}=6x-36x^2+o(x^3)$,带入题目极限有
    \[
    \lim_{x\to 0}\frac{6x+xf(x)+o(x^3)}{x^3} = \lim_{x\to 0}\frac{6x+xf(x)}{x^3} = 36
    \]
    \end{solution}
    
    \item  (2002,数二)设$y=y(x)$是二阶常系数微分方程$y''+py'+qy=e^{3x}$
    满足初始条件$y(0)=y'(0)=0$的特解,
    则当$x\to0$时,函数$\frac{\ln(1+x^2)}{y(x)}$的极限
    \newline
    \text{(A)}不等于\qquad \text{(B)}等于1\qquad \text{(C)}等于2\qquad \text{(D)}等于3
    
    \begin{solution}
    由微分方程和$y(0)=y'(0)=0$可知$y''(0)=1$,则$y(x)=\frac{1}{2}x^2+o(x^2)$,则
    \[
    \lim_{x\to 0}\frac{\ln(1+x^2)}{y(x)} = \lim_{x\to 0}\frac{x^2}{\frac{1}{2}x^2} = 2
    \]
    \end{solution}
\end{enumerate}

\begin{remark}(类型二 $\frac{\infty}{\infty}$型)
\end{remark}

\begin{enumerate}[label=\arabic*.,start=6]
    \item  (2014,数一、数二、数三)求极限
    \[
    \lim_{x\to \infty}\frac{\int_{1}^{x}\left[t^2(e^{\frac{1}{t}} - 1) - t\right]dt}
    {x^2\ln{(1+\frac{1}{x})}}
    \]
    \begin{solution}
    \begin{align*}
        \lim_{x\to \infty}\frac{\int_{1}^{x}\left[t^2(e^{\frac{1}{t}} - 1) - t\right]dt}
        {x} &= \lim_{x\to \infty} x^2(e^{\frac{1}{x}}-1)-x \\
        & =\lim_{t\to 0}\frac{e^t-1-x}{x^2} \\
        & = \frac{1}{2}
    \end{align*}
    \end{solution}
\end{enumerate}

\begin{remark}(类型三 $0\cdot\infty$型)
\end{remark}

\begin{enumerate}[label=\arabic*.,start=7]
    \item  求极限$\lim_{x\to0^+}\ln(1+x)\ln\left(1+e^{1/x}\right)$
    
    \begin{solution}
    
    \end{solution}
\end{enumerate}

\begin{remark}(类型四 $\infty-\infty$型)
\end{remark}

\begin{enumerate}[label=\arabic*.,start=8]
    \item  求极限$\lim_{x\to\infty}\left(x^3\ln\frac{x+1}{x-1}-2x^2\right)$
    
    \begin{solution}
    【详解】
    \end{solution}
\end{enumerate}

\begin{remark}(类型五 $0^0$与$\infty^0$型)
\end{remark}

\begin{enumerate}[label=\arabic*.,start=9]
    \item  (2010,数三)求极限$\lim_{x\to+\infty}\left(x^{1/x}-1\right)^{1/\ln x}$
    
    \begin{solution}
    【详解】
    \end{solution}
\end{enumerate}

\begin{remark}(类型六 $1^\infty$型)
\end{remark}

\begin{enumerate}[label=\arabic*.,start=10]
    \item  求极限$\lim_{x\to0}\left(\frac{a^x+a^{2x}+\cdots+a^{nx}}{n}\right)^{1/x}\ (a>0,n\in\mathbb{N})$
    
    \begin{solution}
    
    \end{solution}
\end{enumerate}

\section{已知极限反求参数}
\begin{remark}(方法)

\end{remark}

\begin{enumerate}[label=\arabic*.,start=11]
    \item  (1998,数二)确定常数$a,b,c$的值,使$\lim_{x\to0}\frac{ax-\sin x}{\int_b^x\frac{\ln(1+t^3)}{t}dt}=c\ (c\neq0)$
    
    \begin{solution}
    【详解】
    \end{solution}
\end{enumerate}

\section{无穷小阶的比较}
\begin{remark}(方法)
\end{remark}
\begin{enumerate}[label=\arabic*.,start=12]
    \item  (2002,数二)设函数$f(x)$在$x=0$的某邻域内具有二阶连续导数,且$f(0)\neq0$,$f'(0)\neq0$,$f''(0)\neq0$。证明:存在唯一的一组实数$\lambda_1,\lambda_2,\lambda_3$,使得当$h\to0$时,$\lambda_1f(h)+\lambda_2f(2h)+\lambda_3f(3h)-f(0)$是比$h^2$高阶的无穷小。
    
    \begin{solution}
    【详解】
    \end{solution}
    
    \item  (2006,数二)试确定$A,B,C$的值,使得$e^x(1+Bx+Cx^2)=1+Ax+o(x^3)$,其中$o(x^3)$是当$x\to0$时比$x^3$高阶的无穷小量。
    
    \begin{solution}
    【详解】
    \end{solution}
    
    \item  (2013,数二、数三)当$x\to0$时,$1-\cos x\cdot\cos2x\cdot\cos3x$与$ax^n$为等价无穷小,求$n$与$a$的值。
    
    \begin{solution}
    【详解】
    \end{solution}
\end{enumerate}

\section{数列极限的计算}
\begin{remark}(方法)
\end{remark}
\begin{enumerate}[label=\arabic*.,start=15]
    \item  (2011,数一、数二)
    \begin{enumerate}[label=(\roman*)]
        \item 证明:对任意正整数$n$,都有$\frac{1}{n+1}<\ln\left(1+\frac{1}{n}\right)<\frac{1}{n}$
        \item 设$a_n=1+\frac{1}{2}+\cdots+\frac{1}{n}-\ln n\ (n=1,2,\cdots)$,证明数列$\{a_n\}$收敛。
    \end{enumerate}
    
    \begin{solution}
    【详解】
    \end{solution}
    
    \item  (2018,数一、数二、数三)设数列$\{x_n\}$满足:$x_1>0$,$x_ne^{x_{n+1}}=e^{x_n}-1\ (n=1,2,\cdots)$。证明$\{x_n\}$收敛,并求$\lim_{n\to\infty}x_n$。
    
    \begin{solution}
    【详解】
    \end{solution}
    
    \item  (2019,数一、数三)设$a_n=\int_0^1 x^n\sqrt{1-x^2}dx\ (n=0,1,2,\cdots)$。
    \begin{enumerate}[label=(\roman*)]
        \item 证明数列$\{a_n\}$单调减少,且$a_n=\frac{n-1}{n+2}a_{n-2}\ (n=2,3,\cdots)$
        \item 求$\lim_{n\to\infty}\frac{a_n}{a_{n-1}}$
    \end{enumerate}
    
    \begin{solution}
    【详解】
    \end{solution}
    
    \item  (2017,数一、数二、数三)求$\lim_{n\to\infty}\sum_{k=1}^n\frac{k}{n^2}\ln\left(1+\frac{k}{n}\right)$
    
    \begin{solution}
    【详解】
    \end{solution}
\end{enumerate}

\section{间断点的判定}

\begin{enumerate}[label=\arabic*.,start=19]
    \item  (2000,数二)设函数$f(x)=\frac{x}{a+e^{bx}}$在$(-\infty,+\infty)$内连续,且$\lim_{x\to-\infty}f(x)=0$,则常数$a,b$满足
    \begin{align*}
        (\text{A})&\ a<0,b<0 \quad (\text{B})\ a>0,b>0 \\
        (\text{C})&\ a\leq0,b>0 \quad (\text{D})\ a\geq0,b<0
    \end{align*}
    
    \begin{solution}
    【详解】
    \end{solution}
\end{enumerate}



\ifx\allfiles\undefined
\end{document}
\fi