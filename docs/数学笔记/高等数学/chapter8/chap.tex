\ifx\allfiles\undefined
\documentclass[12pt, a4paper, oneside, UTF8]{ctexbook}
\def\path{../../config}
\usepackage{amsmath}
\usepackage{amsthm}
\usepackage{amssymb}
\usepackage{array}
\usepackage{xcolor}
\usepackage{graphicx}
\usepackage{mathrsfs}
\usepackage{enumitem}
\usepackage{geometry}
\usepackage[colorlinks, linkcolor=black]{hyperref}
\usepackage{stackengine}
\usepackage{yhmath}
\usepackage{extarrows}
\usepackage{tikz}
\usepackage{pgfplots}
\usepackage{asymptote}
\usepackage{float}
\usepackage{fontspec} % 使用字体

\setmainfont{Times New Roman}
\setCJKmainfont{LXGWWenKai-Light}[
    SlantedFont=*
]

\everymath{\displaystyle}

\usepgfplotslibrary{polar}
\usepackage{subcaption}
\usetikzlibrary{decorations.pathreplacing, positioning}

\usepgfplotslibrary{fillbetween}
\pgfplotsset{compat=1.18}
% \usepackage{unicode-math}
\usepackage{esint}
\usepackage[most]{tcolorbox}

\usepackage{fancyhdr}
\usepackage[dvipsnames, svgnames]{xcolor}
\usepackage{listings}

\definecolor{mygreen}{rgb}{0,0.6,0}
\definecolor{mygray}{rgb}{0.5,0.5,0.5}
\definecolor{mymauve}{rgb}{0.58,0,0.82}
\definecolor{NavyBlue}{RGB}{0,0,128}
\definecolor{Rhodamine}{RGB}{255,0,255}
\definecolor{PineGreen}{RGB}{0,128,0}

\graphicspath{ {figures/},{../figures/}, {config/}, {../config/} }

\linespread{1.6}

\geometry{
    top=25.4mm, 
    bottom=25.4mm, 
    left=20mm, 
    right=20mm, 
    headheight=2.17cm, 
    headsep=4mm, 
    footskip=12mm
}

\setenumerate[1]{itemsep=5pt,partopsep=0pt,parsep=\parskip,topsep=5pt}
\setitemize[1]{itemsep=5pt,partopsep=0pt,parsep=\parskip,topsep=5pt}
\setdescription{itemsep=5pt,partopsep=0pt,parsep=\parskip,topsep=5pt}

\lstset{
    language=Mathematica,
    basicstyle=\tt,
    breaklines=true,
    keywordstyle=\bfseries\color{NavyBlue}, 
    emphstyle=\bfseries\color{Rhodamine},
    commentstyle=\itshape\color{black!50!white}, 
    stringstyle=\bfseries\color{PineGreen!90!black},
    columns=flexible,
    numbers=left,
    numberstyle=\footnotesize,
    frame=tb,
    breakatwhitespace=false,
} 

\lstset{
    language=TeX, % 设置语言为 TeX
    basicstyle=\ttfamily, % 使用等宽字体
    breaklines=true, % 自动换行
    keywordstyle=\bfseries\color{NavyBlue}, % 关键字样式
    emphstyle=\bfseries\color{Rhodamine}, % 强调样式
    commentstyle=\itshape\color{black!50!white}, % 注释样式
    stringstyle=\bfseries\color{PineGreen!90!black}, % 字符串样式
    columns=flexible, % 列的灵活性
    numbers=left, % 行号在左侧
    numberstyle=\footnotesize, % 行号字体大小
    frame=tb, % 顶部和底部边框
    breakatwhitespace=false % 不在空白处断行
}

% \begin{lstlisting}[language=TeX] ... \end{lstlisting}

% 定理环境设置
\usepackage[strict]{changepage} 
\usepackage{framed}

\definecolor{greenshade}{rgb}{0.90,1,0.92}
\definecolor{redshade}{rgb}{1.00,0.88,0.88}
\definecolor{brownshade}{rgb}{0.99,0.95,0.9}
\definecolor{lilacshade}{rgb}{0.95,0.93,0.98}
\definecolor{orangeshade}{rgb}{1.00,0.88,0.82}
\definecolor{lightblueshade}{rgb}{0.8,0.92,1}
\definecolor{purple}{rgb}{0.81,0.85,1}

\theoremstyle{definition}
\newtheorem{myDefn}{\indent Definition}[section]
\newtheorem{myLemma}{\indent Lemma}[section]
\newtheorem{myThm}[myLemma]{\indent Theorem}
\newtheorem{myCorollary}[myLemma]{\indent Corollary}
\newtheorem{myCriterion}[myLemma]{\indent Criterion}
\newtheorem*{myRemark}{\indent Remark}
\newtheorem{myProposition}{\indent Proposition}[section]

\newenvironment{formal}[2][]{%
	\def\FrameCommand{%
		\hspace{1pt}%
		{\color{#1}\vrule width 2pt}%
		{\color{#2}\vrule width 4pt}%
		\colorbox{#2}%
	}%
	\MakeFramed{\advance\hsize-\width\FrameRestore}%
	\noindent\hspace{-4.55pt}%
	\begin{adjustwidth}{}{7pt}\vspace{2pt}\vspace{2pt}}{%
		\vspace{2pt}\end{adjustwidth}\endMakeFramed%
}

\newenvironment{definition}{\vspace{-\baselineskip * 2 / 3}%
	\begin{formal}[Green]{greenshade}\vspace{-\baselineskip * 4 / 5}\begin{myDefn}}
	{\end{myDefn}\end{formal}\vspace{-\baselineskip * 2 / 3}}

\newenvironment{theorem}{\vspace{-\baselineskip * 2 / 3}%
	\begin{formal}[LightSkyBlue]{lightblueshade}\vspace{-\baselineskip * 4 / 5}\begin{myThm}}%
	{\end{myThm}\end{formal}\vspace{-\baselineskip * 2 / 3}}

\newenvironment{lemma}{\vspace{-\baselineskip * 2 / 3}%
	\begin{formal}[Plum]{lilacshade}\vspace{-\baselineskip * 4 / 5}\begin{myLemma}}%
	{\end{myLemma}\end{formal}\vspace{-\baselineskip * 2 / 3}}

\newenvironment{corollary}{\vspace{-\baselineskip * 2 / 3}%
	\begin{formal}[BurlyWood]{brownshade}\vspace{-\baselineskip * 4 / 5}\begin{myCorollary}}%
	{\end{myCorollary}\end{formal}\vspace{-\baselineskip * 2 / 3}}

\newenvironment{criterion}{\vspace{-\baselineskip * 2 / 3}%
	\begin{formal}[DarkOrange]{orangeshade}\vspace{-\baselineskip * 4 / 5}\begin{myCriterion}}%
	{\end{myCriterion}\end{formal}\vspace{-\baselineskip * 2 / 3}}
	

\newenvironment{remark}{\vspace{-\baselineskip * 2 / 3}%
	\begin{formal}[LightCoral]{redshade}\vspace{-\baselineskip * 4 / 5}\begin{myRemark}}%
	{\end{myRemark}\end{formal}\vspace{-\baselineskip * 2 / 3}}

\newenvironment{proposition}{\vspace{-\baselineskip * 2 / 3}%
	\begin{formal}[RoyalPurple]{purple}\vspace{-\baselineskip * 4 / 5}\begin{myProposition}}%
	{\end{myProposition}\end{formal}\vspace{-\baselineskip * 2 / 3}}


\newtheorem{example}{\indent \color{SeaGreen}{Example}}[section]
\renewcommand{\proofname}{\indent\textbf{\textcolor{TealBlue}{Proof}}}
\NewEnviron{solution}{%
	\begin{proof}[\indent\textbf{\textcolor{TealBlue}{Solution}}]%
		\color{blue}% 设置内容为蓝色
		\BODY% 插入环境内容
		\color{black}% 恢复默认颜色(可选,避免影响后续文字)
	\end{proof}%
}

% 自定义命令的文件

\def\d{\mathrm{d}}
\def\R{\mathbb{R}}
%\newcommand{\bs}[1]{\boldsymbol{#1}}
%\newcommand{\ora}[1]{\overrightarrow{#1}}
\newcommand{\myspace}[1]{\par\vspace{#1\baselineskip}}
\newcommand{\xrowht}[2][0]{\addstackgap[.5\dimexpr#2\relax]{\vphantom{#1}}}
\newenvironment{mycases}[1][1]{\linespread{#1} \selectfont \begin{cases}}{\end{cases}}
\newenvironment{myvmatrix}[1][1]{\linespread{#1} \selectfont \begin{vmatrix}}{\end{vmatrix}}
\newcommand{\tabincell}[2]{\begin{tabular}{@{}#1@{}}#2\end{tabular}}
\newcommand{\pll}{\kern 0.56em/\kern -0.8em /\kern 0.56em}
\newcommand{\dive}[1][F]{\mathrm{div}\;\boldsymbol{#1}}
\newcommand{\rotn}[1][A]{\mathrm{rot}\;\boldsymbol{#1}}

\newif\ifshowanswers
\showanswerstrue % 注释掉这行就不显示答案

% 定义答案环境
\newcommand{\answer}[1]{%
    \ifshowanswers
        #1%
    \fi
}

% 修改参数改变封面样式,0 默认原始封面、内置其他1、2、3种封面样式
\def\myIndex{0}


\ifnum\myIndex>0
    \input{\path/cover_package_\myIndex} 
\fi

\def\myTitle{考研数学笔记}
\def\myAuthor{Weary Bird}
\def\myDateCover{\today}
\def\myDateForeword{\today}
\def\myForeword{相见欢·林花谢了春红}
\def\myForewordText{
    林花谢了春红,太匆匆。
    无奈朝来寒雨晚来风。
    胭脂泪,相留醉,几时重。
    自是人生长恨水长东。
}
\def\mySubheading{以姜晓千强化课讲义为底本}


\begin{document}
\else
\fi

\chapter{多元函数积分学}

\begin{tcolorbox}[title=三维向量]
$\vec{a}=(a_x,a_y,a_z), \vec{b}=(b_x,b_y,b_z)$ 
\begin{enumerate}
    \item [数量积] $\vec{a}\cdot\vec{b} = \left|\vec{a}\right|\left|\vec{b}\right|\cos{\theta}=
    a_xb_x+a_yb_y+a_zb_z$ 
    \begin{enumerate}
        \item [性质1] 判断空间向量垂直\ $\vec{a}\cdot\vec{b}=0\iff a\perp b$ 
        \item [性质2] 求空间两直线的夹角\ $\displaystyle \cos\theta = \frac{\vec{a}\cdot\vec{b}}{\left|\vec{a}\right|\left|\vec{b}\right|}$
    \end{enumerate}
    \item [向量积] $a\times b = |a||b|\sin\theta = \begin{vmatrix}
        i & j & k\\
        a_x & a_y & a_z \\
        b_x & b_y & b_z
    \end{vmatrix}$
    \begin{enumerate}
        \item [性质1] 判断空间直线平行 $\vec{a}\times\vec{b}=0\iff a\parallel b$
        \item [性质2] 求平面四边形的面积 $S=\left|\vec{a}\times\vec{b}\right|$ 
    \end{enumerate}
    \item [混合积] $(\vec{a}\vec{b}\vec{c})=(\vec{a}\times\vec{b})\cdot\vec{c} = \left|\vec{a}\times\vec{b}\right|\cdot\left|\vec{c}\right|
    =\begin{vmatrix}
        a_x & a_y & a_z \\
        b_x & b_y & b_z \\
        c_x & c_y & c_z
    \end{vmatrix}$ 
    \begin{enumerate}
        \item [性质1] 判断三个向量是否共面 $\text{共面}\iff (\vec{a}\vec{b}\vec{c})=0$
        \item [性质2] 平行六面体的体积 $V=\left|(\vec{a}\vec{b}\vec{c})\right|$
    \end{enumerate}
\end{enumerate}

\end{tcolorbox}

\begin{tcolorbox}[title=直线与平面]
(一)平面 \\
平面的点法式\ 假设平面过点$(x_0,y_0,z_0)$且该平面的法向量为$\vec{n}=\left\{A,B,C\right\}$ 则平面
方程为
$$
A(x-x_0)+B(y-y_0)+C(z-z_0) = 0
$$
平面的一般式\ 将点法式展开
$$
Ax+By+Cz+D = 0
$$
平面的截距式\  
$$
\frac{x}{a}+\frac{y}{b}+\frac{z}{c} = 1
$$
其中$a,b,c$分别是该平面与$x,y,z$轴的截距 \\
{\color{red} 点到平面的距离公式} 假设平面外一点$(x_0,y_0,z_0)$到平面的距离
$$
d = \frac{\left|Ax_0+By_0+Cz_0+D\right|}{\sqrt{A^2+B^2+C^2}}
$$
(直线) \\
直线的点向式\ 假设直线过点$(x_0,y_0,z_0)$且该直线的方向向量为$\vec{s}=\{l,m,n\}$ 则该直线的直线方程为
$$
\frac{x_0-x}{l}=\frac{y_0-y}{m}=\frac{z-z_0}{n}
$$
直线的参数式 
$$
\begin{cases}
    x = x_0 + lt \\
    y = y_0 + mt \\
    z = z_0 + nt
\end{cases}
$$
直线的一般式(两平面的交线)
$$
\begin{cases}
    A_1x+B_1y+C_1z+D=0 \\
    A_2x+B_2y+C_2z+D=0
\end{cases}
$$
{\color{red} 平面束方程} 过某一直线的所有平面的方程 $\lambda(A_1x+B_1y+C_1z+D)+\mu(A_2x+B_2y+C_2z+D_2) = 0$
其中$\lambda,\mu$不同时为0,$(\ldots)$即该直线一般式的两平面方程
\end{tcolorbox}

\begin{tcolorbox}[title=曲面与曲线]
    假设直线外一点$(x_0,y_0,z_0)$其到直线的距离为
    $$
    d = \frac{\left|(x_1-x_0,y_1-y_0,z_1-z_0)\times(l,m,n)\right|}{\sqrt{l^2+m^2+n^2}}
    $$
    平面与直线的关系基本只需要考察$\vec{n}$和$\vec{s}$的关系即可 \\
    旋转曲面 \\
    假设曲线$L=\begin{cases}
        F(x,y,z) = 0 \\
        G(x,y,z) = 0
    \end{cases}\implies \begin{cases}
        x = x(z) \\
        y = y(z)
    \end{cases}$ 则曲线L绕z轴旋转而来的旋转曲面方程为 
    $$
    {\color{red}x^2 + y ^2 = x^2(z) + y ^2(z)}
    $$
    求旋转曲面的问题,捉住旋转过程中的不变量进行处理,例如绕z轴旋转,则旋转曲面上的点到z轴的距离和z坐标都与原来曲线的点一致即
    $$
    P_0=\begin{cases}
        x_0=x(z_0) \\
        y_0=y(z_0)
    \end{cases}; P=\begin{cases}
        x^2+y^2=x^2_0+y^2_0 \\
        z = z_0
    \end{cases}
    $$ 消去$z_0$即可得到答案 \\
常见曲面的类型 
$$
\begin{cases}
    \text{球面}&x^2+y^2+c^2=R^2 \\
    \text{圆柱面} &x^2+y^2 = R^2 \\
    \text{椭球面} &\frac{x^2}{a^2}+\frac{y^2}{b^2}+\frac{z^2}{c^2} = 1 \\
    \text{抛物面} &\frac{x^2}{2p}+\frac{y^2}{2p} = z (p > 0) \\
    \text{圆锥面} &z=a\sqrt{x^2+y^2}\text{上圆锥面} \\
    \text{单叶双曲面} &\frac{x^2}{a^2}+\frac{y^2}{b^2}-\frac{z^2}{c^2} = 1 \\
    \text{双叶双曲面} &\frac{x^2}{a^2}+\frac{y^2}{b^2}-\frac{z^2}{c^2} = -1
\end{cases}
$$
\end{tcolorbox}

\begin{tcolorbox}[title=曲面与曲线]
    {\color{red} 与线代考点的综合题} 二次型的特征值的正负对应图像的情况 
    $$
    \begin{cases}
        \text{三正}, &\text{球/椭球} \\
        \text{两正}, &\text{圆锥} \\
        \text{两正一负}, &\text{单叶双曲面} \\
        \text{两负一正}, &\text{双叶双曲面}
    \end{cases}
    $$
    投影曲线,往$xoy$面的投影曲线只需要消去$z$即可
    $$
    \begin{cases}
        F(x,y,z) = 0\\
        G(x,y,z) = 0
    \end{cases} \xrightarrow{\text{消去z}} \begin{cases}
        H(x,y) = 0 \\
        {\color{red}} z = 0
    \end{cases}
    $$
{\color{red} 曲面的法向量与切平面}  \\
若曲面是显示给出的即$F(x,y,z)=0$则其法向量为
$$
\vec{n} = \{F'_x,F'_y,F'_z\}
$$
若曲面的是隐式给出的即$z=z(x,y)$则其法向量为 
$$
\vec{n} = \{-z'_x,-z'_y,1\}
$$
切平面方程为
$$
F'_x(x-x_0)+F'_y(y-y_0)+F'z(z-z_0) = 0
$$
{\color{red} 曲线的切向量} \\
若曲线是以参数式给出即$\begin{cases}
    x=x(t) \\
    y=y(t) \\
    z=z(t)
\end{cases}$ 则其切向量为 
$$
\tau = (x'(t),y'(t),z'(t))
$$
若以两曲面的交线形式给出,即$\begin{cases}
    F(x,y,z) =0 \\
    G(x,y,z) = 0
\end{cases}$ 此时切向量为
$$
\tau = \vec{n_1} \times \vec{n_2},\text{其中$n_1,n_2$分别为两曲面的法向量}
$$
\end{tcolorbox}

\begin{tcolorbox}[title=方向导数与三度]
方向导数 
$$
\frac{\partial f}{\partial \vec{l}}\big|_{x_0,y_0} = \lim_{t\to 0^{+}}\frac{f(x_0+t\cos{\alpha},y_0+t\cos{\beta})-f(x_0,y_0)}{t}
$$
其中$\alpha$为与$x$轴正方向的夹角,$\beta$为与$y$轴正方向的夹角$t$是趋于$0^{+}$ \\
若$f(x,y)$可微分,则
$$
\frac{\partial f}{\partial \vec{l}} = f'_x\cos\alpha + f'y\cos\beta = \vec{grad\ f} \cdot \vec{l_0}
$$
梯度,散度,旋度
\begin{align*}
    \vec{grad\ f} &= (\frac{\partial f}{\partial x},\frac{\partial f}{\partial y},\frac{\partial f}{\partial z})\cdot(\vec{i},\vec{j},\vec{k}) \\
    div\vec{A} &= \frac{\partial P}{\partial x} + \frac{\partial Q}{\partial y} + \frac{\partial R}{\partial z} \\
    \vec{rot\ A} &= \begin{vmatrix}
        \vec{i} & \vec{j} & \vec{k} \\
        \frac{\partial}{\partial x} & \frac{\partial}{\partial y} & \frac{\partial}{\partial z} \\
        P & Q & R
    \end{vmatrix}
\end{align*}

{\color{red}方向导数沿梯度方向取得最大值,沿梯度反方向取得最小值},值为
$$
\pm\left|\vec{grad\ f}\right| = \pm\left|(\frac{\partial f}{\partial x},\frac{\partial f}{\partial y},\frac{\partial f}{\partial z})\right|
$$
{\color{red} 三度之间的关系,要求二阶偏导连续}
\begin{align*}
    div\ \vec{grad\ f} &= \frac{\partial^2 f}{\partial x^2} + \frac{\partial^2 f}{\partial y^2} + \frac{\partial^2 f}{\partial z^2} \\
    \vec{rot}\vec{grad\ f} & = \vec{0} \\
    div\vec{rot} = 0
\end{align*}
\end{tcolorbox}
\section{三重积分的计算}
\begin{remark}
    三重积分 \\
    (三重积分的定义) 三维物体的质量
    $$
    \iiint_{\Omega}f(x,y,z)\d V=\lim_{\lambda\to 0}\sum_{i=1}^{n}f(\xi_i,\eta_i,\zeta_i)\Delta V_i
    $$
    三重积分的性质(8条) \\
    线性,区域可加性,比较定理,中值定理,估值定理,轮换对称性,奇偶性,形心公式 \\
    若函数图像关于$xoy$平面对称
    $$
    \iiint_{\Omega}=\begin{cases}
        \displaystyle 2\iiint_{\Omega'}f(x,y,z)\d V, &f(x,y,-z)=f(x,y,z) \\
        0, &f(x,y,-z)=-f(x,y,z)
    \end{cases}
    $$
    直接坐标计算(两种)
    $$
    \begin{cases}
        \displaystyle \int_{a}^{b}\d z\iint_{D_z}f(x,y,z)\d x\d y, &\text{先二后一,截面法} \\
        \displaystyle \iint_{D_{xy}}\d x\d y\int_{z_1(x)}^{z_2(x)}f(x,y,z)\d z, &\text{先一后二,投影法}
    \end{cases}
    $$
    柱坐标($x,y$转换为极坐标) 
    $$
    \begin{cases}
        x = r\cos\theta \\
        y = r\sin\theta \\
        z = z \\
        \d V=r\d r\d x\d y 
    \end{cases}
    $$
    球坐标 
    $$
    \begin{cases}
        x = r\sin\varphi\cos\theta \\
        y = r\sin\varphi\sin\theta \\
        z = r\cos\varphi \\
        \d V = r^2\sin\varphi\d r\d\varphi\d\theta
    \end{cases}
    $$
    其中$\theta$是与$x$轴正方向的夹角,$\varphi$是与$z$轴正反向的夹角
\end{remark}

\begin{enumerate}[label=\arabic*.]
    \item (2013,数一)设直线$L$过$A(1,0,0)$,$B(0,1,1)$两点,将$L$绕$z$轴旋转一周得到曲面$\Sigma$,$\Sigma$与平面$z=0$,$z=2$所围成的立体为$\Omega$.
    \begin{enumerate}
        \item[(I)] 求曲面$\Sigma$的方程;
        \item[(II)] 求$\Omega$的形心坐标.
    \end{enumerate}
    
    \begin{solution}
    (1) 有题设可知直线方程为$\begin{cases}
        z = 1 - x \\
        z = y
    \end{cases}$ 原直线上一点$P_0$满足$\begin{cases}
        x_0 = 1 - z_0 \\
        y_0 = z_0 
    \end{cases}$ 旋转曲面上一点$P$满足 $\begin{cases}
        x^2+y^2=x^2_0+y^2_0 \\
        z=z_0
    \end{cases}$ 带入直线方程消去$(x_0,y_0,z_0)$有曲面方程为
    $$
    x^2+y^2=2z^2-2z+1
    $$
    (2) 对于三重积分以及后面的积分,最大的误区可能就是上来二话不说先画图,然后发现图画不出来就不会做.其实完全没必要画图
    观察曲面方程,容易发现其关于$xoz,yoz$平面对称,故$\bar{x}=\bar{y}=0$ 由形心公式有
    $$
    \bar{z} = \frac{\iiint_{\Omega}z\d V}{\iiint_{\Omega}\d V}
    $$
    由题设条件$z\in[0,2]$已经提示了该用截面法喽,从而有
    \begin{align*}
        \iiint_{\Omega}\d V &= \int_{0}^{2}\d z\iint_{D_z}\d x\d y \\
        &= \int_{0}^{2}\pi\cdot\left(2z^2-2z+1\right)\d z \\
        &=\frac{10}{3}\pi \\
        \iiint_{\Omega}z\d V &= \int_{0}^{2}\d z\iint_{D_z} z\d x\d y \\
        &= \int_{0}^{2}\pi\cdot\left(2z^3-2z^2+z\right)\d z \\
        &= \frac{14}{3}\pi
    \end{align*}
    综上形心坐标为
    \begin{center}
        \fbox{$(0,0,\frac{7}{5})$}
    \end{center}
    \end{solution}
    
    \item (2019,数一)设$\Omega$是由锥面$x^{2}+(y-z)^{2}=(1-z)^{2}(0\leq z\leq 1)$与平面$z=0$围成的锥体,求$\Omega$的形心坐标.
    
    \begin{solution}
    这个图像张啥样,其实也一定都不重要.只要能把握其在某一二维平面的投影即可,观察曲面表达式,显然其关于$yoz$平面堆成故$\bar{x}=0$,而由形心公式可知要求3个三重积分,分别做吧 
    \begin{align*}
        \iiint_{\Omega}\d V &= \int_{0}^{1}\d z\iint_{D_z}\d x\d y \\
        &=\int_{0}^{1}\pi(1-z)^2\d z \\
        &=\frac{1}{3}\pi \\
        \iiint_{\Omega}z\d V &= \int_{0}^{1}\d z\iint_{D_z}z\d x\d y \\
        &=\int_{0}^{1}\pi z(1-z)^2\d z \\ 
        &=\frac{1}{12}\pi \\
        \iiint_{\Omega}y\d V &= \int_{0}^{1}\d z\iint_{D_z}y\d x\d y \\
        &=\int_{0}^{1}\pi z(1-z)^2\d z \\
        &=\frac{1}{12}\pi
    \end{align*}
    综上,该区域的形心为
    \begin{center}
        \fbox{$(0,\frac{1}{4},\frac{1}{4})$}
    \end{center}
    \end{solution}
\end{enumerate}

\newpage

\section{第一类曲线积分的计算}
\begin{remark}
    一类线 \\
    定义 
    $$
    \int_Lf(x,y)\d s=\lim_{\lambda\to 0}\sum_{i=1}^{n}f(\xi_i,\eta_i)\Delta s_i 
    $$
    其中$\d s$是弧微分 \\
    一类线的性质(8条)\\
    线性,区域可加性,比较定理,中值定理,估值定理,轮换对称性,奇偶性,形心公式 \\
    计算公式, \underline{曲线方程带入}
    $$
    \int_{L}f(x,y)\d s
    \begin{cases}
        \displaystyle \int_{\alpha}^{\beta}f(x(t),y(t))\sqrt{(x'(t))^2+(y'(t))^2}\d t, &\text{参数方程} \\
        \displaystyle \int_{a}^{b}f(x,y(x))\sqrt{1+(y'(x))^2}\d x, &\text{直接坐标} \\
        \displaystyle \int_{\alpha}^{\beta}f(r(\theta)\cos\theta,r(\theta)\sin\theta)\sqrt{r^2(\theta)+(r'(\theta))^2}\d\theta, &\text{极坐标}
    \end{cases}
    $$
\end{remark}
\begin{enumerate}[label=\arabic*.,start=3]
    \item (2018,数一)设$L$为球面$x^2+y^2+z^2=1$与平面$x+y+z=0$的交线,则$\oint_L xy ds=$
    
    \begin{solution}
    这道题是比较显然的轮换对称性的题目
    \begin{align*}
        \text{原式}&=\frac{1}{3}\oint_{L}(xy+yz+xz)\d s \\
        &=\frac{1}{6}\oint_{L}\left[(x+y+z)^2-(x^2+y^2+z^2)\right] \\
        &\xlongequal{\text{曲线方程带入}} -\frac{1}{6}\oint_{L}\d s \\
        &=-\frac{1}{3}\pi
    \end{align*}
    \end{solution}
    
    \item 设连续函数$f(x,y)$满足$f(x,y)=(x+3y)^2+\int_L f(x,y) ds$,其中$L$为曲线$y=\sqrt{1-x^2}$,求曲线积分$\int_L f(x,y) ds$.
    
    \begin{solution}
    不妨设$A=\int_Lf(x,y)\d s$ 同时对等式两边同时求一类线有 
    \begin{align*}
        A &= \int_{L}\left[(x+3y)^2+A\right]\d s \\
        &=A\pi + \int_{L}(x+3y)^2\d s \\
        &=A\pi + \int_{L}(x^2+6xy+9y^2)\d s \\
        &=(1+A)\pi + 8\int_{L}y^2\d s \\
        &=(1+A)\pi + 8\int_{0}^{2\pi}\sin^2\theta\d \theta \\
        &=(5+A)\pi \implies A = \frac{5\pi}{1-\pi}
    \end{align*}
    {\color{red} 计算过程中优先考虑使用性质化简,而非直接套公式}
    \end{solution}

    \begin{tcolorbox}
        对于曲线/曲面/定积分/二重积分/三重积分,它在某区域内积分后就是一个\underline{数},变限积分和不定积分仍然是一个\underline{函数}.
    \end{tcolorbox}
\end{enumerate}

\section{第二类曲线积分的计算}
\begin{remark}
    二类线 \\
    二类线的定义:沿曲线做功
    $$
    \int_{L}P(x,y)\d x+ Q(x,y)\d y = \lim_{\lambda\to 0}\sum_{i=1}^{n}\left[P(\xi_i,\eta_i)\Delta x_i + Q(\xi_i,\eta_i)\Delta y_i\right]
    $$
    其中$\d x =\d s\cdot\cos{\alpha},\d y=\d s\cdot\cos\beta$,其中$(\cos\alpha,\cos\beta)$为切向量的单位向量 \\
    性质(3条) \\
    线性,区域可加性,方向性 
    $$
    \int_L=-\int_L',L\text{和}L'\text{方向相反}
    $$
    计算方式(两种) 
    $$
    \int_{L}P(x,y)\d x+ Q(x,y)\d y =
    \begin{cases}
        \displaystyle \int_{\alpha}^{\beta}\left[P(x(t),y(t))x'(t)+Q(x(t),y(t))y'(t)\right]\d t, &\text{参数方程} \\
        \displaystyle \int_{a}^{b}\left[P(x,f(x))+Q(x,f(x))f'(x)\right]\d x, &\text{直角坐标}
    \end{cases}
    $$
    注意此时$\alpha\rightarrow\beta,a\rightarrow b$均为起点指向终点,和大小无关\\
    {\color{red} 格林公式} 设闭区域$D$由分段光滑的曲线$L$围成,$L$取正向,$P(x,y),Q(x,y)$在$D$上有一阶连续偏导数,则 
    $$
    \oint_LP\d x + Q\d y = \iint_{D}\left(\frac{\partial Q}{\partial x}-\frac{\partial P}{\partial y}\right)\d x\d y
    $$
    {\color{red} 积分与路径无关(四个充分条件)} 设$P(x,y),Q(x,y)$在单连通闭区域$D$上有一阶连续偏导数,则 
    \begin{align*}
        & \frac{\partial Q}{\partial x} = \frac{\partial P}{\partial y} \\
        &\iff D\text{内任意曲线}L,\oint_LP\d x+Q\d y = 0 \\
        &\iff D\text{任意两点}A,B,\int_{A}^{B}P\d x+ Q\d y\text{与路径无关} \\
        &\iff \exists u(x,y),\d u= P(x,y)\d x + Q(x,y)\d y, \text{且}u(x,y)=\int_{(x_0,y_0)}^{(x,y)}P\d x + Q\d y
    \end{align*}
    \underline{曲线方程带入}
\end{remark}

\begin{tcolorbox}[title=曲线积分基本定理]
    设$P(x,y),Q(x,y)$在区域$D$内连续,$u(x,y)$满足$\d u=P(x,y)\d x + Q(x,y)\d y$, 则区域$D$内任意两点$A,B$曲线积分
    $\int_{A}^{B}P\d x+Q\d y$与路径无关,且$\int_{A}^{B}P\d x + Q\d y=u(B)-u(A)$
\end{tcolorbox}
\begin{enumerate}[label=\arabic*.,start=5]
    \item (2021,数一)设$D\subset \mathbb{R}^2$是有界单连通闭区域,$I(D)=\iint_D(4-x^2-y^2)dxdy$取得最大值的积分域记为$D_1$.
    \begin{enumerate}
        \item[(I)] 求$I(D_1)$的值;
        \item[(II)] 计算$\int_{\partial D_1}\frac{(xe^{x^2+4y^2}+y)dx+(4ye^{x^2+4y^2}-x)dy}{x^2+4y^2}$,其中$\partial D_1$是$D_1$的正向边界.
    \end{enumerate}
    
    \begin{solution}
    (1)由二重积分的几何意义,使得$4-x^2-y^2\geq 0$始终成立的区域即为积分最大的区域,即
    $$
    D_1=\left\{(x,y)\mid x^2+y^2\leq 4\right\} 
    $$
    此时积分为
    \begin{align*}
        I &= \int_{0}^{2\pi}\d\theta\int_{0}^{2}(4-r^2)r\d r = 8\pi
    \end{align*}
    (2) 显然$(0,0,0)$点是被积函数的奇点,此时考虑挖去该点,即设
    $$
    L':x^2+4y^2=1,\text{取顺时针}
    $$
    此时有
    $$
        I=\oint_{\partial D_1+L'}-\oint_{L'} 
    $$
    对于前一个积分,用Green公式有
    $$
    \oint_{\partial D_1+L'} = \iint_{D_1/D_{L'}}\left(\frac{\partial Q}{\partial x}-\frac{\partial P}{\partial y}\right)\d x\d y = 0
    $$
    对于后一个积分,先将曲线方程带入表达式后有
    \begin{align*}
        \oint_{L'} &=\oint_{L'}\left(ex+y\right)\d x + \left(4ey-x\right)\d y \\
        &\xlongequal{\text{格林公式}} - \iint_{D_{L'}}(-1-1) = 2S_{D_{L'}}=\pi
    \end{align*}
    故
    $$
    I = 0 - \pi = -\pi
    $$
    \end{solution}
\end{enumerate}

\begin{enumerate}[label=\arabic*.,start=6]
    \item (2011,数一)设$L$是柱面$x^2+y^2=1$与平面$z=x+y$的交线,从$z$轴正向往$z$轴负向看去为逆时针方向,则曲线积分$\oint_L xz dx+xdy+\frac{y^2}{2}dz=$
    
    \begin{solution}
    这种问题仅有三种解法,推荐解法3,但三种解法都需要掌握. \\
    (解法一\ 公式法) 设曲线的参数方程为
    $\begin{cases}
        x = \cos{t} \\
        y = \sin{t} \\
        z = \sin{t} + \cos{t}
    \end{cases}$ 由于从$z$轴正向往$z$轴负向看去为逆时针方向,故$t:0\rightarrow 2\pi$,此时原积分等于 
    \begin{align*}
        I &= \int_{0}^{2\pi}\left\{\left[\cos{t}(\sin{t}+\cos{t})(-\sin{t})\right] + \cos^2{t} + \frac{\sin^2{t}}{2}(\cos{t}-\sin{t})\right\}\d t \\
        &=\int_{0}^{2\pi}\cos^2{\theta} = \pi
    \end{align*}
    (解法二\ 斯托克斯公式) 注意斯托克斯公式一般转换为一类面来做(公式法) \\
    曲面法向量为$\vec{n}=(-Z'_x,-Z'_y,1)=(-1,-1,1)$ 其单位向量为$\vec{n_0}=(-\frac{1}{\sqrt{3}},-\frac{1}{\sqrt{3}},\frac{1}{\sqrt{3}})$ 
    此时由斯托克斯公式有 
    \begin{align*}
        \oint_{L} &=\iint_{\sum}\begin{vmatrix}
            -\frac{1}{\sqrt{3}} & -\frac{1}{\sqrt{3}} & \frac{1}{\sqrt{3}} \\
            \frac{\partial}{\partial x} & \frac{\partial}{\partial y} & \frac{\partial}{\partial z} \\
            xz & x & \frac{y^2}{2}
        \end{vmatrix} \d S \\
        &=-\frac{1}{\sqrt{3}}\iint_{\sum}(y-x-1)\d S \\
        &\xlongequal{\text{公式法}}=-\frac{1}{\sqrt{3}}\iint_{D_{xy}}(y+x-1)\sqrt{1+1+1}\d x\d y \\
        &=\pi
    \end{align*}
    (解法三\ 转换为平面二类型) 由$z=x+y$消去原曲线积分中的所有$z$,注意$\d z = \d x + \d y$ 此时积分转换为 
    其中$L':x^2+y^2=1$取逆时针方向
    \begin{align*}
        I = &\oint_{L'}(x^2+xy+\frac{y^2}{2})\d x + (x+\frac{y^2}{2})\d y \\
        &\xlongequal{\text{格林公式}} \iint_{D}(1-x-y)\d x\d y = \pi
    \end{align*}
    \end{solution}
\end{enumerate}

\section{第一类曲面积分的计算}
\begin{remark}
    一类面 \\
    一类面的定义 
    $$
    \iint_{\sum}f(x,y,z)\d S=\lim_{\lambda\to 0}\sum_{i=1}^{n}f(\xi_i,\eta_i,\zeta_i)\Delta S_i
    $$
    性质(8条) \\
    线性,区域可加性,比较定理,中值定理,估值定理,轮换对称性,奇偶性,形心公式 \\
    计算公式(一投,二代) 
    $$
    \iint_{\sum}f(x,y,z)\d S=\iint_{D_{xy}}f(x,y,z(x,y))\sqrt{1+(Z'_x)^2+(Z'_y)^2}\d x\d y 
    $$
    \underline{曲面方程带入}
\end{remark}
\begin{enumerate}[label=\arabic*.,start=7]
    \item (2010,数一)设$P$为椭球面$S:x^2+y^2+z^2-yz=1$上的动点,若$S$在点$P$的切平面与$xOy$面垂直,求$P$点的轨迹$C$,并计算曲面积分
    \begin{align*}
    I=\iint_{\Sigma}\frac{(x+\sqrt{3})|y-2z|}{\sqrt{4+y^2+z^2-4yz}}dS,
    \end{align*}
    其中$\Sigma$是椭球面$S$位于曲线$C$上方的部分.
    
    \begin{solution}
    一类面的难点肯定在于如何求出该平面,计算都是小意思用公式就可以. \\
    S在点P处的切平面,其法向量为$\vec{n_1}=(F'_x,F'_y,F'_z=2x,2y-z,2z-y)$ 而$xoy$面的法向量为$\vec{n_2}=(0,0,1)$由题设知
    $\vec{n_1}\cdot\vec{n_2}=0$ 即$2z-y=0$ 带入S的方程化简有,曲线C的方程为
    $$
        \begin{cases}
            x^2+\frac{3}{4}y^2 = 1 \\
            y = 2z
        \end{cases}
    $$
    即一个椭球柱与平面的交线,将曲线往$xoy$面投影,其区域为$D_{xy}:\{(x,y)\mid x^2+\frac{3}{4}y^2 \leq 1\}$ 
    $$
    \d S = \sqrt{1+(Z'_x)^2+(Z'_y)^2}\d x\d y = \frac{\sqrt{4+y^2+z^2-4yz}}{\left|y-2z\right|}\d x\d y
    $$
    原积分由公式法等于 
    \begin{align*}
        I = \iint_{D_{xy}} (x+\sqrt{3})\d x\d y = 2\pi
    \end{align*}
    \end{solution}
\end{enumerate}

\section{第二类曲面积分的计算}
\begin{remark}
    二类面 \\
    二类面的定义:流量
    \begin{align*}
        &\iint_{\sum}P(x,y,z)\d y\d z + Q(x,y,z)\d z\d x + R(x,y,z)\d x\d y \\
        &=\lim_{\lambda\to 0} \sum_{i=1}^{n}[P(\xi_i,\eta_i,\zeta_i)(\Delta S_i)_{yz}+
    Q(\xi_i,\eta_i,\zeta_i)(\Delta S_i)_{zx}+R(\xi_i,\eta_i,\zeta_i)(\Delta S_i)_{xy}]
    \end{align*}
    其中$\d y\d z = \d S\cdot\cos\alpha$ 其余类似,而$(\cos\alpha,\cos\beta,\cos\gamma)$为平面$\sum$的法向量的单位向量 \\
    性质(3条) \\
    线性,区域可加性,方向性 \\
    计算公式(三合一投影法)
    \begin{align*}
    &\iint_{\sum}P(x,y,z)\d y\d z + Q(x,y,z)\d z\d x + R(x,y,z)\d x\d y \\
    &=\pm (P(x,y),Q(x,y),R(x,y))\cdot (-Z'_x,-Z'_y,1) \\
    &=\pm \iint_{D_{xy}}\left[P(x,y,z(x,y))(-Z'_x)+Q(x,y,z(x,y))(-Z'_y)+R(x,y,z(x,y))\right]\d x\d y
    \end{align*}
    上侧为正,下侧为负 \\
    {\color{red} 高斯公式} 设闭区域$\Omega$由分片光滑的曲面$\sum$围成,$\sum$取{\color{red}外侧},$P,Q,R$在其上有{\color{red} 一阶连续偏导数},则
    $$
    \oiint_{\sum}P\d y\d z + Q\d z\d x + R\d x\d y=
    \iiint_{\Omega}\left(\frac{\partial P}{\partial x} + \frac{\partial Q}{\partial y} + \frac{\partial R}{\partial z}\right)
    $$
    \underline{曲面方程带入} \\
    {\color{red} 斯托克斯公式} 设$P,Q,R$在曲面$\sum$围成的区域$\Omega$内有一阶连续偏导数,$\sum$的边界曲线$L$的方向与$\sum$所取的法向量满足右手法则,则
    $$
    \oint_{L}P\d x + Q\d y + R\d z = \iint_{\sum}\begin{vmatrix}
        \d y\d z & \d z\d x & \d x\d y \\
        \frac{\partial}{\partial x} & \frac{\partial}{\partial y} & \frac{\partial}{\partial z} \\
        P & Q & R
    \end{vmatrix} = \begin{vmatrix}
        \cos\alpha & \cos\beta & \cos\gamma \\
        \frac{\partial}{\partial x} & \frac{\partial}{\partial y} & \frac{\partial}{\partial z} \\
        P & Q & R
    \end{vmatrix}\d S
    $$
    即将三维的二类线转换为一类面或者二类面来做
\end{remark}
\begin{enumerate}[label=\arabic*.,start=8]
    \item (2009,数一)计算曲面积分
    \begin{align*}
    I=\oiint_{\Sigma}\frac{xdydz+ydzdx+zdxdy}{(x^2+y^2+z^2)^{\frac{3}{2}}},
    \end{align*}
    其中$\Sigma$是曲面$2x^2+2y^2+z^2=4$的外侧.
    
    \begin{solution}
    显然点$(0,0,0)$是被积函数的奇点,需要挖去这一个点,不妨设 
    $$
    \sum_1:x^2+y^2+z^2 = 1, \text{取外侧}
    $$
    记
    \begin{align*}
        &\Omega:\left\{(x,y,z)\mid x^2+y^2+z^2 \geq 1, 2x^2+2y^2+z^2\leq 4\right\} \\
        &\Omega_1:\left\{(x,y,z)\mid x^2+y^2+z^2 \leq 1\right\}
    \end{align*}
    此时原积分等于 
    $$
    I = \iint_{\sum+\sum_1}-\iint_{\sum_1} 
    $$
    其中 
    $$
    \iint_{\sum+\sum_1} \xlongequal{\text{高斯定理}}
    \iiint_{\Omega}\left(\frac{\partial P}{\partial x} + 
    \frac{\partial Q}{\partial y} + \frac{\partial R}{\partial z}\right) = 0
    $$
    对于第二个积分,先带入$\sum_1$的曲面方程此时有
    \begin{align*}
        \iint_{\sum_1} &= \iint_{\sum_1}x\d y\d z + y\d z\d x + z\d x\d y \\
        &=-\iiint_{\Omega_1}3\d V \\
        &=-4\pi 
    \end{align*}
    综上有
    $$
    I = 0 + 4\pi = 4\pi
    $$
    \end{solution}
    
    \item 计算
    \begin{align*}
    \iint_{\Sigma}\frac{axdydz+(z+a)^2dxdy}{(x^2+y^2+z^2)^2},
    \end{align*}
    其中$\Sigma$为下半球面$z=-\sqrt{a^2-x^2-y^2}$的上侧,$a$为大于零的常数.
    
    \begin{solution}
    发现这个曲面不是封闭的,立刻补上,即设
    $$
    \sum_1:\begin{cases}
        x^2+y^2\leq a^2 \\
        z = 0
    \end{cases},\text{取下侧}
    $$
    注意,虽然被积函数在(0,0,0)处貌似是奇点,但注意到可以通过带入曲线方程消去分母,就不需要挖点了
    \begin{align*}
        I &= \frac{1}{a}\iint_{\sum}ax\d y\d z + (z+a)^2\d x\d y \\
        &=\frac{1}{a}(\iint_{\sum+\sum_1}-\iint_{\sum_1})
    \end{align*}
    记$\sum_1,\sum$围成的区域为$\Omega$,则有
    $$
    \iint_{\sum+\sum_1} = -\iiint_{\Omega} = -\iiint_{\Omega}\left[a+2(z+a)\right]\d V = -\frac{3}{2}\pi a^4
    $$
    记$D_{xy}:\{(x,y)\mid x^2+y^2\leq a^2\}$ 则有
    $$
    \iint_{\sum_1} \xlongequal{\text{公式}}-\iint_{D_{xy}}a^2\d x\d y = -\pi a^4
    $$
    综上有
    $$
    I = -\frac{\pi a^3}{2}
    $$
    \end{solution}
    
    \item (2020,数一)设$\Sigma$为曲面$z=\sqrt{x^2+y^2}(1\leq x^2+y^2\leq 4)$的下侧,$f(x)$为连续函数,计算
    \begin{align*}
    I=\iint_{\Sigma}[xf(xy)+2x-y]\d y\d z+[yf(xy)+2y+x]\d z\d x+[zf(xy)+z]\d x\d y.
    \end{align*}
    
    \begin{solution}
    因为$f(xy)$仅连续,高斯的条件为\underline{封闭外侧,偏导连续},只能使用三合一投影法 \\
    记区域$D_{xy}:\{(x,y)\mid 1 \leq x^2 + y^2 \leq 4\}$
    \begin{align*}
        I &= -\iint_{D_{xy}}\left(\left[xf(xy)+2x-y\right]\left(-\frac{x}{\sqrt{x^2+y^2}}\right)\right. \\
        &\quad + \left.\left[yf(xy)+2y+x\right]\left(-\frac{y}{\sqrt{x^2+y^2}}\right)\right. \\
        &\quad + \left.\left[\sqrt{x^2+y^2}f(xy)+\sqrt{x^2+y^2}\right]\right)\,\mathrm{d}x\,\mathrm{d}y \\
        &= \iint_{D_{xy}}\sqrt{x^2+y^2}\,\mathrm{d}x\,\mathrm{d}y \\
        &= \int_{0}^{2\pi}\mathrm{d}\theta\int_{1}^{2}r^2\,\mathrm{d}r = \frac{14}{3}\pi
    \end{align*}
    \end{solution}
\end{enumerate}


\ifx\allfiles\undefined
\end{document}
\fi