\ifx\allfiles\undefined
\documentclass[12pt, a4paper, oneside, UTF8]{ctexbook}
\def\path{../../config}
\usepackage{amsthm}
\usepackage{amssymb}
\usepackage{array}
\usepackage{xcolor}
\usepackage{graphicx}
\usepackage{mathrsfs}
\usepackage{enumitem}
\usepackage{geometry}
\usepackage[colorlinks, linkcolor=black]{hyperref}
\usepackage{stackengine}
\usepackage{yhmath}
\usepackage{extarrows}
\usepackage{tikz}
\usepackage{forest}
\usetikzlibrary{decorations.pathreplacing, positioning}
% \usepackage{unicode-math}
\usepackage{esint}
\usepackage{pifont}
\usepackage{tcolorbox}
\tcbuselibrary{skins, breakable}

\usepackage{multicol} 
\usepackage{fontspec} % 使用字体

\setmainfont{Times New Roman}
\setCJKmainfont{LXGWWenKai-Light}[
    SlantedFont=*
]

\usepackage{listings} % 用于插入代码

% 定义代码高亮风格
\lstset{
    basicstyle=\ttfamily\small,        % 基本字体样式(等宽小字体)
    keywordstyle=\color{blue},         % 关键字颜色
    commentstyle=\color{green},        % 注释颜色
    stringstyle=\color{red},           % 字符串颜色
    numbers=none,
    breaklines=true,                   % 自动换行
    frame=single,                      % 代码框边框
    rulecolor=\color{black},           % 边框颜色
    captionpos=b,                      % 标题位置(底部)
    showspaces=false,                  % 不显示空格标记
    showstringspaces=false,            % 不显示字符串中的空格标记
    language=C                         % 设置语言为 C
}

\usepackage{fontawesome5}

\usepackage{amsmath}
\usepackage{booktabs, array}
\usepackage{makecell}
\usepackage{fancyhdr}
\usepackage[dvipsnames, svgnames]{xcolor}
\usepackage{listings}
\usepackage{tasks}[2020/01/11]

\everymath{\displaystyle}

\definecolor{mygreen}{rgb}{0,0.6,0}
\definecolor{mygray}{rgb}{0.5,0.5,0.5}
\definecolor{mymauve}{rgb}{0.58,0,0.82}
\definecolor{NavyBlue}{RGB}{0,0,128}
\definecolor{Rhodamine}{RGB}{255,0,255}
\definecolor{PineGreen}{RGB}{0,128,0}

\graphicspath{ {figures/},{../figures/}, {config/}, {../config/} }

\linespread{1.6}

\geometry{
    top=25.4mm, 
    bottom=25.4mm, 
    left=20mm, 
    right=20mm, 
    headheight=2.17cm, 
    headsep=4mm, 
    footskip=12mm
}

\setenumerate[1]{itemsep=5pt,partopsep=0pt,parsep=\parskip,topsep=5pt}
\setitemize[1]{itemsep=5pt,partopsep=0pt,parsep=\parskip,topsep=5pt}
\setdescription{itemsep=5pt,partopsep=0pt,parsep=\parskip,topsep=5pt}



% \begin{lstlisting}[language=TeX] ... \end{lstlisting}

% 定理环境设置
% ---------- 颜色 ----------
\definecolor{ExBlue}{HTML}{4F81BD}
\definecolor{SolGreen}{HTML}{77933C}
\definecolor{DefRed}{HTML}{C5504B}
\definecolor{ThmOrange}{HTML}{E97132}
\definecolor{RemGray}{HTML}{7F7F7F}
\definecolor{CorPurple}{HTML}{7030A0}
\definecolor{ForGray}{HTML}{595959}

% ---------- 通用“变色”模板 ----------
\tcbset{
    mybox/.style n args={1}{
        enhanced, breakable,
        arc=6pt,
        boxrule=0.6pt,
        left=8pt, right=8pt, top=6pt, bottom=6pt,
        drop shadow={black!25},
        fonttitle=\bfseries,
        coltitle=white,
        colbacktitle=#1!85,
        colback=#1!10,
        colframe=#1,
    }
}

% ---------- 各环境 ----------
% 例题
\newtcolorbox{example}[1][]{mybox={ExBlue}, title={\ifstrempty{#1}{Example}{#1}}}
% 解答
\newtcolorbox{solution}[1][]{mybox={SolGreen}, title={\ifstrempty{#1}{Solution}{#1}}}
% 定义
\newtcolorbox{definition}[1][]{mybox={DefRed}, title={\ifstrempty{#1}{Definition}{#1}}}
% 定理
\newtcolorbox{theorem}[1][]{mybox={ThmOrange}, title={\ifstrempty{#1}{Theorem}{#1}}}
% 标注
\newtcolorbox{remark}[1][]{mybox={RemGray}, title={\ifstrempty{#1}{Remark}{#1}}}
% 推论
\newtcolorbox{corollary}[1][]{mybox={CorPurple}, title={\ifstrempty{#1}{Corollary}{#1}}}
% 公式
\newtcolorbox{formula}[1][]{mybox={ForGray}, title={\ifstrempty{#1}{Formula}{#1}}}


\settasks{
    label-format = \bfseries,
    label        = \Alph*.,
    label-width  = 1.2em,
    label-offset = 0.3em,
    item-indent  = 1.9em,
    column-sep   = 0.5em
}

\newenvironment{choices}[1][4]   % 默认 4 栏
    {\begin{tasks}(#1)}
    {\end{tasks}}

% 自定义命令的文件

\def\d{\mathrm{d}}
\def\R{\mathbb{R}}
\def\P{\partial} 
\newcommand{\bs}[1]{\begin{solution}#1\end{solution}}
\newcommand{\bt}[1][1]{% 默认参数为1
    \ensuremath{% 确保数学模式
        \foreach \n in {1,...,#1} {\blacktriangle}% 循环输出 #1 个黑色三角形
    }%
}

\newcommand{\bl}[1][1]{% 默认参数为1
    \ensuremath{% 确保数学模式
        \foreach \n in {1,...,#1} {\blacklozenge}% 循环输出 #1 个黑色三角形
    }%
}
\newif\ifshowanswers
%\showanswerstrue % 注释掉这行就不显示答案

% 定义答案环境
\newcommand{\answer}[1]{%
    \ifshowanswers
        #1%
    \fi
}




% 修改参数改变封面样式,0 默认原始封面、内置其他1、2、3种封面样式
\def\myIndex{3}


\ifnum\myIndex>0
    \input{\path/cover_package_\myIndex} 
\fi

\def\myTitle{冲刺150笔记}
\def\myAuthor{Weary Bird}
\def\myDateCover{\today}
\def\myDateForeword{\today}
\def\myForeword{行香子}
\def\myForewordText{
树绕村庄,水满陂塘;倚东风、豪兴徜徉。小园几许,收尽春光。有桃花红,李花白,菜花黄。 \\
远远苔墙,隐隐茅堂;飏青旗、流水桥旁。偶然乘兴,步过东冈。正莺儿啼,燕儿舞,蝶儿忙。 \\
}
\def\mySubheading{知错能改善莫大焉}


\begin{document}
\input{../../config/cover}
\else
\fi

\chapter{一元函数积分学}
\section{ 定积分的概念}

\begin{enumerate}[label=\arabic*.]
    \item 例1 (2007,数一、数二、数三)如图,连续函数$y=f(x)$在区间[-3,-2],[2,3]上的图形分别是直径为1的上、下半圆周,在区间[-2,0],[0,2]的图形分别是直径为2的下、上半圆周.
    设$F(x)=\int_0^x f(t) dt$,则下列结论正确的是:
    \begin{align*}
        (A) F(3)=-\frac{3}{4} F(-2)
    \end{align*}
    
    \begin{solution}
    【详解】
    \end{solution}
    
    \item 例2 (2009,数三)使不等式$\int_1^x\frac{\sin t}{t} dt>\ln x$成立的$x$的范围是
    \begin{align*}
        (A)\ (0,1)\quad(B)\left(1,\frac{\pi}{2}\right)\quad(C)\left(\frac{\pi}{2},\pi\right)\quad(D)(\pi,+\infty)
    \end{align*}
    
    \begin{solution}
    【详解】
    \end{solution}
    
    \item 例3 (2003,数二)设$I_1=\int_0^{\frac{\pi}{4}}\frac{\tan x}{x} dx, I_2=\int_0^{\frac{\pi}{4}}\frac{x}{\tan x} dx$,则
    \begin{align*}
        (A) I_1>I_2>1\quad(B) 1>I_1>I_2 \\
        (C) I_2>I_1>1\quad(D) 1>I_2>I_1
    \end{align*}
    
    \begin{solution}
    【详解】
    \end{solution}
\end{enumerate}

\section{ 不定积分的计算}

\begin{enumerate}[label=\arabic*.,start=4]
    \item 例5 (2009,数二、数三)计算不定积分$\int\frac{1}{1+\sqrt{\frac{1+x}{x}}}dx(x>0)$
    
    \begin{solution}
    【详解】
    \end{solution}
    
    \item 例6 求$\int\frac{1}{1+\sin x+\cos x} dx$
    
    \begin{solution}
    【详解】
    \end{solution}
\end{enumerate}

\section{ 定积分的计算}

\begin{enumerate}[label=\arabic*.,start=6]
    \item 例7 (2013,数一)计算$\int_0^1\frac{f(x)}{\sqrt{x}} dx$,其中$f(x)=\int_1^x\frac{\ln(t+1)}{t} dt$
    
    \begin{solution}
    【详解】
    \end{solution}
    
    \item 例8 求下列积分:
    \begin{align*}
        (1)\ \int_0^{\frac{\pi}{2}}\frac{1}{1+(\tan x)^{\sqrt{2}}} dx
    \end{align*}
    
    \begin{solution}
    【详解】
    \end{solution}
    
    \item 例9 求$\int_0^{\frac{\pi}{4}}\ln(1+\tan x) dx$
    
    \begin{solution}
    【详解】
    \end{solution}
\end{enumerate}

\section{ 反常积分的计算}

\begin{enumerate}[label=\arabic*.,start=9]
    \item 例10 (1998,数二)计算积分(题目内容缺失)
    
    \begin{solution}
    【详解】
    \end{solution}
\end{enumerate}

\section{ 反常积分敛散性的判定}

\begin{enumerate}[label=\arabic*.,start=10]
    \item 例11 (2016,数一)若反常积分$\int_0^{+\infty}\frac{1}{x^a(1+x)^b} dx$收敛,则
    \begin{align*}
        (A)\ a<1且\ b>1 \\
        (B)\ a>1且\ b>1 \\
        (C)\ a<1且\ a+b>1 \\
        (D)\ a>1且\ a+b>1
    \end{align*}
    
    \begin{solution}
    【详解】
    \end{solution}
    
    \item 例12 (2010,数一、数二)设$m,n$均为正整数,则反常积分$\int_0^1\frac{\sqrt[n]{\ln^2(1-x)}}{\sqrt[n]{x}} dx$的收敛性
    \begin{align*}
        (A)\ 仅与\ m\ 的取值有关 \\
        (B)\ 仅与\ n\ 的取值有关 \\
        (C)\ 与\ m,n\ 的取值都有关 \\
        (D)\ 与\ m,n\ 的取值都无关
    \end{align*}
    
    \begin{solution}
    【详解】
    \end{solution}
\end{enumerate}

\section{ 变限积分函数}

\begin{enumerate}[label=\arabic*.,start=12]
    \item 例13 (2013,数二)设函数$f(x)=\begin{cases}
        \sin x, & 0\leq x<\pi \\
        2, & \pi\leq x\leq 2\pi
    \end{cases}$,$F(x)=\int_0^x f(t) dt$,则
    \begin{align*}
        (A)\ x=\pi\ 是函数\ F(x)\ 的跳跃间断点 \\
        (B)\ x=\pi\ 是函数\ F(x)\ 的可去间断点 \\
        (C)\ F(x)\ 在\ x=\pi\ 处连续但不可导 \\
        (D)\ F(x)\ 在\ x=\pi\ 处可导
    \end{align*}
    
    \begin{solution}
    【详解】
    \end{solution}
    
    \item 例14 (2016,数二)已知函数$f(x)$在$[0,3\pi]$上连续,在$(0,3\pi)$内是函数的一个原函数,且$f(0)=0$.
    \begin{enumerate}[label=(\roman*)]
        \item 求$f(x)$在区间$[0,\frac{3\pi}{2}]$上的平均值;
        \item 证明$f(x)$在区间$[0,\frac{3\pi}{2}]$内存在唯一零点.
    \end{enumerate}
    
    \begin{solution}
    【详解】
    \end{solution}
\end{enumerate}

\section{ 定积分应用求面积}

\begin{enumerate}[label=\arabic*.,start=14]
    \item 例15 (2019,数一、数二、数三)求曲线$y=e^{-x}\sin x(x\geq 0)$与$x$轴之间图形的面积.
    
    \begin{solution}
    【详解】
    \end{solution}
\end{enumerate}

\section{ 定积分应用求体积}

\begin{enumerate}[label=\arabic*.,start=15]
    \item 例16 (2003,数一)过原点作曲线$y=\ln x$的切线,该切线与曲线$y=\ln x$及$x$轴围成平面图形$D$.
    \begin{enumerate}[label=(\roman*)]
        \item 求$D$的面积$A$;
        \item 求$D$绕直线$x=e$旋转一周所得旋转体的体积$V$.
    \end{enumerate}
    
    \begin{solution}
    【详解】
    \end{solution}
    
    \item 例17 (2014,数二)已知函数$f(x, y)$满足$\frac{\partial f}{\partial y}=2(y+1)$,且$f(y, y)=(y+1)^2-(2-y)\ln y$,求曲线$f(x, y)=0$所围图形绕直线$y=-1$旋转所成旋转体的体积.
    
    \begin{solution}
    【详解】
    \end{solution}
\end{enumerate}

\section{ 定积分应用求弧长}

\begin{enumerate}[label=\arabic*.,start=17]
    \item 例18 求心形线$r=a(1+\cos\theta)(a>0)$的全长.
    
    \begin{solution}
    【详解】
    \end{solution}
\end{enumerate}

\section{ 定积分应用求侧面积}

\begin{enumerate}[label=\arabic*.,start=18]
    \item 例19 (2016,数二)设$D$是由曲线$y=\sqrt{1-x^2}(0\leq x\leq 1)$与$x=\cos^3 t$围成的平面区域,求$D$绕$x$轴旋转一周所得旋转体的体积和表面积.
    
    \begin{solution}
    【详解】
    \end{solution}
\end{enumerate}

\section{一 定积分物理应用}

\begin{enumerate}[label=\arabic*.,start=19]
    \item 例20 (2020,数二)设边长为$2a$等腰直角三角形平板铅直地沉没在水中,且斜边与水面相齐,设重力加速度为$g$,水密度为$\rho$,则该平板一侧所受的水压力为
    
    \begin{solution}
    【详解】
    \end{solution}
\end{enumerate}

\section{二 证明含有积分的等式或不等式}

\begin{enumerate}[label=\arabic*.,start=20]
    \item 例21 (2000,数二)设函数$S(x)=\int_0^x|\cos t| dt$.
    \begin{enumerate}[label=(\roman*)]
        \item 当$n$为正整数,且$n\pi\leq x<(n+1)\pi$时,证明$2n\leq S(x)<2(n+1)$;
        \item 求$\lim_{x\to+\infty}\frac{S(x)}{x}$
    \end{enumerate}
    
    \begin{solution}
    【详解】
    \end{solution}
    
    \item 例22 (2014,数二、数三)设函数$f(x), g(x)$在区间$[a, b]$上连续,且$f(x)$单调增加,$0\leq g(x)\leq 1$.
    证明:
    \begin{enumerate}[label=(\roman*)]
        \item $0\leq\int_a^x g(t) dt\leq x-a, x\in[a, b]$;
        \item $\int_a^{a+\int_a^b g(t) dt} f(x) dx\leq\int_a^b f(x) g(x) dx$.
    \end{enumerate}
    
    \begin{solution}
    【详解】
    \end{solution}
\end{enumerate}

\ifx\allfiles\undefined
\end{document}
\fi