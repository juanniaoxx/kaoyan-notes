\ifx\allfiles\undefined
\documentclass[12pt, a4paper, oneside, UTF8]{ctexbook}
\def\path{../config}
\usepackage{amsmath}
\usepackage{amsthm}
\usepackage{amssymb}
\usepackage{array}
\usepackage{xcolor}
\usepackage{graphicx}
\usepackage{mathrsfs}
\usepackage{enumitem}
\usepackage{geometry}
\usepackage[colorlinks, linkcolor=black]{hyperref}
\usepackage{stackengine}
\usepackage{yhmath}
\usepackage{extarrows}
\usepackage{tikz}
\usepackage{pgfplots}
\usepackage{asymptote}
\usepackage{float}
\usepackage{fontspec} % 使用字体

\setmainfont{Times New Roman}
\setCJKmainfont{LXGWWenKai-Light}[
    SlantedFont=*
]

\everymath{\displaystyle}

\usepgfplotslibrary{polar}
\usepackage{subcaption}
\usetikzlibrary{decorations.pathreplacing, positioning}

\usepgfplotslibrary{fillbetween}
\pgfplotsset{compat=1.18}
% \usepackage{unicode-math}
\usepackage{esint}
\usepackage[most]{tcolorbox}

\usepackage{fancyhdr}
\usepackage[dvipsnames, svgnames]{xcolor}
\usepackage{listings}

\definecolor{mygreen}{rgb}{0,0.6,0}
\definecolor{mygray}{rgb}{0.5,0.5,0.5}
\definecolor{mymauve}{rgb}{0.58,0,0.82}
\definecolor{NavyBlue}{RGB}{0,0,128}
\definecolor{Rhodamine}{RGB}{255,0,255}
\definecolor{PineGreen}{RGB}{0,128,0}

\graphicspath{ {figures/},{../figures/}, {config/}, {../config/} }

\linespread{1.6}

\geometry{
    top=25.4mm, 
    bottom=25.4mm, 
    left=20mm, 
    right=20mm, 
    headheight=2.17cm, 
    headsep=4mm, 
    footskip=12mm
}

\setenumerate[1]{itemsep=5pt,partopsep=0pt,parsep=\parskip,topsep=5pt}
\setitemize[1]{itemsep=5pt,partopsep=0pt,parsep=\parskip,topsep=5pt}
\setdescription{itemsep=5pt,partopsep=0pt,parsep=\parskip,topsep=5pt}

\lstset{
    language=Mathematica,
    basicstyle=\tt,
    breaklines=true,
    keywordstyle=\bfseries\color{NavyBlue}, 
    emphstyle=\bfseries\color{Rhodamine},
    commentstyle=\itshape\color{black!50!white}, 
    stringstyle=\bfseries\color{PineGreen!90!black},
    columns=flexible,
    numbers=left,
    numberstyle=\footnotesize,
    frame=tb,
    breakatwhitespace=false,
} 

\lstset{
    language=TeX, % 设置语言为 TeX
    basicstyle=\ttfamily, % 使用等宽字体
    breaklines=true, % 自动换行
    keywordstyle=\bfseries\color{NavyBlue}, % 关键字样式
    emphstyle=\bfseries\color{Rhodamine}, % 强调样式
    commentstyle=\itshape\color{black!50!white}, % 注释样式
    stringstyle=\bfseries\color{PineGreen!90!black}, % 字符串样式
    columns=flexible, % 列的灵活性
    numbers=left, % 行号在左侧
    numberstyle=\footnotesize, % 行号字体大小
    frame=tb, % 顶部和底部边框
    breakatwhitespace=false % 不在空白处断行
}

% \begin{lstlisting}[language=TeX] ... \end{lstlisting}

% 定理环境设置
\usepackage[strict]{changepage} 
\usepackage{framed}

\definecolor{greenshade}{rgb}{0.90,1,0.92}
\definecolor{redshade}{rgb}{1.00,0.88,0.88}
\definecolor{brownshade}{rgb}{0.99,0.95,0.9}
\definecolor{lilacshade}{rgb}{0.95,0.93,0.98}
\definecolor{orangeshade}{rgb}{1.00,0.88,0.82}
\definecolor{lightblueshade}{rgb}{0.8,0.92,1}
\definecolor{purple}{rgb}{0.81,0.85,1}

\theoremstyle{definition}
\newtheorem{myDefn}{\indent Definition}[section]
\newtheorem{myLemma}{\indent Lemma}[section]
\newtheorem{myThm}[myLemma]{\indent Theorem}
\newtheorem{myCorollary}[myLemma]{\indent Corollary}
\newtheorem{myCriterion}[myLemma]{\indent Criterion}
\newtheorem*{myRemark}{\indent Remark}
\newtheorem{myProposition}{\indent Proposition}[section]

\newenvironment{formal}[2][]{%
	\def\FrameCommand{%
		\hspace{1pt}%
		{\color{#1}\vrule width 2pt}%
		{\color{#2}\vrule width 4pt}%
		\colorbox{#2}%
	}%
	\MakeFramed{\advance\hsize-\width\FrameRestore}%
	\noindent\hspace{-4.55pt}%
	\begin{adjustwidth}{}{7pt}\vspace{2pt}\vspace{2pt}}{%
		\vspace{2pt}\end{adjustwidth}\endMakeFramed%
}

\newenvironment{definition}{\vspace{-\baselineskip * 2 / 3}%
	\begin{formal}[Green]{greenshade}\vspace{-\baselineskip * 4 / 5}\begin{myDefn}}
	{\end{myDefn}\end{formal}\vspace{-\baselineskip * 2 / 3}}

\newenvironment{theorem}{\vspace{-\baselineskip * 2 / 3}%
	\begin{formal}[LightSkyBlue]{lightblueshade}\vspace{-\baselineskip * 4 / 5}\begin{myThm}}%
	{\end{myThm}\end{formal}\vspace{-\baselineskip * 2 / 3}}

\newenvironment{lemma}{\vspace{-\baselineskip * 2 / 3}%
	\begin{formal}[Plum]{lilacshade}\vspace{-\baselineskip * 4 / 5}\begin{myLemma}}%
	{\end{myLemma}\end{formal}\vspace{-\baselineskip * 2 / 3}}

\newenvironment{corollary}{\vspace{-\baselineskip * 2 / 3}%
	\begin{formal}[BurlyWood]{brownshade}\vspace{-\baselineskip * 4 / 5}\begin{myCorollary}}%
	{\end{myCorollary}\end{formal}\vspace{-\baselineskip * 2 / 3}}

\newenvironment{criterion}{\vspace{-\baselineskip * 2 / 3}%
	\begin{formal}[DarkOrange]{orangeshade}\vspace{-\baselineskip * 4 / 5}\begin{myCriterion}}%
	{\end{myCriterion}\end{formal}\vspace{-\baselineskip * 2 / 3}}
	

\newenvironment{remark}{\vspace{-\baselineskip * 2 / 3}%
	\begin{formal}[LightCoral]{redshade}\vspace{-\baselineskip * 4 / 5}\begin{myRemark}}%
	{\end{myRemark}\end{formal}\vspace{-\baselineskip * 2 / 3}}

\newenvironment{proposition}{\vspace{-\baselineskip * 2 / 3}%
	\begin{formal}[RoyalPurple]{purple}\vspace{-\baselineskip * 4 / 5}\begin{myProposition}}%
	{\end{myProposition}\end{formal}\vspace{-\baselineskip * 2 / 3}}


\newtheorem{example}{\indent \color{SeaGreen}{Example}}[section]
\renewcommand{\proofname}{\indent\textbf{\textcolor{TealBlue}{Proof}}}
\NewEnviron{solution}{%
	\begin{proof}[\indent\textbf{\textcolor{TealBlue}{Solution}}]%
		\color{blue}% 设置内容为蓝色
		\BODY% 插入环境内容
		\color{black}% 恢复默认颜色(可选,避免影响后续文字)
	\end{proof}%
}

% 自定义命令的文件

\def\d{\mathrm{d}}
\def\R{\mathbb{R}}
%\newcommand{\bs}[1]{\boldsymbol{#1}}
%\newcommand{\ora}[1]{\overrightarrow{#1}}
\newcommand{\myspace}[1]{\par\vspace{#1\baselineskip}}
\newcommand{\xrowht}[2][0]{\addstackgap[.5\dimexpr#2\relax]{\vphantom{#1}}}
\newenvironment{mycases}[1][1]{\linespread{#1} \selectfont \begin{cases}}{\end{cases}}
\newenvironment{myvmatrix}[1][1]{\linespread{#1} \selectfont \begin{vmatrix}}{\end{vmatrix}}
\newcommand{\tabincell}[2]{\begin{tabular}{@{}#1@{}}#2\end{tabular}}
\newcommand{\pll}{\kern 0.56em/\kern -0.8em /\kern 0.56em}
\newcommand{\dive}[1][F]{\mathrm{div}\;\boldsymbol{#1}}
\newcommand{\rotn}[1][A]{\mathrm{rot}\;\boldsymbol{#1}}

\newif\ifshowanswers
\showanswerstrue % 注释掉这行就不显示答案

% 定义答案环境
\newcommand{\answer}[1]{%
    \ifshowanswers
        #1%
    \fi
}

% 修改参数改变封面样式,0 默认原始封面、内置其他1、2、3种封面样式
\def\myIndex{0}


\ifnum\myIndex>0
    \input{\path/cover_package_\myIndex} 
\fi

\def\myTitle{考研数学笔记}
\def\myAuthor{Weary Bird}
\def\myDateCover{\today}
\def\myDateForeword{\today}
\def\myForeword{相见欢·林花谢了春红}
\def\myForewordText{
    林花谢了春红,太匆匆。
    无奈朝来寒雨晚来风。
    胭脂泪,相留醉,几时重。
    自是人生长恨水长东。
}
\def\mySubheading{以姜晓千强化课讲义为底本}


\begin{document}
% \input{\path/cover_text_\myIndex.tex}

\newpage
\thispagestyle{empty}
\begin{center}
    \Huge\textbf{\myForeword}
\end{center}
\myForewordText
\begin{flushright}
    \begin{tabular}{c}
        \myDateForeword
    \end{tabular}
\end{flushright}

\newpage
\pagestyle{plain}
\setcounter{page}{1}
\pagenumbering{Roman}
\tableofcontents

\newpage
\pagenumbering{arabic}
% \setcounter{chapter}{-1}
\setcounter{page}{1}

\pagestyle{fancy}
\fancyfoot[C]{\thepage}
\renewcommand{\headrulewidth}{0.4pt}
\renewcommand{\footrulewidth}{0pt}








\else
\fi
\chapter{高等数学}

\section{极限与连续}
\begin{enumerate}
    \item $\star$ 设函数$f(x)=\cos{(\sin{x})},g(x)=\sin{(\cos{x})}$当$\displaystyle x\in\left(0,\frac{\pi}{2}\right)$时(  ) \\
    A.$f(x)$单调递增,$g(x)$单调递减\qquad B.$f(x)$单调递减,$g(x)$单调递增 \\
    C.$f(x),g(x)$均单调递减\qquad\qquad\quad D.$f(x),g(x)$均单调递增

    \item $\star\star$讨论函数$f(x)=\displaystyle \lim_{x\to\infty}\frac{x^{n+2}-x^{-n}}{x^n+x^{-n}}$的连续性

    \item $\star\star$设$f(x)$在$\left[a,b\right]$上连续,且$a<c<d<b$证明:在$\left(a,b\right)$内必定存在一点$\xi$使得
    $mf(c)+nf(d)=(m+n)f(\xi)$,其中$m,n$为任意给定的自然数 

    \item $\star\star$设$x_1=\sqrt{a}(a>0),x_{n+1}=\sqrt{a+x_n}$证明$\lim_{n\to\infty}x_n$存在,并求出其值. 
    
    \item $\star\star\star$设$x_1=a\geq 0,y_1=b\geq 0, a\leq b, x_{n+1}=\sqrt{x_ny_n},\displaystyle y_{n+1}=\frac{x_n+y_n}{2}(n=1,2,\ldots)$证明
    $\displaystyle \lim_{n\to\infty}x_n=\lim_{n\to\infty}y_n$

    \item $\star\star$设$\left\{x_n\right\}$为数列,则下列数据结论正确的是(  ) 
    \begin{enumerate}
        \item [\ding{172}] 若$\{\arctan{x_n}\}$收敛,则$\{x_n\}$收敛 
        \item [\ding{173}] 若$\{\arctan{x_n}\}$单调,则$\{x_n\}$收敛 
        \item [\ding{174}] 若$x_n\in\left[-1,1\right]$,且$\{x_n\}$收敛,则$\{\arctan{x_n}\}$收敛
        \item [\ding{175}] 若$x_n\in\left[-1,1\right]$,且$\{x_n\}$单调,则$\{\arctan{x_n}\}$收敛
    \end{enumerate}
    A.\ding{172}\ding{173}\qquad B.\ding{174}\ding{175}\qquad C.\ding{172}\ding{174}\qquad D.\ding{173}\ding{175}

    \item $\star$极限$\displaystyle \lim_{x\to 0}\frac{(\cos{x}-e^{x^2})\sin x^2}{\frac{x^2}{2}+1-\sqrt{1+x^2}}$=\_\_\_\_ 
    
    \item $\star$设$\displaystyle a_n=\frac{3}{2}\int_{0}^{\frac{n}{n+1}}x^{n-1}\sqrt{1+x^n}\d x,$则$\displaystyle \lim_{n\to\infty}na_n$=\_\_\_\_
    
    \item $\star\star$设$\displaystyle \lim_{x\to 0}\left\{a\left[x\right]+\frac{\ln\left(1+e^{\frac{2}{x}}\right)}{\ln\left(1+e^{\frac{1}{x}}\right)}\right\}=b$则$a$=\_\_\_,$b$=\_\_\_

    \item $\star$ 设$x_1=1,x_2=2,x_{n+2}=\displaystyle \frac{1}{2}(x_n+x_{n+1})$,求$\displaystyle\lim_{n\to\infty}x_n$ 

    \item $\star\star\star$设$f(x)$在$\left[0,1\right]$上连续,且$f(0)=f(1)$证明 
    \begin{enumerate}
        \item [(I)] 至少存在一点$\xi\in\left(0,1\right)$使得$f(\xi)=f(\xi+\displaystyle \frac{1}{2})$ 
        \item [(II)] 至少存在一点$\xi\in\left(0,1\right)$使得$f(\xi)=f(\xi+\displaystyle \frac{1}{n})(n\geq 2, n\in\mathbb{N})$
    \end{enumerate}

    \item $\star\star\star\star$(2011.数一) 
    \begin{enumerate}
        \item[(I)] 证明$\displaystyle \frac{1}{n+1}<\ln{(1+\frac{1}{n})}<\frac{1}{n}$
        \item[(II)] 证明极限$\displaystyle \lim_{n\to\infty}\left(1+\frac{1}{2}+\ldots+\frac{1}{n}-\ln{n}\right)$存在
    \end{enumerate}
\end{enumerate}

\section{一元函数微分学/积分学(除证明题)/多元函数微分学}
\begin{enumerate}
    \item $\star$设$f'_x(x_0,y_0),f'_y(x_0,y_0)$均存在,则下列结论正确的是(  ) \\
    A.$ \displaystyle \lim_{\substack{x\to x_0\\ y\to y_0}}f(x,y)$ 存在 \qquad
    B.$f(x,y)$在$(x_0,y_0)$处连续 \\
    C.$\lim_{x\to x_0}f(x,y_0)$存在 \qquad
    D.$f(x,y)$在去心邻域$(x_0,y_0)$内有定义 

    \item $\star$设$z=(1+xy)^y,$则$\d z\big|_{1,1}=\_\_\_$ 
    
    \item $\star\star$ 设$\begin{cases}
        y = f(x, t) \\
        F(x,y,t) = 0
    \end{cases}f,F$有一阶连续偏导数,则$\displaystyle\frac{\d y}{\d x}=\_\_\_\_$ 

    \item $\star\star$设$y=f(x,t),t=t(x,t)$由方程$G(x,y,t)=0$确定,$f,G$可微,则$\displaystyle\frac{\d y}{\d x}=\_\_\_\_$

    \item $\star$设$z=z(x,y)$有方程$e^{2yz}+x+y^2+z=\frac{7}{4}$确定,则$\d z\big|_{\frac{1}{2},\frac{1}{2}}=\_\_\_\_$

    \item $\star$曲面$z=x^2+y^2-1$在点$P(2,1,4)$处的且平面方程为$\_\_\_$法线方程$\_\_\_$ 

    \item $\star$求$f(x,y)=(1+e^y)\cos{x}-ye^y$的极值 
    
    \item $\star\star$求双曲线$xy=4$与直线$2x+y=1$之间的最短距离
\end{enumerate}
\section{空间解析几何/多元函数积分学}
\begin{enumerate}
    \item $\star$设向量$\vec{a}=(1,2,1),\vec{b}=(-1,0,2),\vec{c}=(0,k,-3)$共面,则$k=\_\_\_$ 
    
    \item $\star\star$设非零向量$\vec{\alpha},\vec{\beta}$满足$\vec{\alpha}-\vec{\beta}$于$\vec{\alpha}+\vec{\beta}$的模相等,则必有(  ) \\
    A.$\vec{\alpha}-\vec{\beta}=\vec{\alpha}+\vec{\beta}$ \qquad B.$\vec{\alpha}=\vec{\beta}$ \qquad
    C.$\vec{\alpha}\times\vec{\beta}=\vec{0}$\qquad D.$\vec{\alpha}\cdot\vec{\beta}=0$

    \item $\star\star$直线$L_1:\begin{cases}
        x - 1 = 0 \\
        y = z
    \end{cases}$与$L_2:\begin{cases}
        x+2y = 0 \\
        z + 2 = 0
    \end{cases}$的距离$d=\_\_\_$

    \item $\star\star$设$\alpha,\beta$均为单位向量,其夹角为$\displaystyle \frac{\pi}{6}$则$\alpha+2\beta$与$3\alpha+\beta$为邻边的
    平行四边形的面积为\_\_\_ 

    \item $\star\star$设$\alpha,\beta$是非零常向量,夹角为$\displaystyle \frac{\pi}{3}$, 且$\left|\beta\right|=2$
    求$\displaystyle \lim_{x\to 0}\frac{\left|\alpha+x\beta\right|-\left|\alpha\right|}{x}=\_\_\_$

    \item $\star$求平行于平面$x+y+z=9$且与球面$x^2+y^2+z^2=4$相切的平面方程. 
    
    \item $\star$设平面$\pi$过直线$L:\begin{cases}
        x + 5y + z = 0 \\
        x - z + 4 = 0
    \end{cases}$且与平面$\pi_1:x-4y-8z+12=0$的夹角为$\displaystyle\frac{\pi}{4}$求平面$\pi$的方程

    \item $\star$求与直线$L_1:x+2=3-y=z+1$与$L_2:\displaystyle \frac{x+4}{2}=y=\frac{z-4}{3}$都垂直相交的直线方程
    
    \item $\star$求直线$L_1:\displaystyle\frac{x-3}{2}=y=\frac{z-1}{0}$与$L_2:\displaystyle \frac{x+1}{1}=\frac{y-2}{0}=z$的公垂线方程
    
    \item $\star\star$求直线$L:\displaystyle \frac{x-1}{3}=\frac{y-2}{4}=\frac{z+1}{1}$绕直线$\begin{cases}
        x = 2\\
        y = 3
    \end{cases}$旋转一周所得到的曲面方程


\end{enumerate}
\section{常微分方程}

\section{无穷级数}


\section{证明题}
\ifx\allfiles\undefined
\end{document}
\fi