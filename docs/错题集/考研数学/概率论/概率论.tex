\ifx\allfiles\undefined
\documentclass[12pt, a4paper, oneside, UTF8]{ctexbook}
\def\path{../config}
\usepackage{amsmath}
\usepackage{amsthm}
\usepackage{amssymb}
\usepackage{array}
\usepackage{xcolor}
\usepackage{graphicx}
\usepackage{mathrsfs}
\usepackage{enumitem}
\usepackage{geometry}
\usepackage[colorlinks, linkcolor=black]{hyperref}
\usepackage{stackengine}
\usepackage{yhmath}
\usepackage{extarrows}
\usepackage{tikz}
\usepackage{pgfplots}
\usepackage{asymptote}
\usepackage{float}
\usepackage{fontspec} % 使用字体

\setmainfont{Times New Roman}
\setCJKmainfont{LXGWWenKai-Light}[
    SlantedFont=*
]

\everymath{\displaystyle}

\usepgfplotslibrary{polar}
\usepackage{subcaption}
\usetikzlibrary{decorations.pathreplacing, positioning}

\usepgfplotslibrary{fillbetween}
\pgfplotsset{compat=1.18}
% \usepackage{unicode-math}
\usepackage{esint}
\usepackage[most]{tcolorbox}

\usepackage{fancyhdr}
\usepackage[dvipsnames, svgnames]{xcolor}
\usepackage{listings}

\definecolor{mygreen}{rgb}{0,0.6,0}
\definecolor{mygray}{rgb}{0.5,0.5,0.5}
\definecolor{mymauve}{rgb}{0.58,0,0.82}
\definecolor{NavyBlue}{RGB}{0,0,128}
\definecolor{Rhodamine}{RGB}{255,0,255}
\definecolor{PineGreen}{RGB}{0,128,0}

\graphicspath{ {figures/},{../figures/}, {config/}, {../config/} }

\linespread{1.6}

\geometry{
    top=25.4mm, 
    bottom=25.4mm, 
    left=20mm, 
    right=20mm, 
    headheight=2.17cm, 
    headsep=4mm, 
    footskip=12mm
}

\setenumerate[1]{itemsep=5pt,partopsep=0pt,parsep=\parskip,topsep=5pt}
\setitemize[1]{itemsep=5pt,partopsep=0pt,parsep=\parskip,topsep=5pt}
\setdescription{itemsep=5pt,partopsep=0pt,parsep=\parskip,topsep=5pt}

\lstset{
    language=Mathematica,
    basicstyle=\tt,
    breaklines=true,
    keywordstyle=\bfseries\color{NavyBlue}, 
    emphstyle=\bfseries\color{Rhodamine},
    commentstyle=\itshape\color{black!50!white}, 
    stringstyle=\bfseries\color{PineGreen!90!black},
    columns=flexible,
    numbers=left,
    numberstyle=\footnotesize,
    frame=tb,
    breakatwhitespace=false,
} 

\lstset{
    language=TeX, % 设置语言为 TeX
    basicstyle=\ttfamily, % 使用等宽字体
    breaklines=true, % 自动换行
    keywordstyle=\bfseries\color{NavyBlue}, % 关键字样式
    emphstyle=\bfseries\color{Rhodamine}, % 强调样式
    commentstyle=\itshape\color{black!50!white}, % 注释样式
    stringstyle=\bfseries\color{PineGreen!90!black}, % 字符串样式
    columns=flexible, % 列的灵活性
    numbers=left, % 行号在左侧
    numberstyle=\footnotesize, % 行号字体大小
    frame=tb, % 顶部和底部边框
    breakatwhitespace=false % 不在空白处断行
}

% \begin{lstlisting}[language=TeX] ... \end{lstlisting}

% 定理环境设置
\usepackage[strict]{changepage} 
\usepackage{framed}

\definecolor{greenshade}{rgb}{0.90,1,0.92}
\definecolor{redshade}{rgb}{1.00,0.88,0.88}
\definecolor{brownshade}{rgb}{0.99,0.95,0.9}
\definecolor{lilacshade}{rgb}{0.95,0.93,0.98}
\definecolor{orangeshade}{rgb}{1.00,0.88,0.82}
\definecolor{lightblueshade}{rgb}{0.8,0.92,1}
\definecolor{purple}{rgb}{0.81,0.85,1}

\theoremstyle{definition}
\newtheorem{myDefn}{\indent Definition}[section]
\newtheorem{myLemma}{\indent Lemma}[section]
\newtheorem{myThm}[myLemma]{\indent Theorem}
\newtheorem{myCorollary}[myLemma]{\indent Corollary}
\newtheorem{myCriterion}[myLemma]{\indent Criterion}
\newtheorem*{myRemark}{\indent Remark}
\newtheorem{myProposition}{\indent Proposition}[section]

\newenvironment{formal}[2][]{%
	\def\FrameCommand{%
		\hspace{1pt}%
		{\color{#1}\vrule width 2pt}%
		{\color{#2}\vrule width 4pt}%
		\colorbox{#2}%
	}%
	\MakeFramed{\advance\hsize-\width\FrameRestore}%
	\noindent\hspace{-4.55pt}%
	\begin{adjustwidth}{}{7pt}\vspace{2pt}\vspace{2pt}}{%
		\vspace{2pt}\end{adjustwidth}\endMakeFramed%
}

\newenvironment{definition}{\vspace{-\baselineskip * 2 / 3}%
	\begin{formal}[Green]{greenshade}\vspace{-\baselineskip * 4 / 5}\begin{myDefn}}
	{\end{myDefn}\end{formal}\vspace{-\baselineskip * 2 / 3}}

\newenvironment{theorem}{\vspace{-\baselineskip * 2 / 3}%
	\begin{formal}[LightSkyBlue]{lightblueshade}\vspace{-\baselineskip * 4 / 5}\begin{myThm}}%
	{\end{myThm}\end{formal}\vspace{-\baselineskip * 2 / 3}}

\newenvironment{lemma}{\vspace{-\baselineskip * 2 / 3}%
	\begin{formal}[Plum]{lilacshade}\vspace{-\baselineskip * 4 / 5}\begin{myLemma}}%
	{\end{myLemma}\end{formal}\vspace{-\baselineskip * 2 / 3}}

\newenvironment{corollary}{\vspace{-\baselineskip * 2 / 3}%
	\begin{formal}[BurlyWood]{brownshade}\vspace{-\baselineskip * 4 / 5}\begin{myCorollary}}%
	{\end{myCorollary}\end{formal}\vspace{-\baselineskip * 2 / 3}}

\newenvironment{criterion}{\vspace{-\baselineskip * 2 / 3}%
	\begin{formal}[DarkOrange]{orangeshade}\vspace{-\baselineskip * 4 / 5}\begin{myCriterion}}%
	{\end{myCriterion}\end{formal}\vspace{-\baselineskip * 2 / 3}}
	

\newenvironment{remark}{\vspace{-\baselineskip * 2 / 3}%
	\begin{formal}[LightCoral]{redshade}\vspace{-\baselineskip * 4 / 5}\begin{myRemark}}%
	{\end{myRemark}\end{formal}\vspace{-\baselineskip * 2 / 3}}

\newenvironment{proposition}{\vspace{-\baselineskip * 2 / 3}%
	\begin{formal}[RoyalPurple]{purple}\vspace{-\baselineskip * 4 / 5}\begin{myProposition}}%
	{\end{myProposition}\end{formal}\vspace{-\baselineskip * 2 / 3}}


\newtheorem{example}{\indent \color{SeaGreen}{Example}}[section]
\renewcommand{\proofname}{\indent\textbf{\textcolor{TealBlue}{Proof}}}
\NewEnviron{solution}{%
	\begin{proof}[\indent\textbf{\textcolor{TealBlue}{Solution}}]%
		\color{blue}% 设置内容为蓝色
		\BODY% 插入环境内容
		\color{black}% 恢复默认颜色(可选,避免影响后续文字)
	\end{proof}%
}

% 自定义命令的文件

\def\d{\mathrm{d}}
\def\R{\mathbb{R}}
%\newcommand{\bs}[1]{\boldsymbol{#1}}
%\newcommand{\ora}[1]{\overrightarrow{#1}}
\newcommand{\myspace}[1]{\par\vspace{#1\baselineskip}}
\newcommand{\xrowht}[2][0]{\addstackgap[.5\dimexpr#2\relax]{\vphantom{#1}}}
\newenvironment{mycases}[1][1]{\linespread{#1} \selectfont \begin{cases}}{\end{cases}}
\newenvironment{myvmatrix}[1][1]{\linespread{#1} \selectfont \begin{vmatrix}}{\end{vmatrix}}
\newcommand{\tabincell}[2]{\begin{tabular}{@{}#1@{}}#2\end{tabular}}
\newcommand{\pll}{\kern 0.56em/\kern -0.8em /\kern 0.56em}
\newcommand{\dive}[1][F]{\mathrm{div}\;\boldsymbol{#1}}
\newcommand{\rotn}[1][A]{\mathrm{rot}\;\boldsymbol{#1}}

\newif\ifshowanswers
\showanswerstrue % 注释掉这行就不显示答案

% 定义答案环境
\newcommand{\answer}[1]{%
    \ifshowanswers
        #1%
    \fi
}

% 修改参数改变封面样式,0 默认原始封面、内置其他1、2、3种封面样式
\def\myIndex{0}


\ifnum\myIndex>0
    \input{\path/cover_package_\myIndex} 
\fi

\def\myTitle{考研数学笔记}
\def\myAuthor{Weary Bird}
\def\myDateCover{\today}
\def\myDateForeword{\today}
\def\myForeword{相见欢·林花谢了春红}
\def\myForewordText{
    林花谢了春红,太匆匆。
    无奈朝来寒雨晚来风。
    胭脂泪,相留醉,几时重。
    自是人生长恨水长东。
}
\def\mySubheading{以姜晓千强化课讲义为底本}


\begin{document}
\else
\fi
\chapter{概率论}

\section{事件与概率,随机变量及其分布}
\begin{enumerate}
    \item 每箱产品有 10 件,其中次品数从 0 到 2 是等可能的,开箱检验时,从中任取一件,
    如果检验为次品,则认为该箱产品不合格而拒收. 由于检验误差,一件正品被误判为次品的概率为 2\%,
    一件次品被误判为正品的概率为 10\%. 试求:
        \item[(I)]随机检验一箱产品,它能通过验收的概率$p$ ;
        \item[(II)]检验10箱产品通过率不低于90\%的概率$q$ .
    \answer{
    \begin{solution}
        (1)设$B=\{\text{任取一件为正品}\}$,$A=\{\text{一箱产品能通过验收}\}$
        则由全概率公式有$$P(A)=P(A\mid B)P(B)+P(A\mid\bar{B})P(\bar{B})$$
        而其中$$P(A\mid B)= 1 - 0.02 = 0.98,P(A\mid\bar{B})=1-0.1=0.9$$代入全概率公式有
        $p=P(A)=1+0.88P(B)$,为求$P(B)$,记$C_i$为每箱中包含$i$件次品,且$C_0,C_1,C_2$为完备事件组,再
        由全概率公式可以求出$$P(B)=\sum_{i=0}^{2}P(C_i)P(B\mid C_i)=0.9$$故$P(A)=0.892$ \\
        (2)$q=P\{X/10\geq 0.9\}=P\{X\geq 9\}=P\{X=9\}+P\{X=10\}\approx 0.705$
    \end{solution}
    }

    \item 一条自动生产线生产 $n$ 件产品不出故障的概率为 
    $\frac{{\lambda }^{n}}{n!}{\mathrm{e}}^{-\lambda },n = 0,1,2,\cdots$ . 
    假设产品的优质品率为 $p\left( {0 < p < 1}\right)$ . 如果各件产品是否为优质品相互独立.
        \item[(I)]计算生产线在两次故障间共生产 $k$ 件 $\left( {k = 0,1,2,\cdots }\right)$ 优质品的概率;
        \item[(II)]若已知在某两次故障间该生产线生产了 $k$ 件优质品,求它共生产 $m$ 件产品的概率.
    
    \answer{
    \begin{solution}
        (1)不妨令\\
        $B_k=\{\text{两次故障公生产了k件优质品}\}$,$A_n=\{\text{两次故障间总共生产了n件产品}\}$,显然$A_0,A_1,\ldots$构成了
        一个完备事件组,故利用全概率公式有
        \begin{align*}
        P(B_k) &=\sum_{n=0}^{\infty}P(A_n)P(B_k\mid A_n) \\
        &\xlongequal{\text{前k-1次不可能产生k件优质品}}\sum_{n=k}^{\infty}P(A_n)P(B_k\mid A_n)\\
        &=\frac{(\lambda p)^k}{k!}e^{-\lambda p}\sum_{n=k}^{\infty}\frac{(\lambda p)^{n-k}}{(n-k)!}e^{-\lambda p} \\
        &\xlongequal{\text{Possion分布}}\frac{(\lambda p)^k}{k!}e^{-\lambda p}
        \end{align*}
        (2)当$m<k$的时候,$P(A_m\mid B_k)=0$,当$m\geq k$,
        \begin{align*}
            P(A_m\mid B_k) &=\frac{P(A_m)P(B_k\mid A_m)}{P(B_k)} \\
            &= \frac{(\lambda q)^{m-k}}{(m-k)!}e^{-\lambda q}, m\in (k,k+1,\ldots)
        \end{align*}
    \end{solution}
    \begin{tcolorbox}[title=总结]
        关于全概率公式与贝叶斯公式的总结 \\
        这种问题的关键在于寻找一个合适的完备事件组,当问题涉及"原因推结果/结果推原因"大概率要用贝叶斯公式(条件概率是贝叶斯的特殊情况)
    \end{tcolorbox}
    }

    \item 甲、乙二人轮流投篮,游戏规则规定为甲先开始,且甲每轮只投一次,而乙每轮连续投两次,先投中者为胜. 设甲、乙每次投篮的命中率分别是 $p$ 与 0.5,
    则 $p =$\_\_\_时,甲、乙胜负概率相同.
    
    \answer{
    \begin{solution}
        这道题和笔记中的交替射击模型一致,记$A=\{\text{甲获胜}\},B=\{\text{乙获胜}\}$,则由题意有 
        \[
        P(A)=p+(1-p)(1-0.5)(1-0.5)P(A)\implies P(A)=\frac{p}{1-0.25(1-p)}
        \]
        再由题意可知,要使得甲乙获胜概率一致,则$P(A)=P(B)=0.5\implies p = \frac{3}{7}$
    \end{solution}
    }
    
    \item (非离散非连续的概率)设随机变量X的绝对值不大于1,且$P\{X=0\}=\frac{1}{4}$,已知当$X\neq 0$的时候,X在其他取值范围内
    满足均匀分布,求X的分布函数$F_{X}(x)$.
    \answer{
    \begin{solution}
        由题意有$P\{|X|\leq 1\}=1,P\{X=0\}=\frac{1}{4}\implies P\{X\neq = 0\}=\frac{3}{4}$,又因为区间长度为$2$,有
        \[F_{X}(x) =
        \begin{cases}
            0, & x < -1 \\
            \frac{3}{8}(x+1), & -1 <= x < 0 \\
            \frac{3x+5}{8}, & 0\leq x < 1 \\
            1, & 0 \geq 1
        \end{cases}
        \]
    \end{solution}
    }
    \item 设有四个编号分别为$1,2,3,4$的盒子和三只球,现将每个球随机地放入四个盒子,记$X$为至少有一只球的盒子的最小号码.
    \begin{enumerate}
        \item [(1)] 求X的分布律; 
        \item [(2)] 若当$X=k$的时候,随机变量在$\left[0,k\right]$上服从均匀分布,求$P\{Y\leq 2\}$;
    \end{enumerate}
    \answer{
    \begin{solution}
        \item[(1)] 由题有$P\{X=1\}=\frac{C_{3}^{1}3^2+C_{3}^{2}3+C_{3}^{3}}{4^3}=\frac{37}{64}$
        解释:总共有$4^3$种方案,若1是最小的有球的盒子,则其中可以有$1,2,3$三种可能,$C_{3}^{1}3^2$表示选择一个球加入1号盒子,其余两个球可以从剩余
        3个盒子中随机选择两个放入.
        同理可以求出X=2,3,4,故有
        \[
        X\sim\left(\begin{array}{cccc}
            1 & 2 & 3 & 4\\
            \frac{37}{64} & \frac{19}{64} & \frac{7}{64} & \frac{1}{61}
        \end{array}
        \right)
        \]
        \item[(2)] 由全概率公式$P\{Y\leq 2\}=\sum_{i=1}^{4}P\{Y\leq 2\mid X=k\}=\frac{367}{384}$
    \end{solution}
    }
    \item 有一根长为$L$的木棒,将其任意折成三段,记事件$A=\{\text{中间一段为三段中的最长者}\}$,则$P(A)=\_\_\_\_$
    \answer{
    \begin{solution}
    
    \end{solution}
    }

    \item 设甲乙两人独立对同一目标进行一次设计,其命中率分别为0.5和0.4,已知目标被命中,则它是乙射中的概率为\_\_\_\_
    
    \answer{
    \begin{solution}
        
    \end{solution}
    }

    \item 已知$10$部手机中有$7$个合格品和$3$个次品,每次任取一个作测试,测试后不放回,直到将$3$个次品都找到为止
    ,则需要测试$7$次的概率为\_\_\_\_
    
    \answer{
    \begin{solution}
        
    \end{solution}
    }

    \item 在 \( n \) 重伯努利试验中,事件 \( A \) 发生的概率为 \( p \) ,则事件 \( A \) 发生奇数次的概率为\_\_\_\_ 
    \answer{
    \begin{solution}
        \item [(方法一)]
        首先考虑第n次试验,A发生奇数次的情况有两种.(1)前n-1次成功率偶数次,第n次成功;(2)前n-1次成功了奇数次,第n次失败了.
        则不发令$A_k=\{\text{第k次试验成功}\},P(A_k)=p;B_k=\{\text{k次实验中成功奇数次}\}$,记$P(B_k)=p_k$,则有
        \[
        B_n=B_{n-1}\bar{A_n}+\overline{B_{n-1}}A_n
        \]
        显然$B_{n-1}\bar{A_n}$与$\overline{B_{n-1}}A_n$互斥,则有
        \[
        p_n=P(B_{n-1}\bar{A_n}+\overline{B_{n-1}}A_n)=P(B_{n-1}\bar{A_n})+P(\overline{B_{n-1}}A_n)
        \]
        又由于伯努利试验的独立性,有
        \begin{align*}
            \text{上式}&=P(B_{n-1})P(\bar{A})+P(\overline{B_{n-1}})P(A_n) \\
            &=(1-p)p_{n-1}+p(1-p_{n-1}) \\
            &=p+(1-2p)p_{n-1}
        \end{align*}
        有递推关系式,可以得到
        \[
        p_n-\frac{1}{2}=(1-2p)[p_{n-1}-\frac{1}{2}]\xlongequal{\text{等比数列}}{-\frac{(1-2p)^n}{2}}
        \]

        \item [(方法二) 利用奇偶]
        设$X\sim B(n,p)$,则$P(X=k)=C_{n}^{k}p^k(1-p)^(n-k),k=0,1,2,\ldots$\\
        若n为偶数则
        \begin{align*}
            P(X=odd) &=P(X=1)+P(X=3)+\ldots+P(X=n-1) \\
            &=C_{n}^{1}(1-p)^{n-1}+C_{n}^{3}p^3(1-p)^{n-3}+\ldots+C_{n}^{n-1}p^{n-1}(1-p) \\
            P(X=even) &=P(X=0)+P(X=2)+\ldots+P(X=n) \\
            &=C_{n}^{0}p^0(1-p)^{n}+\ldots+C_{n}^{n}p^n(1-p)^0
        \end{align*}
        且$P(X=odd)+P(X=even)=1$,有注意到\\
        \begin{align*}
            P(X=odd) &=P(X=1)+P(X=3)+\ldots+P(X=n-1) \\
            &=-C_{n}^{1}(p-1)^{n-1}-C_{n}^{3}p^3(p-1)^{n-3}-\ldots-C_{n}^{n-1}p^{n-1}(p-1) \\
            P(X=even) &=P(X=0)+P(X=2)+\ldots+P(X=n) \\
            &=C_{n}^{0}p^0(p-1)^{n}+\ldots+C_{n}^{n}p^n(p-1)^0
        \end{align*}
        则
        \begin{align*}
        P(X=even)-P(X=odd)
        &=C_{n}^{0}p^0(p-1)^{n}+C_{n}^{1}p^1(p-1)^{n-1}+\ldots+C_{n}^{n}p^n(p-1)^0 \\
        &\xlongequal{\text{二项式定理}}{}(2p-1)^n 
        \end{align*}
        则$2P(X=odd)=1-(2p-1)^n\implies P(X=odd)=\frac{1-(2p-1)^n}{2}$\\
        同理当n为奇数的时候,上述也成立,故$P(X=\text{奇数})=\frac{1-(2p-1)^n}{2}$
        \item [(方法三)]
        设 $ X \sim B(n, p) $,则 $ P(X = k) = C_{n}^{k} p^k (1 - p)^{n - k} $,$ k = 0, 1, 2, \ldots $\\
        令 $ Y = \frac{1}{2} [1 + (-1)^X] $,当 $ X $ 为奇数时,$ Y = 0 $;当 $ X $ 为偶数时,$ Y = 1 $
        \\于是原问题转换为求 $ P(X \text{为奇数}) = P(Y = 0) $
        注意到 $ E[Y] = 0 \cdot P(Y = 0) + 1 \cdot P(Y = 1) = P(Y = 1) = 1 - P(Y = 0) $,故只需要求 $ E[Y] $
        \begin{align*}
            EY&=E(\frac{1}{2} [1 + (-1)^X])=\frac{1}{2}+E(-1)^X \\
            &=\frac{1}{2}+\frac{1}{2}\sum_{k=0}^{n}(-1)^kC_n^kp^k(1-p)^{n-k} \\
            &\xlongequal{\text{\text{逆用二项式定理}}}{}=\frac{1}{2}+\frac{1}{2}(1-2p)^n
        \end{align*}
        故$P(Y=0)=1-P(Y=1)=\frac{1-(1-2p)^n}{2}$
    \end{solution}
    }


    \item 设甲盒中有 4 个红球和 2 个白球,乙盒中有 2 个红球和 4 个白球,掷一枚均匀的硬币,若正面出现,
    则从甲盒中任取一球,若反面出现,则从乙盒中任取一球,设每次取出的球观看颜色后放回原盒中.
    \begin{enumerate}
        \item [(I)]若前两次都取得红球,求第三次也取得红球的概率;
        \item [(II)]若前两次都取得红球,求红球都来自甲盒的概率.
    \end{enumerate}

    \answer{
    \begin{solution}
        设$A_i=\{\text{第i次取得红球}\}(i=1,2,3),B_i=\{\text{第j次投掷银币出现正面}\}(j=1,2,3)$ 
        \begin{enumerate}
            \item [(1)] 显然$A_i$与$B_j$之间是相互独立的,所求概率为
            \begin{align*}
                &P(A_3\mid A_1A_2)=\frac{P(A_1A_2A_3)}{P(A_1A_2)}=P(A_3)=P(A_1) \\
                &P(A_1)\xlongequal{\text{全概率公式}}{}P(A\mid B_1)P(B_1)+P(A\mid \bar{B_1})P(\bar{B_1}) \\
                &=\frac{1}{2}\frac{4}{6}+\frac{1}{2}\frac{2}{6}=\frac{1}{2} 
            \end{align*}
            \item [(2)] 由于两次试验都是独立重复的所以$A_1B_1$与$A_2B_2$是相互独立的\\
            则$P(A_1B_1)=P(A_2B_2)=P(B_1)P(A_1\mid P(B_1))=\frac{1}{3}$ \\
            则所求概率为
            \[
            P(B_1B_2\mid A_1A_2)=\frac{P(B_1B_2A_1A_2)}{P(A_1A_2)}=\frac{(\frac{1}{3})^2}{(\frac{1}{2})^2}=\frac{4}{9}
            \]
        \end{enumerate}
    \end{solution}
    }
    \item (考的可能性比较低)设一批产品中有$15\%$的次品,进行独立重复抽样检验,若抽取 20 个样品,则抽出的20个样品中,
    可能性最大的次品数是多少?并求其概率.

    \answer{
        \begin{solution}
        设20次抽取其中出现次品的次数为X,其显然满足$X\sim B(20,0.15)$,不妨假设当$X=k$的时候物品的可能性最大,则有
        $P(X=k)\geq P(X=k-1),P(X=k)\geq P(X=k+1)$即
        \[
        \frac{C_{20}^{k}0.15^{k}0.85^{(20-k)}}{C_{20}^{(k-1)}0.15^{k-1}0.85^{(21-k)}} \geq 1
        \]
        与
        \[
        \frac{C_{20}^{k}0.15^{k}0.85^{(20-k)}}{C_{20}^{(k+1)}0.15^{k+1}0.85^{(19-k)}} \geq 1
        \]
        得到如下结果
        \[
        \begin{cases}
            300-15k+15\geq 85k \\
            85k+85\geq 300-15k
        \end{cases}
        \]
        即$2.15\leq k \leq 3.15$ 故$k=3$,其概率为$P(X=3)=C_{20}^{3}0.15^{3}0.85^{17}$
        \end{solution}
    }
    
    \item 设自动机床在任意时长为$t$的时间间隔内发生故障的此时为$X$服从参数为$\lambda_t$的泊松分布,$Y$表示相继两次
    故障之间的时间间隔,则当$t>0$时,$P\{Y>t\}=\_\_\_\_$
    \answer{
        \begin{solution}
        是一个文字游戏,所谓$P\{Y>t\}$转换为$X$的话其实就是在$t$时间内没有发生故障,$P\{X=0\}=\frac{\lambda_t^0}{0!}e^{-\lambda_t}=e^{-\lambda_t}$
        \end{solution}
    }

    \item 设随机变量$X\sim N(\mu,\sigma^2),\sigma>0,(x_0,y_0)$为分布函数曲线$y=F(x)$的拐点,则$x_0=\_\_\_,y_0=\_\_\_$ 
    
    \answer{
        \begin{solution}
            这道题本身并没啥,但要注意题目,$y_0$是$F(x_0)$而不是$f(x_0)$,答案是$\mu,\frac{1}{2}$
        \end{solution}
    }

    \item 设离散型随机变量X的分布律为$P\{X=k\}=\frac{a}{k!}e^{-2},k=0,1,2\ldots$则常数$a=\_\_\_\_$ 
    
    \answer{
        \begin{solution}
        这道题由两个解法,需要注意对比泊松分布时候系数的确定 
        \begin{enumerate}
            \item [(方法一)] 由规范性有 
            \[
            \sum_{k=0}^{\infty}\frac{a}{k!}e^{-2} = 1 =\implies a = e
            \]
            \item [(方法二)] 有泊松分布有
            \[
            P\{X=k\}=\frac{\lambda^k}{k!}e^{-\lambda} 
            \]
            注意到$\lambda=-1$的时候与题设要求接近,故有$ae^{-2}=e^{-1}\implies a = e$
        \end{enumerate}
    \end{solution}
    }

    \item 设$X\sim N(0,\sigma^2)$,$X$在区间$(a,b)$内取值的概率最大,其中$a>0$则$\sigma^2=\_\_\_\_$
    
    \answer{
        \begin{solution}
        这道题还真实蛮奇怪的,有题可知所求概率为
        $$P\{a/\sigma < X < b/\sigma\}=\Phi(b/\sigma)-\Phi(a/\sigma)$$
        不妨记
        \[
        f(\sigma)=\Phi(b/\sigma) - \Phi(a/\sigma)
        \]
        问题等效为去上面函数的最值问题. 
        \[
            f'(\sigma) = \frac{a}{\sigma^2}\phi(a/\sigma)-\frac{b}{\sigma^2}\phi(b/\sigma)
        \]
        令$f'(\sigma)=0$,则有$be^{-(b^2/(2\sigma^2))}=ae^{-(a^2/(2\sigma^2))}$
        两边取对数,可以得到 
        \[
        \sigma^2=\frac{a^2-b^2}{2(\ln{a}-\ln{b})}
        \]
        当$\sigma^2>$所求值的时候$f'(\sigma)>0$反之则有$f'(\sigma)<0$故所求值即为最大值 
        \[
        \fbox{$\sigma^2=\frac{a^2-b^2}{2(\ln{a}-\ln{b})}$}
        \]
    \end{solution}
    }
\end{enumerate}


\section{多维随机变量}

\section{数字特征}

\section{后三章}

\ifx\allfiles\undefined
\end{document}
\fi