\ifx\allfiles\undefined
\documentclass[12pt, a4paper, oneside, UTF8]{ctexbook}
\def\path{../config}
\usepackage{amsthm}
\usepackage{amssymb}
\usepackage{array}
\usepackage{xcolor}
\usepackage{graphicx}
\usepackage{mathrsfs}
\usepackage{enumitem}
\usepackage{geometry}
\usepackage[colorlinks, linkcolor=black]{hyperref}
\usepackage{stackengine}
\usepackage{yhmath}
\usepackage{extarrows}
\usepackage{tikz}
\usepackage{forest}
\usetikzlibrary{decorations.pathreplacing, positioning}
% \usepackage{unicode-math}
\usepackage{esint}
\usepackage{pifont}
\usepackage{tcolorbox}
\tcbuselibrary{skins, breakable}

\usepackage{multicol} 
\usepackage{fontspec} % 使用字体

\setmainfont{Times New Roman}
\setCJKmainfont{LXGWWenKai-Light}[
    SlantedFont=*
]

\usepackage{listings} % 用于插入代码

% 定义代码高亮风格
\lstset{
    basicstyle=\ttfamily\small,        % 基本字体样式(等宽小字体)
    keywordstyle=\color{blue},         % 关键字颜色
    commentstyle=\color{green},        % 注释颜色
    stringstyle=\color{red},           % 字符串颜色
    numbers=none,
    breaklines=true,                   % 自动换行
    frame=single,                      % 代码框边框
    rulecolor=\color{black},           % 边框颜色
    captionpos=b,                      % 标题位置(底部)
    showspaces=false,                  % 不显示空格标记
    showstringspaces=false,            % 不显示字符串中的空格标记
    language=C                         % 设置语言为 C
}

\usepackage{fontawesome5}

\usepackage{amsmath}
\usepackage{booktabs, array}
\usepackage{makecell}
\usepackage{fancyhdr}
\usepackage[dvipsnames, svgnames]{xcolor}
\usepackage{listings}
\usepackage{tasks}[2020/01/11]

\everymath{\displaystyle}

\definecolor{mygreen}{rgb}{0,0.6,0}
\definecolor{mygray}{rgb}{0.5,0.5,0.5}
\definecolor{mymauve}{rgb}{0.58,0,0.82}
\definecolor{NavyBlue}{RGB}{0,0,128}
\definecolor{Rhodamine}{RGB}{255,0,255}
\definecolor{PineGreen}{RGB}{0,128,0}

\graphicspath{ {figures/},{../figures/}, {config/}, {../config/} }

\linespread{1.6}

\geometry{
    top=25.4mm, 
    bottom=25.4mm, 
    left=20mm, 
    right=20mm, 
    headheight=2.17cm, 
    headsep=4mm, 
    footskip=12mm
}

\setenumerate[1]{itemsep=5pt,partopsep=0pt,parsep=\parskip,topsep=5pt}
\setitemize[1]{itemsep=5pt,partopsep=0pt,parsep=\parskip,topsep=5pt}
\setdescription{itemsep=5pt,partopsep=0pt,parsep=\parskip,topsep=5pt}



% \begin{lstlisting}[language=TeX] ... \end{lstlisting}

% 定理环境设置
% ---------- 颜色 ----------
\definecolor{ExBlue}{HTML}{4F81BD}
\definecolor{SolGreen}{HTML}{77933C}
\definecolor{DefRed}{HTML}{C5504B}
\definecolor{ThmOrange}{HTML}{E97132}
\definecolor{RemGray}{HTML}{7F7F7F}
\definecolor{CorPurple}{HTML}{7030A0}
\definecolor{ForGray}{HTML}{595959}

% ---------- 通用“变色”模板 ----------
\tcbset{
    mybox/.style n args={1}{
        enhanced, breakable,
        arc=6pt,
        boxrule=0.6pt,
        left=8pt, right=8pt, top=6pt, bottom=6pt,
        drop shadow={black!25},
        fonttitle=\bfseries,
        coltitle=white,
        colbacktitle=#1!85,
        colback=#1!10,
        colframe=#1,
    }
}

% ---------- 各环境 ----------
% 例题
\newtcolorbox{example}[1][]{mybox={ExBlue}, title={\ifstrempty{#1}{Example}{#1}}}
% 解答
\newtcolorbox{solution}[1][]{mybox={SolGreen}, title={\ifstrempty{#1}{Solution}{#1}}}
% 定义
\newtcolorbox{definition}[1][]{mybox={DefRed}, title={\ifstrempty{#1}{Definition}{#1}}}
% 定理
\newtcolorbox{theorem}[1][]{mybox={ThmOrange}, title={\ifstrempty{#1}{Theorem}{#1}}}
% 标注
\newtcolorbox{remark}[1][]{mybox={RemGray}, title={\ifstrempty{#1}{Remark}{#1}}}
% 推论
\newtcolorbox{corollary}[1][]{mybox={CorPurple}, title={\ifstrempty{#1}{Corollary}{#1}}}
% 公式
\newtcolorbox{formula}[1][]{mybox={ForGray}, title={\ifstrempty{#1}{Formula}{#1}}}


\settasks{
    label-format = \bfseries,
    label        = \Alph*.,
    label-width  = 1.2em,
    label-offset = 0.3em,
    item-indent  = 1.9em,
    column-sep   = 0.5em
}

\newenvironment{choices}[1][4]   % 默认 4 栏
    {\begin{tasks}(#1)}
    {\end{tasks}}

% 自定义命令的文件

\def\d{\mathrm{d}}
\def\R{\mathbb{R}}
\def\P{\partial} 
\newcommand{\bs}[1]{\begin{solution}#1\end{solution}}
\newcommand{\bt}[1][1]{% 默认参数为1
    \ensuremath{% 确保数学模式
        \foreach \n in {1,...,#1} {\blacktriangle}% 循环输出 #1 个黑色三角形
    }%
}

\newcommand{\bl}[1][1]{% 默认参数为1
    \ensuremath{% 确保数学模式
        \foreach \n in {1,...,#1} {\blacklozenge}% 循环输出 #1 个黑色三角形
    }%
}
\newif\ifshowanswers
%\showanswerstrue % 注释掉这行就不显示答案

% 定义答案环境
\newcommand{\answer}[1]{%
    \ifshowanswers
        #1%
    \fi
}




% 修改参数改变封面样式,0 默认原始封面、内置其他1、2、3种封面样式
\def\myIndex{3}


\ifnum\myIndex>0
    \input{\path/cover_package_\myIndex} 
\fi

\def\myTitle{冲刺150笔记}
\def\myAuthor{Weary Bird}
\def\myDateCover{\today}
\def\myDateForeword{\today}
\def\myForeword{行香子}
\def\myForewordText{
树绕村庄,水满陂塘;倚东风、豪兴徜徉。小园几许,收尽春光。有桃花红,李花白,菜花黄。 \\
远远苔墙,隐隐茅堂;飏青旗、流水桥旁。偶然乘兴,步过东冈。正莺儿啼,燕儿舞,蝶儿忙。 \\
}
\def\mySubheading{知错能改善莫大焉}


\begin{document}
% \input{../config/cover}
\else
\fi
\chapter{高等数学}

\section{极限与连续}
\begin{enumerate}
    \item $\star$ 设函数$f(x)=\cos{(\sin{x})},g(x)=\sin{(\cos{x})}$当$\displaystyle x\in\left(0,\frac{\pi}{2}\right)$时(  ) \\
    A.$f(x)$单调递增,$g(x)$单调递减\qquad B.$f(x)$单调递减,$g(x)$单调递增 \\
    C.$f(x),g(x)$均单调递减\qquad\qquad\quad D.$f(x),g(x)$均单调递增

    \item $\star\star$讨论函数$f(x)=\displaystyle \lim_{x\to\infty}\frac{x^{n+2}-x^{-n}}{x^n+x^{-n}}$的连续性

    \item $\star\star$设$f(x)$在$\left[a,b\right]$上连续,且$a<c<d<b$证明:在$\left(a,b\right)$内必定存在一点$\xi$使得
    $mf(c)+nf(d)=(m+n)f(\xi)$,其中$m,n$为任意给定的自然数 

    \item $\star\star$设$x_1=\sqrt{a}(a>0),x_{n+1}=\sqrt{a+x_n}$证明$\lim_{n\to\infty}x_n$存在,并求出其值. 
    
    \item $\star\star\star$设$x_1=a\geq 0,y_1=b\geq 0, a\leq b, x_{n+1}=\sqrt{x_ny_n},\displaystyle y_{n+1}=\frac{x_n+y_n}{2}(n=1,2,\ldots)$证明
    $\displaystyle \lim_{n\to\infty}x_n=\lim_{n\to\infty}y_n$

    \item $\star\star$设$\left\{x_n\right\}$为数列,则下列数据结论正确的是(  ) 
    \begin{enumerate}
        \item [\ding{172}] 若$\{\arctan{x_n}\}$收敛,则$\{x_n\}$收敛 
        \item [\ding{173}] 若$\{\arctan{x_n}\}$单调,则$\{x_n\}$收敛 
        \item [\ding{174}] 若$x_n\in\left[-1,1\right]$,且$\{x_n\}$收敛,则$\{\arctan{x_n}\}$收敛
        \item [\ding{175}] 若$x_n\in\left[-1,1\right]$,且$\{x_n\}$单调,则$\{\arctan{x_n}\}$收敛
    \end{enumerate}
    A.\ding{172}\ding{173}\qquad B.\ding{174}\ding{175}\qquad C.\ding{172}\ding{174}\qquad D.\ding{173}\ding{175}

    \item $\star$极限$\displaystyle \lim_{x\to 0}\frac{(\cos{x}-e^{x^2})\sin x^2}{\frac{x^2}{2}+1-\sqrt{1+x^2}}$=\_\_\_\_ 
    
    \item $\star$设$\displaystyle a_n=\frac{3}{2}\int_{0}^{\frac{n}{n+1}}x^{n-1}\sqrt{1+x^n}\d x,$则$\displaystyle \lim_{n\to\infty}na_n$=\_\_\_\_
    
    \item $\star\star$设$\displaystyle \lim_{x\to 0}\left\{a\left[x\right]+\frac{\ln\left(1+e^{\frac{2}{x}}\right)}{\ln\left(1+e^{\frac{1}{x}}\right)}\right\}=b$则$a$=\_\_\_,$b$=\_\_\_

    \item $\star$ 设$x_1=1,x_2=2,x_{n+2}=\displaystyle \frac{1}{2}(x_n+x_{n+1})$,求$\displaystyle\lim_{n\to\infty}x_n$ 

    \item $\star\star\star$设$f(x)$在$\left[0,1\right]$上连续,且$f(0)=f(1)$证明 
    \begin{enumerate}
        \item [(I)] 至少存在一点$\xi\in\left(0,1\right)$使得$f(\xi)=f(\xi+\displaystyle \frac{1}{2})$ 
        \item [(II)] 至少存在一点$\xi\in\left(0,1\right)$使得$f(\xi)=f(\xi+\displaystyle \frac{1}{n})(n\geq 2, n\in\mathbb{N})$
    \end{enumerate}

    \item $\star\star\star\star$(2011.数一) 
    \begin{enumerate}
        \item[(I)] 证明$\displaystyle \frac{1}{n+1}<\ln{(1+\frac{1}{n})}<\frac{1}{n}$
        \item[(II)] 证明极限$\displaystyle \lim_{n\to\infty}\left(1+\frac{1}{2}+\ldots+\frac{1}{n}-\ln{n}\right)$存在
    \end{enumerate}
\end{enumerate}

\section{一元函数微分学/积分学(除证明题)/多元函数微分学}
\begin{enumerate}
    \item $\star$设$f'_x(x_0,y_0),f'_y(x_0,y_0)$均存在,则下列结论正确的是(  ) \\
    A.$ \displaystyle \lim_{\substack{x\to x_0\\ y\to y_0}}f(x,y)$ 存在 \qquad
    B.$f(x,y)$在$(x_0,y_0)$处连续 \\
    C.$\lim_{x\to x_0}f(x,y_0)$存在 \qquad
    D.$f(x,y)$在去心邻域$(x_0,y_0)$内有定义 

    \item $\star$设$z=(1+xy)^y,$则$\d z\big|_{1,1}=\_\_\_$ 
    
    \item $\star\star$ 设$\begin{cases}
        y = f(x, t) \\
        F(x,y,t) = 0
    \end{cases}f,F$有一阶连续偏导数,则$\displaystyle\frac{\d y}{\d x}=\_\_\_\_$ 

    \item $\star\star$设$y=f(x,t),t=t(x,t)$由方程$G(x,y,t)=0$确定,$f,G$可微,则$\displaystyle\frac{\d y}{\d x}=\_\_\_\_$

    \item $\star$设$z=z(x,y)$有方程$e^{2yz}+x+y^2+z=\frac{7}{4}$确定,则$\d z\big|_{\frac{1}{2},\frac{1}{2}}=\_\_\_\_$

    \item $\star$曲面$z=x^2+y^2-1$在点$P(2,1,4)$处的且平面方程为$\_\_\_$法线方程$\_\_\_$ 

    \item $\star$求$f(x,y)=(1+e^y)\cos{x}-ye^y$的极值 
    
    \item $\star\star$求双曲线$xy=4$与直线$2x+y=1$之间的最短距离
\end{enumerate}
\section{空间解析几何/多元函数积分学}
\begin{enumerate}
    \item $\star$设向量$\vec{a}=(1,2,1),\vec{b}=(-1,0,2),\vec{c}=(0,k,-3)$共面,则$k=\_\_\_$ 
    
    \item $\star\star$设非零向量$\vec{\alpha},\vec{\beta}$满足$\vec{\alpha}-\vec{\beta}$于$\vec{\alpha}+\vec{\beta}$的模相等,则必有(  ) \\
    A.$\vec{\alpha}-\vec{\beta}=\vec{\alpha}+\vec{\beta}$ \qquad B.$\vec{\alpha}=\vec{\beta}$ \qquad
    C.$\vec{\alpha}\times\vec{\beta}=\vec{0}$\qquad D.$\vec{\alpha}\cdot\vec{\beta}=0$

    \item $\star\star$直线$L_1:\begin{cases}
        x - 1 = 0 \\
        y = z
    \end{cases}$与$L_2:\begin{cases}
        x+2y = 0 \\
        z + 2 = 0
    \end{cases}$的距离$d=\_\_\_$

    \item $\star\star$设$\alpha,\beta$均为单位向量,其夹角为$\displaystyle \frac{\pi}{6}$则$\alpha+2\beta$与$3\alpha+\beta$为邻边的
    平行四边形的面积为\_\_\_ 

    \item $\star\star$设$\alpha,\beta$是非零常向量,夹角为$\displaystyle \frac{\pi}{3}$, 且$\left|\beta\right|=2$
    求$\displaystyle \lim_{x\to 0}\frac{\left|\alpha+x\beta\right|-\left|\alpha\right|}{x}=\_\_\_$

    \item $\star$求平行于平面$x+y+z=9$且与球面$x^2+y^2+z^2=4$相切的平面方程. 
    
    \item $\star$设平面$\pi$过直线$L:\begin{cases}
        x + 5y + z = 0 \\
        x - z + 4 = 0
    \end{cases}$且与平面$\pi_1:x-4y-8z+12=0$的夹角为$\displaystyle\frac{\pi}{4}$求平面$\pi$的方程

    \item $\star$求与直线$L_1:x+2=3-y=z+1$与$L_2:\displaystyle \frac{x+4}{2}=y=\frac{z-4}{3}$都垂直相交的直线方程
    
    \item $\star$求直线$L_1:\displaystyle\frac{x-3}{2}=y=\frac{z-1}{0}$与$L_2:\displaystyle \frac{x+1}{1}=\frac{y-2}{0}=z$的公垂线方程
    
    \item $\star\star$求直线$L:\displaystyle \frac{x-1}{3}=\frac{y-2}{4}=\frac{z+1}{1}$绕直线$\begin{cases}
        x = 2\\
        y = 3
    \end{cases}$旋转一周所得到的曲面方程


\end{enumerate}
\section{常微分方程}

\section{无穷级数}


\section{证明题}
\ifx\allfiles\undefined
\end{document}
\fi