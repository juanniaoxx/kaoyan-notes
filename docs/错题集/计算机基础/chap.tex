\ifx\allfiles\undefined
\documentclass[12pt, a4paper, oneside, UTF8]{ctexbook}
\def\path{../config}
\usepackage{amsthm}
\usepackage{amssymb}
\usepackage{array}
\usepackage{xcolor}
\usepackage{graphicx}
\usepackage{mathrsfs}
\usepackage{enumitem}
\usepackage{geometry}
\usepackage[colorlinks, linkcolor=black]{hyperref}
\usepackage{stackengine}
\usepackage{yhmath}
\usepackage{extarrows}
\usepackage{tikz}
\usepackage{forest}
\usetikzlibrary{decorations.pathreplacing, positioning}
% \usepackage{unicode-math}
\usepackage{esint}
\usepackage{pifont}
\usepackage{tcolorbox}
\tcbuselibrary{skins, breakable}

\usepackage{multicol} 
\usepackage{fontspec} % 使用字体

\setmainfont{Times New Roman}
\setCJKmainfont{LXGWWenKai-Light}[
    SlantedFont=*
]

\usepackage{listings} % 用于插入代码

% 定义代码高亮风格
\lstset{
    basicstyle=\ttfamily\small,        % 基本字体样式(等宽小字体)
    keywordstyle=\color{blue},         % 关键字颜色
    commentstyle=\color{green},        % 注释颜色
    stringstyle=\color{red},           % 字符串颜色
    numbers=none,
    breaklines=true,                   % 自动换行
    frame=single,                      % 代码框边框
    rulecolor=\color{black},           % 边框颜色
    captionpos=b,                      % 标题位置(底部)
    showspaces=false,                  % 不显示空格标记
    showstringspaces=false,            % 不显示字符串中的空格标记
    language=C                         % 设置语言为 C
}

\usepackage{fontawesome5}

\usepackage{amsmath}
\usepackage{booktabs, array}
\usepackage{makecell}
\usepackage{fancyhdr}
\usepackage[dvipsnames, svgnames]{xcolor}
\usepackage{listings}
\usepackage{tasks}[2020/01/11]

\everymath{\displaystyle}

\definecolor{mygreen}{rgb}{0,0.6,0}
\definecolor{mygray}{rgb}{0.5,0.5,0.5}
\definecolor{mymauve}{rgb}{0.58,0,0.82}
\definecolor{NavyBlue}{RGB}{0,0,128}
\definecolor{Rhodamine}{RGB}{255,0,255}
\definecolor{PineGreen}{RGB}{0,128,0}

\graphicspath{ {figures/},{../figures/}, {config/}, {../config/} }

\linespread{1.6}

\geometry{
    top=25.4mm, 
    bottom=25.4mm, 
    left=20mm, 
    right=20mm, 
    headheight=2.17cm, 
    headsep=4mm, 
    footskip=12mm
}

\setenumerate[1]{itemsep=5pt,partopsep=0pt,parsep=\parskip,topsep=5pt}
\setitemize[1]{itemsep=5pt,partopsep=0pt,parsep=\parskip,topsep=5pt}
\setdescription{itemsep=5pt,partopsep=0pt,parsep=\parskip,topsep=5pt}



% \begin{lstlisting}[language=TeX] ... \end{lstlisting}

% 定理环境设置
% ---------- 颜色 ----------
\definecolor{ExBlue}{HTML}{4F81BD}
\definecolor{SolGreen}{HTML}{77933C}
\definecolor{DefRed}{HTML}{C5504B}
\definecolor{ThmOrange}{HTML}{E97132}
\definecolor{RemGray}{HTML}{7F7F7F}
\definecolor{CorPurple}{HTML}{7030A0}
\definecolor{ForGray}{HTML}{595959}

% ---------- 通用“变色”模板 ----------
\tcbset{
    mybox/.style n args={1}{
        enhanced, breakable,
        arc=6pt,
        boxrule=0.6pt,
        left=8pt, right=8pt, top=6pt, bottom=6pt,
        drop shadow={black!25},
        fonttitle=\bfseries,
        coltitle=white,
        colbacktitle=#1!85,
        colback=#1!10,
        colframe=#1,
    }
}

% ---------- 各环境 ----------
% 例题
\newtcolorbox{example}[1][]{mybox={ExBlue}, title={\ifstrempty{#1}{Example}{#1}}}
% 解答
\newtcolorbox{solution}[1][]{mybox={SolGreen}, title={\ifstrempty{#1}{Solution}{#1}}}
% 定义
\newtcolorbox{definition}[1][]{mybox={DefRed}, title={\ifstrempty{#1}{Definition}{#1}}}
% 定理
\newtcolorbox{theorem}[1][]{mybox={ThmOrange}, title={\ifstrempty{#1}{Theorem}{#1}}}
% 标注
\newtcolorbox{remark}[1][]{mybox={RemGray}, title={\ifstrempty{#1}{Remark}{#1}}}
% 推论
\newtcolorbox{corollary}[1][]{mybox={CorPurple}, title={\ifstrempty{#1}{Corollary}{#1}}}
% 公式
\newtcolorbox{formula}[1][]{mybox={ForGray}, title={\ifstrempty{#1}{Formula}{#1}}}


\settasks{
    label-format = \bfseries,
    label        = \Alph*.,
    label-width  = 1.2em,
    label-offset = 0.3em,
    item-indent  = 1.9em,
    column-sep   = 0.5em
}

\newenvironment{choices}[1][4]   % 默认 4 栏
    {\begin{tasks}(#1)}
    {\end{tasks}}

% 自定义命令的文件

\def\d{\mathrm{d}}
\def\R{\mathbb{R}}
\def\P{\partial} 
\newcommand{\bs}[1]{\begin{solution}#1\end{solution}}
\newcommand{\bt}[1][1]{% 默认参数为1
    \ensuremath{% 确保数学模式
        \foreach \n in {1,...,#1} {\blacktriangle}% 循环输出 #1 个黑色三角形
    }%
}

\newcommand{\bl}[1][1]{% 默认参数为1
    \ensuremath{% 确保数学模式
        \foreach \n in {1,...,#1} {\blacklozenge}% 循环输出 #1 个黑色三角形
    }%
}
\newif\ifshowanswers
%\showanswerstrue % 注释掉这行就不显示答案

% 定义答案环境
\newcommand{\answer}[1]{%
    \ifshowanswers
        #1%
    \fi
}




% 修改参数改变封面样式,0 默认原始封面、内置其他1、2、3种封面样式
\def\myIndex{3}


\ifnum\myIndex>0
    \input{\path/cover_package_\myIndex} 
\fi

\def\myTitle{冲刺150笔记}
\def\myAuthor{Weary Bird}
\def\myDateCover{\today}
\def\myDateForeword{\today}
\def\myForeword{行香子}
\def\myForewordText{
树绕村庄,水满陂塘;倚东风、豪兴徜徉。小园几许,收尽春光。有桃花红,李花白,菜花黄。 \\
远远苔墙,隐隐茅堂;飏青旗、流水桥旁。偶然乘兴,步过东冈。正莺儿啼,燕儿舞,蝶儿忙。 \\
}
\def\mySubheading{知错能改善莫大焉}


\usepackage{listings} % 用于插入代码

% 定义代码高亮风格
\lstset{
    basicstyle=\ttfamily\small,        % 基本字体样式(等宽小字体)
    keywordstyle=\color{blue},         % 关键字颜色
    commentstyle=\color{green},        % 注释颜色
    stringstyle=\color{red},           % 字符串颜色
    numbers=right,
    breaklines=true,                   % 自动换行
    frame=single,                      % 代码框边框
    rulecolor=\color{black},           % 边框颜色
    captionpos=b,                      % 标题位置(底部)
    showspaces=false,                  % 不显示空格标记
    showstringspaces=false,            % 不显示字符串中的空格标记
    language=C                         % 设置语言为 C
}
\begin{document}
% \input{../config/cover}
\else
\fi
\chapter{计算机基础}
\section{数据结构}
\begin{enumerate}
    \item  评估下面这段代码的时间复杂度()
\begin{lstlisting}[language=C]
    int func(int n) {
        int i = 0, sum = 0;
        while(sum < n) sum += ++i;
        return i;
    }
\end{lstlisting}
    \begin{solution}
        
    \end{solution}

    \item 评估下面这段代码的时间复杂度()
\begin{lstlisting}[language=C]
    int sum = 0;
        for(int i = 1; i < n; i *= 2)
            for (int j = 0; j < i; j++)
                sum++;
\end{lstlisting}

    \begin{solution}
        
    \end{solution}

    \item 一个栈的入栈序列为\underline{$1,2,3,\ldots,n$},出栈序列是\underline{$P_1,P_2,P_3,\ldots,P_n$}.若
    $P_2=3$,则$P_3$的可能取值的个数可能是() \\
    A.n-1\qquad B.n-2\qquad C.n-3\qquad D.无法确认 

    \item 已知\underline{循环队列}存储在一维数组$A[0,\ldots,n-1]$中,且队列非空的时候front和rear分别指向队头和队尾.若初始时
    队列为空,且要求第一个进入队列的元素存储在$A[0]$,则初始时front和rear的值分别为() \\
    A.0,0\qquad\qquad B.0,n-1\qquad\qquad C.n-1,0\qquad\qquad D.n-1,n-1

    \item \underline{循环队列}放在一维数组$A[0,\ldots,M-1]$中,end1指向队头元素,end2指向队尾元素的后一个位置.
    假设队列两端都可以进行入队和出队操作,队列中最多能容纳$M-1$个元素.初始队列不为空.下列判断对空和队满的条件中,正确的是() \\
    A. 对空:end1\ ==\ end2; \qquad\qquad\qquad\qquad\quad 队满:end1\ ==\ (end2+1)mod\ M\\
    B. 对空:end1\ ==\ end2; \qquad\qquad\qquad\qquad\quad 队满:end2\ ==\ (end1+1)mod\ M-1\\
    C. 对空:end2\ ==\ (end1+1)mod\ M; \qquad\qquad 队满:end1\ ==\ (end2+1)mod\ M\\
    D. 对空:end1\ ==\ (end2+1)mod\ M; \qquad\qquad 队满:end2\ ==\ (end1+1)mod\ (M-1)
    
    \item \underline{火车重排问题}\\
    假设火车入口和出口之间有n条轨道,列车驶入的顺序为\underline{$8,4,2,5,3,9,1,6,7$}若希望
    得到的驶出顺序为 \underline{$1\sim 9$}则n至少为()\\
    A.2\qquad\qquad B.3\qquad\qquad C.4\qquad\qquad D.5

    \item 在一颗度为4的树T中,若有20个度为4的结点,10个度为3的结点,1个度为2的结点,10个度为1的结点,则树T
    的叶结点个数为() \\
    A.41\qquad\qquad B.82\qquad\qquad C.113\qquad\qquad D.122

    \item 已知一颗完全二叉树的第六层(设根为第一层)由8个叶结点,则该完全二叉树的结点个数\underline{最多为}() \\
    A.39\qquad\qquad B.52\qquad\qquad C.111\qquad\qquad D.119
    
    \item 若一颗完全二叉树有786个结点,则该二叉树中叶结点的个数为() \\
    A.257\qquad\qquad B.258\qquad\qquad C.384\qquad\qquad D.385

    \item 先序序列为\underline{$a,b,c,d$}的不同二叉树的个数为() \\
    A.13 \qquad\qquad B.14 \qquad\qquad C.15\qquad\qquad D.16

    \item 将森林转换为对应的二叉树,若在二叉树中,结点u是结点v的父结点的父结点,则在原来的森林中,u和v可能的关系是()
    \begin{enumerate}
        \item [(I)] 父子关系
        \item [(II)] 兄弟关系
        \item [(III)] u的父结点与v的父结点是兄弟关系
    \end{enumerate}

    \item 已知一颗有2011个结点的树,其叶结点的个数为116,该树对应的二叉树中\underline{无右孩子}的结点个数为()
    
    \item 已知森林F及与之对应的二叉树T,若F的先根遍历序列为\underline{$a,b,c,d,e,f$},中根遍历序列为\underline{$b,a,d,f,e,c$},则T的后根
    遍历序列为()

    \item 对任意给定的含n(n>2)个字符的有限集合S,用二叉树表示S的哈夫曼编码集与定长编码集,分别得到二叉树
    $T_1,T_2$.下列叙述中,正确的是() \\
    A. $T_1$和$T_2$的结点个数相同 \\
    B. $T_1$的高度大于$T_2$的高度 \\
    C. 出现频次不同的字符在$T_1$中处于不同的层 \\
    D. 出现频次不同的字符在$T_2$中处于相同的层

    \item 在由6个字符构成的字符集S中,各字符出现的频次为\underline{$3,4,5,6,8,10$},为S构造的哈夫曼树的
    加权平均长度为() \\
    A.2.4\qquad\qquad B.2.5\qquad\qquad C.2.67\qquad\qquad D.2.75
    \item 对于任意一棵高度为5且有10个结点的二叉树,若采用顺序存储结构保存,每个结点占一个存储单元,则存放该二叉树
    至少需要多少存储单元? 
    \begin{solution}
    对应顺序存储,应该按照满二叉树存储,故需要的存储空间为$2^h-1=2^5-1=31$个
    \end{solution}
    \item 在下列关于二叉树遍历的说法中, 正确的是 (   ).
    \begin{enumerate}
        \item[(A)]若有一个结点是二叉树中某个子树的中序遍历结果序列的最后一个结点, 则它一定
        是该子树的前序遍历结果序列的最后一个结点
        \item[(B)] 若有一个结点是二叉树中某个子树的前序遍历结果序列的最后一个结点, 则它一定
        是该子树的中序遍历结果序列的最后一个结点
        \item[(C)]若有一个叶结点是二叉树中某个子树的中序遍历结果序列的最后一个结点, 则它一
        定是该子树的前序遍历结果序列的最后一个结点
        \item[(D)] 若有一个叶结点是二叉树中某个子树的前序遍历结果序列的最后一个结点, 则它一
        定是该子树的中序遍历结果序列的最后一个结点
    \end{enumerate}
    \begin{solution}
        二叉树中序遍历的最后一个结点必然是从根开始沿着起右指针走到底的结点,记其为$p$. 
        
        若其不是叶子结点,不妨假设其左子树非空,则在先序遍历中最后的结点必然在其左子树中,故ABD都不对.  

        若其是叶子结点,则前序遍历和中序遍历的最后一个结点都会是它,故C正确.
    \end{solution}
\end{enumerate}
\section{计算机网络}
\begin{enumerate}
    \item TCP/IP参考模型分为四个层次 (正确) 
    \begin{solution}
        虽然一直讲物理层-数据链路层-网络层-传输层-应用层,TCP/IP五层结构;但如果问层次,其实物理层和数据链路层做的
        事情可以被概况的,通常称为网络接口层.
    \end{solution}
    \item 二进制信号在信噪比为127:1的4kHz的信道上传输,最大数据传输速率可达到() \\
    A.28000bps\qquad B.8000bps\qquad C.4000bps\qquad D.无限大
    \begin{solution}
        考虑奈氏定理,有 $B=2W=8kBand$ 此时最大数据传输速率为 $$R_{N-max}= B\times\log_{2}(n)=8kbps$$ 
        考虑香农定理,有 $$R_{S-max}=W\times\log_{2}(1+S/N)=4kHz\times\log_{2}(1+127)=28kbps$$ 
        最终的传输速率有二者的{\textbf{最小值}}限制,故$R_{max}=\min{(R_{Nmax},R_{Smax})}=8kpbs$
    \end{solution}
\end{enumerate}

\section{计算机组成原理}

\begin{enumerate}
    \item 某计算机字长为8位,CPU中有一个8位加法器.已知无符号数x=69,y=38,如果在该加法器中计算x-y,则加法
    器的两个输入端入端信息和低位进位信息分别是() \\
    A.0100\ 0101,0010\ 0110, 0 \qquad B.0100\ 0101,1101\ 1001, 1 \\
    C.A.0100\ 0101,1101\ 10110, 0 \qquad D.0100\ 0101,1101\ 1010, 1
    \item 某计算机存储器按字节编制,采用小端方式存放数据.假定编译器规定int型和short型长度分别为32位和16位
    并且数据按边界对齐存储.某C语言程序段如下
\begin{lstlisting}[language=C]
    struct {
        int a;
        char b;
        short c;
    }record;
    record.a = 273;
\end{lstlisting}
    若record变量的首地址为0xC008地址0xC008中的内容及record.c的地址分别是()
    A.0x00, 0xC00D\qquad B.0x00,0xC00E\qquad C.0x11,0xC00D \qquad D.0x11,0xC00E 

    \item 有如下C语言序段:
\begin{lstlisting}[language=C]
    short si = -32767;
    unsigned short usi = si;
\end{lstlisting}
    这执行上述两条语句后,usi的值是\_\_\_\_ 
\end{enumerate}
\section{操作系统}
\ifx\allfiles\undefined
\end{document}
\fi