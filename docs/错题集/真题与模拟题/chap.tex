\ifx\allfiles\undefined
\documentclass[12pt, a4paper, oneside, UTF8]{ctexbook}
\def\path{../config}
\usepackage{amsthm}
\usepackage{amssymb}
\usepackage{array}
\usepackage{xcolor}
\usepackage{graphicx}
\usepackage{mathrsfs}
\usepackage{enumitem}
\usepackage{geometry}
\usepackage[colorlinks, linkcolor=black]{hyperref}
\usepackage{stackengine}
\usepackage{yhmath}
\usepackage{extarrows}
\usepackage{tikz}
\usepackage{forest}
\usetikzlibrary{decorations.pathreplacing, positioning}
% \usepackage{unicode-math}
\usepackage{esint}
\usepackage{pifont}
\usepackage{tcolorbox}
\tcbuselibrary{skins, breakable}

\usepackage{multicol} 
\usepackage{fontspec} % 使用字体

\setmainfont{Times New Roman}
\setCJKmainfont{LXGWWenKai-Light}[
    SlantedFont=*
]

\usepackage{listings} % 用于插入代码

% 定义代码高亮风格
\lstset{
    basicstyle=\ttfamily\small,        % 基本字体样式(等宽小字体)
    keywordstyle=\color{blue},         % 关键字颜色
    commentstyle=\color{green},        % 注释颜色
    stringstyle=\color{red},           % 字符串颜色
    numbers=none,
    breaklines=true,                   % 自动换行
    frame=single,                      % 代码框边框
    rulecolor=\color{black},           % 边框颜色
    captionpos=b,                      % 标题位置(底部)
    showspaces=false,                  % 不显示空格标记
    showstringspaces=false,            % 不显示字符串中的空格标记
    language=C                         % 设置语言为 C
}

\usepackage{fontawesome5}

\usepackage{amsmath}
\usepackage{booktabs, array}
\usepackage{makecell}
\usepackage{fancyhdr}
\usepackage[dvipsnames, svgnames]{xcolor}
\usepackage{listings}
\usepackage{tasks}[2020/01/11]

\everymath{\displaystyle}

\definecolor{mygreen}{rgb}{0,0.6,0}
\definecolor{mygray}{rgb}{0.5,0.5,0.5}
\definecolor{mymauve}{rgb}{0.58,0,0.82}
\definecolor{NavyBlue}{RGB}{0,0,128}
\definecolor{Rhodamine}{RGB}{255,0,255}
\definecolor{PineGreen}{RGB}{0,128,0}

\graphicspath{ {figures/},{../figures/}, {config/}, {../config/} }

\linespread{1.6}

\geometry{
    top=25.4mm, 
    bottom=25.4mm, 
    left=20mm, 
    right=20mm, 
    headheight=2.17cm, 
    headsep=4mm, 
    footskip=12mm
}

\setenumerate[1]{itemsep=5pt,partopsep=0pt,parsep=\parskip,topsep=5pt}
\setitemize[1]{itemsep=5pt,partopsep=0pt,parsep=\parskip,topsep=5pt}
\setdescription{itemsep=5pt,partopsep=0pt,parsep=\parskip,topsep=5pt}



% \begin{lstlisting}[language=TeX] ... \end{lstlisting}

% 定理环境设置
% ---------- 颜色 ----------
\definecolor{ExBlue}{HTML}{4F81BD}
\definecolor{SolGreen}{HTML}{77933C}
\definecolor{DefRed}{HTML}{C5504B}
\definecolor{ThmOrange}{HTML}{E97132}
\definecolor{RemGray}{HTML}{7F7F7F}
\definecolor{CorPurple}{HTML}{7030A0}
\definecolor{ForGray}{HTML}{595959}

% ---------- 通用“变色”模板 ----------
\tcbset{
    mybox/.style n args={1}{
        enhanced, breakable,
        arc=6pt,
        boxrule=0.6pt,
        left=8pt, right=8pt, top=6pt, bottom=6pt,
        drop shadow={black!25},
        fonttitle=\bfseries,
        coltitle=white,
        colbacktitle=#1!85,
        colback=#1!10,
        colframe=#1,
    }
}

% ---------- 各环境 ----------
% 例题
\newtcolorbox{example}[1][]{mybox={ExBlue}, title={\ifstrempty{#1}{Example}{#1}}}
% 解答
\newtcolorbox{solution}[1][]{mybox={SolGreen}, title={\ifstrempty{#1}{Solution}{#1}}}
% 定义
\newtcolorbox{definition}[1][]{mybox={DefRed}, title={\ifstrempty{#1}{Definition}{#1}}}
% 定理
\newtcolorbox{theorem}[1][]{mybox={ThmOrange}, title={\ifstrempty{#1}{Theorem}{#1}}}
% 标注
\newtcolorbox{remark}[1][]{mybox={RemGray}, title={\ifstrempty{#1}{Remark}{#1}}}
% 推论
\newtcolorbox{corollary}[1][]{mybox={CorPurple}, title={\ifstrempty{#1}{Corollary}{#1}}}
% 公式
\newtcolorbox{formula}[1][]{mybox={ForGray}, title={\ifstrempty{#1}{Formula}{#1}}}


\settasks{
    label-format = \bfseries,
    label        = \Alph*.,
    label-width  = 1.2em,
    label-offset = 0.3em,
    item-indent  = 1.9em,
    column-sep   = 0.5em
}

\newenvironment{choices}[1][4]   % 默认 4 栏
    {\begin{tasks}(#1)}
    {\end{tasks}}

% 自定义命令的文件

\def\d{\mathrm{d}}
\def\R{\mathbb{R}}
\def\P{\partial} 
\newcommand{\bs}[1]{\begin{solution}#1\end{solution}}
\newcommand{\bt}[1][1]{% 默认参数为1
    \ensuremath{% 确保数学模式
        \foreach \n in {1,...,#1} {\blacktriangle}% 循环输出 #1 个黑色三角形
    }%
}

\newcommand{\bl}[1][1]{% 默认参数为1
    \ensuremath{% 确保数学模式
        \foreach \n in {1,...,#1} {\blacklozenge}% 循环输出 #1 个黑色三角形
    }%
}
\newif\ifshowanswers
%\showanswerstrue % 注释掉这行就不显示答案

% 定义答案环境
\newcommand{\answer}[1]{%
    \ifshowanswers
        #1%
    \fi
}




% 修改参数改变封面样式,0 默认原始封面、内置其他1、2、3种封面样式
\def\myIndex{3}


\ifnum\myIndex>0
    \input{\path/cover_package_\myIndex} 
\fi

\def\myTitle{冲刺150笔记}
\def\myAuthor{Weary Bird}
\def\myDateCover{\today}
\def\myDateForeword{\today}
\def\myForeword{行香子}
\def\myForewordText{
树绕村庄,水满陂塘;倚东风、豪兴徜徉。小园几许,收尽春光。有桃花红,李花白,菜花黄。 \\
远远苔墙,隐隐茅堂;飏青旗、流水桥旁。偶然乘兴,步过东冈。正莺儿啼,燕儿舞,蝶儿忙。 \\
}
\def\mySubheading{知错能改善莫大焉}


\begin{document}

% \input{../config/cover}
\else
\fi
\chapter{真题与模拟题}
\begin{tcolorbox}
    真题全刷结束后才开始套卷练习(8月)\\
    真题卷要保证刷3遍(9月,10月,11月)各一次 \\
    模拟卷从25年开始往前刷
\end{tcolorbox}
\section{数学真题一网打尽} 
\begin{enumerate}
    \item $\star$ 求 $\displaystyle \lim_{n\to\infty}\left(\frac{\sin\frac{\pi}{n}}{n+1}+
    \frac{\sin\frac{2\pi}{n}}{n+\frac{1}{2}}+\ldots+\frac{\sin\pi}{n+\frac{1}{n}}\right)$ 

    \answer{
        \begin{solution}
        
        \end{solution}
    }
    
    \item $\star\star$ 设函数$f(x)$在区间$\left[0,1\right]$连续,则$\int_{0}^{1}f(x)\d x = $ (   ) \\
    (A).$\displaystyle\lim_{n\to\infty}\sum_{k=1}^{n}f\left(\frac{2k-1}{2n}\right)\cdot\frac{1}{2n}$\qquad
    (B).$\displaystyle\lim_{n\to\infty}\sum_{k=1}^{n}f\left(\frac{2k-1}{2n}\right)\cdot\frac{1}{n}$\\
    (C).$\displaystyle\lim_{n\to\infty}\sum_{k=1}^{n}f\left(\frac{k-1}{2n}\right)\cdot\frac{1}{2n}$\qquad
    (D).$\displaystyle\lim_{n\to\infty}\sum_{k=1}^{n}f\left(\frac{k}{2n}\right)\cdot\frac{2}{n}$\qquad

    \item $\star\star$ 设$f(x)$是区间$[0,+\infty)$上单调递减且非负的连续函数,$\displaystyle a_n=\sum_{k=1}^{n}f(k)-\int_{1}^{n}f(x)\d x(n=1,2,\ldots)$证明
    数列$\{a_n\}$极限存在 

    \item 
    \begin{enumerate}
        \item [(I)] 证明方程$x^{n}+x^{n-1}+\ldots+x=1(n>1,n\in\mathbf{N})$在区间$\left(\frac{1}{2},1\right)$内仅有一个实根 
        \item [(II)] $\star\star$记$(I)$中的实根为$x_n$证明$\displaystyle \lim_{n\to\infty}x_n$存在,并求出此极限 
    \end{enumerate}

    \item $\star\star$设函数$f(x)=\ln{x}+\frac{1}{x}$ 
    \begin{enumerate}
        \item [(1)] 求$f(x)$的最小值 
        \item [(2)] 设数列$\{x_n\}$满足$\ln{x_n}+\frac{1}{x_{n+1}}<1$证明$\displaystyle\lim_{n\to\infty}x_n$存在,并求此极限
    \end{enumerate}

    \item $\star\star$ 当$x\to 0$时,$\alpha(x),\beta(x)$是非零无穷小量,则下列命题中 
    \begin{enumerate}
        \item [(1)] 若$\alpha(x)\sim\beta(x)$,则$\alpha^2(x)\sim\beta^2(x)$ 
        \item [(2)] 若$\alpha^2(x)\sim\beta^2(x)$,则$\alpha(x)\sim\beta(x)$
        \item [(3)] 若$\alpha(x)\sim\beta(x)$,则$\alpha(x)-\beta(x)=o(\alpha(x))$
        \item [(4)] 若$\alpha(x)-\beta(x)=o(\alpha(x))$,则$\alpha(x)\sim\beta(x)$
    \end{enumerate}
    A.1,3\qquad B.1,4\qquad C.1,3,4\qquad D.2,3,4 

    \item $\star\star$设对任意的$x$,总有$\varphi(x)\leq f(x)\leq g(x)$,且$\displaystyle \lim_{n\to\infty}\left[g(x)-\varphi(x)\right]=0$,则
    $\displaystyle\lim_{n\to\infty}f(x)$(   )  \\
    A. 存在且等于零 \qquad B.存在但不一定为零 \\
    C.一定不存在\qquad\quad\ D.不一定存在 

    \item $\star\star$设函数$f(x)$在$(0,+\infty)$内具有二阶导数,且$f''(x)>0$,
    令$u_n=f(n)(n=1,2,\ldots)$则下列结论正确的是(  ) \\
    A.若$u_1>u_2$,则$\{u_n\}$必收敛\qquad B.若$u_1>u_2$,则$\{u_n\}$必发散\\
    C.若$u_1<u_2$,则$\{u_n\}$必收敛\qquad D.若$u_1<u_2$,则$\{u_n\}$必发散

    \item $\star\star\star$ 设$\displaystyle\lim_{n\to\infty}a_n=a$且$a\neq 0$则当$n$充分大的时候,有(   )\\
    A. $\left|a_n\right|>\frac{\left|a\right|}{2}$\qquad B.$\left|a_n\right|<\frac{\left|a\right|}{2}$\qquad
    C. $a_n>a-\frac{1}{n}$\qquad D.$a_n<a+\frac{1}{n}$

    \item $\star\star$设有数列$\left\{x_n\right\},-\frac{\pi}{2}\leq x_n\leq \frac{\pi}{2}$则(    )\\
    A.若$\displaystyle\lim_{n\to\infty}\cos{(\sin{x_n})}$存在,则$\displaystyle\lim_{n\to\infty}x_n$存在 \\
    B.若$\displaystyle\lim_{n\to\infty}\sin{(\cos{x_n})}$存在,则$\displaystyle\lim_{n\to\infty}x_n$存在 \\
    C.若$\displaystyle\lim_{n\to\infty}\cos{(\sin{x_n})}$存在,则$\displaystyle\lim_{n\to\infty}\sin{x_n}$存在,但$\displaystyle\lim_{n\to\infty}x_n$不存在 \\
    C.若$\displaystyle\lim_{n\to\infty}\sin{(\cos{x_n})}$存在,则$\displaystyle\lim_{n\to\infty}\cos{x_n}$存在,但$\displaystyle\lim_{n\to\infty}x_n$不存在

    \item $\star$已知$a_n=\sqrt[n]{n}-\frac{(-1)^n}{n}(n=1,2,\ldots)$则$\{a_n\}$(    )\\
    A.有最大值与最小值 \qquad B.有最大值无最小值\\ 
    C.有最小值无最大值 \qquad D.无最大值与最小值
\end{enumerate}


\section{计算机基础真题套卷}
\section{合工大}
\ifx\allfiles\undefined
\end{document}
\fi