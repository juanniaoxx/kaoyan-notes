\ifx\allfiles\undefined
\documentclass[12pt, a4paper, oneside, UTF8]{ctexbook}
\def\path{../config}
\usepackage{amsmath}
\usepackage{amsthm}
\usepackage{amssymb}
\usepackage{array}
\usepackage{xcolor}
\usepackage{graphicx}
\usepackage{mathrsfs}
\usepackage{enumitem}
\usepackage{geometry}
\usepackage[colorlinks, linkcolor=black]{hyperref}
\usepackage{stackengine}
\usepackage{yhmath}
\usepackage{extarrows}
\usepackage{tikz}
\usepackage{pgfplots}
\usepackage{asymptote}
\usepackage{float}
\usepackage{fontspec} % 使用字体

\setmainfont{Times New Roman}
\setCJKmainfont{LXGWWenKai-Light}[
    SlantedFont=*
]

\everymath{\displaystyle}

\usepgfplotslibrary{polar}
\usepackage{subcaption}
\usetikzlibrary{decorations.pathreplacing, positioning}

\usepgfplotslibrary{fillbetween}
\pgfplotsset{compat=1.18}
% \usepackage{unicode-math}
\usepackage{esint}
\usepackage[most]{tcolorbox}

\usepackage{fancyhdr}
\usepackage[dvipsnames, svgnames]{xcolor}
\usepackage{listings}

\definecolor{mygreen}{rgb}{0,0.6,0}
\definecolor{mygray}{rgb}{0.5,0.5,0.5}
\definecolor{mymauve}{rgb}{0.58,0,0.82}
\definecolor{NavyBlue}{RGB}{0,0,128}
\definecolor{Rhodamine}{RGB}{255,0,255}
\definecolor{PineGreen}{RGB}{0,128,0}

\graphicspath{ {figures/},{../figures/}, {config/}, {../config/} }

\linespread{1.6}

\geometry{
    top=25.4mm, 
    bottom=25.4mm, 
    left=20mm, 
    right=20mm, 
    headheight=2.17cm, 
    headsep=4mm, 
    footskip=12mm
}

\setenumerate[1]{itemsep=5pt,partopsep=0pt,parsep=\parskip,topsep=5pt}
\setitemize[1]{itemsep=5pt,partopsep=0pt,parsep=\parskip,topsep=5pt}
\setdescription{itemsep=5pt,partopsep=0pt,parsep=\parskip,topsep=5pt}

\lstset{
    language=Mathematica,
    basicstyle=\tt,
    breaklines=true,
    keywordstyle=\bfseries\color{NavyBlue}, 
    emphstyle=\bfseries\color{Rhodamine},
    commentstyle=\itshape\color{black!50!white}, 
    stringstyle=\bfseries\color{PineGreen!90!black},
    columns=flexible,
    numbers=left,
    numberstyle=\footnotesize,
    frame=tb,
    breakatwhitespace=false,
} 

\lstset{
    language=TeX, % 设置语言为 TeX
    basicstyle=\ttfamily, % 使用等宽字体
    breaklines=true, % 自动换行
    keywordstyle=\bfseries\color{NavyBlue}, % 关键字样式
    emphstyle=\bfseries\color{Rhodamine}, % 强调样式
    commentstyle=\itshape\color{black!50!white}, % 注释样式
    stringstyle=\bfseries\color{PineGreen!90!black}, % 字符串样式
    columns=flexible, % 列的灵活性
    numbers=left, % 行号在左侧
    numberstyle=\footnotesize, % 行号字体大小
    frame=tb, % 顶部和底部边框
    breakatwhitespace=false % 不在空白处断行
}

% \begin{lstlisting}[language=TeX] ... \end{lstlisting}

% 定理环境设置
\usepackage[strict]{changepage} 
\usepackage{framed}

\definecolor{greenshade}{rgb}{0.90,1,0.92}
\definecolor{redshade}{rgb}{1.00,0.88,0.88}
\definecolor{brownshade}{rgb}{0.99,0.95,0.9}
\definecolor{lilacshade}{rgb}{0.95,0.93,0.98}
\definecolor{orangeshade}{rgb}{1.00,0.88,0.82}
\definecolor{lightblueshade}{rgb}{0.8,0.92,1}
\definecolor{purple}{rgb}{0.81,0.85,1}

\theoremstyle{definition}
\newtheorem{myDefn}{\indent Definition}[section]
\newtheorem{myLemma}{\indent Lemma}[section]
\newtheorem{myThm}[myLemma]{\indent Theorem}
\newtheorem{myCorollary}[myLemma]{\indent Corollary}
\newtheorem{myCriterion}[myLemma]{\indent Criterion}
\newtheorem*{myRemark}{\indent Remark}
\newtheorem{myProposition}{\indent Proposition}[section]

\newenvironment{formal}[2][]{%
	\def\FrameCommand{%
		\hspace{1pt}%
		{\color{#1}\vrule width 2pt}%
		{\color{#2}\vrule width 4pt}%
		\colorbox{#2}%
	}%
	\MakeFramed{\advance\hsize-\width\FrameRestore}%
	\noindent\hspace{-4.55pt}%
	\begin{adjustwidth}{}{7pt}\vspace{2pt}\vspace{2pt}}{%
		\vspace{2pt}\end{adjustwidth}\endMakeFramed%
}

\newenvironment{definition}{\vspace{-\baselineskip * 2 / 3}%
	\begin{formal}[Green]{greenshade}\vspace{-\baselineskip * 4 / 5}\begin{myDefn}}
	{\end{myDefn}\end{formal}\vspace{-\baselineskip * 2 / 3}}

\newenvironment{theorem}{\vspace{-\baselineskip * 2 / 3}%
	\begin{formal}[LightSkyBlue]{lightblueshade}\vspace{-\baselineskip * 4 / 5}\begin{myThm}}%
	{\end{myThm}\end{formal}\vspace{-\baselineskip * 2 / 3}}

\newenvironment{lemma}{\vspace{-\baselineskip * 2 / 3}%
	\begin{formal}[Plum]{lilacshade}\vspace{-\baselineskip * 4 / 5}\begin{myLemma}}%
	{\end{myLemma}\end{formal}\vspace{-\baselineskip * 2 / 3}}

\newenvironment{corollary}{\vspace{-\baselineskip * 2 / 3}%
	\begin{formal}[BurlyWood]{brownshade}\vspace{-\baselineskip * 4 / 5}\begin{myCorollary}}%
	{\end{myCorollary}\end{formal}\vspace{-\baselineskip * 2 / 3}}

\newenvironment{criterion}{\vspace{-\baselineskip * 2 / 3}%
	\begin{formal}[DarkOrange]{orangeshade}\vspace{-\baselineskip * 4 / 5}\begin{myCriterion}}%
	{\end{myCriterion}\end{formal}\vspace{-\baselineskip * 2 / 3}}
	

\newenvironment{remark}{\vspace{-\baselineskip * 2 / 3}%
	\begin{formal}[LightCoral]{redshade}\vspace{-\baselineskip * 4 / 5}\begin{myRemark}}%
	{\end{myRemark}\end{formal}\vspace{-\baselineskip * 2 / 3}}

\newenvironment{proposition}{\vspace{-\baselineskip * 2 / 3}%
	\begin{formal}[RoyalPurple]{purple}\vspace{-\baselineskip * 4 / 5}\begin{myProposition}}%
	{\end{myProposition}\end{formal}\vspace{-\baselineskip * 2 / 3}}


\newtheorem{example}{\indent \color{SeaGreen}{Example}}[section]
\renewcommand{\proofname}{\indent\textbf{\textcolor{TealBlue}{Proof}}}
\NewEnviron{solution}{%
	\begin{proof}[\indent\textbf{\textcolor{TealBlue}{Solution}}]%
		\color{blue}% 设置内容为蓝色
		\BODY% 插入环境内容
		\color{black}% 恢复默认颜色(可选,避免影响后续文字)
	\end{proof}%
}

% 自定义命令的文件

\def\d{\mathrm{d}}
\def\R{\mathbb{R}}
%\newcommand{\bs}[1]{\boldsymbol{#1}}
%\newcommand{\ora}[1]{\overrightarrow{#1}}
\newcommand{\myspace}[1]{\par\vspace{#1\baselineskip}}
\newcommand{\xrowht}[2][0]{\addstackgap[.5\dimexpr#2\relax]{\vphantom{#1}}}
\newenvironment{mycases}[1][1]{\linespread{#1} \selectfont \begin{cases}}{\end{cases}}
\newenvironment{myvmatrix}[1][1]{\linespread{#1} \selectfont \begin{vmatrix}}{\end{vmatrix}}
\newcommand{\tabincell}[2]{\begin{tabular}{@{}#1@{}}#2\end{tabular}}
\newcommand{\pll}{\kern 0.56em/\kern -0.8em /\kern 0.56em}
\newcommand{\dive}[1][F]{\mathrm{div}\;\boldsymbol{#1}}
\newcommand{\rotn}[1][A]{\mathrm{rot}\;\boldsymbol{#1}}

\newif\ifshowanswers
\showanswerstrue % 注释掉这行就不显示答案

% 定义答案环境
\newcommand{\answer}[1]{%
    \ifshowanswers
        #1%
    \fi
}

% 修改参数改变封面样式,0 默认原始封面、内置其他1、2、3种封面样式
\def\myIndex{0}


\ifnum\myIndex>0
    \input{\path/cover_package_\myIndex} 
\fi

\def\myTitle{考研数学笔记}
\def\myAuthor{Weary Bird}
\def\myDateCover{\today}
\def\myDateForeword{\today}
\def\myForeword{相见欢·林花谢了春红}
\def\myForewordText{
    林花谢了春红,太匆匆。
    无奈朝来寒雨晚来风。
    胭脂泪,相留醉,几时重。
    自是人生长恨水长东。
}
\def\mySubheading{以姜晓千强化课讲义为底本}


\begin{document}

% \input{\path/cover_text_\myIndex.tex}

\newpage
\thispagestyle{empty}
\begin{center}
    \Huge\textbf{\myForeword}
\end{center}
\myForewordText
\begin{flushright}
    \begin{tabular}{c}
        \myDateForeword
    \end{tabular}
\end{flushright}

\newpage
\pagestyle{plain}
\setcounter{page}{1}
\pagenumbering{Roman}
\tableofcontents

\newpage
\pagenumbering{arabic}
% \setcounter{chapter}{-1}
\setcounter{page}{1}

\pagestyle{fancy}
\fancyfoot[C]{\thepage}
\renewcommand{\headrulewidth}{0.4pt}
\renewcommand{\footrulewidth}{0pt}








\else
\fi
\chapter{真题与模拟题}
\begin{tcolorbox}
    真题全刷结束后才开始套卷练习(8月)\\
    真题卷要保证刷3遍(9月,10月,11月)各一次 \\
    模拟卷从25年开始往前刷
\end{tcolorbox}
\section{数学真题一网打尽} 
\begin{enumerate}
    \item $\star$ 求 $\displaystyle \lim_{n\to\infty}\left(\frac{\sin\frac{\pi}{n}}{n+1}+
    \frac{\sin\frac{2\pi}{n}}{n+\frac{1}{2}}+\ldots+\frac{\sin\pi}{n+\frac{1}{n}}\right)$ 

    \answer{
        \begin{solution}
        
        \end{solution}
    }
    
    \item $\star\star$ 设函数$f(x)$在区间$\left[0,1\right]$连续,则$\int_{0}^{1}f(x)\d x = $ (   ) \\
    (A).$\displaystyle\lim_{n\to\infty}\sum_{k=1}^{n}f\left(\frac{2k-1}{2n}\right)\cdot\frac{1}{2n}$\qquad
    (B).$\displaystyle\lim_{n\to\infty}\sum_{k=1}^{n}f\left(\frac{2k-1}{2n}\right)\cdot\frac{1}{n}$\\
    (C).$\displaystyle\lim_{n\to\infty}\sum_{k=1}^{n}f\left(\frac{k-1}{2n}\right)\cdot\frac{1}{2n}$\qquad
    (D).$\displaystyle\lim_{n\to\infty}\sum_{k=1}^{n}f\left(\frac{k}{2n}\right)\cdot\frac{2}{n}$\qquad

    \item $\star\star$ 设$f(x)$是区间$[0,+\infty)$上单调递减且非负的连续函数,$\displaystyle a_n=\sum_{k=1}^{n}f(k)-\int_{1}^{n}f(x)\d x(n=1,2,\ldots)$证明
    数列$\{a_n\}$极限存在 

    \item 
    \begin{enumerate}
        \item [(I)] 证明方程$x^{n}+x^{n-1}+\ldots+x=1(n>1,n\in\mathbf{N})$在区间$\left(\frac{1}{2},1\right)$内仅有一个实根 
        \item [(II)] $\star\star$记$(I)$中的实根为$x_n$证明$\displaystyle \lim_{n\to\infty}x_n$存在,并求出此极限 
    \end{enumerate}

    \item $\star\star$设函数$f(x)=\ln{x}+\frac{1}{x}$ 
    \begin{enumerate}
        \item [(1)] 求$f(x)$的最小值 
        \item [(2)] 设数列$\{x_n\}$满足$\ln{x_n}+\frac{1}{x_{n+1}}<1$证明$\displaystyle\lim_{n\to\infty}x_n$存在,并求此极限
    \end{enumerate}

    \item $\star\star$ 当$x\to 0$时,$\alpha(x),\beta(x)$是非零无穷小量,则下列命题中 
    \begin{enumerate}
        \item [(1)] 若$\alpha(x)\sim\beta(x)$,则$\alpha^2(x)\sim\beta^2(x)$ 
        \item [(2)] 若$\alpha^2(x)\sim\beta^2(x)$,则$\alpha(x)\sim\beta(x)$
        \item [(3)] 若$\alpha(x)\sim\beta(x)$,则$\alpha(x)-\beta(x)=o(\alpha(x))$
        \item [(4)] 若$\alpha(x)-\beta(x)=o(\alpha(x))$,则$\alpha(x)\sim\beta(x)$
    \end{enumerate}
    A.1,3\qquad B.1,4\qquad C.1,3,4\qquad D.2,3,4 

    \item $\star\star$设对任意的$x$,总有$\varphi(x)\leq f(x)\leq g(x)$,且$\displaystyle \lim_{n\to\infty}\left[g(x)-\varphi(x)\right]=0$,则
    $\displaystyle\lim_{n\to\infty}f(x)$(   )  \\
    A. 存在且等于零 \qquad B.存在但不一定为零 \\
    C.一定不存在\qquad\quad\ D.不一定存在 

    \item $\star\star$设函数$f(x)$在$(0,+\infty)$内具有二阶导数,且$f''(x)>0$,
    令$u_n=f(n)(n=1,2,\ldots)$则下列结论正确的是(  ) \\
    A.若$u_1>u_2$,则$\{u_n\}$必收敛\qquad B.若$u_1>u_2$,则$\{u_n\}$必发散\\
    C.若$u_1<u_2$,则$\{u_n\}$必收敛\qquad D.若$u_1<u_2$,则$\{u_n\}$必发散

    \item $\star\star\star$ 设$\displaystyle\lim_{n\to\infty}a_n=a$且$a\neq 0$则当$n$充分大的时候,有(   )\\
    A. $\left|a_n\right|>\frac{\left|a\right|}{2}$\qquad B.$\left|a_n\right|<\frac{\left|a\right|}{2}$\qquad
    C. $a_n>a-\frac{1}{n}$\qquad D.$a_n<a+\frac{1}{n}$

    \item $\star\star$设有数列$\left\{x_n\right\},-\frac{\pi}{2}\leq x_n\leq \frac{\pi}{2}$则(    )\\
    A.若$\displaystyle\lim_{n\to\infty}\cos{(\sin{x_n})}$存在,则$\displaystyle\lim_{n\to\infty}x_n$存在 \\
    B.若$\displaystyle\lim_{n\to\infty}\sin{(\cos{x_n})}$存在,则$\displaystyle\lim_{n\to\infty}x_n$存在 \\
    C.若$\displaystyle\lim_{n\to\infty}\cos{(\sin{x_n})}$存在,则$\displaystyle\lim_{n\to\infty}\sin{x_n}$存在,但$\displaystyle\lim_{n\to\infty}x_n$不存在 \\
    C.若$\displaystyle\lim_{n\to\infty}\sin{(\cos{x_n})}$存在,则$\displaystyle\lim_{n\to\infty}\cos{x_n}$存在,但$\displaystyle\lim_{n\to\infty}x_n$不存在

    \item $\star$已知$a_n=\sqrt[n]{n}-\frac{(-1)^n}{n}(n=1,2,\ldots)$则$\{a_n\}$(    )\\
    A.有最大值与最小值 \qquad B.有最大值无最小值\\ 
    C.有最小值无最大值 \qquad D.无最大值与最小值
\end{enumerate}


\section{计算机基础真题套卷}
\section{合工大}
\ifx\allfiles\undefined
\end{document}
\fi