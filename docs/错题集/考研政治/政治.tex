\ifx\allfiles\undefined
\documentclass[12pt, a4paper, oneside, UTF8]{ctexbook}
\usepackage{ctex}
\usepackage{verse}
\def\path{../config}
\usepackage{amsthm}
\usepackage{amssymb}
\usepackage{array}
\usepackage{xcolor}
\usepackage{graphicx}
\usepackage{mathrsfs}
\usepackage{enumitem}
\usepackage{geometry}
\usepackage[colorlinks, linkcolor=black]{hyperref}
\usepackage{stackengine}
\usepackage{yhmath}
\usepackage{extarrows}
\usepackage{tikz}
\usepackage{forest}
\usetikzlibrary{decorations.pathreplacing, positioning}
% \usepackage{unicode-math}
\usepackage{esint}
\usepackage{pifont}
\usepackage{tcolorbox}
\tcbuselibrary{skins, breakable}

\usepackage{multicol} 
\usepackage{fontspec} % 使用字体

\setmainfont{Times New Roman}
\setCJKmainfont{LXGWWenKai-Light}[
    SlantedFont=*
]

\usepackage{listings} % 用于插入代码

% 定义代码高亮风格
\lstset{
    basicstyle=\ttfamily\small,        % 基本字体样式(等宽小字体)
    keywordstyle=\color{blue},         % 关键字颜色
    commentstyle=\color{green},        % 注释颜色
    stringstyle=\color{red},           % 字符串颜色
    numbers=none,
    breaklines=true,                   % 自动换行
    frame=single,                      % 代码框边框
    rulecolor=\color{black},           % 边框颜色
    captionpos=b,                      % 标题位置(底部)
    showspaces=false,                  % 不显示空格标记
    showstringspaces=false,            % 不显示字符串中的空格标记
    language=C                         % 设置语言为 C
}

\usepackage{fontawesome5}

\usepackage{amsmath}
\usepackage{booktabs, array}
\usepackage{makecell}
\usepackage{fancyhdr}
\usepackage[dvipsnames, svgnames]{xcolor}
\usepackage{listings}
\usepackage{tasks}[2020/01/11]

\everymath{\displaystyle}

\definecolor{mygreen}{rgb}{0,0.6,0}
\definecolor{mygray}{rgb}{0.5,0.5,0.5}
\definecolor{mymauve}{rgb}{0.58,0,0.82}
\definecolor{NavyBlue}{RGB}{0,0,128}
\definecolor{Rhodamine}{RGB}{255,0,255}
\definecolor{PineGreen}{RGB}{0,128,0}

\graphicspath{ {figures/},{../figures/}, {config/}, {../config/} }

\linespread{1.6}

\geometry{
    top=25.4mm, 
    bottom=25.4mm, 
    left=20mm, 
    right=20mm, 
    headheight=2.17cm, 
    headsep=4mm, 
    footskip=12mm
}

\setenumerate[1]{itemsep=5pt,partopsep=0pt,parsep=\parskip,topsep=5pt}
\setitemize[1]{itemsep=5pt,partopsep=0pt,parsep=\parskip,topsep=5pt}
\setdescription{itemsep=5pt,partopsep=0pt,parsep=\parskip,topsep=5pt}



% \begin{lstlisting}[language=TeX] ... \end{lstlisting}

% 定理环境设置
% ---------- 颜色 ----------
\definecolor{ExBlue}{HTML}{4F81BD}
\definecolor{SolGreen}{HTML}{77933C}
\definecolor{DefRed}{HTML}{C5504B}
\definecolor{ThmOrange}{HTML}{E97132}
\definecolor{RemGray}{HTML}{7F7F7F}
\definecolor{CorPurple}{HTML}{7030A0}
\definecolor{ForGray}{HTML}{595959}

% ---------- 通用“变色”模板 ----------
\tcbset{
    mybox/.style n args={1}{
        enhanced, breakable,
        arc=6pt,
        boxrule=0.6pt,
        left=8pt, right=8pt, top=6pt, bottom=6pt,
        drop shadow={black!25},
        fonttitle=\bfseries,
        coltitle=white,
        colbacktitle=#1!85,
        colback=#1!10,
        colframe=#1,
    }
}

% ---------- 各环境 ----------
% 例题
\newtcolorbox{example}[1][]{mybox={ExBlue}, title={\ifstrempty{#1}{Example}{#1}}}
% 解答
\newtcolorbox{solution}[1][]{mybox={SolGreen}, title={\ifstrempty{#1}{Solution}{#1}}}
% 定义
\newtcolorbox{definition}[1][]{mybox={DefRed}, title={\ifstrempty{#1}{Definition}{#1}}}
% 定理
\newtcolorbox{theorem}[1][]{mybox={ThmOrange}, title={\ifstrempty{#1}{Theorem}{#1}}}
% 标注
\newtcolorbox{remark}[1][]{mybox={RemGray}, title={\ifstrempty{#1}{Remark}{#1}}}
% 推论
\newtcolorbox{corollary}[1][]{mybox={CorPurple}, title={\ifstrempty{#1}{Corollary}{#1}}}
% 公式
\newtcolorbox{formula}[1][]{mybox={ForGray}, title={\ifstrempty{#1}{Formula}{#1}}}


\settasks{
    label-format = \bfseries,
    label        = \Alph*.,
    label-width  = 1.2em,
    label-offset = 0.3em,
    item-indent  = 1.9em,
    column-sep   = 0.5em
}

\newenvironment{choices}[1][4]   % 默认 4 栏
    {\begin{tasks}(#1)}
    {\end{tasks}}

% 自定义命令的文件

\def\d{\mathrm{d}}
\def\R{\mathbb{R}}
\def\P{\partial} 
\newcommand{\bs}[1]{\begin{solution}#1\end{solution}}
\newcommand{\bt}[1][1]{% 默认参数为1
    \ensuremath{% 确保数学模式
        \foreach \n in {1,...,#1} {\blacktriangle}% 循环输出 #1 个黑色三角形
    }%
}

\newcommand{\bl}[1][1]{% 默认参数为1
    \ensuremath{% 确保数学模式
        \foreach \n in {1,...,#1} {\blacklozenge}% 循环输出 #1 个黑色三角形
    }%
}
\newif\ifshowanswers
%\showanswerstrue % 注释掉这行就不显示答案

% 定义答案环境
\newcommand{\answer}[1]{%
    \ifshowanswers
        #1%
    \fi
}




% 修改参数改变封面样式,0 默认原始封面、内置其他1、2、3种封面样式
\def\myIndex{3}


\ifnum\myIndex>0
    \input{\path/cover_package_\myIndex} 
\fi

\def\myTitle{冲刺150笔记}
\def\myAuthor{Weary Bird}
\def\myDateCover{\today}
\def\myDateForeword{\today}
\def\myForeword{行香子}
\def\myForewordText{
树绕村庄,水满陂塘;倚东风、豪兴徜徉。小园几许,收尽春光。有桃花红,李花白,菜花黄。 \\
远远苔墙,隐隐茅堂;飏青旗、流水桥旁。偶然乘兴,步过东冈。正莺儿啼,燕儿舞,蝶儿忙。 \\
}
\def\mySubheading{知错能改善莫大焉}


\begin{document}
% \input{../config/cover}
\else
\fi
\chapter{考研政治}
\section{马克思主义基本原理}
\begin{enumerate}
    \item (单选)进入21世纪以来,社会化大生产在世界范围内更大规模、
    更广范围、更深层次展开,世界格局深度调整,资本主义呈现一些新变化新特征.
    当代资本主义最突出、最鲜明、最主要的特征是(   ) \\
    A. 输出、渗透资本主义价值观 \qquad
    B. 工人阶级内部层级结构逐渐分化 \\
    C. 国际金融资本的垄断 \qquad\qquad\quad
    D. 科技创新加速资本主义生产方式变化

    \item 垄断是在自由竞争中形成的,是作为自由竞争的对立面产生的.
    但是,垄断并不能消除竞争,反而使竞争更加复杂和剧烈.这是因为(   ) \\
    A. 垄断没有改变生产资料的资本主义私有制 \\
    B. 垄断企业必须不断增强自己的实力,巩固自己的垄断地位\\
    C. 如果竞争不复存在,垄断企业就没有动力和压力壮大自己的实力\\
    D. 垄断企业不可能把全部社会生产都包下来

    \item 国家垄断资本主义是国家政权和私人垄断资本融合在一起的垄断资本主义.第二次世界大战结束以来,
    在国家垄断资本主义获得充分发展的同时,资本主义国家通过宏观调节和微观规制对生产,流通,分配和消费各个环节
    的干预也更加加深.其中,微观规制的类型主要有(   ) \\
    A.社会经济规制\qquad B.公共事业规制 \\
    C.公共生活规制\qquad D.反托拉斯法 

    \item 20世纪80年代以来,随着冷战的结束,分割的世界经济体系也随之被打破,技术、资本、商品等真
    正实现了全球范围的流动,各国之间的经济联系日益密切,相互合作、相互依存大大加强,世界进入到经济全
    球化迅猛发展的新时代.促进经济全球化迅猛发展的因素有(   ) \\
    A.各国经济体制变革给出的有利制度条件 \\
    B.出现了适宜于全球化的企业组织形式 \\
    C.企业不断进行的技术创新与管理创新 \\
    D.科学技术的进步和生产力的快速发展

    \item 第二次世界大战结束后,资本主义国家对经济进行的干预明显加强,从而使得资本主义社会的经济调节机制发生了显著变化.
    与这种变化相适应,资本主义政治制度也发生了很大变化.其主要表现包括(   ) \\
    A. 政治制度出现多元化的趋势 \qquad
    B. 法治建设得到重视和加强 \\
    C. 社会阶层和阶级结构的变化 \qquad
    D. 改良主义政党的影响日益扩大

    \item 放眼当今世界,新一轮科技革命和产业变革深入发展,国际力量对比深刻调整,中国发展奇迹同西方资本主义的衰落形成了鲜明对比,“东升西降”已成为百年变局中的大势所趋.
    进入21世纪以来,由于错综复杂的原因,当代资本主义呈现一些不同以往的变化态势与特点.
    这些新变化新特征(   ) \\
    A. 并未改变资本主义的经济基础和追求利润最大化的本性\\
    B. 不断引发世界范围内对资本主义制度和价值观的质疑\\
    C. 使得资本主义的基本矛盾发生了根本性改变\\
    D. 不断诱发更激烈的世界性问题和全球性矛盾

    \item 《共产党宣言》指出:“一切所有制关系都经历了经常的历史更替、经常的历史变更以及社会主义必然代替资本主义,这是历史发展的客观规律,
    也是科学社会主义最基本的结论.”资本主义为社会主义所代替的历史必然性的依据有(   ) \\
    A. 资本主义基本矛盾“包含着现代的一切冲突的萌芽” \\
    B. 资本积累推动资本主义基本矛盾不断激化并最终否定资本主义自身 \\
    C. 国家垄断资本主义是资本社会化的更高形式,将成为社会主义的前奏 \\
    D. 资本主义社会存在着资产阶级和无产阶级两大阶级之间的矛盾和斗争

    \item (单选)恩格斯指出:“我认为,所谓‘社会主义社会’不是一种一成不变的东西,
    而应当和任何其他社会制度一样,把它看成经常变化和改革的社会.”社会主义改革的根源是(   ) \\
    A. 改革是社会主义社会发展的动力 \\
    B. 社会生产力发展水平不够高 \\
    C. 社会主义制度没有根本克服资本主义制度下生产力与生产关系的对抗性矛盾 \\
    D. 社会主义社会的基本矛盾 

    \item (单选)列宁指出,不能“为死教条而牺牲活的马克思主义”.习近平全面总结社会主义历史进程,
    得出“社会主义从来都是在开拓中前进的”.这些表述对我们的深刻启示是(   ) \\
    A. 必须始终“坚持科学社会主义基本原则” \\
    B. 要把科学社会主义基本原则与本国实际相结合 \\
    C. 科学社会主义基本原则要紧跟时代和实践的发展而发展 \\
    D. 时代和实践的不断发展使社会主义发展道路具有多样性

    \item 习近平指出:当代中国的伟大社会变革,不是简单延续我国历史文化的母版,
    不是简单套用马克思主义经典作家设想的模板,不是其他国家社会主义实践的再版,
    不是国外现代化发展的翻版.这对我们理解科学社会主义一般原则的启示是(   ) \\
    A. 科学社会主义是人类优秀文化传统的历史延续 \\
    B. 科学社会主义与资本主义生产方式没有必然的联系 \\
    C. 科学社会主义绝不是一成不变的教条 \\
    D. 科学社会主义在不同的时代具有不同的内容和形式 

    \item 世界上没有放之四海而皆准的发展道路和发展模式,也没有一成不变的发展道路和发展模式.
    30多年前,印有五角星和镰刀锤头的红旗在克里姆林宫上空悄然滑落,社会主义阵营老大哥消失,西亚北非地区陷入动荡.如今,中国共产党已然走过100多个春秋,中国特色社会主义
    比任何时期都要焕发生机与活力,社会主义发展的生机悄然而至.历史证明,
    社会主义之所以在曲折中发展,是因为(   ) \\
    A. 社会主义作为新生事物,其成长不会一帆风顺 \\
    B. 经济全球化对于社会主义的发展既有机遇又有挑战 \\
    C. 社会主义社会的基本矛盾推动社会发展,是作为一个过程而展开的 \\
    D. 各国历史文化传统的差异性决定社会主义发展方向

    \item 资本主义必然为社会主义所代替,并不意味着资本主义将在短期内自行消亡.
    资本主义向社会主义的过渡必然是一个复杂、长期的历史进程,其原因在于(   ) \\
    A. 资本主义社会具有一定的自我调节能力 \\
    B. 资本主义的发展具有不平衡性 \\
    C. 任何社会形态的存在都有绝对稳定性 \\
    D. 当代资本主义的发展还显示出生产关系对生产力容纳的空间

    \item (单选)共产主义社会是人类社会发展的最高社会形态,这一社会实现的必要条件是(   ) \\
    A.社会关系的高度和谐 \qquad
    B.人自由而全面的发展 \\
    C.生产力的高度发展 \qquad
    D.阶级和国家的消亡

    \item 马克思主义最崇高的社会理想是实现共产主义社会,即实现(   ) \\
    A.无矛盾的和谐社会 \qquad
    B.物质财富极大丰富 \\
    C.人们精神境界极大提高 \qquad
    D.每个人自由而全面的发展

    \item 马克思在表述共产主义社会的基本特征时指出,共产主义社会是社会关系高度和谐,
    人们精神境界极大提高的社会.社会关系的高度和谐表现在(   ) \\
    A.国家消亡\qquad
    B.阶级消亡 \\
    C.工业与农业、城市与乡村、脑力劳动与体力劳动的差别——"三大差别"消失 \\
    D.人、自然及社会都达成和谐 

    \item 马克思、恩格斯在《共产党宣言》中明确提出:
    "资产阶级的灭亡和无产阶级的胜利是同样不可避免的.""资本主义必然灭亡,社会主义必然胜利
    "是科学社会主义的核心命题.这"两个必然"是他们研究人类历史发展,
    特别是资本主义历史发展所得出的基本结论.这一科学论断(   ) \\
    A.在科学社会主义理论与实践中具有首要和基础的地位 \\
    B.是共产主义理想信念的核心要义\\
    C.是马克思主义追求的根本价值目标\\
    D.揭示了人类社会从资本主义向社会主义转变的历史必然性
\end{enumerate}
\section{思道法} 

\begin{enumerate}
    \item (单选) 人的生命是有限的,但生命的意义和价值却可以不同.实现人生价值的根本途径是  \\
    A.培养积极进取的人生态度   \\
    B.自觉提高自我的主体素质和能力   \\
    C.正确认识自我价值和社会价值的关系  \\ 
    D.进行有意识、有目的的创造性实践活动  

    \item 导弹技术专家沈忠芳隐姓埋名60多载,直到2022年4月中国航天科工集团二院正式发布《导弹人生》一书,
    才首次向全社会公开12位此前隐姓埋名的中国导弹武器型号总指挥、总设计师,沈忠芳在列.《感动中国》组委会给予沈忠芳的颁奖词这样写道:"从无到有,从近到远,从长缨在手,到红旗如画.这一代人从没有在乎过自己的得与失,这一代人一辈子都在砺国家的剑与盾.
    今天,后辈们终于能听到你们的传奇,隐秘而伟大,平静而神圣."这对们的人生启示是 \\
    A. 评价人生价值的根本尺度,是看一个人的实践活动是否符合社会发展的客观规律,是否进了历史的进步 \\
    B. 社会价值的实现总是以个人价值的牺牲为代价 \\
    C. 社会对于个人的价值评判主要是以个人对国家和社会所作的奉献为衡量标准 \\
    D. 人生社会价值的实现是个体自我完善、全面发展的保障

    \item (单选) 信念是认知、情感和意志的有机统一体,是人们在一定的认识基础上确立的对某种思想或事物坚信不疑并
    身体力行的精神状态.信念是人们追求理想目标的强大动力,决定事业的成败.信念有不同的层次和类型,其中 \\
    A. 高层次的信念决定低层次的信念   \\
    B. 低层次的信念代表了一个人的基本信仰   \\
    C. 相同社会环境中生活的人们的信念始终一致   \\
    D. 各种信念没有科学与非科学之分 

    \item (单选) "立志当高远,立志做大事."大量的事实告诉我们,那些在事业上取得伟大成就、对人类作出
    卓越贡献的人,都是在青年时期就立下了鸿鹄之志,并为之坚持不懈、努力奋斗.下列名言能体现这一说法的是 \\
    A. "功崇惟志,业广惟勤" \qquad
    B. "夙夜在公" \\
    C. "己所不欲,勿施于人" \qquad
    D. "己欲立而立人,己欲达而达人"

    \item 周恩来就读东关模范学校时,正值中国社会发生剧烈变动的时期.
    校长亲自为学生上修身课,题目是"立命".校长讲到精彩处突然停顿下来,
    问道:"诸生为何读书啊?"有人回答为名利而读书,有人回答为做官而读书.12岁的周恩来响亮地回答:
    "为中华之崛起而读书."校长赞叹道:"有志者,当效周生啊!"周总理的故事告诉我们
    要正确处理好个人理想和社会理想的关系,就要认识到 \\
    A.要坚持个人理想与社会理想的统一,在为实现社会理想而奋斗的过程中实现个人理想 \\
    B.个人理想以社会理想为指引,社会理想是对个人理想的凝练和升华 \\
    C.社会理想是最根本、最重要的,个人理想从属于社会理想 \\
    D.社会理想的实现必须以个人理想的实现为前提和基础 
\end{enumerate}
\section{毛中特} 

\section{史纲}

\begin{enumerate}
    \item 认识中国近代一切社会问题和革命问题的最基本的依据是(\qquad)
    \begin{choices}[1]
    \task 反帝反封建的革命任务
    \task 近代中国的基本国情
    \task 资本—帝国主义的侵略
    \task 民族资产阶级的软弱性
    \end{choices}
    \item 《四洲志》是一部世界地理著作,该书简要叙述了世界五大洲30多个国家的地理、历史和政治状况,是近代中国第一部相对完整、比较系统的世界地理志书。在此基础上,后人编写了《海国图志》。组织编写《四洲志》,且为近代中国睁眼看世界的第一人的是
    \begin{choices}
    \task 林则徐
    \task 魏源
    \task 马建忠
    \task 郑观应
    \end{choices}
    \item 在近代,王韬首次提出“变法”的主张,他在介绍西方国家的“君主”“民主”“君民共主”这三种制度时,最早提出废除封建君主专制,建立“与众民共政事并治天下”的君主立宪制。该思想
    \begin{choices}[1]
    \task 最早提出发展资本主义
    \task 推动了维新变法的兴起
    \task 反思了当时中国近代化的问题
    \task 引入了社会进化论的思想
    \end{choices}

    \item \bl 鸦片战争以后,西方列强通过发动侵略战争,强迫中国签订了一系列不平等条约,使中国沦为半殖民地半封建社会。中国逐步沦为半殖民地社会的原因包括
    \begin{choices}[1]
    \task 中国已经丧失了完全独立的地位
    \task 列强把中国卷入世界资本主义经济体系和世界市场之中
    \task 中国仍然维持着独立国家和政府的名义
    \task 西方列强不愿意中国成为独立的资本主义国家
    \end{choices}

    \item \bl 资本—帝国主义列强在对中国实行军事侵略、政治控制、经济掠夺的同时,还对中国进行文化渗透,具体包括
    \begin{choices}[2]
    \task 以传教为掩护进行间谍活动
    \task 创办《万国公报》
    \task 炮制“中国威胁论”
    \task 允许外国公使常驻北京
    \end{choices}

    \item \bl 魏源在其所著的《海国图志》一书中提出了“师夷长技以制夷”的思想,19世纪60年代开始的洋务运动则提出了“自强”“求富”的口号。二者的相同点包括
    \begin{choices}[1]
    \task 有内在的一致性和继承性
    \task 都有抵御外来侵略的意图
    \task 主要体现了地主阶级的要求
    \task 意识到了中国落后挨打的根本原因
    \end{choices}

    \item \bl 1894年7月丰岛海战后,中日两国相互宣战,战至1895年2月北洋海军全军覆没,中日两国各自改革30年后的决战以清政府惨败而告终,1895年4月清政府被迫签署《马关条约》。这场战争对中国产生了极其严重的后果,表现在
    \begin{choices}[1]
    \task 清政府已经彻底沦为“洋人的朝廷”
    \task 中断了清政府通过洋务运动向近代化转型的努力
    \task 中国付出了丧失领土主权的极大代价
    \task 清政府损失了海军主力
    \end{choices}

    \item 中国民族资产阶级登上政治舞台的第一次表演是
    \begin{choices}
    \task 洋务运动
    \task 戊戌维新运动
    \task 辛亥革命
    \task 五四运动
    \end{choices}

    \item \bl 在向西方学习的过程中.戊戌维新运动与洋务运动的不同点在于
    \begin{choices}[2]
    \task 学习西方的科学技术
    \task 学习西方的政治制度
    \task 批判封建的伦理道德
    \task 主张采取君主立宪制
    \end{choices}

    \item 1904年.孙中山发表了《中国问题的真解决》一文.指出只有推翻清政府的统治.“一个新的、开明的、进步的政府来代替旧政府”“把过时的满清君主政体改变为‘中华民国’”.才能真正解决中国问题。这表明革命派与改良派的根本不同之处是
    \begin{choices}[2]
    \task 平均地权.进行社会革命
    \task 兴民权.实行君主立宪
    \task 推翻帝制.实行共和
    \task 武装起义.推翻清王朝统治
    \end{choices}

    \item 1911年10月10日晚.驻武昌的新军工程第八营的革命党人打响了起义的第一枪。起义军一夜之间占领武昌.取得首义的胜利。武昌起义
    \begin{choices}[1]
    \task 是同盟会成立后发动的第一次武装起义
    \task 使得在中国延续了两千多年的封建帝制终于覆灭
    \task 拉开了中国完全意义上的近代民族民主革命的序幕
    \task 使得革命势力发展到了长江流域和黄河流域的大部分地区
    \end{choices}

    \item 中国近代思想主要经历了“师夷长技以制夷”“中体西用”“维新变法”“民主共和”“民主与科学”及“马克思主义”的演进过程。这些思想反映的共同主题是
    \begin{choices}[2]
    \task 发展资本主义
    \task 救亡图存
    \task 否定封建文化
    \task 民族独立、人民解放
    \end{choices}
    
    \item \bl 辛亥革命为中国的进步打开了闸门.为中国共产党的诞生准备了客观社会条件。辛亥革命给中国共产党的成立准备了
    \begin{choices}[2]
    \task 干部条件
    \task 理论基础
    \task 阶级基础
    \task 革命方法
    \end{choices}
    
    \item \bl 在评价辛亥革命,毛泽东指出,辛亥革命"有他胜利的地方,也有它失败的地方"得出这一结论的根据有
    \begin{choices}[1]
    \task 辛亥革命沉重打击了帝国主义侵略势力
    \task 辛亥革命推翻了封建君主专制制度
    \task 辛亥革命改变了中国半殖民地半封建的社会性质
    \task 辛亥革命是一次比较完全意义上的资产阶级民主革命
    \end{choices}

\end{enumerate}
\section{新思想}

\section{毛中特}
\begin{enumerate}
    \item 在党的二十大报告中指出,马克思主义是我们立党立国、兴党兴国的根本指导思想。实践告诉我们,中国共产党为什么能,中国特色社会主义为什么好,归根到底是
    \begin{choices}[1]
    \task 马克思主义行
    \task 中华优秀传统文化行
    \task 国际化的科学社会主义行
    \task 中国化时代
    \end{choices}

    \item 1938年11月,毛泽东在党的六届六中全会上明确指出:“共产党的任务,基本地不是经过长期合法斗争以进入起义和战争,也不是先占城市后取乡村,而是走相反的道路。”从此把经过长期武装斗争,先占乡村,后取城市,最后夺取全国胜利,作为革命道路确立下来。毛泽东提出该观点的著作是
    \begin{choices}[1]
    \task 《战争和战略问题》
    \task 《中国革命和中国共产党》
    \task 《在晋绥干部会议上的讲话》
    \task 《新民主主义论》
    \end{choices}

    
    \item 将新民主主义革命总路线的内容完整地表述为“无产阶级领导的,人民大众的,反对帝国主义、封建主义和官僚资本主义的革命”,是在毛泽东的著作
    \begin{choices}[1]
    \task 1938年的《论新阶段》中
    \task 1939年的《中国革命和中国共产党》中
    \task 1940年的《新民主主义论》中
    \task 1948年的《在晋绥干部会议上的讲话》中
    \end{choices}

    
    \item 延安时期,为了解决党内存在的思想分歧、宗派主义等问题,我们党开展了大规模的整风运动,使全党达到了空前的团结和统一,为夺取抗战胜利和全国解放奠定了强大思想政治基础。从此实事求是的思想路线在全党内确立起来。下列关于“实事求是”的说法正确的是
    \begin{choices}[1]
    \task 实事求是思想路线是我们党全部理论和实践的立足点
    \task 实事求是思想路线是我们党的基本思想方法、工作方法、领导方法
    \task 实事求是思想路线是我们党的生命线和根本工作路线
    \task 实事求是思想路线是中国共产党在中国革命中战胜敌人的三大法宝之一
    \end{choices}

    
    \item “实事求是”本是一句古老的中国格言,最早见于《汉书·河间献王传》,言称汉景帝之子河间献王刘德“修学好古,实事求是”。1941年5月,毛泽东在《改造我们的学习》一文中把这一原指治学态度的格言从哲学的高度作了新的解释,赋予了它新的含义。毛泽东指出,“实事”就是客观存在着的一切事物,“是”就是客观事物的内部联系,即规律性,“求”就是我们去研究。关于“实事求是”,下列表述正确的有
    \begin{choices}[1]
    \task 中共八大在全党确立了实事求是的思想路线
    \task 中共十一届三中全会重新确立了实事求是的思想路线
    \task 实事求是作为中国共产党的思想路线,始终是马克思主义中国化时代化理论成果的精髓和灵魂
    \task 实事求是是中国共产党带领人民推动中国革命、建设和改革事业不断取得胜利的重要法宝
    \end{choices}

    
    \item 中国共产党领导的新民主主义革命的根本目的是
    \begin{choices}[1]
    \task 推翻帝国主义、封建主义和官僚资本主义的压迫
    \task 改变中国半殖民地半封建社会的面貌
    \task 建立新民主主义的人民共和国
    \task 解放被束缚的生产力
    \end{choices}

    
    \item 新民主主义革命总路线表述为“无产阶级领导的,人民大众的,反对帝国主义、封建主义和官僚资本主义的革命”。其中“人民大众的”是指
    \begin{choices}[2]
    \task 革命的动力
    \task 革命的对象
    \task 革命的性质
    \task 革命的领导权
    \end{choices}

    
    \item 新民主主义革命总路线反映了中国革命的基本规律,指明了中国革命的对象、动力、领导力量、性质和前途。毛泽东在《新民主主义论》中指出,中国革命的基本问题是
    \begin{choices}[2]
    \task 政权问题
    \task 农民问题
    \task 无产阶级的领导权
    \task 分清敌友
    \end{choices}

    
    \item 中国共产党在领导人民革命的过程中,积累了丰富的经验,锻造出了有效的克敌制胜的武器。武装斗争就是中国共产党在中国革命中战胜敌人的重要法宝之一,其实质是
    \begin{choices}[1]
    \task 工人阶级领导的农民战争
    \task 资产阶级领导的反封建战争
    \task 工农联合的反军阀战争
    \task 无产阶级领导的反帝国主义战争
    \end{choices}

    
    \item 在中国革命和建设的长期实践中,以毛泽东同志为主要代表的中国共产党人,把马克思列宁主义基本原理同中国具体实际结合起来,形成了适合中国情况的科学指导思想,即毛泽东思想。在毛泽东思想的指导下,中国共产党团结带领中国人民,创造了新民主主义革命、社会主义革命和建设的伟大成就。其中,新民主主义革命的伟大成就,为实现中华民族伟大复兴创造了
    \begin{choices}[1]
    \task 根本社会条件
    \task 根本政治前提和制度基础
    \task 充满新的活力的体制保证和快速发展的物质条件
    \task 更为完善的制度保证、更为坚实的物质基础、更为主动的精神力量
    \end{choices}

    
    \item 毛泽东在《关于目前党的政策中的几个重要问题》中认为,中国革命的领导权是在与资产阶级争夺领导权的斗争中实现的,“领导的阶级和政党,要实现自己对于被领导的阶级、阶层、政党和人民团体的领导”,必须具备的条件包括
    \begin{choices}[1]
    \task 率领被领导者(同盟者)向着共同敌人作坚决的斗争,并取得胜利
    \task 对被领导者给以物质福利,至少不损害其利益
    \task 对被领导者给以政治教育
    \task 对被领导者采取“有理、有利、有节”的斗争原则
    \end{choices}

    
    \item 1925年,毛泽东在《中国社会各阶级的分析》中指出,谁是我们的敌人?谁是我们的朋友?这个问题是革命的首要问题。因此,党在不同时期的革命的主要对象有
    \begin{choices}[1]
    \task 在国共合作的大革命时期,革命的主要对象是帝国主义支持下的国民党新军阀
    \task 在土地革命战争时期,革命的主要对象是北洋军阀
    \task 在抗日战争时期,革命的主要对象是日本帝国主义
    \task 在解放战争时期,革命的主要对象是美帝国主义支持下的国民党反动派
    \end{choices}

    
    \item 在长期的革命斗争中,毛泽东始终关注着中国农民问题,没有毛泽东关于农民问题的理论,就不可能有农村包围城市的有中国特色的革命道路,就不可能夺取新民主主义革命和农业社会主义改造的胜利。农民问题之所以是中国革命的基本问题,是因为
    \begin{choices}[1]
    \task 农民是中国革命的主力军
    \task 新民主主义革命实质上就是中国共产党领导下的农民革命
    \task 中国革命战争实质上就是党领导下的农民战争
    \task 农民是中国革命最基本的动力
    \end{choices}

    
    \item 以马克思列宁主义的理论思想武装起来的中国共产党具备了三大优良作风,这就是
    \begin{choices}[1]
    \task 自力更生、艰苦奋斗的作风
    \task 理论和实践相结合的作风
    \task 和人民群众紧密地联系在一起的作风
    \task 批评与自我批评的作风
    \end{choices}

    
    \item 1938年11月,毛泽东在《战争和战略问题》中明确指出:“共产党的任务,基本地不是经过长期合法斗争以进入起义和战争,也不是先占城市后取乡村,而是走相反的道路。”中国革命“走相反的道路”的原因有
    \begin{choices}[1]
    \task 半殖民地半封建的中国社会内无民主外无独立
    \task 近代中国是一个农业大国,农民占全国人口的绝大多数
    \task 近代中国政治经济发展极端不平衡,存在不少的统治薄弱环节
    \task 广大农村深受压迫和剥削,人民革命愿望强烈
    \end{choices}

    
    \item 新民主主义文化,就是无产阶级领导的人民大众的反帝反封建的文化,即民族的科学的大众的文化。其中,科学的文化要求我们
    \begin{choices}[1]
    \task 反对一切封建思想和迷信思想
    \task 对于封建时代创造的文化,应剔除其封建糟粕,吸收其民主性精华
    \task 反对民族虚无主义
    \task 具有鲜明的民族风格、民族形式和民族特色
    \end{choices}
    
    \item 新民主主义社会不是一个独立的社会形态,而是由新民主主义转变到社会主义的过渡性的社会形态。其过渡性质,在新民主主义社会中的表现包括
    \begin{choices}[1]
    \task 个体经济经由合作社经济向社会主义集体经济过渡
    \task 私人资本主义经济由国家资本主义经济向社会主义国营经济过渡
    \task 土地改革基本完成后必须解决工人阶级和资产阶级的矛盾
    \task 国营经济成分不断扩大,掌握主要经济命脉,居于主体地位
    \end{choices}

    \item 中国共产党根据马克思、恩格斯和列宁关于采用和平方式变革所有制的设想,结合中国的具体情况,提出了对资本主义工商业实行和平赎买的方针。所谓赎买,就是国家有偿地将私营企业改变为国营企业,将资本主义私有制改变为社会主义公有制。对资本主义工商业实行和平赎买,有利于
    \begin{choices}[1]
    \task 发挥民族资产阶级中大多数人的知识、才能、技术专长和管理经验
    \task 团结各民主党派和各界爱国民主人士,巩固和发展统一战线
    \task 发挥私营工商业在国计民生方面的积极作用
    \task 争取和团结那些原来同资产阶级相联系的知识分子为社会主义建设服务
    \end{choices}

    \item 毛泽东指出:“现在我们能造什么?能造桌子椅子,能造茶碗茶壶,能种粮食,还能磨成面粉,还能造纸,但是,一辆汽车、一架飞机、一辆坦克、一辆拖拉机都不能造。”社会主义改造的目的包括
    \begin{choices}[1]
    \task 变革旧的生产关系
    \task 确立社会主义生产关系
    \task 实现民族独立
    \task 继续解放和发展生产力
    \end{choices}

    \item 毛泽东同志毕生最突出最伟大的贡献,就是领导我们党和人民找到了新民主主义革命的正确道路,完成了反帝反封建的任务,建立了中华人民共和国,确立了社会主义基本制度,取得了社会主义建设的基础性成就。关于社会主义基本制度确立的意义,下列说法正确的是
    \begin{choices}[1]
    \task 是中国历史上最深刻最伟大的社会变革
    \task 使中国人民站起来了,真正成为国家的主人
    \task 为当代中国一切发展进步奠定了制度基础
    \task 极大地提高了工人阶级和广大劳动人民的积极性、创造性
    \end{choices}

    \item 1956年4月,毛泽东在召集中央其他领导同志开会时就提出:“最重要的是要独立思考,把马列主义的基本原理同中国革命和建设的具体实际相结合。”为此,他先后在中央政治局扩大会议和最高国务会议上,作了《论十大关系》的报告。该报告
    \begin{choices}[1]
    \task 初步总结了我国社会主义建设的经验
    \task 系统论述了社会主义社会矛盾的理论
    \task 明确提出要以苏为鉴
    \task 提出要独立自主地探索适合中国情况的社会主义建设道路
    \end{choices}

    \item 1957年,毛泽东运用对立统一规律,以正确处理人民内部矛盾为总题目,总结了我国社会主义革命和建设的经验,全面分析了社会主义社会的矛盾,提出了正确区分和解决两类不同性质的矛盾的方法,规定了正确处理人民内部矛盾的一系列正确方针。其中,对于政治思想领域的人民内部矛盾,具体的方针政策包括
    \begin{choices}[1]
    \task 实行“百花齐放、百家争鸣”的方针
    \task 坚持民主集中制原则
    \task 坚持说服教育、讨论的方法
    \task 实行“团结—批评—团结”的方针
    \end{choices}

    \item 中华人民共和国成立后,以毛泽东为主要代表的中央领导集体对中国社会主义建设道路作出了孜孜不倦的思考和探索,虽然经历严重曲折,但也取得了独创性理论成果和巨大成就。回顾毛泽东对中国社会主义建设道路的探索成果,对于在新的历史起点上高举中国特色社会主义伟大旗帜,坚持和发展中国特色社会主义,具有的重要意义包括
    \begin{choices}[1]
    \task 标志着马克思主义基本原理和中国具体实际“第二次结合”任务的完成
    \task 丰富了科学社会主义的理论和实践
    \task 巩固和发展了中国特色社会主义制度
    \task 为开创中国特色社会主义提供了重要的理论准备、宝贵经验和物质基础
    \end{choices}

    \item 实践是理论创新的源泉。实践发展永无止境,我们认识真理、进行理论创新就永无止境。中国特色社会主义理论体系不是从天上掉下来的。中国特色社会主义理论体系形成发展的实践基础是
    \begin{choices}[1]
    \task 改革开放的伟大实践
    \task 社会主义改造的成功实践
    \task 社会主义现代化建设的生动实践
    \task 社会主义建设初步探索的实践
    \end{choices}

    \item 南方谈话是邓小平理论的集大成之作,邓小平理论也逐步走向成熟。邓小平南方谈话重申了深化改革、加速发展的必要性和重要性,并从中国实际出发,站在时代的高度,深刻地总结了十多年改革开放的经验教训,在一系列重大理论和实践问题上,提出了重要论断,主要包括
    \begin{choices}[1]
    \task 社会主义本质理论
    \task “三个有利于”标准
    \task 社会主义可以搞市场经济
    \task 革命是解放生产力,改革也是解放生产力
    \end{choices}

    \item 1985年,邓小平指出:“现在世界上真正大的问题,带全球性的战略问题,一个是和平问题,一个是经济问题或者说是发展问题。和平问题是东西问题,发展问题是南北问题。概括起来,就是东西南北四个字。”据此分析,当今世界所有问题的核心问题是
    \begin{choices}[2]
    \task 南北问题
    \task 维护世界和平问题
    \task 反对霸权主义、强权政治问题
    \task 建立国际政治经济新秩序问题
    \end{choices}

    \item 中国共产党的思想路线是一切从实际出发,理论联系实际,实事求是,在实践中检验和发展真理。其中,党的思想路线的根本途径和方法是
    \begin{choices}[2]
    \task 一切从实际出发
    \task 理论联系实际
    \task 实事求是
    \task 在实践中检验真理和发展真理
    \end{choices}

    \item 邓小平曾说过:“一个党,一个国家,一个民族,如果一切从本本出发,思想僵化,迷信盛行,那它就不能前进,它的生机就停止了,就要亡党亡国。”这段话深刻阐明了
    \begin{choices}[1]
    \task 恢复实事求是思想路线的极端重要性
    \task 坚持解放思想的极端重要性
    \task 坚持与时俱进的极端重要性
    \task 坚持求真务实的极端重要性
    \end{choices}

    \item 党的十三大从我国社会主义初级阶段的基本国情出发,提出了党在社会主义初级阶段的基本路线,把建设“富强、民主、文明的社会主义现代化国家”作为党在社会主义初级阶段的奋斗目标。实现奋斗目标的根本立足点是
    \begin{choices}[1]
    \task 坚持改革开放
    \task 以经济建设为中心
    \task 领导和团结各族人民
    \task 自力更生,艰苦创业
    \end{choices}

    \item 在我国改革开放全面展开的历史进程中,邓小平反复强调的,是中国实现社会主义现代化发展战略的必要前提,也是中国的最高利益的是
    \begin{choices}[1]
    \task 发展
    \task 改革
    \task 稳定
    \task 开放
    \end{choices}

    \item 习近平在主持中央政治局集体学习时强调:“在市场作用和政府作用的问题上,要讲辩证法、两点论,‘看不见的手’和‘看得见的手’都要用好,努力形成市场作用和政府作用有机统一、相互补充、相互协调、相互促进的格局,推动经济社会持续健康发展。”“看不见的手”和“看得见的手”实际上指的是
    \begin{choices}[1]
    \task 资源配置的不同方式
    \task 经济制度的表现形式
    \task 生产关系的具体范畴
    \task 分配方式的主要类型
    \end{choices}

    \item 邓小平曾指出:“现在虽说我们也在搞社会主义,但事实上不够格。”所谓“不够格”,也就是不够马克思所讲的“共产主义低级阶段”即社会主义阶段的“资格”。这里的“不够格”主要是物质技术基础方面不够格,也表现为社会经济制度和上层建筑方面的不成熟和不完善。社会主义初级阶段的长期性,从根本上说取决于
    \begin{choices}[1]
    \task 中国进入社会主义的历史条件
    \task 建成社会主义所需要的物质基础
    \task 党的基本路线和基本纲领
    \task 社会主义初级阶段的主要矛盾
    \end{choices}

    \item 社会主义本质理论揭示了社会主义的根本任务是解放和发展生产力,这合乎科学社会主义基本原则,因为
    \begin{choices}[1]
    \task 高度发达的生产力是实现社会主义的物质基础
    \task 社会主义的根本任务与根本目标是一致的
    \task 解放生产力为促进生产力的发展开辟了道路
    \task 要实现共同富裕,根本途径是解放和发展生产力
    \end{choices}

    \item 市场经济作为资源配置的一种方式,本身不具有制度属性,可以和不同的社会制度结合,从而表现出不同的性质。社会主义与市场经济的结合,其基本特征主要体现在
    \begin{choices}[1]
    \task 所有制结构方面
    \task 分配制度方面
    \task 宏观调控方面
    \task 资源配置方式方面
    \end{choices}

    \item 党的十九大报告对社会主义初级阶段的基本路线作出了修改,不仅将“美丽”纳入了基本路线,而且将“现代化国家”提升为“现代化强国”,扩展了党的基本路线的内涵,提升了社会主义初级阶段的奋斗目标。之所以在不同的历史阶段都反复强调坚持社会主义初级阶段基本路线的重要性,是因为党的基本路线
    \begin{choices}[1]
    \task 是党和国家的生命线、人民的幸福线
    \task 紧紧抓住了中国现阶段的主要矛盾
    \task 体现了运用社会主义社会基本矛盾运动的规律
    \task 是全面推动历史进步,实现民富国强、民族振兴的要求
    \end{choices}

    \item 在纪念邓小平同志诞辰120周年座谈会上,习近平高度评价了邓小平同志的伟大历史功勋。邓小平同志带领党和人民实现了伟大历史转折,开辟了社会主义现代化建设新局面以及
    \begin{choices}[1]
    \task 推动实现了马克思主义中国化新的飞跃
    \task 提出马克思主义同中国实际“第二次结合”
    \task 确立了实现祖国完全统一的正确路径
    \task 坚定捍卫了光辉的社会主义旗帜
    \end{choices}
\end{enumerate}
\section{时政}


\section{答案}

\subsection{史纲}
\begin{enumerate}
    \item B, 一切以实际出发,对于一个国家来说最大的实际就是国情
    \item A, 林则徐是开眼看世界的第一人
    \item C, 王韬是洋务运动的参与者
    \item AC, BD是半封建的原因
    \item ABC
    \item ABD
    \item BCD, A:八国联军侵华,签订辛丑条约标志着清政府完全沦为洋人的朝廷
    \item B, A:是封建地主阶级的自救运动
    \item BCD, 戊戌维新匹配了封建君权和封建伦理
    \item D, 课本原话不是C
    \item C,
    \item B, 注意前三个并没有提出民族独立与人民解法
    \item ABC, 三民主义
    \item ABD
\end{enumerate}
\ifx\allfiles\undefined
\end{document}
\fi