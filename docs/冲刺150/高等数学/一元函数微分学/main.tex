\ifx\allfiles\undefined
\documentclass[12pt, a4paper, oneside, UTF8]{ctexbook}
\def\path{../config}
\usepackage{amsmath}
\usepackage{amsthm}
\usepackage{amssymb}
\usepackage{array}
\usepackage{xcolor}
\usepackage{graphicx}
\usepackage{mathrsfs}
\usepackage{enumitem}
\usepackage{geometry}
\usepackage[colorlinks, linkcolor=black]{hyperref}
\usepackage{stackengine}
\usepackage{yhmath}
\usepackage{extarrows}
\usepackage{tikz}
\usepackage{pgfplots}
\usepackage{asymptote}
\usepackage{float}
\usepackage{fontspec} % 使用字体

\setmainfont{Times New Roman}
\setCJKmainfont{LXGWWenKai-Light}[
    SlantedFont=*
]

\everymath{\displaystyle}

\usepgfplotslibrary{polar}
\usepackage{subcaption}
\usetikzlibrary{decorations.pathreplacing, positioning}

\usepgfplotslibrary{fillbetween}
\pgfplotsset{compat=1.18}
% \usepackage{unicode-math}
\usepackage{esint}
\usepackage[most]{tcolorbox}

\usepackage{fancyhdr}
\usepackage[dvipsnames, svgnames]{xcolor}
\usepackage{listings}

\definecolor{mygreen}{rgb}{0,0.6,0}
\definecolor{mygray}{rgb}{0.5,0.5,0.5}
\definecolor{mymauve}{rgb}{0.58,0,0.82}
\definecolor{NavyBlue}{RGB}{0,0,128}
\definecolor{Rhodamine}{RGB}{255,0,255}
\definecolor{PineGreen}{RGB}{0,128,0}

\graphicspath{ {figures/},{../figures/}, {config/}, {../config/} }

\linespread{1.6}

\geometry{
    top=25.4mm, 
    bottom=25.4mm, 
    left=20mm, 
    right=20mm, 
    headheight=2.17cm, 
    headsep=4mm, 
    footskip=12mm
}

\setenumerate[1]{itemsep=5pt,partopsep=0pt,parsep=\parskip,topsep=5pt}
\setitemize[1]{itemsep=5pt,partopsep=0pt,parsep=\parskip,topsep=5pt}
\setdescription{itemsep=5pt,partopsep=0pt,parsep=\parskip,topsep=5pt}

\lstset{
    language=Mathematica,
    basicstyle=\tt,
    breaklines=true,
    keywordstyle=\bfseries\color{NavyBlue}, 
    emphstyle=\bfseries\color{Rhodamine},
    commentstyle=\itshape\color{black!50!white}, 
    stringstyle=\bfseries\color{PineGreen!90!black},
    columns=flexible,
    numbers=left,
    numberstyle=\footnotesize,
    frame=tb,
    breakatwhitespace=false,
} 

\lstset{
    language=TeX, % 设置语言为 TeX
    basicstyle=\ttfamily, % 使用等宽字体
    breaklines=true, % 自动换行
    keywordstyle=\bfseries\color{NavyBlue}, % 关键字样式
    emphstyle=\bfseries\color{Rhodamine}, % 强调样式
    commentstyle=\itshape\color{black!50!white}, % 注释样式
    stringstyle=\bfseries\color{PineGreen!90!black}, % 字符串样式
    columns=flexible, % 列的灵活性
    numbers=left, % 行号在左侧
    numberstyle=\footnotesize, % 行号字体大小
    frame=tb, % 顶部和底部边框
    breakatwhitespace=false % 不在空白处断行
}

% \begin{lstlisting}[language=TeX] ... \end{lstlisting}

% 定理环境设置
\usepackage[strict]{changepage} 
\usepackage{framed}

\definecolor{greenshade}{rgb}{0.90,1,0.92}
\definecolor{redshade}{rgb}{1.00,0.88,0.88}
\definecolor{brownshade}{rgb}{0.99,0.95,0.9}
\definecolor{lilacshade}{rgb}{0.95,0.93,0.98}
\definecolor{orangeshade}{rgb}{1.00,0.88,0.82}
\definecolor{lightblueshade}{rgb}{0.8,0.92,1}
\definecolor{purple}{rgb}{0.81,0.85,1}

\theoremstyle{definition}
\newtheorem{myDefn}{\indent Definition}[section]
\newtheorem{myLemma}{\indent Lemma}[section]
\newtheorem{myThm}[myLemma]{\indent Theorem}
\newtheorem{myCorollary}[myLemma]{\indent Corollary}
\newtheorem{myCriterion}[myLemma]{\indent Criterion}
\newtheorem*{myRemark}{\indent Remark}
\newtheorem{myProposition}{\indent Proposition}[section]

\newenvironment{formal}[2][]{%
	\def\FrameCommand{%
		\hspace{1pt}%
		{\color{#1}\vrule width 2pt}%
		{\color{#2}\vrule width 4pt}%
		\colorbox{#2}%
	}%
	\MakeFramed{\advance\hsize-\width\FrameRestore}%
	\noindent\hspace{-4.55pt}%
	\begin{adjustwidth}{}{7pt}\vspace{2pt}\vspace{2pt}}{%
		\vspace{2pt}\end{adjustwidth}\endMakeFramed%
}

\newenvironment{definition}{\vspace{-\baselineskip * 2 / 3}%
	\begin{formal}[Green]{greenshade}\vspace{-\baselineskip * 4 / 5}\begin{myDefn}}
	{\end{myDefn}\end{formal}\vspace{-\baselineskip * 2 / 3}}

\newenvironment{theorem}{\vspace{-\baselineskip * 2 / 3}%
	\begin{formal}[LightSkyBlue]{lightblueshade}\vspace{-\baselineskip * 4 / 5}\begin{myThm}}%
	{\end{myThm}\end{formal}\vspace{-\baselineskip * 2 / 3}}

\newenvironment{lemma}{\vspace{-\baselineskip * 2 / 3}%
	\begin{formal}[Plum]{lilacshade}\vspace{-\baselineskip * 4 / 5}\begin{myLemma}}%
	{\end{myLemma}\end{formal}\vspace{-\baselineskip * 2 / 3}}

\newenvironment{corollary}{\vspace{-\baselineskip * 2 / 3}%
	\begin{formal}[BurlyWood]{brownshade}\vspace{-\baselineskip * 4 / 5}\begin{myCorollary}}%
	{\end{myCorollary}\end{formal}\vspace{-\baselineskip * 2 / 3}}

\newenvironment{criterion}{\vspace{-\baselineskip * 2 / 3}%
	\begin{formal}[DarkOrange]{orangeshade}\vspace{-\baselineskip * 4 / 5}\begin{myCriterion}}%
	{\end{myCriterion}\end{formal}\vspace{-\baselineskip * 2 / 3}}
	

\newenvironment{remark}{\vspace{-\baselineskip * 2 / 3}%
	\begin{formal}[LightCoral]{redshade}\vspace{-\baselineskip * 4 / 5}\begin{myRemark}}%
	{\end{myRemark}\end{formal}\vspace{-\baselineskip * 2 / 3}}

\newenvironment{proposition}{\vspace{-\baselineskip * 2 / 3}%
	\begin{formal}[RoyalPurple]{purple}\vspace{-\baselineskip * 4 / 5}\begin{myProposition}}%
	{\end{myProposition}\end{formal}\vspace{-\baselineskip * 2 / 3}}


\newtheorem{example}{\indent \color{SeaGreen}{Example}}[section]
\renewcommand{\proofname}{\indent\textbf{\textcolor{TealBlue}{Proof}}}
\NewEnviron{solution}{%
	\begin{proof}[\indent\textbf{\textcolor{TealBlue}{Solution}}]%
		\color{blue}% 设置内容为蓝色
		\BODY% 插入环境内容
		\color{black}% 恢复默认颜色(可选,避免影响后续文字)
	\end{proof}%
}

% 自定义命令的文件

\def\d{\mathrm{d}}
\def\R{\mathbb{R}}
%\newcommand{\bs}[1]{\boldsymbol{#1}}
%\newcommand{\ora}[1]{\overrightarrow{#1}}
\newcommand{\myspace}[1]{\par\vspace{#1\baselineskip}}
\newcommand{\xrowht}[2][0]{\addstackgap[.5\dimexpr#2\relax]{\vphantom{#1}}}
\newenvironment{mycases}[1][1]{\linespread{#1} \selectfont \begin{cases}}{\end{cases}}
\newenvironment{myvmatrix}[1][1]{\linespread{#1} \selectfont \begin{vmatrix}}{\end{vmatrix}}
\newcommand{\tabincell}[2]{\begin{tabular}{@{}#1@{}}#2\end{tabular}}
\newcommand{\pll}{\kern 0.56em/\kern -0.8em /\kern 0.56em}
\newcommand{\dive}[1][F]{\mathrm{div}\;\boldsymbol{#1}}
\newcommand{\rotn}[1][A]{\mathrm{rot}\;\boldsymbol{#1}}

\newif\ifshowanswers
\showanswerstrue % 注释掉这行就不显示答案

% 定义答案环境
\newcommand{\answer}[1]{%
    \ifshowanswers
        #1%
    \fi
}

% 修改参数改变封面样式,0 默认原始封面、内置其他1、2、3种封面样式
\def\myIndex{0}


\ifnum\myIndex>0
    \input{\path/cover_package_\myIndex} 
\fi

\def\myTitle{考研数学笔记}
\def\myAuthor{Weary Bird}
\def\myDateCover{\today}
\def\myDateForeword{\today}
\def\myForeword{相见欢·林花谢了春红}
\def\myForewordText{
    林花谢了春红,太匆匆。
    无奈朝来寒雨晚来风。
    胭脂泪,相留醉,几时重。
    自是人生长恨水长东。
}
\def\mySubheading{以姜晓千强化课讲义为底本}


\begin{document}
% \input{\path/cover_text_\myIndex.tex}

\newpage
\thispagestyle{empty}
\begin{center}
    \Huge\textbf{\myForeword}
\end{center}
\myForewordText
\begin{flushright}
    \begin{tabular}{c}
        \myDateForeword
    \end{tabular}
\end{flushright}

\newpage
\pagestyle{plain}
\setcounter{page}{1}
\pagenumbering{Roman}
\tableofcontents

\newpage
\pagenumbering{arabic}
% \setcounter{chapter}{-1}
\setcounter{page}{1}

\pagestyle{fancy}
\fancyfoot[C]{\thepage}
\renewcommand{\headrulewidth}{0.4pt}
\renewcommand{\footrulewidth}{0pt}








\else
\fi

\chapter{一元函数微分学}
\begin{enumerate}
    \item $x$为以 $T$ 为周期的连续函数,则下列结论中正确的个数为 ( ).
    \begin{enumerate}
        \item [(I)] ${\int }_{0}^{x}f\left( t\right) \mathrm{d}t$ 以 $T$ 为周期
        \item [(II)] ${\int }_{0}^{x}f\left( t\right) \mathrm{d}t - \frac{x}{T}{\int }_{0}^{T}f\left( t\right) \mathrm{d}t$ 以 $T$ 为周期
        \item [(III)] 若 $f\left( x\right)$ 为奇函数,则 ${\int }_{0}^{x}f\left( t\right) \mathrm{d}t$ 以 $T$ 为周期
        \item [(IV)] ${\int }_{0}^{x}\left\lbrack  {f\left( t\right)  - f\left( {-t}\right) }\right\rbrack  \mathrm{d}t$ 以 $T$ 为周期
        \item [(V)] 若${\int }_{0}^{+\infty }f\left( x\right) \mathrm{d}x$ 收敛,则 ${\int }_{0}^{x}f\left( t\right) \mathrm{d}t$ 以 $T$ 为周期
    \end{enumerate}
    \begin{choices}
        \task 1
        \task 2
        \task 3
        \task 4
    \end{choices}
    
    
    \newpage
    

    \item 设$f(x)$为$x$的三次多项式,且$\lim_{x \to 2a}\frac{f(x)}{x-2a}=1,\lim_{x\to 4a}\frac{f(x)}{x-4a}=1(a\neq 0)$,则$\lim_{x\to 3a}\frac{f(x)}{x-3a}=\_\_\_\_\_$
    
    \newpage
    
    \item 设$y=y(x)$为微分方程$y''+(x+1)y'+x^2y=e^x$满足初始条件$y(0)=0,y'(0)=1$的特解,若$\lim_{x\to 0}\frac{y(x)-x}{x^k}=c(c\neq 0)$则$c=\_\_\_\_\_,k=\_\_\_\_\_$
    
    \newpage
    
    \item 设$f(x)$在点$x=0$处三阶可导,且$f(0)=f'(0)=f''(0)=0,f'''(0)\neq 0$求极限$\lim_{x\to 0}\frac{\int_{0}^{x}tf(x-t)\d t}{x\int_{0}^{x}f(x-t)\d t}.$
    
    \newpage
    
    \item 
    \begin{enumerate}
        \item[(1)] 设$f(x),g(x)$连续,且$\lim_{x\to 0}\frac{f(x)}{g(x)}=1,\lim_{x\to x_0}\varphi(x)=0$证明当$x\to x_0$的时候,$\int_{0}^{\varphi(x)}f(t)\d t \sim \int_{0}^{\varphi(x)}g(t)\d t$
        \item[(2)] 求极限$\lim_{x\to 0}\frac{\int_{0}^{x^2}\ln{(1+2\tan{t})}}{\left[\int_{0}^{x}\ln{(1+2\tan{t}\d t)}^2\right]}$
    \end{enumerate}
    
    \newpage
    
    \item 设极限$\lim_{x\to 0}\frac{1}{x}\int_{-x}^{x}\left(1-\frac{\left|t\right|}{x}\right)\cos{(\theta - t)}\d t$存在,求$\theta$的值.
    
    \newpage
    
    \item 设$f(x)=(1+x)^{\frac{1}{x}},$当$x\to 0^+$时,$f(x)=e+Ax+Bx^2+o(x^2),$求$A,B$的值.
    
    \newpage
    
    \item 求下列极限
    \begin{enumerate}
        \item [(1)] $\lim_{x\to 0}\frac{\tan{\tan{x}}-\sin{\sin{x}}}{\tan{x}-\sin{x}}$
        \item [(2)] $\lim_{x\to 0}\frac{\sin\sin{x}-\sin\tan{x}}{x^2(\sqrt{1+x}-e^x)}$
        \item [(3)] $\lim_{x\to 0}\frac{\cos\sin{x} - \cos\tan{x}}{x^3(\sqrt{1+x}-e^x)}$
    \end{enumerate}
    
    \newpage
        
    \item 
    \begin{enumerate}
        \item [(1)] 设$\lim_{x\to 0}\frac{(1+\sin{2x^2})^{\frac{1}{x^2}}-e^2}{x^n}=a(a\neq 0)$求$a,n$
        \item [(2)] 设$\lim_{x\to 0}\frac{(1+\tan{3x^2})^{\frac{1}{x^2}}-e^3}{x^n}=a(a\neq 0)$求$a,n$
    \end{enumerate}
    
    \newpage
    
    \item 求下列极限
    $$
    \lim_{x\to 0}\frac{\sqrt{\frac{1+x}{1-x}}\sqrt[4]{\frac{1+2x}{1-2x}}\sqrt[6]{\frac{1+3x}{1-3x}}\ldots\sqrt[2n]{\frac{1+nx}{1-nx}}-1}{3\pi\arcsin{x}-(x^2+1)\arctan^3{x}}(n\geq 1)
    $$
    
    \newpage
    
    \item 求下列极限
    \begin{enumerate}
        \item [(1)] $\lim_{x\to 0}\left[\frac{1}{\ln{(x+\sqrt{1+x^2})}}-\frac{1}{\ln{(1+x)}}\right]$
        \item [(2)] $\lim_{x\to 0}\left[\frac{\ln{(x+\sqrt{1+x^2})}}{\ln{(1+x)}}\right]^{\frac{1}{\ln{(1+x)}}}$
    \end{enumerate}
    
    \newpage
    
    \item 求极限$\lim_{x\to +\infty}\left[\frac{x^{1+x}}{(1+x)^x}-\frac{x}{e}\right]$
    
    \newpage
    
    \item 设极限$\lim_{x\to +\infty}\left[\left(x^3-x^2+\frac{x}{2}\right)e^{\frac{1}{x}}-\sqrt{x^n+1}\right]$存在,求$n$的值并求出该极限.
    
    \newpage
        
    \item 设$f(x)$在$x=x_0$处二阶可导,且$f''(x_0)\neq 0$若$f(x)=f(x_0)+f'\left[x_0+\theta(x-x_0)\right](x-x_0)(0<\theta<1),$求$\lim_{x\to x_0}\theta$
    
    \newpage
    
    \item 证明数列$2,2+\frac{1}{2},2+\frac{1}{2+\frac{1}{2}}+\ldots$收敛,并求出其极限.
    
    \newpage
    
    \item 设$f(x)=x+\ln{(2-x)}.$
    \begin{enumerate}
        \item [(1)] 求$f(x)$的最大值;
        \item [(2)] 若$x_1=\ln{2},x_{n+1}=f(x_n)(n=1,2,\ldots),$证明数列$\{x_n\}$收敛,并求出其极限.
    \end{enumerate}
    
    \newpage
    
    \item 
    \begin{enumerate}
        \item [(1)] 设$x_1>-6,x_{n+1}=\sqrt{6+x_n}(n=1,2,\ldots)$证明数列$\{x_n\}$收敛,并求出其极限.
        \item [(2)] 设$x_1>0,x_{n+1}=\frac{c(1+x_n)}{c+x_n}(n=1,2,\ldots),$其中$c>1$证明数列$\{x_n\}$收敛,并求出其极限.
    \end{enumerate}
    
    \newpage
    
    \item 求下列极限
    \begin{enumerate}
        \item [(1)] $\lim_{n\to\infty}\sqrt[n]{(1+1)^n+(1+\frac{1}{2})^{2n}+\ldots+(1+\frac{1}{n})^{n^2}}$
        \item [(2)] $\lim_{n\to\infty}\sqrt[n]{(n+1)+\sqrt{n^2+1}+\ldots+\sqrt[n]{n^n+1}}$
        \item [(3)] $\lim_{n\to\infty}\sqrt[n]{1+\sqrt{2}+\ldots+\sqrt[n]{n}}$
    \end{enumerate}
    
    \newpage
    
    \item 求下列极限
    \begin{enumerate}
        \item [(1)] $\lim_{n\to\infty}\left(\frac{2^{\frac{1}{n}}}{n+1}+\frac{2^{\frac{2}{n}}}{n+\frac{1}{2}}+\ldots+\frac{2^{\frac{n}{n}}}{n+\frac{1}{n}}\right)$
        \item [(2)] $\lim_{n\to\infty}\left(\frac{1}{n^2+n+1}+\frac{2}{n^2+n+2}+\ldots+\frac{n}{n^2+n+n^2}\right)$ 
        \item [(3)] $\lim_{n\to\infty}\sqrt{n}\left(1-\sum_{i=1}^{n}\frac{1}{n+\sqrt{i}}\right)$
    \end{enumerate}
    
    \newpage
    
    \item 
    \begin{enumerate}
        \item [(1)] 证明:当$x>0$时,$x-\frac{1}{2}x^2<\ln{(1+x)}<x$.
        \item [(2)] 求极限$\lim_{n\to\infty}\left(1+\frac{1}{n^2}\right)\left(1+\frac{2}{n^2}\right)\ldots\left(1+\frac{n}{n^2}\right)$
    \end{enumerate}
    
    \newpage
    
    \item 
    \begin{enumerate}
        \item [(1)] 求极限$\lim_{n\to\infty}\frac{\sqrt[n]{n!}}{n}$
        \item [(2)] 求极限$\lim_{n\to\infty}\frac{1}{n}\int_{0}^{\ln{n}}\left[e^x\right]\d x$其中$[x]$表示不超过x的最大整数
    \end{enumerate}
    
    \newpage
    
    \item 求下列极限
    \begin{enumerate}
        \item [(1)] $\lim_{n\to\infty}\sum_{i=1}^{n}\cos{\frac{(2i-1)\pi}{4n}}\cdot\frac{1}{n}$
        \item [(2)] $\lim_{n\to\infty}\sum_{i=1}^{n}\cos{\frac{(3i-1)\pi}{6n}}\cdot\frac{1}{n}$
        \item [(3)] $\lim_{n\to\infty}\sum_{i=1}^{n}\frac{i-\sin^2{i}}{n^2}\left[\ln{(n+i-\sin^2{i})-\ln{n}}\right]$
    \end{enumerate}
    
    \newpage
    
    \item 求下列极限
    \begin{enumerate}
        \item [(1)] $\lim_{n\to\infty}\sum_{i=n}^{2n}\frac{n}{i(n+i)}$
        \item [(2)] $\lim_{n\to\infty}\sum_{i=n+1}^{3n}\frac{n}{i(n+i)}$
    \end{enumerate}
    
    \newpage
    
    \item 
    \begin{enumerate}
        \item [(1)] 设$f(x)$在$x=0$与$x=1$处连续,满足$f(x^2)=f(x)$且$f(0)=0,$则$f(x)=\_\_\_\_$
        \item [(2)] 设$f(x)$在$[0,1]$上可导,满足$\left|f'(x)\right|\leq k\left|f(x)\right|(0<k<1),$且$f(0)=0$则$f(x)=\_\_\_\_$
    \end{enumerate}
    
    \newpage
    
    \item 判定下列函数的间断点及其类型
    \begin{enumerate}
        \item [(1)] $f(x)=\lim_{n\to\infty}\frac{\arctan{e^{nx}}}{x^{2n}+1}$
        \item [(2)] $f(x)=\lim_{n\to\infty}\frac{2e^{(n+1)x}+1}{e^{nx}+x^n+1}$
    \end{enumerate}
    
    \newpage
    
\end{enumerate}
\ifx\allfiles\undefined
\end{document}
\fi