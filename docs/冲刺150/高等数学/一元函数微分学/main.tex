\ifx\allfiles\undefined
\documentclass[12pt, a4paper, oneside, UTF8]{ctexbook}
\def\path{../config}
\usepackage{amsthm}
\usepackage{amssymb}
\usepackage{array}
\usepackage{xcolor}
\usepackage{graphicx}
\usepackage{mathrsfs}
\usepackage{enumitem}
\usepackage{geometry}
\usepackage[colorlinks, linkcolor=black]{hyperref}
\usepackage{stackengine}
\usepackage{yhmath}
\usepackage{extarrows}
\usepackage{tikz}
\usepackage{forest}
\usetikzlibrary{decorations.pathreplacing, positioning}
% \usepackage{unicode-math}
\usepackage{esint}
\usepackage{pifont}
\usepackage{tcolorbox}
\tcbuselibrary{skins, breakable}

\usepackage{multicol} 
\usepackage{fontspec} % 使用字体

\setmainfont{Times New Roman}
\setCJKmainfont{LXGWWenKai-Light}[
    SlantedFont=*
]

\usepackage{listings} % 用于插入代码

% 定义代码高亮风格
\lstset{
    basicstyle=\ttfamily\small,        % 基本字体样式(等宽小字体)
    keywordstyle=\color{blue},         % 关键字颜色
    commentstyle=\color{green},        % 注释颜色
    stringstyle=\color{red},           % 字符串颜色
    numbers=none,
    breaklines=true,                   % 自动换行
    frame=single,                      % 代码框边框
    rulecolor=\color{black},           % 边框颜色
    captionpos=b,                      % 标题位置(底部)
    showspaces=false,                  % 不显示空格标记
    showstringspaces=false,            % 不显示字符串中的空格标记
    language=C                         % 设置语言为 C
}

\usepackage{fontawesome5}

\usepackage{amsmath}
\usepackage{booktabs, array}
\usepackage{makecell}
\usepackage{fancyhdr}
\usepackage[dvipsnames, svgnames]{xcolor}
\usepackage{listings}
\usepackage{tasks}[2020/01/11]

\everymath{\displaystyle}

\definecolor{mygreen}{rgb}{0,0.6,0}
\definecolor{mygray}{rgb}{0.5,0.5,0.5}
\definecolor{mymauve}{rgb}{0.58,0,0.82}
\definecolor{NavyBlue}{RGB}{0,0,128}
\definecolor{Rhodamine}{RGB}{255,0,255}
\definecolor{PineGreen}{RGB}{0,128,0}

\graphicspath{ {figures/},{../figures/}, {config/}, {../config/} }

\linespread{1.6}

\geometry{
    top=25.4mm, 
    bottom=25.4mm, 
    left=20mm, 
    right=20mm, 
    headheight=2.17cm, 
    headsep=4mm, 
    footskip=12mm
}

\setenumerate[1]{itemsep=5pt,partopsep=0pt,parsep=\parskip,topsep=5pt}
\setitemize[1]{itemsep=5pt,partopsep=0pt,parsep=\parskip,topsep=5pt}
\setdescription{itemsep=5pt,partopsep=0pt,parsep=\parskip,topsep=5pt}



% \begin{lstlisting}[language=TeX] ... \end{lstlisting}

% 定理环境设置
% ---------- 颜色 ----------
\definecolor{ExBlue}{HTML}{4F81BD}
\definecolor{SolGreen}{HTML}{77933C}
\definecolor{DefRed}{HTML}{C5504B}
\definecolor{ThmOrange}{HTML}{E97132}
\definecolor{RemGray}{HTML}{7F7F7F}
\definecolor{CorPurple}{HTML}{7030A0}
\definecolor{ForGray}{HTML}{595959}

% ---------- 通用“变色”模板 ----------
\tcbset{
    mybox/.style n args={1}{
        enhanced, breakable,
        arc=6pt,
        boxrule=0.6pt,
        left=8pt, right=8pt, top=6pt, bottom=6pt,
        drop shadow={black!25},
        fonttitle=\bfseries,
        coltitle=white,
        colbacktitle=#1!85,
        colback=#1!10,
        colframe=#1,
    }
}

% ---------- 各环境 ----------
% 例题
\newtcolorbox{example}[1][]{mybox={ExBlue}, title={\ifstrempty{#1}{Example}{#1}}}
% 解答
\newtcolorbox{solution}[1][]{mybox={SolGreen}, title={\ifstrempty{#1}{Solution}{#1}}}
% 定义
\newtcolorbox{definition}[1][]{mybox={DefRed}, title={\ifstrempty{#1}{Definition}{#1}}}
% 定理
\newtcolorbox{theorem}[1][]{mybox={ThmOrange}, title={\ifstrempty{#1}{Theorem}{#1}}}
% 标注
\newtcolorbox{remark}[1][]{mybox={RemGray}, title={\ifstrempty{#1}{Remark}{#1}}}
% 推论
\newtcolorbox{corollary}[1][]{mybox={CorPurple}, title={\ifstrempty{#1}{Corollary}{#1}}}
% 公式
\newtcolorbox{formula}[1][]{mybox={ForGray}, title={\ifstrempty{#1}{Formula}{#1}}}


\settasks{
    label-format = \bfseries,
    label        = \Alph*.,
    label-width  = 1.2em,
    label-offset = 0.3em,
    item-indent  = 1.9em,
    column-sep   = 0.5em
}

\newenvironment{choices}[1][4]   % 默认 4 栏
    {\begin{tasks}(#1)}
    {\end{tasks}}

% 自定义命令的文件

\def\d{\mathrm{d}}
\def\R{\mathbb{R}}
\def\P{\partial} 
\newcommand{\bs}[1]{\begin{solution}#1\end{solution}}
\newcommand{\bt}[1][1]{% 默认参数为1
    \ensuremath{% 确保数学模式
        \foreach \n in {1,...,#1} {\blacktriangle}% 循环输出 #1 个黑色三角形
    }%
}

\newcommand{\bl}[1][1]{% 默认参数为1
    \ensuremath{% 确保数学模式
        \foreach \n in {1,...,#1} {\blacklozenge}% 循环输出 #1 个黑色三角形
    }%
}
\newif\ifshowanswers
%\showanswerstrue % 注释掉这行就不显示答案

% 定义答案环境
\newcommand{\answer}[1]{%
    \ifshowanswers
        #1%
    \fi
}




% 修改参数改变封面样式,0 默认原始封面、内置其他1、2、3种封面样式
\def\myIndex{3}


\ifnum\myIndex>0
    \input{\path/cover_package_\myIndex} 
\fi

\def\myTitle{冲刺150笔记}
\def\myAuthor{Weary Bird}
\def\myDateCover{\today}
\def\myDateForeword{\today}
\def\myForeword{行香子}
\def\myForewordText{
树绕村庄,水满陂塘;倚东风、豪兴徜徉。小园几许,收尽春光。有桃花红,李花白,菜花黄。 \\
远远苔墙,隐隐茅堂;飏青旗、流水桥旁。偶然乘兴,步过东冈。正莺儿啼,燕儿舞,蝶儿忙。 \\
}
\def\mySubheading{知错能改善莫大焉}


\begin{document}
% \input{../config/cover}
\else
\fi

\chapter{一元函数微分学}
\begin{enumerate}
    \item $x$为以 $T$ 为周期的连续函数,则下列结论中正确的个数为 ( ).
    \begin{enumerate}
        \item [(I)] ${\int }_{0}^{x}f\left( t\right) \mathrm{d}t$ 以 $T$ 为周期
        \item [(II)] ${\int }_{0}^{x}f\left( t\right) \mathrm{d}t - \frac{x}{T}{\int }_{0}^{T}f\left( t\right) \mathrm{d}t$ 以 $T$ 为周期
        \item [(III)] 若 $f\left( x\right)$ 为奇函数,则 ${\int }_{0}^{x}f\left( t\right) \mathrm{d}t$ 以 $T$ 为周期
        \item [(IV)] ${\int }_{0}^{x}\left\lbrack  {f\left( t\right)  - f\left( {-t}\right) }\right\rbrack  \mathrm{d}t$ 以 $T$ 为周期
        \item [(V)] 若${\int }_{0}^{+\infty }f\left( x\right) \mathrm{d}x$ 收敛,则 ${\int }_{0}^{x}f\left( t\right) \mathrm{d}t$ 以 $T$ 为周期
    \end{enumerate}
    \begin{choices}
        \task 1
        \task 2
        \task 3
        \task 4
    \end{choices}
    
    
    \newpage
    

    \item 设$f(x)$为$x$的三次多项式,且$\lim_{x \to 2a}\frac{f(x)}{x-2a}=1,\lim_{x\to 4a}\frac{f(x)}{x-4a}=1(a\neq 0)$,则$\lim_{x\to 3a}\frac{f(x)}{x-3a}=\_\_\_\_\_$
    
    \newpage
    
    \item 设$y=y(x)$为微分方程$y''+(x+1)y'+x^2y=e^x$满足初始条件$y(0)=0,y'(0)=1$的特解,若$\lim_{x\to 0}\frac{y(x)-x}{x^k}=c(c\neq 0)$则$c=\_\_\_\_\_,k=\_\_\_\_\_$
    
    \newpage
    
    \item 设$f(x)$在点$x=0$处三阶可导,且$f(0)=f'(0)=f''(0)=0,f'''(0)\neq 0$求极限$\lim_{x\to 0}\frac{\int_{0}^{x}tf(x-t)\d t}{x\int_{0}^{x}f(x-t)\d t}.$
    
    \newpage
    
    \item 
    \begin{enumerate}
        \item[(1)] 设$f(x),g(x)$连续,且$\lim_{x\to 0}\frac{f(x)}{g(x)}=1,\lim_{x\to x_0}\varphi(x)=0$证明当$x\to x_0$的时候,$\int_{0}^{\varphi(x)}f(t)\d t \sim \int_{0}^{\varphi(x)}g(t)\d t$
        \item[(2)] 求极限$\lim_{x\to 0}\frac{\int_{0}^{x^2}\ln{(1+2\tan{t})}}{\left[\int_{0}^{x}\ln{(1+2\tan{t}\d t)}^2\right]}$
    \end{enumerate}
    
    \newpage
    
    \item 设极限$\lim_{x\to 0}\frac{1}{x}\int_{-x}^{x}\left(1-\frac{\left|t\right|}{x}\right)\cos{(\theta - t)}\d t$存在,求$\theta$的值.
    
    \newpage
    
    \item 设$f(x)=(1+x)^{\frac{1}{x}},$当$x\to 0^+$时,$f(x)=e+Ax+Bx^2+o(x^2),$求$A,B$的值.
    
    \newpage
    
    \item 求下列极限
    \begin{enumerate}
        \item [(1)] $\lim_{x\to 0}\frac{\tan{\tan{x}}-\sin{\sin{x}}}{\tan{x}-\sin{x}}$
        \item [(2)] $\lim_{x\to 0}\frac{\sin\sin{x}-\sin\tan{x}}{x^2(\sqrt{1+x}-e^x)}$
        \item [(3)] $\lim_{x\to 0}\frac{\cos\sin{x} - \cos\tan{x}}{x^3(\sqrt{1+x}-e^x)}$
    \end{enumerate}
    
    \newpage
        
    \item 
    \begin{enumerate}
        \item [(1)] 设$\lim_{x\to 0}\frac{(1+\sin{2x^2})^{\frac{1}{x^2}}-e^2}{x^n}=a(a\neq 0)$求$a,n$
        \item [(2)] 设$\lim_{x\to 0}\frac{(1+\tan{3x^2})^{\frac{1}{x^2}}-e^3}{x^n}=a(a\neq 0)$求$a,n$
    \end{enumerate}
    
    \newpage
    
    \item 求下列极限
    $$
    \lim_{x\to 0}\frac{\sqrt{\frac{1+x}{1-x}}\sqrt[4]{\frac{1+2x}{1-2x}}\sqrt[6]{\frac{1+3x}{1-3x}}\ldots\sqrt[2n]{\frac{1+nx}{1-nx}}-1}{3\pi\arcsin{x}-(x^2+1)\arctan^3{x}}(n\geq 1)
    $$
    
    \newpage
    
    \item 求下列极限
    \begin{enumerate}
        \item [(1)] $\lim_{x\to 0}\left[\frac{1}{\ln{(x+\sqrt{1+x^2})}}-\frac{1}{\ln{(1+x)}}\right]$
        \item [(2)] $\lim_{x\to 0}\left[\frac{\ln{(x+\sqrt{1+x^2})}}{\ln{(1+x)}}\right]^{\frac{1}{\ln{(1+x)}}}$
    \end{enumerate}
    
    \newpage
    
    \item 求极限$\lim_{x\to +\infty}\left[\frac{x^{1+x}}{(1+x)^x}-\frac{x}{e}\right]$
    
    \newpage
    
    \item 设极限$\lim_{x\to +\infty}\left[\left(x^3-x^2+\frac{x}{2}\right)e^{\frac{1}{x}}-\sqrt{x^n+1}\right]$存在,求$n$的值并求出该极限.
    
    \newpage
        
    \item 设$f(x)$在$x=x_0$处二阶可导,且$f''(x_0)\neq 0$若$f(x)=f(x_0)+f'\left[x_0+\theta(x-x_0)\right](x-x_0)(0<\theta<1),$求$\lim_{x\to x_0}\theta$
    
    \newpage
    
    \item 证明数列$2,2+\frac{1}{2},2+\frac{1}{2+\frac{1}{2}}+\ldots$收敛,并求出其极限.
    
    \newpage
    
    \item 设$f(x)=x+\ln{(2-x)}.$
    \begin{enumerate}
        \item [(1)] 求$f(x)$的最大值;
        \item [(2)] 若$x_1=\ln{2},x_{n+1}=f(x_n)(n=1,2,\ldots),$证明数列$\{x_n\}$收敛,并求出其极限.
    \end{enumerate}
    
    \newpage
    
    \item 
    \begin{enumerate}
        \item [(1)] 设$x_1>-6,x_{n+1}=\sqrt{6+x_n}(n=1,2,\ldots)$证明数列$\{x_n\}$收敛,并求出其极限.
        \item [(2)] 设$x_1>0,x_{n+1}=\frac{c(1+x_n)}{c+x_n}(n=1,2,\ldots),$其中$c>1$证明数列$\{x_n\}$收敛,并求出其极限.
    \end{enumerate}
    
    \newpage
    
    \item 求下列极限
    \begin{enumerate}
        \item [(1)] $\lim_{n\to\infty}\sqrt[n]{(1+1)^n+(1+\frac{1}{2})^{2n}+\ldots+(1+\frac{1}{n})^{n^2}}$
        \item [(2)] $\lim_{n\to\infty}\sqrt[n]{(n+1)+\sqrt{n^2+1}+\ldots+\sqrt[n]{n^n+1}}$
        \item [(3)] $\lim_{n\to\infty}\sqrt[n]{1+\sqrt{2}+\ldots+\sqrt[n]{n}}$
    \end{enumerate}
    
    \newpage
    
    \item 求下列极限
    \begin{enumerate}
        \item [(1)] $\lim_{n\to\infty}\left(\frac{2^{\frac{1}{n}}}{n+1}+\frac{2^{\frac{2}{n}}}{n+\frac{1}{2}}+\ldots+\frac{2^{\frac{n}{n}}}{n+\frac{1}{n}}\right)$
        \item [(2)] $\lim_{n\to\infty}\left(\frac{1}{n^2+n+1}+\frac{2}{n^2+n+2}+\ldots+\frac{n}{n^2+n+n^2}\right)$ 
        \item [(3)] $\lim_{n\to\infty}\sqrt{n}\left(1-\sum_{i=1}^{n}\frac{1}{n+\sqrt{i}}\right)$
    \end{enumerate}
    
    \newpage
    
    \item 
    \begin{enumerate}
        \item [(1)] 证明:当$x>0$时,$x-\frac{1}{2}x^2<\ln{(1+x)}<x$.
        \item [(2)] 求极限$\lim_{n\to\infty}\left(1+\frac{1}{n^2}\right)\left(1+\frac{2}{n^2}\right)\ldots\left(1+\frac{n}{n^2}\right)$
    \end{enumerate}
    
    \newpage
    
    \item 
    \begin{enumerate}
        \item [(1)] 求极限$\lim_{n\to\infty}\frac{\sqrt[n]{n!}}{n}$
        \item [(2)] 求极限$\lim_{n\to\infty}\frac{1}{n}\int_{0}^{\ln{n}}\left[e^x\right]\d x$其中$[x]$表示不超过x的最大整数
    \end{enumerate}
    
    \newpage
    
    \item 求下列极限
    \begin{enumerate}
        \item [(1)] $\lim_{n\to\infty}\sum_{i=1}^{n}\cos{\frac{(2i-1)\pi}{4n}}\cdot\frac{1}{n}$
        \item [(2)] $\lim_{n\to\infty}\sum_{i=1}^{n}\cos{\frac{(3i-1)\pi}{6n}}\cdot\frac{1}{n}$
        \item [(3)] $\lim_{n\to\infty}\sum_{i=1}^{n}\frac{i-\sin^2{i}}{n^2}\left[\ln{(n+i-\sin^2{i})-\ln{n}}\right]$
    \end{enumerate}
    
    \newpage
    
    \item 求下列极限
    \begin{enumerate}
        \item [(1)] $\lim_{n\to\infty}\sum_{i=n}^{2n}\frac{n}{i(n+i)}$
        \item [(2)] $\lim_{n\to\infty}\sum_{i=n+1}^{3n}\frac{n}{i(n+i)}$
    \end{enumerate}
    
    \newpage
    
    \item 
    \begin{enumerate}
        \item [(1)] 设$f(x)$在$x=0$与$x=1$处连续,满足$f(x^2)=f(x)$且$f(0)=0,$则$f(x)=\_\_\_\_$
        \item [(2)] 设$f(x)$在$[0,1]$上可导,满足$\left|f'(x)\right|\leq k\left|f(x)\right|(0<k<1),$且$f(0)=0$则$f(x)=\_\_\_\_$
    \end{enumerate}
    
    \newpage
    
    \item 判定下列函数的间断点及其类型
    \begin{enumerate}
        \item [(1)] $f(x)=\lim_{n\to\infty}\frac{\arctan{e^{nx}}}{x^{2n}+1}$
        \item [(2)] $f(x)=\lim_{n\to\infty}\frac{2e^{(n+1)x}+1}{e^{nx}+x^n+1}$
    \end{enumerate}
    
    \newpage
    
\end{enumerate}
\ifx\allfiles\undefined
\end{document}
\fi