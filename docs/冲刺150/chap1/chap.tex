\ifx\allfiles\undefined
\documentclass[12pt, a4paper, oneside, UTF8]{ctexbook}
\def\path{../config}
\usepackage{amsthm}
\usepackage{amssymb}
\usepackage{array}
\usepackage{xcolor}
\usepackage{graphicx}
\usepackage{mathrsfs}
\usepackage{enumitem}
\usepackage{geometry}
\usepackage[colorlinks, linkcolor=black]{hyperref}
\usepackage{stackengine}
\usepackage{yhmath}
\usepackage{extarrows}
\usepackage{tikz}
\usepackage{forest}
\usetikzlibrary{decorations.pathreplacing, positioning}
% \usepackage{unicode-math}
\usepackage{esint}
\usepackage{pifont}
\usepackage{tcolorbox}
\tcbuselibrary{skins, breakable}

\usepackage{multicol} 
\usepackage{fontspec} % 使用字体

\setmainfont{Times New Roman}
\setCJKmainfont{LXGWWenKai-Light}[
    SlantedFont=*
]

\usepackage{listings} % 用于插入代码

% 定义代码高亮风格
\lstset{
    basicstyle=\ttfamily\small,        % 基本字体样式(等宽小字体)
    keywordstyle=\color{blue},         % 关键字颜色
    commentstyle=\color{green},        % 注释颜色
    stringstyle=\color{red},           % 字符串颜色
    numbers=none,
    breaklines=true,                   % 自动换行
    frame=single,                      % 代码框边框
    rulecolor=\color{black},           % 边框颜色
    captionpos=b,                      % 标题位置(底部)
    showspaces=false,                  % 不显示空格标记
    showstringspaces=false,            % 不显示字符串中的空格标记
    language=C                         % 设置语言为 C
}

\usepackage{fontawesome5}

\usepackage{amsmath}
\usepackage{booktabs, array}
\usepackage{makecell}
\usepackage{fancyhdr}
\usepackage[dvipsnames, svgnames]{xcolor}
\usepackage{listings}
\usepackage{tasks}[2020/01/11]

\everymath{\displaystyle}

\definecolor{mygreen}{rgb}{0,0.6,0}
\definecolor{mygray}{rgb}{0.5,0.5,0.5}
\definecolor{mymauve}{rgb}{0.58,0,0.82}
\definecolor{NavyBlue}{RGB}{0,0,128}
\definecolor{Rhodamine}{RGB}{255,0,255}
\definecolor{PineGreen}{RGB}{0,128,0}

\graphicspath{ {figures/},{../figures/}, {config/}, {../config/} }

\linespread{1.6}

\geometry{
    top=25.4mm, 
    bottom=25.4mm, 
    left=20mm, 
    right=20mm, 
    headheight=2.17cm, 
    headsep=4mm, 
    footskip=12mm
}

\setenumerate[1]{itemsep=5pt,partopsep=0pt,parsep=\parskip,topsep=5pt}
\setitemize[1]{itemsep=5pt,partopsep=0pt,parsep=\parskip,topsep=5pt}
\setdescription{itemsep=5pt,partopsep=0pt,parsep=\parskip,topsep=5pt}



% \begin{lstlisting}[language=TeX] ... \end{lstlisting}

% 定理环境设置
% ---------- 颜色 ----------
\definecolor{ExBlue}{HTML}{4F81BD}
\definecolor{SolGreen}{HTML}{77933C}
\definecolor{DefRed}{HTML}{C5504B}
\definecolor{ThmOrange}{HTML}{E97132}
\definecolor{RemGray}{HTML}{7F7F7F}
\definecolor{CorPurple}{HTML}{7030A0}
\definecolor{ForGray}{HTML}{595959}

% ---------- 通用“变色”模板 ----------
\tcbset{
    mybox/.style n args={1}{
        enhanced, breakable,
        arc=6pt,
        boxrule=0.6pt,
        left=8pt, right=8pt, top=6pt, bottom=6pt,
        drop shadow={black!25},
        fonttitle=\bfseries,
        coltitle=white,
        colbacktitle=#1!85,
        colback=#1!10,
        colframe=#1,
    }
}

% ---------- 各环境 ----------
% 例题
\newtcolorbox{example}[1][]{mybox={ExBlue}, title={\ifstrempty{#1}{Example}{#1}}}
% 解答
\newtcolorbox{solution}[1][]{mybox={SolGreen}, title={\ifstrempty{#1}{Solution}{#1}}}
% 定义
\newtcolorbox{definition}[1][]{mybox={DefRed}, title={\ifstrempty{#1}{Definition}{#1}}}
% 定理
\newtcolorbox{theorem}[1][]{mybox={ThmOrange}, title={\ifstrempty{#1}{Theorem}{#1}}}
% 标注
\newtcolorbox{remark}[1][]{mybox={RemGray}, title={\ifstrempty{#1}{Remark}{#1}}}
% 推论
\newtcolorbox{corollary}[1][]{mybox={CorPurple}, title={\ifstrempty{#1}{Corollary}{#1}}}
% 公式
\newtcolorbox{formula}[1][]{mybox={ForGray}, title={\ifstrempty{#1}{Formula}{#1}}}


\settasks{
    label-format = \bfseries,
    label        = \Alph*.,
    label-width  = 1.2em,
    label-offset = 0.3em,
    item-indent  = 1.9em,
    column-sep   = 0.5em
}

\newenvironment{choices}[1][4]   % 默认 4 栏
    {\begin{tasks}(#1)}
    {\end{tasks}}

% 自定义命令的文件

\def\d{\mathrm{d}}
\def\R{\mathbb{R}}
\def\P{\partial} 
\newcommand{\bs}[1]{\begin{solution}#1\end{solution}}
\newcommand{\bt}[1][1]{% 默认参数为1
    \ensuremath{% 确保数学模式
        \foreach \n in {1,...,#1} {\blacktriangle}% 循环输出 #1 个黑色三角形
    }%
}

\newcommand{\bl}[1][1]{% 默认参数为1
    \ensuremath{% 确保数学模式
        \foreach \n in {1,...,#1} {\blacklozenge}% 循环输出 #1 个黑色三角形
    }%
}
\newif\ifshowanswers
%\showanswerstrue % 注释掉这行就不显示答案

% 定义答案环境
\newcommand{\answer}[1]{%
    \ifshowanswers
        #1%
    \fi
}




% 修改参数改变封面样式,0 默认原始封面、内置其他1、2、3种封面样式
\def\myIndex{3}


\ifnum\myIndex>0
    \input{\path/cover_package_\myIndex} 
\fi

\def\myTitle{冲刺150笔记}
\def\myAuthor{Weary Bird}
\def\myDateCover{\today}
\def\myDateForeword{\today}
\def\myForeword{行香子}
\def\myForewordText{
树绕村庄,水满陂塘;倚东风、豪兴徜徉。小园几许,收尽春光。有桃花红,李花白,菜花黄。 \\
远远苔墙,隐隐茅堂;飏青旗、流水桥旁。偶然乘兴,步过东冈。正莺儿啼,燕儿舞,蝶儿忙。 \\
}
\def\mySubheading{知错能改善莫大焉}


\begin{document}
% \input{../config/cover}
\else
\fi

\chapter{高等数学 第二讲}
\begin{example}(莫斯科1975年竞赛题)
证明数列 $ 2,2+\frac{1}{2},2+\frac{1}{2+\frac{1}{2+\frac{1}{2}}} \cdots $ 收敛,并求其极限。
\end{example}

\begin{example}
设 $ f(x)=x+\ln(2-x) $.
\begin{enumerate}
    \item[(\uppercase\expandafter{\romannumeral1})] 求 $ f(x) $ 的最大值;
    \item[(\uppercase\expandafter{\romannumeral2})] 若 $ x_{1}=\ln 2,x_{n+1}=f(x_{n})(n=1,2,\cdots) $,证明数列 $ \{x_{n}\} $ 收敛,并求其极限。
\end{enumerate}
\end{example}

\begin{example}
\begin{enumerate}
    \item[(1)] 设 $ x_{1}>-6,x_{n+1}=\sqrt{6+x_{n}}(n=1,2,\cdots) $,证明数列 $ \{x_{n}\} $ 收敛,并求其极限。
    \item[(2)] (南京大学2000年,武汉大学2004年,天津大学2004年,浙江大学2007年) 设 $ x_{1}>0,x_{n+1}=\frac{c(1+x_{n})}{c+x_{n}}(n=1,2,\cdots) $,其中 $ c>1 $,证明数列 $ \{x_{n}\} $ 收敛,并求其极限。
\end{enumerate}
\end{example}

\begin{example}求下列极限:
\begin{enumerate}
    \item[(1)] $ \lim_{n\to\infty}\sqrt{(1+1)^{n}+\left(1+\frac{1}{2}\right)^{2n}+\cdots+\left(1+\frac{1}{n}\right)^{n^{2}}} $.
    \item[(2)] $ \lim_{n\to\infty}\sqrt[n]{(n+1)+\sqrt{n^{2}+1}+\cdots+\sqrt[n]{n^{n}+1}} $.
    \item[(3)] $ \lim _{n\rightarrow\infty}\sqrt[n]{1+\sqrt{2}+\cdots+\sqrt[n]{n}} $.
\end{enumerate}
\end{example}

\begin{example}求下列极限:
\begin{enumerate}
    \item[(1)] (莫斯科1976年竞赛题) $ \lim_{n\rightarrow\infty}\left(\frac{2^{\frac{1}{n}}}{n+1}+\frac{2^{\frac{2}{n}}}{n+\frac{1}{2}}+\cdots+\frac{2^{\frac{n}{n}}}{n+\frac{1}{n}}\right) $.
    \item[(2)] $ \lim_{n\rightarrow\infty}\left(\frac{1}{n^{2}+n+1}+\frac{2}{n^{2}+n+2^{2}}+\cdots+\frac{n}{n^{2}+n+n^{2}}\right) $.
    \item[(3)] (第十一届中国大学生数学竞赛题,2020年) $ \lim_{n\rightarrow\infty}\sqrt{n}\left(1-\sum_{i=1}^{n}\frac{1}{n+\sqrt{i}}\right) $.
\end{enumerate}
\end{example}

\begin{example}
\begin{enumerate}
    \item[(1)] 证明:当 $ x>0 $ 时,$ x-\frac{1}{2}x^{2}<\ln(1+x)<x $.
    \item[(2)] 求极限 $ \lim_{n\to\infty}(1+\frac{1}{n^{2}})(1+\frac{2}{n^{2}})\cdots(1+\frac{n}{n^{2}}) $.
\end{enumerate}
\end{example}

\begin{example}
\begin{enumerate}
    \item[(1)] 求极限 $ \lim_{n\to\infty}\frac{\sqrt[n]{n!}}{n} $.
    \item[(2)] 求极限 $ \lim_{n\to\infty}\frac{1}{n}\int_{0}^{\ln n}\left[e^{x}\right] dx $,其中 $ [x] $ 表示不超过 $ x $ 的最大整数.
\end{enumerate}
\end{example}

\begin{example}【例1.22】求下列极限:
\begin{enumerate}
    \item[(1)] $ \lim_{n\to\infty}\sum_{i=1}^{n}\cos\frac{(2i-1)\pi}{4 n}\cdot\frac{1}{n} $.
    \item[(2)] $ \lim_{n\to\infty}\sum_{i=1}^{n}\cos\frac{(3i-1)\pi}{6 n}\cdot\frac{1}{n} $.
    \item[(3)] (浙江省高等数学竞赛题,2013年) $ \lim_{n\to\infty}\sum_{i=1}^{n}\frac{i-\sin^{2}i}{n^{2}}\left[\ln\left(n+i-\sin^{2}i\right)-\ln n\right] $.
\end{enumerate}
\end{example}

\begin{example}(浙江省高等数学竞赛题,2009年)求下列极限:
\begin{enumerate}
    \item[(1)] $ \lim_{n\to\infty}\sum_{i=n}^{2 n}\frac{n}{i(n+i)} $.
    \item[(2)] $ \lim_{n\to\infty}\sum_{i=n+1}^{3n}\frac{n}{i(n+i)} $.
\end{enumerate}
\end{example}


\ifx\allfiles\undefined
\end{document}
\fi