\ifx\allfiles\undefined
\documentclass[12pt, a4paper, oneside, UTF8]{ctexbook}
\usepackage{ctex}
\usepackage{verse}
\def\path{../config}
\usepackage{amsthm}
\usepackage{amssymb}
\usepackage{array}
\usepackage{xcolor}
\usepackage{graphicx}
\usepackage{mathrsfs}
\usepackage{enumitem}
\usepackage{geometry}
\usepackage[colorlinks, linkcolor=black]{hyperref}
\usepackage{stackengine}
\usepackage{yhmath}
\usepackage{extarrows}
\usepackage{tikz}
\usepackage{forest}
\usetikzlibrary{decorations.pathreplacing, positioning}
% \usepackage{unicode-math}
\usepackage{esint}
\usepackage{pifont}
\usepackage{tcolorbox}
\tcbuselibrary{skins, breakable}

\usepackage{multicol} 
\usepackage{fontspec} % 使用字体

\setmainfont{Times New Roman}
\setCJKmainfont{LXGWWenKai-Light}[
    SlantedFont=*
]

\usepackage{listings} % 用于插入代码

% 定义代码高亮风格
\lstset{
    basicstyle=\ttfamily\small,        % 基本字体样式(等宽小字体)
    keywordstyle=\color{blue},         % 关键字颜色
    commentstyle=\color{green},        % 注释颜色
    stringstyle=\color{red},           % 字符串颜色
    numbers=none,
    breaklines=true,                   % 自动换行
    frame=single,                      % 代码框边框
    rulecolor=\color{black},           % 边框颜色
    captionpos=b,                      % 标题位置(底部)
    showspaces=false,                  % 不显示空格标记
    showstringspaces=false,            % 不显示字符串中的空格标记
    language=C                         % 设置语言为 C
}

\usepackage{fontawesome5}

\usepackage{amsmath}
\usepackage{booktabs, array}
\usepackage{makecell}
\usepackage{fancyhdr}
\usepackage[dvipsnames, svgnames]{xcolor}
\usepackage{listings}
\usepackage{tasks}[2020/01/11]

\everymath{\displaystyle}

\definecolor{mygreen}{rgb}{0,0.6,0}
\definecolor{mygray}{rgb}{0.5,0.5,0.5}
\definecolor{mymauve}{rgb}{0.58,0,0.82}
\definecolor{NavyBlue}{RGB}{0,0,128}
\definecolor{Rhodamine}{RGB}{255,0,255}
\definecolor{PineGreen}{RGB}{0,128,0}

\graphicspath{ {figures/},{../figures/}, {config/}, {../config/} }

\linespread{1.6}

\geometry{
    top=25.4mm, 
    bottom=25.4mm, 
    left=20mm, 
    right=20mm, 
    headheight=2.17cm, 
    headsep=4mm, 
    footskip=12mm
}

\setenumerate[1]{itemsep=5pt,partopsep=0pt,parsep=\parskip,topsep=5pt}
\setitemize[1]{itemsep=5pt,partopsep=0pt,parsep=\parskip,topsep=5pt}
\setdescription{itemsep=5pt,partopsep=0pt,parsep=\parskip,topsep=5pt}



% \begin{lstlisting}[language=TeX] ... \end{lstlisting}

% 定理环境设置
% ---------- 颜色 ----------
\definecolor{ExBlue}{HTML}{4F81BD}
\definecolor{SolGreen}{HTML}{77933C}
\definecolor{DefRed}{HTML}{C5504B}
\definecolor{ThmOrange}{HTML}{E97132}
\definecolor{RemGray}{HTML}{7F7F7F}
\definecolor{CorPurple}{HTML}{7030A0}
\definecolor{ForGray}{HTML}{595959}

% ---------- 通用“变色”模板 ----------
\tcbset{
    mybox/.style n args={1}{
        enhanced, breakable,
        arc=6pt,
        boxrule=0.6pt,
        left=8pt, right=8pt, top=6pt, bottom=6pt,
        drop shadow={black!25},
        fonttitle=\bfseries,
        coltitle=white,
        colbacktitle=#1!85,
        colback=#1!10,
        colframe=#1,
    }
}

% ---------- 各环境 ----------
% 例题
\newtcolorbox{example}[1][]{mybox={ExBlue}, title={\ifstrempty{#1}{Example}{#1}}}
% 解答
\newtcolorbox{solution}[1][]{mybox={SolGreen}, title={\ifstrempty{#1}{Solution}{#1}}}
% 定义
\newtcolorbox{definition}[1][]{mybox={DefRed}, title={\ifstrempty{#1}{Definition}{#1}}}
% 定理
\newtcolorbox{theorem}[1][]{mybox={ThmOrange}, title={\ifstrempty{#1}{Theorem}{#1}}}
% 标注
\newtcolorbox{remark}[1][]{mybox={RemGray}, title={\ifstrempty{#1}{Remark}{#1}}}
% 推论
\newtcolorbox{corollary}[1][]{mybox={CorPurple}, title={\ifstrempty{#1}{Corollary}{#1}}}
% 公式
\newtcolorbox{formula}[1][]{mybox={ForGray}, title={\ifstrempty{#1}{Formula}{#1}}}


\settasks{
    label-format = \bfseries,
    label        = \Alph*.,
    label-width  = 1.2em,
    label-offset = 0.3em,
    item-indent  = 1.9em,
    column-sep   = 0.5em
}

\newenvironment{choices}[1][4]   % 默认 4 栏
    {\begin{tasks}(#1)}
    {\end{tasks}}

% 自定义命令的文件

\def\d{\mathrm{d}}
\def\R{\mathbb{R}}
\def\P{\partial} 
\newcommand{\bs}[1]{\begin{solution}#1\end{solution}}
\newcommand{\bt}[1][1]{% 默认参数为1
    \ensuremath{% 确保数学模式
        \foreach \n in {1,...,#1} {\blacktriangle}% 循环输出 #1 个黑色三角形
    }%
}

\newcommand{\bl}[1][1]{% 默认参数为1
    \ensuremath{% 确保数学模式
        \foreach \n in {1,...,#1} {\blacklozenge}% 循环输出 #1 个黑色三角形
    }%
}
\newif\ifshowanswers
%\showanswerstrue % 注释掉这行就不显示答案

% 定义答案环境
\newcommand{\answer}[1]{%
    \ifshowanswers
        #1%
    \fi
}




% 修改参数改变封面样式,0 默认原始封面、内置其他1、2、3种封面样式
\def\myIndex{3}


\ifnum\myIndex>0
    \input{\path/cover_package_\myIndex} 
\fi

\def\myTitle{冲刺150笔记}
\def\myAuthor{Weary Bird}
\def\myDateCover{\today}
\def\myDateForeword{\today}
\def\myForeword{行香子}
\def\myForewordText{
树绕村庄,水满陂塘;倚东风、豪兴徜徉。小园几许,收尽春光。有桃花红,李花白,菜花黄。 \\
远远苔墙,隐隐茅堂;飏青旗、流水桥旁。偶然乘兴,步过东冈。正莺儿啼,燕儿舞,蝶儿忙。 \\
}
\def\mySubheading{知错能改善莫大焉}


\begin{document}
% \input{../config/cover}
\else
\fi
\chapter{考研政治}
\section{马克思主义基本原理}
\begin{enumerate}
    \item (单选)进入21世纪以来,社会化大生产在世界范围内更大规模、
    更广范围、更深层次展开,世界格局深度调整,资本主义呈现一些新变化新特征.
    当代资本主义最突出、最鲜明、最主要的特征是(   ) \\
    A. 输出、渗透资本主义价值观 \qquad
    B. 工人阶级内部层级结构逐渐分化 \\
    C. 国际金融资本的垄断 \qquad\qquad\quad
    D. 科技创新加速资本主义生产方式变化

    \item 垄断是在自由竞争中形成的,是作为自由竞争的对立面产生的.
    但是,垄断并不能消除竞争,反而使竞争更加复杂和剧烈.这是因为(   ) \\
    A. 垄断没有改变生产资料的资本主义私有制 \\
    B. 垄断企业必须不断增强自己的实力,巩固自己的垄断地位\\
    C. 如果竞争不复存在,垄断企业就没有动力和压力壮大自己的实力\\
    D. 垄断企业不可能把全部社会生产都包下来

    \item 国家垄断资本主义是国家政权和私人垄断资本融合在一起的垄断资本主义.第二次世界大战结束以来,
    在国家垄断资本主义获得充分发展的同时,资本主义国家通过宏观调节和微观规制对生产,流通,分配和消费各个环节
    的干预也更加加深.其中,微观规制的类型主要有(   ) \\
    A.社会经济规制\qquad B.公共事业规制 \\
    C.公共生活规制\qquad D.反托拉斯法 

    \item 20世纪80年代以来,随着冷战的结束,分割的世界经济体系也随之被打破,技术、资本、商品等真
    正实现了全球范围的流动,各国之间的经济联系日益密切,相互合作、相互依存大大加强,世界进入到经济全
    球化迅猛发展的新时代.促进经济全球化迅猛发展的因素有(   ) \\
    A.各国经济体制变革给出的有利制度条件 \\
    B.出现了适宜于全球化的企业组织形式 \\
    C.企业不断进行的技术创新与管理创新 \\
    D.科学技术的进步和生产力的快速发展

    \item 第二次世界大战结束后,资本主义国家对经济进行的干预明显加强,从而使得资本主义社会的经济调节机制发生了显著变化.
    与这种变化相适应,资本主义政治制度也发生了很大变化.其主要表现包括(   ) \\
    A. 政治制度出现多元化的趋势 \qquad
    B. 法治建设得到重视和加强 \\
    C. 社会阶层和阶级结构的变化 \qquad
    D. 改良主义政党的影响日益扩大

    \item 放眼当今世界,新一轮科技革命和产业变革深入发展,国际力量对比深刻调整,中国发展奇迹同西方资本主义的衰落形成了鲜明对比,“东升西降”已成为百年变局中的大势所趋.
    进入21世纪以来,由于错综复杂的原因,当代资本主义呈现一些不同以往的变化态势与特点.
    这些新变化新特征(   ) \\
    A. 并未改变资本主义的经济基础和追求利润最大化的本性\\
    B. 不断引发世界范围内对资本主义制度和价值观的质疑\\
    C. 使得资本主义的基本矛盾发生了根本性改变\\
    D. 不断诱发更激烈的世界性问题和全球性矛盾

    \item 《共产党宣言》指出:“一切所有制关系都经历了经常的历史更替、经常的历史变更以及社会主义必然代替资本主义,这是历史发展的客观规律,
    也是科学社会主义最基本的结论.”资本主义为社会主义所代替的历史必然性的依据有(   ) \\
    A. 资本主义基本矛盾“包含着现代的一切冲突的萌芽” \\
    B. 资本积累推动资本主义基本矛盾不断激化并最终否定资本主义自身 \\
    C. 国家垄断资本主义是资本社会化的更高形式,将成为社会主义的前奏 \\
    D. 资本主义社会存在着资产阶级和无产阶级两大阶级之间的矛盾和斗争

    \item (单选)恩格斯指出:“我认为,所谓‘社会主义社会’不是一种一成不变的东西,
    而应当和任何其他社会制度一样,把它看成经常变化和改革的社会.”社会主义改革的根源是(   ) \\
    A. 改革是社会主义社会发展的动力 \\
    B. 社会生产力发展水平不够高 \\
    C. 社会主义制度没有根本克服资本主义制度下生产力与生产关系的对抗性矛盾 \\
    D. 社会主义社会的基本矛盾 

    \item (单选)列宁指出,不能“为死教条而牺牲活的马克思主义”.习近平全面总结社会主义历史进程,
    得出“社会主义从来都是在开拓中前进的”.这些表述对我们的深刻启示是(   ) \\
    A. 必须始终“坚持科学社会主义基本原则” \\
    B. 要把科学社会主义基本原则与本国实际相结合 \\
    C. 科学社会主义基本原则要紧跟时代和实践的发展而发展 \\
    D. 时代和实践的不断发展使社会主义发展道路具有多样性

    \item 习近平指出:当代中国的伟大社会变革,不是简单延续我国历史文化的母版,
    不是简单套用马克思主义经典作家设想的模板,不是其他国家社会主义实践的再版,
    不是国外现代化发展的翻版.这对我们理解科学社会主义一般原则的启示是(   ) \\
    A. 科学社会主义是人类优秀文化传统的历史延续 \\
    B. 科学社会主义与资本主义生产方式没有必然的联系 \\
    C. 科学社会主义绝不是一成不变的教条 \\
    D. 科学社会主义在不同的时代具有不同的内容和形式 

    \item 世界上没有放之四海而皆准的发展道路和发展模式,也没有一成不变的发展道路和发展模式.
    30多年前,印有五角星和镰刀锤头的红旗在克里姆林宫上空悄然滑落,社会主义阵营老大哥消失,西亚北非地区陷入动荡.如今,中国共产党已然走过100多个春秋,中国特色社会主义
    比任何时期都要焕发生机与活力,社会主义发展的生机悄然而至.历史证明,
    社会主义之所以在曲折中发展,是因为(   ) \\
    A. 社会主义作为新生事物,其成长不会一帆风顺 \\
    B. 经济全球化对于社会主义的发展既有机遇又有挑战 \\
    C. 社会主义社会的基本矛盾推动社会发展,是作为一个过程而展开的 \\
    D. 各国历史文化传统的差异性决定社会主义发展方向

    \item 资本主义必然为社会主义所代替,并不意味着资本主义将在短期内自行消亡.
    资本主义向社会主义的过渡必然是一个复杂、长期的历史进程,其原因在于(   ) \\
    A. 资本主义社会具有一定的自我调节能力 \\
    B. 资本主义的发展具有不平衡性 \\
    C. 任何社会形态的存在都有绝对稳定性 \\
    D. 当代资本主义的发展还显示出生产关系对生产力容纳的空间

    \item (单选)共产主义社会是人类社会发展的最高社会形态,这一社会实现的必要条件是(   ) \\
    A.社会关系的高度和谐 \qquad
    B.人自由而全面的发展 \\
    C.生产力的高度发展 \qquad
    D.阶级和国家的消亡

    \item 马克思主义最崇高的社会理想是实现共产主义社会,即实现(   ) \\
    A.无矛盾的和谐社会 \qquad
    B.物质财富极大丰富 \\
    C.人们精神境界极大提高 \qquad
    D.每个人自由而全面的发展

    \item 马克思在表述共产主义社会的基本特征时指出,共产主义社会是社会关系高度和谐,
    人们精神境界极大提高的社会.社会关系的高度和谐表现在(   ) \\
    A.国家消亡\qquad
    B.阶级消亡 \\
    C.工业与农业、城市与乡村、脑力劳动与体力劳动的差别——"三大差别"消失 \\
    D.人、自然及社会都达成和谐 

    \item 马克思、恩格斯在《共产党宣言》中明确提出:
    "资产阶级的灭亡和无产阶级的胜利是同样不可避免的.""资本主义必然灭亡,社会主义必然胜利
    "是科学社会主义的核心命题.这"两个必然"是他们研究人类历史发展,
    特别是资本主义历史发展所得出的基本结论.这一科学论断(   ) \\
    A.在科学社会主义理论与实践中具有首要和基础的地位 \\
    B.是共产主义理想信念的核心要义\\
    C.是马克思主义追求的根本价值目标\\
    D.揭示了人类社会从资本主义向社会主义转变的历史必然性
\end{enumerate}
\section{思道法} 

\begin{enumerate}
    \item (单选) 人的生命是有限的,但生命的意义和价值却可以不同.实现人生价值的根本途径是  \\
    A.培养积极进取的人生态度   \\
    B.自觉提高自我的主体素质和能力   \\
    C.正确认识自我价值和社会价值的关系  \\ 
    D.进行有意识、有目的的创造性实践活动  

    \item 导弹技术专家沈忠芳隐姓埋名60多载,直到2022年4月中国航天科工集团二院正式发布《导弹人生》一书,
    才首次向全社会公开12位此前隐姓埋名的中国导弹武器型号总指挥、总设计师,沈忠芳在列.《感动中国》组委会给予沈忠芳的颁奖词这样写道:"从无到有,从近到远,从长缨在手,到红旗如画.这一代人从没有在乎过自己的得与失,这一代人一辈子都在砺国家的剑与盾.
    今天,后辈们终于能听到你们的传奇,隐秘而伟大,平静而神圣."这对们的人生启示是 \\
    A. 评价人生价值的根本尺度,是看一个人的实践活动是否符合社会发展的客观规律,是否进了历史的进步 \\
    B. 社会价值的实现总是以个人价值的牺牲为代价 \\
    C. 社会对于个人的价值评判主要是以个人对国家和社会所作的奉献为衡量标准 \\
    D. 人生社会价值的实现是个体自我完善、全面发展的保障

    \item (单选) 信念是认知、情感和意志的有机统一体,是人们在一定的认识基础上确立的对某种思想或事物坚信不疑并
    身体力行的精神状态.信念是人们追求理想目标的强大动力,决定事业的成败.信念有不同的层次和类型,其中 \\
    A. 高层次的信念决定低层次的信念   \\
    B. 低层次的信念代表了一个人的基本信仰   \\
    C. 相同社会环境中生活的人们的信念始终一致   \\
    D. 各种信念没有科学与非科学之分 

    \item (单选) "立志当高远,立志做大事."大量的事实告诉我们,那些在事业上取得伟大成就、对人类作出
    卓越贡献的人,都是在青年时期就立下了鸿鹄之志,并为之坚持不懈、努力奋斗.下列名言能体现这一说法的是 \\
    A. "功崇惟志,业广惟勤" \qquad
    B. "夙夜在公" \\
    C. "己所不欲,勿施于人" \qquad
    D. "己欲立而立人,己欲达而达人"

    \item 周恩来就读东关模范学校时,正值中国社会发生剧烈变动的时期.
    校长亲自为学生上修身课,题目是"立命".校长讲到精彩处突然停顿下来,
    问道:"诸生为何读书啊?"有人回答为名利而读书,有人回答为做官而读书.12岁的周恩来响亮地回答:
    "为中华之崛起而读书."校长赞叹道:"有志者,当效周生啊!"周总理的故事告诉我们
    要正确处理好个人理想和社会理想的关系,就要认识到 \\
    A.要坚持个人理想与社会理想的统一,在为实现社会理想而奋斗的过程中实现个人理想 \\
    B.个人理想以社会理想为指引,社会理想是对个人理想的凝练和升华 \\
    C.社会理想是最根本、最重要的,个人理想从属于社会理想 \\
    D.社会理想的实现必须以个人理想的实现为前提和基础 
\end{enumerate}
\section{毛中特} 

\section{史纲}

\begin{enumerate}
    \item 认识中国近代一切社会问题和革命问题的最基本的依据是(\qquad)
    \begin{choices}[1]
    \task 反帝反封建的革命任务
    \task 近代中国的基本国情
    \task 资本—帝国主义的侵略
    \task 民族资产阶级的软弱性
    \end{choices}
    \item 《四洲志》是一部世界地理著作,该书简要叙述了世界五大洲30多个国家的地理、历史和政治状况,是近代中国第一部相对完整、比较系统的世界地理志书。在此基础上,后人编写了《海国图志》。组织编写《四洲志》,且为近代中国睁眼看世界的第一人的是
    \begin{choices}
    \task 林则徐
    \task 魏源
    \task 马建忠
    \task 郑观应
    \end{choices}
    \item 在近代,王韬首次提出“变法”的主张,他在介绍西方国家的“君主”“民主”“君民共主”这三种制度时,最早提出废除封建君主专制,建立“与众民共政事并治天下”的君主立宪制。该思想
    \begin{choices}[1]
    \task 最早提出发展资本主义
    \task 推动了维新变法的兴起
    \task 反思了当时中国近代化的问题
    \task 引入了社会进化论的思想
    \end{choices}

    \item \bl 鸦片战争以后,西方列强通过发动侵略战争,强迫中国签订了一系列不平等条约,使中国沦为半殖民地半封建社会。中国逐步沦为半殖民地社会的原因包括
    \begin{choices}[1]
    \task 中国已经丧失了完全独立的地位
    \task 列强把中国卷入世界资本主义经济体系和世界市场之中
    \task 中国仍然维持着独立国家和政府的名义
    \task 西方列强不愿意中国成为独立的资本主义国家
    \end{choices}

    \item \bl 资本—帝国主义列强在对中国实行军事侵略、政治控制、经济掠夺的同时,还对中国进行文化渗透,具体包括
    \begin{choices}[2]
    \task 以传教为掩护进行间谍活动
    \task 创办《万国公报》
    \task 炮制“中国威胁论”
    \task 允许外国公使常驻北京
    \end{choices}

    \item \bl 魏源在其所著的《海国图志》一书中提出了“师夷长技以制夷”的思想,19世纪60年代开始的洋务运动则提出了“自强”“求富”的口号。二者的相同点包括
    \begin{choices}[1]
    \task 有内在的一致性和继承性
    \task 都有抵御外来侵略的意图
    \task 主要体现了地主阶级的要求
    \task 意识到了中国落后挨打的根本原因
    \end{choices}

    \item \bl 1894年7月丰岛海战后,中日两国相互宣战,战至1895年2月北洋海军全军覆没,中日两国各自改革30年后的决战以清政府惨败而告终,1895年4月清政府被迫签署《马关条约》。这场战争对中国产生了极其严重的后果,表现在
    \begin{choices}[1]
    \task 清政府已经彻底沦为“洋人的朝廷”
    \task 中断了清政府通过洋务运动向近代化转型的努力
    \task 中国付出了丧失领土主权的极大代价
    \task 清政府损失了海军主力
    \end{choices}

    \item 中国民族资产阶级登上政治舞台的第一次表演是
    \begin{choices}
    \task 洋务运动
    \task 戊戌维新运动
    \task 辛亥革命
    \task 五四运动
    \end{choices}

    \item \bl 在向西方学习的过程中.戊戌维新运动与洋务运动的不同点在于
    \begin{choices}[2]
    \task 学习西方的科学技术
    \task 学习西方的政治制度
    \task 批判封建的伦理道德
    \task 主张采取君主立宪制
    \end{choices}

    \item 1904年.孙中山发表了《中国问题的真解决》一文.指出只有推翻清政府的统治.“一个新的、开明的、进步的政府来代替旧政府”“把过时的满清君主政体改变为‘中华民国’”.才能真正解决中国问题。这表明革命派与改良派的根本不同之处是
    \begin{choices}[2]
    \task 平均地权.进行社会革命
    \task 兴民权.实行君主立宪
    \task 推翻帝制.实行共和
    \task 武装起义.推翻清王朝统治
    \end{choices}

    \item 1911年10月10日晚.驻武昌的新军工程第八营的革命党人打响了起义的第一枪。起义军一夜之间占领武昌.取得首义的胜利。武昌起义
    \begin{choices}[1]
    \task 是同盟会成立后发动的第一次武装起义
    \task 使得在中国延续了两千多年的封建帝制终于覆灭
    \task 拉开了中国完全意义上的近代民族民主革命的序幕
    \task 使得革命势力发展到了长江流域和黄河流域的大部分地区
    \end{choices}

    \item 中国近代思想主要经历了“师夷长技以制夷”“中体西用”“维新变法”“民主共和”“民主与科学”及“马克思主义”的演进过程。这些思想反映的共同主题是
    \begin{choices}[2]
    \task 发展资本主义
    \task 救亡图存
    \task 否定封建文化
    \task 民族独立、人民解放
    \end{choices}
    
    \item \bl 辛亥革命为中国的进步打开了闸门.为中国共产党的诞生准备了客观社会条件。辛亥革命给中国共产党的成立准备了
    \begin{choices}[2]
    \task 干部条件
    \task 理论基础
    \task 阶级基础
    \task 革命方法
    \end{choices}
    
    \item \bl 在评价辛亥革命,毛泽东指出,辛亥革命"有他胜利的地方,也有它失败的地方"得出这一结论的根据有
    \begin{choices}[1]
    \task 辛亥革命沉重打击了帝国主义侵略势力
    \task 辛亥革命推翻了封建君主专制制度
    \task 辛亥革命改变了中国半殖民地半封建的社会性质
    \task 辛亥革命是一次比较完全意义上的资产阶级民主革命
    \end{choices}
    
\end{enumerate}
\section{新思想}

\section{时政}


\section{答案}

\subsection{史纲}
\begin{enumerate}
    \item B, 一切以实际出发,对于一个国家来说最大的实际就是国情
    \item A, 林则徐是开眼看世界的第一人
    \item C, 王韬是洋务运动的参与者
    \item AC, BD是半封建的原因
    \item ABC
    \item ABD
    \item BCD, A:八国联军侵华,签订辛丑条约标志着清政府完全沦为洋人的朝廷
    \item B, A:是封建地主阶级的自救运动
    \item BCD, 戊戌维新匹配了封建君权和封建伦理
    \item D, 课本原话不是C
    \item C,
    \item B, 注意前三个并没有提出民族独立与人民解法
    \item ABC, 三民主义
    \item ABD
\end{enumerate}
\ifx\allfiles\undefined
\end{document}
\fi