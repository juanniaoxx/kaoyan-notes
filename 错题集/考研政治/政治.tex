\ifx\allfiles\undefined
\documentclass[12pt, a4paper, oneside, UTF8]{ctexbook}
\usepackage{ctex}
\usepackage{verse}
\def\path{../config}
\usepackage{amsmath}
\usepackage{amsthm}
\usepackage{amssymb}
\usepackage{array}
\usepackage{xcolor}
\usepackage{graphicx}
\usepackage{mathrsfs}
\usepackage{enumitem}
\usepackage{geometry}
\usepackage[colorlinks, linkcolor=black]{hyperref}
\usepackage{stackengine}
\usepackage{yhmath}
\usepackage{extarrows}
\usepackage{tikz}
\usepackage{pgfplots}
\usepackage{asymptote}
\usepackage{float}
\usepackage{fontspec} % 使用字体

\setmainfont{Times New Roman}
\setCJKmainfont{LXGWWenKai-Light}[
    SlantedFont=*
]

\everymath{\displaystyle}

\usepgfplotslibrary{polar}
\usepackage{subcaption}
\usetikzlibrary{decorations.pathreplacing, positioning}

\usepgfplotslibrary{fillbetween}
\pgfplotsset{compat=1.18}
% \usepackage{unicode-math}
\usepackage{esint}
\usepackage[most]{tcolorbox}

\usepackage{fancyhdr}
\usepackage[dvipsnames, svgnames]{xcolor}
\usepackage{listings}

\definecolor{mygreen}{rgb}{0,0.6,0}
\definecolor{mygray}{rgb}{0.5,0.5,0.5}
\definecolor{mymauve}{rgb}{0.58,0,0.82}
\definecolor{NavyBlue}{RGB}{0,0,128}
\definecolor{Rhodamine}{RGB}{255,0,255}
\definecolor{PineGreen}{RGB}{0,128,0}

\graphicspath{ {figures/},{../figures/}, {config/}, {../config/} }

\linespread{1.6}

\geometry{
    top=25.4mm, 
    bottom=25.4mm, 
    left=20mm, 
    right=20mm, 
    headheight=2.17cm, 
    headsep=4mm, 
    footskip=12mm
}

\setenumerate[1]{itemsep=5pt,partopsep=0pt,parsep=\parskip,topsep=5pt}
\setitemize[1]{itemsep=5pt,partopsep=0pt,parsep=\parskip,topsep=5pt}
\setdescription{itemsep=5pt,partopsep=0pt,parsep=\parskip,topsep=5pt}

\lstset{
    language=Mathematica,
    basicstyle=\tt,
    breaklines=true,
    keywordstyle=\bfseries\color{NavyBlue}, 
    emphstyle=\bfseries\color{Rhodamine},
    commentstyle=\itshape\color{black!50!white}, 
    stringstyle=\bfseries\color{PineGreen!90!black},
    columns=flexible,
    numbers=left,
    numberstyle=\footnotesize,
    frame=tb,
    breakatwhitespace=false,
} 

\lstset{
    language=TeX, % 设置语言为 TeX
    basicstyle=\ttfamily, % 使用等宽字体
    breaklines=true, % 自动换行
    keywordstyle=\bfseries\color{NavyBlue}, % 关键字样式
    emphstyle=\bfseries\color{Rhodamine}, % 强调样式
    commentstyle=\itshape\color{black!50!white}, % 注释样式
    stringstyle=\bfseries\color{PineGreen!90!black}, % 字符串样式
    columns=flexible, % 列的灵活性
    numbers=left, % 行号在左侧
    numberstyle=\footnotesize, % 行号字体大小
    frame=tb, % 顶部和底部边框
    breakatwhitespace=false % 不在空白处断行
}

% \begin{lstlisting}[language=TeX] ... \end{lstlisting}

% 定理环境设置
\usepackage[strict]{changepage} 
\usepackage{framed}

\definecolor{greenshade}{rgb}{0.90,1,0.92}
\definecolor{redshade}{rgb}{1.00,0.88,0.88}
\definecolor{brownshade}{rgb}{0.99,0.95,0.9}
\definecolor{lilacshade}{rgb}{0.95,0.93,0.98}
\definecolor{orangeshade}{rgb}{1.00,0.88,0.82}
\definecolor{lightblueshade}{rgb}{0.8,0.92,1}
\definecolor{purple}{rgb}{0.81,0.85,1}

\theoremstyle{definition}
\newtheorem{myDefn}{\indent Definition}[section]
\newtheorem{myLemma}{\indent Lemma}[section]
\newtheorem{myThm}[myLemma]{\indent Theorem}
\newtheorem{myCorollary}[myLemma]{\indent Corollary}
\newtheorem{myCriterion}[myLemma]{\indent Criterion}
\newtheorem*{myRemark}{\indent Remark}
\newtheorem{myProposition}{\indent Proposition}[section]

\newenvironment{formal}[2][]{%
	\def\FrameCommand{%
		\hspace{1pt}%
		{\color{#1}\vrule width 2pt}%
		{\color{#2}\vrule width 4pt}%
		\colorbox{#2}%
	}%
	\MakeFramed{\advance\hsize-\width\FrameRestore}%
	\noindent\hspace{-4.55pt}%
	\begin{adjustwidth}{}{7pt}\vspace{2pt}\vspace{2pt}}{%
		\vspace{2pt}\end{adjustwidth}\endMakeFramed%
}

\newenvironment{definition}{\vspace{-\baselineskip * 2 / 3}%
	\begin{formal}[Green]{greenshade}\vspace{-\baselineskip * 4 / 5}\begin{myDefn}}
	{\end{myDefn}\end{formal}\vspace{-\baselineskip * 2 / 3}}

\newenvironment{theorem}{\vspace{-\baselineskip * 2 / 3}%
	\begin{formal}[LightSkyBlue]{lightblueshade}\vspace{-\baselineskip * 4 / 5}\begin{myThm}}%
	{\end{myThm}\end{formal}\vspace{-\baselineskip * 2 / 3}}

\newenvironment{lemma}{\vspace{-\baselineskip * 2 / 3}%
	\begin{formal}[Plum]{lilacshade}\vspace{-\baselineskip * 4 / 5}\begin{myLemma}}%
	{\end{myLemma}\end{formal}\vspace{-\baselineskip * 2 / 3}}

\newenvironment{corollary}{\vspace{-\baselineskip * 2 / 3}%
	\begin{formal}[BurlyWood]{brownshade}\vspace{-\baselineskip * 4 / 5}\begin{myCorollary}}%
	{\end{myCorollary}\end{formal}\vspace{-\baselineskip * 2 / 3}}

\newenvironment{criterion}{\vspace{-\baselineskip * 2 / 3}%
	\begin{formal}[DarkOrange]{orangeshade}\vspace{-\baselineskip * 4 / 5}\begin{myCriterion}}%
	{\end{myCriterion}\end{formal}\vspace{-\baselineskip * 2 / 3}}
	

\newenvironment{remark}{\vspace{-\baselineskip * 2 / 3}%
	\begin{formal}[LightCoral]{redshade}\vspace{-\baselineskip * 4 / 5}\begin{myRemark}}%
	{\end{myRemark}\end{formal}\vspace{-\baselineskip * 2 / 3}}

\newenvironment{proposition}{\vspace{-\baselineskip * 2 / 3}%
	\begin{formal}[RoyalPurple]{purple}\vspace{-\baselineskip * 4 / 5}\begin{myProposition}}%
	{\end{myProposition}\end{formal}\vspace{-\baselineskip * 2 / 3}}


\newtheorem{example}{\indent \color{SeaGreen}{Example}}[section]
\renewcommand{\proofname}{\indent\textbf{\textcolor{TealBlue}{Proof}}}
\NewEnviron{solution}{%
	\begin{proof}[\indent\textbf{\textcolor{TealBlue}{Solution}}]%
		\color{blue}% 设置内容为蓝色
		\BODY% 插入环境内容
		\color{black}% 恢复默认颜色(可选,避免影响后续文字)
	\end{proof}%
}

% 自定义命令的文件

\def\d{\mathrm{d}}
\def\R{\mathbb{R}}
%\newcommand{\bs}[1]{\boldsymbol{#1}}
%\newcommand{\ora}[1]{\overrightarrow{#1}}
\newcommand{\myspace}[1]{\par\vspace{#1\baselineskip}}
\newcommand{\xrowht}[2][0]{\addstackgap[.5\dimexpr#2\relax]{\vphantom{#1}}}
\newenvironment{mycases}[1][1]{\linespread{#1} \selectfont \begin{cases}}{\end{cases}}
\newenvironment{myvmatrix}[1][1]{\linespread{#1} \selectfont \begin{vmatrix}}{\end{vmatrix}}
\newcommand{\tabincell}[2]{\begin{tabular}{@{}#1@{}}#2\end{tabular}}
\newcommand{\pll}{\kern 0.56em/\kern -0.8em /\kern 0.56em}
\newcommand{\dive}[1][F]{\mathrm{div}\;\boldsymbol{#1}}
\newcommand{\rotn}[1][A]{\mathrm{rot}\;\boldsymbol{#1}}

\newif\ifshowanswers
\showanswerstrue % 注释掉这行就不显示答案

% 定义答案环境
\newcommand{\answer}[1]{%
    \ifshowanswers
        #1%
    \fi
}

% 修改参数改变封面样式,0 默认原始封面、内置其他1、2、3种封面样式
\def\myIndex{0}


\ifnum\myIndex>0
    \input{\path/cover_package_\myIndex} 
\fi

\def\myTitle{考研数学笔记}
\def\myAuthor{Weary Bird}
\def\myDateCover{\today}
\def\myDateForeword{\today}
\def\myForeword{相见欢·林花谢了春红}
\def\myForewordText{
    林花谢了春红,太匆匆。
    无奈朝来寒雨晚来风。
    胭脂泪,相留醉,几时重。
    自是人生长恨水长东。
}
\def\mySubheading{以姜晓千强化课讲义为底本}


\begin{document}
% \input{\path/cover_text_\myIndex.tex}

\newpage
\thispagestyle{empty}
\begin{center}
    \Huge\textbf{\myForeword}
\end{center}
\myForewordText
\begin{flushright}
    \begin{tabular}{c}
        \myDateForeword
    \end{tabular}
\end{flushright}

\newpage
\pagestyle{plain}
\setcounter{page}{1}
\pagenumbering{Roman}
\tableofcontents

\newpage
\pagenumbering{arabic}
% \setcounter{chapter}{-1}
\setcounter{page}{1}

\pagestyle{fancy}
\fancyfoot[C]{\thepage}
\renewcommand{\headrulewidth}{0.4pt}
\renewcommand{\footrulewidth}{0pt}








\else
\fi
\chapter{考研政治}
\section{马克思主义基本原理}
\begin{enumerate}
    \item (单选)进入21世纪以来,社会化大生产在世界范围内更大规模、
    更广范围、更深层次展开,世界格局深度调整,资本主义呈现一些新变化新特征.
    当代资本主义最突出、最鲜明、最主要的特征是(   ) \\
    A. 输出、渗透资本主义价值观 \qquad
    B. 工人阶级内部层级结构逐渐分化 \\
    C. 国际金融资本的垄断 \qquad\qquad\quad
    D. 科技创新加速资本主义生产方式变化

    \item 垄断是在自由竞争中形成的,是作为自由竞争的对立面产生的.
    但是,垄断并不能消除竞争,反而使竞争更加复杂和剧烈.这是因为(   ) \\
    A. 垄断没有改变生产资料的资本主义私有制 \\
    B. 垄断企业必须不断增强自己的实力,巩固自己的垄断地位\\
    C. 如果竞争不复存在,垄断企业就没有动力和压力壮大自己的实力\\
    D. 垄断企业不可能把全部社会生产都包下来

    \item 国家垄断资本主义是国家政权和私人垄断资本融合在一起的垄断资本主义.第二次世界大战结束以来,
    在国家垄断资本主义获得充分发展的同时,资本主义国家通过宏观调节和微观规制对生产,流通,分配和消费各个环节
    的干预也更加加深.其中,微观规制的类型主要有(   ) \\
    A.社会经济规制\qquad B.公共事业规制 \\
    C.公共生活规制\qquad D.反托拉斯法 

    \item 20世纪80年代以来,随着冷战的结束,分割的世界经济体系也随之被打破,技术、资本、商品等真
    正实现了全球范围的流动,各国之间的经济联系日益密切,相互合作、相互依存大大加强,世界进入到经济全
    球化迅猛发展的新时代.促进经济全球化迅猛发展的因素有(   ) \\
    A.各国经济体制变革给出的有利制度条件 \\
    B.出现了适宜于全球化的企业组织形式 \\
    C.企业不断进行的技术创新与管理创新 \\
    D.科学技术的进步和生产力的快速发展

    \item 第二次世界大战结束后,资本主义国家对经济进行的干预明显加强,从而使得资本主义社会的经济调节机制发生了显著变化.
    与这种变化相适应,资本主义政治制度也发生了很大变化.其主要表现包括(   ) \\
    A. 政治制度出现多元化的趋势 \qquad
    B. 法治建设得到重视和加强 \\
    C. 社会阶层和阶级结构的变化 \qquad
    D. 改良主义政党的影响日益扩大

    \item 放眼当今世界,新一轮科技革命和产业变革深入发展,国际力量对比深刻调整,中国发展奇迹同西方资本主义的衰落形成了鲜明对比,“东升西降”已成为百年变局中的大势所趋.
    进入21世纪以来,由于错综复杂的原因,当代资本主义呈现一些不同以往的变化态势与特点.
    这些新变化新特征(   ) \\
    A. 并未改变资本主义的经济基础和追求利润最大化的本性\\
    B. 不断引发世界范围内对资本主义制度和价值观的质疑\\
    C. 使得资本主义的基本矛盾发生了根本性改变\\
    D. 不断诱发更激烈的世界性问题和全球性矛盾

    \item 《共产党宣言》指出:“一切所有制关系都经历了经常的历史更替、经常的历史变更以及社会主义必然代替资本主义,这是历史发展的客观规律,
    也是科学社会主义最基本的结论.”资本主义为社会主义所代替的历史必然性的依据有(   ) \\
    A. 资本主义基本矛盾“包含着现代的一切冲突的萌芽” \\
    B. 资本积累推动资本主义基本矛盾不断激化并最终否定资本主义自身 \\
    C. 国家垄断资本主义是资本社会化的更高形式,将成为社会主义的前奏 \\
    D. 资本主义社会存在着资产阶级和无产阶级两大阶级之间的矛盾和斗争

    \item (单选)恩格斯指出:“我认为,所谓‘社会主义社会’不是一种一成不变的东西,
    而应当和任何其他社会制度一样,把它看成经常变化和改革的社会.”社会主义改革的根源是(   ) \\
    A. 改革是社会主义社会发展的动力 \\
    B. 社会生产力发展水平不够高 \\
    C. 社会主义制度没有根本克服资本主义制度下生产力与生产关系的对抗性矛盾 \\
    D. 社会主义社会的基本矛盾 

    \item (单选)列宁指出,不能“为死教条而牺牲活的马克思主义”.习近平全面总结社会主义历史进程,
    得出“社会主义从来都是在开拓中前进的”.这些表述对我们的深刻启示是(   ) \\
    A. 必须始终“坚持科学社会主义基本原则” \\
    B. 要把科学社会主义基本原则与本国实际相结合 \\
    C. 科学社会主义基本原则要紧跟时代和实践的发展而发展 \\
    D. 时代和实践的不断发展使社会主义发展道路具有多样性

    \item 习近平指出:当代中国的伟大社会变革,不是简单延续我国历史文化的母版,
    不是简单套用马克思主义经典作家设想的模板,不是其他国家社会主义实践的再版,
    不是国外现代化发展的翻版.这对我们理解科学社会主义一般原则的启示是(   ) \\
    A. 科学社会主义是人类优秀文化传统的历史延续 \\
    B. 科学社会主义与资本主义生产方式没有必然的联系 \\
    C. 科学社会主义绝不是一成不变的教条 \\
    D. 科学社会主义在不同的时代具有不同的内容和形式 

    \item 世界上没有放之四海而皆准的发展道路和发展模式,也没有一成不变的发展道路和发展模式.
    30多年前,印有五角星和镰刀锤头的红旗在克里姆林宫上空悄然滑落,社会主义阵营老大哥消失,西亚北非地区陷入动荡.如今,中国共产党已然走过100多个春秋,中国特色社会主义
    比任何时期都要焕发生机与活力,社会主义发展的生机悄然而至.历史证明,
    社会主义之所以在曲折中发展,是因为(   ) \\
    A. 社会主义作为新生事物,其成长不会一帆风顺 \\
    B. 经济全球化对于社会主义的发展既有机遇又有挑战 \\
    C. 社会主义社会的基本矛盾推动社会发展,是作为一个过程而展开的 \\
    D. 各国历史文化传统的差异性决定社会主义发展方向

    \item 资本主义必然为社会主义所代替,并不意味着资本主义将在短期内自行消亡.
    资本主义向社会主义的过渡必然是一个复杂、长期的历史进程,其原因在于(   ) \\
    A. 资本主义社会具有一定的自我调节能力 \\
    B. 资本主义的发展具有不平衡性 \\
    C. 任何社会形态的存在都有绝对稳定性 \\
    D. 当代资本主义的发展还显示出生产关系对生产力容纳的空间

    \item (单选)共产主义社会是人类社会发展的最高社会形态,这一社会实现的必要条件是(   ) \\
    A.社会关系的高度和谐 \qquad
    B.人自由而全面的发展 \\
    C.生产力的高度发展 \qquad
    D.阶级和国家的消亡

    \item 马克思主义最崇高的社会理想是实现共产主义社会,即实现(   ) \\
    A.无矛盾的和谐社会 \qquad
    B.物质财富极大丰富 \\
    C.人们精神境界极大提高 \qquad
    D.每个人自由而全面的发展

    \item 马克思在表述共产主义社会的基本特征时指出,共产主义社会是社会关系高度和谐,
    人们精神境界极大提高的社会.社会关系的高度和谐表现在(   ) \\
    A.国家消亡\qquad
    B.阶级消亡 \\
    C.工业与农业、城市与乡村、脑力劳动与体力劳动的差别——"三大差别"消失 \\
    D.人、自然及社会都达成和谐 

    \item 马克思、恩格斯在《共产党宣言》中明确提出:
    "资产阶级的灭亡和无产阶级的胜利是同样不可避免的.""资本主义必然灭亡,社会主义必然胜利
    "是科学社会主义的核心命题.这"两个必然"是他们研究人类历史发展,
    特别是资本主义历史发展所得出的基本结论.这一科学论断(   ) \\
    A.在科学社会主义理论与实践中具有首要和基础的地位 \\
    B.是共产主义理想信念的核心要义\\
    C.是马克思主义追求的根本价值目标\\
    D.揭示了人类社会从资本主义向社会主义转变的历史必然性
\end{enumerate}
\section{思道法} 

\begin{enumerate}
    \item (单选) 人的生命是有限的,但生命的意义和价值却可以不同.实现人生价值的根本途径是  \\
    A.培养积极进取的人生态度   \\
    B.自觉提高自我的主体素质和能力   \\
    C.正确认识自我价值和社会价值的关系  \\ 
    D.进行有意识、有目的的创造性实践活动  

    \item 导弹技术专家沈忠芳隐姓埋名60多载,直到2022年4月中国航天科工集团二院正式发布《导弹人生》一书,
    才首次向全社会公开12位此前隐姓埋名的中国导弹武器型号总指挥、总设计师,沈忠芳在列.《感动中国》组委会给予沈忠芳的颁奖词这样写道:"从无到有,从近到远,从长缨在手,到红旗如画.这一代人从没有在乎过自己的得与失,这一代人一辈子都在砺国家的剑与盾.
    今天,后辈们终于能听到你们的传奇,隐秘而伟大,平静而神圣."这对们的人生启示是 \\
    A. 评价人生价值的根本尺度,是看一个人的实践活动是否符合社会发展的客观规律,是否进了历史的进步 \\
    B. 社会价值的实现总是以个人价值的牺牲为代价 \\
    C. 社会对于个人的价值评判主要是以个人对国家和社会所作的奉献为衡量标准 \\
    D. 人生社会价值的实现是个体自我完善、全面发展的保障

    \item (单选) 信念是认知、情感和意志的有机统一体,是人们在一定的认识基础上确立的对某种思想或事物坚信不疑并
    身体力行的精神状态.信念是人们追求理想目标的强大动力,决定事业的成败.信念有不同的层次和类型,其中 \\
    A. 高层次的信念决定低层次的信念   \\
    B. 低层次的信念代表了一个人的基本信仰   \\
    C. 相同社会环境中生活的人们的信念始终一致   \\
    D. 各种信念没有科学与非科学之分 

    \item (单选) "立志当高远,立志做大事."大量的事实告诉我们,那些在事业上取得伟大成就、对人类作出
    卓越贡献的人,都是在青年时期就立下了鸿鹄之志,并为之坚持不懈、努力奋斗.下列名言能体现这一说法的是 \\
    A. "功崇惟志,业广惟勤" \qquad
    B. "夙夜在公" \\
    C. "己所不欲,勿施于人" \qquad
    D. "己欲立而立人,己欲达而达人"

    \item 周恩来就读东关模范学校时,正值中国社会发生剧烈变动的时期.
    校长亲自为学生上修身课,题目是"立命".校长讲到精彩处突然停顿下来,
    问道:"诸生为何读书啊?"有人回答为名利而读书,有人回答为做官而读书.12岁的周恩来响亮地回答:
    "为中华之崛起而读书."校长赞叹道:"有志者,当效周生啊!"周总理的故事告诉我们
    要正确处理好个人理想和社会理想的关系,就要认识到 \\
    A.要坚持个人理想与社会理想的统一,在为实现社会理想而奋斗的过程中实现个人理想 \\
    B.个人理想以社会理想为指引,社会理想是对个人理想的凝练和升华 \\
    C.社会理想是最根本、最重要的,个人理想从属于社会理想 \\
    D.社会理想的实现必须以个人理想的实现为前提和基础 
\end{enumerate}
\section{毛中特} 

\section{史纲}

\section{新思想}

\section{时政}

\ifx\allfiles\undefined
\end{document}
\fi