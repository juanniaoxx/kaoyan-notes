\ifx\allfiles\undefined
\documentclass[12pt, a4paper, oneside, UTF8]{ctexbook}
\def\path{../config}
\usepackage{amsmath}
\usepackage{amsthm}
\usepackage{amssymb}
\usepackage{array}
\usepackage{xcolor}
\usepackage{graphicx}
\usepackage{mathrsfs}
\usepackage{enumitem}
\usepackage{geometry}
\usepackage[colorlinks, linkcolor=black]{hyperref}
\usepackage{stackengine}
\usepackage{yhmath}
\usepackage{extarrows}
\usepackage{tikz}
\usepackage{pgfplots}
\usepackage{asymptote}
\usepackage{float}
\usepackage{fontspec} % 使用字体

\setmainfont{Times New Roman}
\setCJKmainfont{LXGWWenKai-Light}[
    SlantedFont=*
]

\everymath{\displaystyle}

\usepgfplotslibrary{polar}
\usepackage{subcaption}
\usetikzlibrary{decorations.pathreplacing, positioning}

\usepgfplotslibrary{fillbetween}
\pgfplotsset{compat=1.18}
% \usepackage{unicode-math}
\usepackage{esint}
\usepackage[most]{tcolorbox}

\usepackage{fancyhdr}
\usepackage[dvipsnames, svgnames]{xcolor}
\usepackage{listings}

\definecolor{mygreen}{rgb}{0,0.6,0}
\definecolor{mygray}{rgb}{0.5,0.5,0.5}
\definecolor{mymauve}{rgb}{0.58,0,0.82}
\definecolor{NavyBlue}{RGB}{0,0,128}
\definecolor{Rhodamine}{RGB}{255,0,255}
\definecolor{PineGreen}{RGB}{0,128,0}

\graphicspath{ {figures/},{../figures/}, {config/}, {../config/} }

\linespread{1.6}

\geometry{
    top=25.4mm, 
    bottom=25.4mm, 
    left=20mm, 
    right=20mm, 
    headheight=2.17cm, 
    headsep=4mm, 
    footskip=12mm
}

\setenumerate[1]{itemsep=5pt,partopsep=0pt,parsep=\parskip,topsep=5pt}
\setitemize[1]{itemsep=5pt,partopsep=0pt,parsep=\parskip,topsep=5pt}
\setdescription{itemsep=5pt,partopsep=0pt,parsep=\parskip,topsep=5pt}

\lstset{
    language=Mathematica,
    basicstyle=\tt,
    breaklines=true,
    keywordstyle=\bfseries\color{NavyBlue}, 
    emphstyle=\bfseries\color{Rhodamine},
    commentstyle=\itshape\color{black!50!white}, 
    stringstyle=\bfseries\color{PineGreen!90!black},
    columns=flexible,
    numbers=left,
    numberstyle=\footnotesize,
    frame=tb,
    breakatwhitespace=false,
} 

\lstset{
    language=TeX, % 设置语言为 TeX
    basicstyle=\ttfamily, % 使用等宽字体
    breaklines=true, % 自动换行
    keywordstyle=\bfseries\color{NavyBlue}, % 关键字样式
    emphstyle=\bfseries\color{Rhodamine}, % 强调样式
    commentstyle=\itshape\color{black!50!white}, % 注释样式
    stringstyle=\bfseries\color{PineGreen!90!black}, % 字符串样式
    columns=flexible, % 列的灵活性
    numbers=left, % 行号在左侧
    numberstyle=\footnotesize, % 行号字体大小
    frame=tb, % 顶部和底部边框
    breakatwhitespace=false % 不在空白处断行
}

% \begin{lstlisting}[language=TeX] ... \end{lstlisting}

% 定理环境设置
\usepackage[strict]{changepage} 
\usepackage{framed}

\definecolor{greenshade}{rgb}{0.90,1,0.92}
\definecolor{redshade}{rgb}{1.00,0.88,0.88}
\definecolor{brownshade}{rgb}{0.99,0.95,0.9}
\definecolor{lilacshade}{rgb}{0.95,0.93,0.98}
\definecolor{orangeshade}{rgb}{1.00,0.88,0.82}
\definecolor{lightblueshade}{rgb}{0.8,0.92,1}
\definecolor{purple}{rgb}{0.81,0.85,1}

\theoremstyle{definition}
\newtheorem{myDefn}{\indent Definition}[section]
\newtheorem{myLemma}{\indent Lemma}[section]
\newtheorem{myThm}[myLemma]{\indent Theorem}
\newtheorem{myCorollary}[myLemma]{\indent Corollary}
\newtheorem{myCriterion}[myLemma]{\indent Criterion}
\newtheorem*{myRemark}{\indent Remark}
\newtheorem{myProposition}{\indent Proposition}[section]

\newenvironment{formal}[2][]{%
	\def\FrameCommand{%
		\hspace{1pt}%
		{\color{#1}\vrule width 2pt}%
		{\color{#2}\vrule width 4pt}%
		\colorbox{#2}%
	}%
	\MakeFramed{\advance\hsize-\width\FrameRestore}%
	\noindent\hspace{-4.55pt}%
	\begin{adjustwidth}{}{7pt}\vspace{2pt}\vspace{2pt}}{%
		\vspace{2pt}\end{adjustwidth}\endMakeFramed%
}

\newenvironment{definition}{\vspace{-\baselineskip * 2 / 3}%
	\begin{formal}[Green]{greenshade}\vspace{-\baselineskip * 4 / 5}\begin{myDefn}}
	{\end{myDefn}\end{formal}\vspace{-\baselineskip * 2 / 3}}

\newenvironment{theorem}{\vspace{-\baselineskip * 2 / 3}%
	\begin{formal}[LightSkyBlue]{lightblueshade}\vspace{-\baselineskip * 4 / 5}\begin{myThm}}%
	{\end{myThm}\end{formal}\vspace{-\baselineskip * 2 / 3}}

\newenvironment{lemma}{\vspace{-\baselineskip * 2 / 3}%
	\begin{formal}[Plum]{lilacshade}\vspace{-\baselineskip * 4 / 5}\begin{myLemma}}%
	{\end{myLemma}\end{formal}\vspace{-\baselineskip * 2 / 3}}

\newenvironment{corollary}{\vspace{-\baselineskip * 2 / 3}%
	\begin{formal}[BurlyWood]{brownshade}\vspace{-\baselineskip * 4 / 5}\begin{myCorollary}}%
	{\end{myCorollary}\end{formal}\vspace{-\baselineskip * 2 / 3}}

\newenvironment{criterion}{\vspace{-\baselineskip * 2 / 3}%
	\begin{formal}[DarkOrange]{orangeshade}\vspace{-\baselineskip * 4 / 5}\begin{myCriterion}}%
	{\end{myCriterion}\end{formal}\vspace{-\baselineskip * 2 / 3}}
	

\newenvironment{remark}{\vspace{-\baselineskip * 2 / 3}%
	\begin{formal}[LightCoral]{redshade}\vspace{-\baselineskip * 4 / 5}\begin{myRemark}}%
	{\end{myRemark}\end{formal}\vspace{-\baselineskip * 2 / 3}}

\newenvironment{proposition}{\vspace{-\baselineskip * 2 / 3}%
	\begin{formal}[RoyalPurple]{purple}\vspace{-\baselineskip * 4 / 5}\begin{myProposition}}%
	{\end{myProposition}\end{formal}\vspace{-\baselineskip * 2 / 3}}


\newtheorem{example}{\indent \color{SeaGreen}{Example}}[section]
\renewcommand{\proofname}{\indent\textbf{\textcolor{TealBlue}{Proof}}}
\NewEnviron{solution}{%
	\begin{proof}[\indent\textbf{\textcolor{TealBlue}{Solution}}]%
		\color{blue}% 设置内容为蓝色
		\BODY% 插入环境内容
		\color{black}% 恢复默认颜色(可选,避免影响后续文字)
	\end{proof}%
}

% 自定义命令的文件

\def\d{\mathrm{d}}
\def\R{\mathbb{R}}
%\newcommand{\bs}[1]{\boldsymbol{#1}}
%\newcommand{\ora}[1]{\overrightarrow{#1}}
\newcommand{\myspace}[1]{\par\vspace{#1\baselineskip}}
\newcommand{\xrowht}[2][0]{\addstackgap[.5\dimexpr#2\relax]{\vphantom{#1}}}
\newenvironment{mycases}[1][1]{\linespread{#1} \selectfont \begin{cases}}{\end{cases}}
\newenvironment{myvmatrix}[1][1]{\linespread{#1} \selectfont \begin{vmatrix}}{\end{vmatrix}}
\newcommand{\tabincell}[2]{\begin{tabular}{@{}#1@{}}#2\end{tabular}}
\newcommand{\pll}{\kern 0.56em/\kern -0.8em /\kern 0.56em}
\newcommand{\dive}[1][F]{\mathrm{div}\;\boldsymbol{#1}}
\newcommand{\rotn}[1][A]{\mathrm{rot}\;\boldsymbol{#1}}

\newif\ifshowanswers
\showanswerstrue % 注释掉这行就不显示答案

% 定义答案环境
\newcommand{\answer}[1]{%
    \ifshowanswers
        #1%
    \fi
}

% 修改参数改变封面样式,0 默认原始封面、内置其他1、2、3种封面样式
\def\myIndex{0}


\ifnum\myIndex>0
    \input{\path/cover_package_\myIndex} 
\fi

\def\myTitle{考研数学笔记}
\def\myAuthor{Weary Bird}
\def\myDateCover{\today}
\def\myDateForeword{\today}
\def\myForeword{相见欢·林花谢了春红}
\def\myForewordText{
    林花谢了春红,太匆匆。
    无奈朝来寒雨晚来风。
    胭脂泪,相留醉,几时重。
    自是人生长恨水长东。
}
\def\mySubheading{以姜晓千强化课讲义为底本}


\begin{document}

% \input{\path/cover_text_\myIndex.tex}

\newpage
\thispagestyle{empty}
\begin{center}
    \Huge\textbf{\myForeword}
\end{center}
\myForewordText
\begin{flushright}
    \begin{tabular}{c}
        \myDateForeword
    \end{tabular}
\end{flushright}

\newpage
\pagestyle{plain}
\setcounter{page}{1}
\pagenumbering{Roman}
\tableofcontents

\newpage
\pagenumbering{arabic}
% \setcounter{chapter}{-1}
\setcounter{page}{1}

\pagestyle{fancy}
\fancyfoot[C]{\thepage}
\renewcommand{\headrulewidth}{0.4pt}
\renewcommand{\footrulewidth}{0pt}








\else
\fi
\chapter{真题与模拟题}

\begin{tcolorbox}[title=备注]
    \bt 表示难度,越多越难
    \bl 表示计算量,越多计算量越大
\end{tcolorbox}

\section{数学真题一网打尽} 
\begin{enumerate}
    \item \bt[2] 求 $ \lim_{n\to\infty}\left(\frac{\sin\frac{\pi}{n}}{n+1}+
    \frac{\sin\frac{2\pi}{n}}{n+\frac{1}{2}}+\ldots+\frac{\sin\pi}{n+\frac{1}{n}}\right)$ 

    \answer{
        \begin{solution}
        显然是一道夹逼定理的题目,但有几点需要注意. 
        $$
        \text{原式} < \frac{1}{n}\sum_{i=1}^{n}\sin{\frac{i}{n}\pi} \xlongequal{n\to\infty} \int_{0}^{1}\sin{\pi x}\d x 
        $$
        放大这一方向是比较好想的,重点在于缩小.
        $$
        \text{原式} > \frac{1}{n+1}\sum_{i=1}^{n}\sin{\frac{i}{n}\pi} = 
        {\color{red} \frac{n}{n+1}\cdot\frac{1}{n}}\sum_{i=1}^{n}\sin{\frac{i}{n}\pi} \xlongequal{n\to\infty} \int_{0}^{1}\sin{\pi x}\d x
        $$
        \begin{align*}
            \int_{0}^{1}\sin{\pi x} = \frac{2}{\pi}
        \end{align*}
        由夹逼定理有
        $$
        \lim_{n\to\infty} \text{原式} = \frac{2}{\pi}
        $$
        \end{solution}
    }
    
    \item \bt[2] 设函数$f(x)$在区间$\left[0,1\right]$连续,则$\int_{0}^{1}f(x)\d x = $ (   )
    \begin{choices}[2]
        \task $\lim_{n\to\infty}\sum_{k=1}^{n}f\left(\frac{2k-1}{2n}\right)\cdot\frac{1}{2n}$ 
        \task $\lim_{n\to\infty}\sum_{k=1}^{n}f\left(\frac{2k-1}{2n}\right)\cdot\frac{1}{n}$
        \task $\lim_{n\to\infty}\sum_{k=1}^{n}f\left(\frac{k-1}{2n}\right)\cdot\frac{1}{n}$
        \task $\lim_{n\to\infty}\sum_{k=1}^{n}f\left(\frac{k}{2n}\right)\cdot\frac{2}{n}$
    \end{choices}


    \answer{
        \begin{solution}[解法一\ 正面突破]
            这道题显然是考察定积分的定义,但考察的比较细节.
            \begin{enumerate}
                \item[i] 其中(A)(B)选项是将区间进行$n$等分的划分,且取的是区间重点,如何得知呢? 考虑
                端点$0,\frac{1}{n},\frac{2}{n},\ldots,\frac{k}{n},\frac{k+1}{n},\ldots,\frac{n}{n}$而
                $$
                    \frac{k-1}{n}=\frac{2k-2}{2n} < \frac{2k - 1}{2n} < \frac{2k}{2n} = \frac{k}{n}
                $$
                故由定积分的定义,此时有
                $$
                \int_{0}^{1}f(x)\d x = \lim_{n\to\infty}\sum_{k=1}^{n}f(\frac{2k-1}{2n})\cdot \frac{1}{n}
                $$
                \item[ii] 其中(C)(D)是将区间进行$2n$等分的划分,取的分别是左/右端点,这并不影响定积分形式,应该为
                $$
                \int_{0}^{1}f(x)\d x = \lim_{n\to\infty}\sum_{k=1}^{2n}f(\frac{k}{2n})\cdot\frac{1}{2n} 
                $$
            \end{enumerate}
        \end{solution}

        \begin{solution}[解法二\ 选择题不客气!]
            取$f(x) = 1$ 则$\int_{0}^{1}1\d x = 1$,对应的选项可以直接计算,结果为
            \begin{enumerate}
                \item [(A)] $ \text{原式}=\lim_{n\to\infty}\sum_{i=1}^{n}\frac{1}{2n}=\frac{1}{2}\neq 1$ 
                \item [(B)] $ \text{原式}=\lim_{n\to\infty}\sum_{i=1}^{n}\frac{1}{n}= 1$
                \item [(C)] $ \text{原式}=\lim_{n\to\infty}\sum_{i=1}^{2n}\frac{1}{n}= 2\neq 1$
                \item [(D)] $ \text{原式}=\lim_{n\to\infty}\sum_{i=1}^{2n}\frac{2}{n}= 4\neq 1$ 
            \end{enumerate}
        \end{solution}

        \newpage
        \begin{remark}[定积分的定义]
            定积分的定义有如下几个要点
            \begin{enumerate}
                \item [(1)] 将区间$[a,b]$划分为$n$个区域,其中记
                $$
                a=x_0<x_1<x_2<\ldots<x_n=b 
                $$
                记自区间长度即模为 
                $$
                \lambda = \max\{\Delta_1,\Delta_2,\ldots,\Delta_{n-1},\Delta_{n}\}
                $$
                \item [(2)] 在每个子区间上取任意一点$\xi_i$取其函数值$f(\xi_i)$,则Riemann和为
                $$
                S=\sum_{i=1}^{n}f(\xi_i)\Delta x_i
                $$
                当$\lambda\to 0$时,若$S$极限存在,且\underline{分割方式与$\xi_i$无关},则称该极限为$f$在$[a,b]$上的定积分,如下
                $$
                \int_{a}^{b}f(x)\d x = \lim_{\lambda\to 0} \sum_{i=1}^{n}f(\xi_i)\Delta x_i
                $$
            \end{enumerate}
        \end{remark}
    }
    \item  \bt (1999.2)$ a_n=\sum_{k=1}^{n}f(k)-\int_{1}^{n}f(x)\d x(n=1,2,\ldots)$,设$f(x)$是区间$[0,+\infty)$
    上单调递减且非负的连续函数证明数列$\{a_n\}$极限存在 

    \answer{
        \begin{solution}
            先证明单调性,作差有
            \begin{align*}
                a_{n+1}-a_n &= f(n+1) - \int_{n}^{n+1}f(x)\d x  \\
                &\xlongequal{\text{积分中值定理}}f(n+1)-f(\xi), \xi\in(n,n+1) 
            \end{align*}
            由于$f(x)$在$[0,+\infty)$上单调递减故$a_{n+1}-a_n < 0 \implies$原数列单调递减. \\
            再证明有界性 由于
            $$
            \color{red} \sum_{k=1}^{n-1}\int_{k}^{k+1}f(x)\d x = \int_{1}^{n}f(x)\d x
            $$
            原式化为
            $$
            \sum_{k=1}^{n-1}\left[f(k)-\int_{k}^{k+1}f(x)\d x\right] + f(n) 
            $$
            由于$f(x)$非负且单调递减,容易直到$f(k) > \int_{k}^{k+1}f(x)\d x$ 故原式一定有
            $$
            \text{原式} \geq 0
            $$
            即原数列单调递减有下界,故原数列收敛. 
        \end{solution}
    }
    \item (2011-12) \bt[2]
    \begin{enumerate}
        \item [(1)] 证明:对于任意的正整数$n$,都有$\frac{1}{n+1}<\ln{\left(1+\frac{1}{n}\right)}<\frac{1}{n}$
        \item [(2)] 设$a_n=1+\frac{1}{2}+\ldots+\frac{1}{n}-\ln{n}$ 证明数列$\{a_n\}$收敛 
    \end{enumerate}

    \answer{
        \begin{solution}[拉氏中值+单调有界证明]
            (1)令$f(x)=\ln{(1+x)}$ 则 
            $$
            \ln{(1+\frac{1}{n})} - \ln{1} = f'(\xi)\cdot\frac{1}{n},\xi\in(0,\frac{1}{n})
            $$
            即
            $$
            \frac{n}{n+1}<\frac{1}{1+\xi}<1
            $$
            综上有
            $$
            \frac{1}{n+1}<\ln{\left(1+\frac{1}{n}\right)}<\frac{1}{n}
            $$
            (2)首先证明其单调,作差有
            $$
            a_{n+1}-a_{n} = \frac{1}{n+1}-\ln{(n+1)}+\ln{n} = \frac{1}{n+1}-\ln{(1+\frac{1}{n})} < 0
            $$
            即原数列单调递减,只需证明其有下界即可. 考虑
            \begin{align*}
                a_n &= 1 + \frac{1}{2} + \ldots + \frac{1}{n} - ln{n} \\
                &> \ln{(1+1)} + \ln{(1 + \frac{1}{2})} + \ldots + \ln{(1+\frac{1}{n})} - \ln{n} \\
                &= \ln{(2\cdot\frac{3}{2}\cdot\frac{4}{3}\ldots\frac{n+1}{n})} - \ln{n} \\
                &= \ln{(n+1)} - \ln{n} > 0
            \end{align*}
            故原数列单调递减有下界,即其极限值存在.
        \end{solution}

        \begin{solution}[积分放缩法]
            {\color{blue}由积分保号性,若需证明$\int_{a}^{b}f(x)\d x>\int_{a}^{b}f(x)\d x$只需证明$f(x) > g(x)$}\\
            (1)考虑如下操作
            \begin{align*}
                \frac{1}{n+1} &< \ln{(1+n)} - \ln{n} < \frac{1}{n} \\
                \frac{1}{n+1} &< \int_{n}^{n+1}\frac{1}{x}\d x <\frac{1}{n} \\
                \int_{n}^{n+1}\frac{1}{n+1}\d x &< \int_{n}^{n+1}\frac{1}{x}\d x < \int_{n}^{n+1} \frac{1}{n}\d x
            \end{align*}
            显然在$(n,n+1)$上有$\frac{1}{n+1}<\frac{1}{x}<\frac{1}{n}$,故原不等式得证  \\
            (2) 证明单调性,作差有 
            $$
            a_{n+1}-a_{n}=\int_{n}^{n+1}\left(\frac{1}{n+1}-\frac{1}{x}\right)\d x < 0
            $$
            证明有下界有
            $$
            a_n = \sum_{k=1}^{n}\frac{1}{k} - \int_{1}^{n}\frac{1}{x}\d x > 0
            $$
            有没有很眼熟,没错,正是上一题(1999.2)的所考察的证明!  \\
            故原数列单调递减有下界,其极限存在.
        \end{solution}

        \begin{solution}[收敛级数]
            (1) 不等式最基本的方法应该想到构建函数,证明单调性. 不妨令$x=\frac{1}{n}$,原不等式等价于证明
            $$
            \frac{x}{1+x} < \ln{(1+x)} < x, x\in(0,1)
            $$
            令$f(x) = x - \ln(1+x)$ 则$f'(x)=1-\frac{1}{x+1} > 0$,故$f'(x)$单调递增,即$f(x)>f(0) = 0$,同理可证明
            左边不等式. \\
            (2) 基于如下结论
            \begin{center}
                \fbox{$\lim_{n\to\infty}a_n\text{存在}\iff\sum_{n=1}^{\infty}(a_{n+1}-a_n)\text{收敛}$}
            \end{center}
            由于
            $$
            a_n = a_1 + (a_2 - a_1) + \ldots + (a_n - a_{n-1}) = a_1 + \sum_{k=2}^{n}(a_k-a_{k-1})
            $$
            故数列$\{a_n\}$与级数$\sum_{k=2}^{n}(a_k-a_{k-1})$同敛散. \\
            由于$\left|a_n-a_{n-1}\right| = \left|\frac{1}{n}+\ln\left(1-\frac{1}{n}\right)\right|$做Taylor展开有 
            \begin{align*}
                \left|a_n-a_{n-1}\right| &= \left|\frac{1}{n}\left[-\frac{1}{n}-\frac{1}{2}\frac{1}{n^2}+o(\frac{1}{n^2})\right]\right| \\
                &=\left|-\frac{1}{2}\frac{1}{n^2}+o(\frac{1}{n^2})\right| \sim \frac{1}{n^2}
            \end{align*}
            又因为$\sum_{n=1}^{\infty}\frac{1}{n^2}$收敛,由比较判别法可知原级数绝对收敛,故而原级数收敛.从而数列极限存在
        \end{solution}
    }
    \item (2012-2) \bt 
    \begin{enumerate}
        \item [(1)] 证明方程$x^{n}+x^{n-1}+\ldots+x=1(n>1,n\in\mathbf{N})$在区间$\left(\frac{1}{2},1\right)$内仅有一个实根 
        \item [(2)] 记$(I)$中的实根为$x_n$证明$ \lim_{n\to\infty}x_n$存在,并求出此极限 
    \end{enumerate}

    \answer{
        \begin{solution}
            (1) 令$f(x) = x^n+\ldots+x-1,f'(x)=nx^{n-1}+\ldots + 2x + 1 > 0$ 故f(x) 在$(\frac{1}{2},1)$上
            单调递增,又有
            $$
            \begin{cases}
                f(1) = n - 1 > 0 \\
                f(\frac{1}{2}) = -\frac{1}{2^n} < 0
            \end{cases}
            $$
            由零点存在定理可知,在区间$(\frac{1}{2},1)$上仅有唯一零点 \\
            (2) 考虑$f(x)_{n+1} = x^{n+1}+x^n+\ldots+x - 1$由(1)可知
            $$
            \begin{cases}
            f(x_n)_{n+1} = x_n^{n+1} > 0 \\
            f(\frac{1}{2})_{n+1} = -\frac{1}{2^{n+1}} < 0
            \end{cases}
            $$
            故在区间$(\frac{1}{2},x_n)$中有唯一零点$x_{n+1}$因此有
            $$
            \frac{1}{2}<x_{n+1}<x_n
            $$
            即数列$\{x_n\}$单调递减有下界故极限存在. \\
            不妨令$\lim_{n\to\infty}x_n=a$带入$f(x)$有
            $$
            \lim_{n\to\infty}\frac{x_n-x_{n+1}}{1-x_n} = 1
            $$
            即
            $$
            \frac{a-0}{1-a} = 1 \implies a = \frac{1}{2}
            $$
        \end{solution}
    }

    \item \bt (2013.2)设函数$f(x)=\ln{x}+\frac{1}{x}$ 
    \begin{enumerate}
        \item [(1)] 求$f(x)$的最小值 
        \item [(2)] 设数列$\{x_n\}$满足$\ln{x_n}+\frac{1}{x_{n+1}}<1$证明$\lim_{n\to\infty}x_n$存在,并求此极限
    \end{enumerate}

        \answer{
        \begin{solution}
            (1) $f'(x) = \frac{1}{x}-\frac{1}{x^2} = \frac{x-1}{x^2}(x>0)$ 有$f(x)$在$(0,1)$上递减,
            在$(1,+\infty)$上递增,故$f(1)=1$为$f(x)$的最小值 \\
            (2)由题设有 
            $$
            \ln{x_n} < \ln{x_n}+\frac{1}{x_{n+1}}<1=\ln{e}
            $$
            有$\ln{x}$单调,故$0 < x_n < e$又由于(1)可知$1=f(1) < f(x_n) \implies x_{n+1} < x_n$故原数列单调递减有下界
            故其极限存在,不妨设$\lim_{n\to\infty}x_n=a$,有题设有
            $$
            \ln{a} + \frac{1}{a} {\color{red} \leq } 1 
            $$
            又因为
            $$
            \ln{x} + \frac{1}{x} \geq 1
            $$
            故$a=1$即
            $$
            \fbox{$\lim_{n\to\infty}x_n = 1$}
            $$
        \end{solution}
    }

    \item \bt 设对任意的$x$,总有$\varphi(x)\leq f(x)\leq g(x)$,且$ \lim_{n\to\infty}\left[g(x)-\varphi(x)\right]=0$,则
    $\lim_{n\to\infty}f(x)$(   )
    \begin{choices}[2]
        \task 存在且等于零
        \task 存在但不一定为零
        \task 一定不存在
        \task 不一定存在
    \end{choices}

    \answer{
        \bs{
            对于A,B选项,不妨取$f(x)=g(x)=\varphi(x)=x$ 但是$\lim_{x\to\infty}f(x)=\infty$不存在 \\
            对于C选项,不妨取$f(x)=g(x)=\varphi(x)=1$,此时$\lim_{x\to\infty}f(x)=1$
        }

        \begin{remark}[夹逼定理]
            原式形式
            $$
            n\text{充分大时}, \begin{cases}
                \varphi(n) \leq f(n) \leq g(n)  \\
                \lim_{n\to\infty}\varphi(n) = \lim_{n\to\infty}g(n) = A 
            \end{cases} \implies \lim_{n\to\infty} f(n) = A
            $$
            考虑题设的$\lim_{n\to\infty}\left[g(x)-\varphi(x)\right]=0$ 则有
            $$
            0\leq f(x)-\varphi(x) \leq g(x)-\varphi(x) \implies \lim_{n\to\infty} \left[f(x)-\varphi(x)\right] = 0 
            $$
            也可以看出$f(x)$的极限与$\varphi(x)$有关,若$\varphi(x)$存在则$f(x)$极限也存在否则不存在.
        \end{remark}
    }

    \item \bt (2007-12)设函数$f(x)$在$(0,+\infty)$内具有二阶导数,且$f''(x)>0$,
    令$u_n=f(n)(n=1,2,\ldots)$则下列结论正确的是(  )
    \begin{choices}[2]
        \task 若$u_1>u_2$,则$\{u_n\}$必收敛 
        \task 若$u_1>u_2$,则$\{u_n\}$必发散
        \task 若$u_1<u_2$,则$\{u_n\}$必收敛
        \task 若$u_1<u_2$,则$\{u_n\}$必发散
    \end{choices}

    \answer{
        \begin{solution}[拉格朗日中值定理]
        存在$\xi_n\in(n,n+1),u_{n+1}-u_n=f(n+1)-f(n)=f'(\xi_n)$进而有
        $u_{n+1}=u_n+f'(\xi_n)$,由于$f''(x)>0\implies f'(x)$单调递增,此时有
        \begin{align*}
            u_{n+1} &= u_n + f'(\xi_n) \\
            &= u_{n-1} + f'(\xi_{n-1}) + f'(\xi_n) \\
            &\ldots \\
            &=u_1 + \sum_{i=1}^{n}f'(\xi_i) \\
            &> u_1 + nf'(\xi_i) = u_1 + n(u_2-u_1)
        \end{align*}
        显然当$u_2>u_1$ 当$n\to\infty,u_n > +\infty$显然极限不存在.
        \end{solution} 

        \begin{solution}[选择题不客气]
            对于选项A,$f(x)=\frac{1}{x}-x$ \\
            对于选项B,$f(x)=\frac{1}{x}$ \\
            对于选项C,$f(x)=x^2$ 
        \end{solution}

        \begin{solution}[级数]
            由于$u_{n+1}-u_n=f'(\xi_n) > f'(\xi_i) = u_2 - u_1 > 0$ 此时$\lim_{n\to\infty}u_{n+1}-u_n \neq 0$ 
            从而级数$\sum_{n=1}^{\infty}\left(a_{n+1}-a_n\right)$极限不存在,由定义有其部分和不存在,即
            $$
            \lim_{n\to\infty}S_n = \lim_{n\to\infty} u_{n+1} - u_1 
            $$
            进而可知$\lim_{n\to\infty}u_n$不存在.
        \end{solution}
    }
    \item  设$\lim_{n\to\infty}a_n=a$且$a\neq 0$则当$n$充分大的时候,有(   )
    \begin{choices}
        \task $\left|a_n\right|>\frac{\left|a\right|}{2}$
        \task $\left|a_n\right|<\frac{\left|a\right|}{2}$
        \task $a_n>a-\frac{1}{n}$
        \task $a_n<a+\frac{1}{n}$
    \end{choices}

    \item 设有数列$\left\{x_n\right\},-\frac{\pi}{2}\leq x_n\leq \frac{\pi}{2}$则(    )
    \begin{choices}[1]
        \task 若$\lim_{n\to\infty}\cos{(\sin{x_n})}$存在,则$\lim_{n\to\infty}x_n$存在
        \task 若$\lim_{n\to\infty}\sin{(\cos{x_n})}$存在,则$\lim_{n\to\infty}x_n$存在
        \task 若$\lim_{n\to\infty}\cos{(\sin{x_n})}$存在,则$\lim_{n\to\infty}\sin{x_n}$存在,但$\lim_{n\to\infty}x_n$不存在
        \task 若$\lim_{n\to\infty}\sin{(\cos{x_n})}$存在,则$\lim_{n\to\infty}\cos{x_n}$存在,但$\lim_{n\to\infty}x_n$不存在
    \end{choices}


    \item 已知$a_n=\sqrt[n]{n}-\frac{(-1)^n}{n}(n=1,2,\ldots)$则$\{a_n\}$(    )
    \begin{choices}[2]
        \task 有最大值与最小值
        \task 有最大值无最小值
        \task 有最小值无最大值
        \task 无最大值与最小值
    \end{choices}

    \item \bl[2] 设$z=z(x,y)$是由方程$x^2+y^2-z=\varphi(x+y+z)$所确定的函数,其中$\varphi$具有2阶导数且$\varphi'\neq -1$ 
    \begin{enumerate}
        \item [(1)] 求$\d z$ 
        \item [(2)] 记$ u(x,y)=\frac{1}{x-y}\left(\frac{\P z}{\P x} - \frac{\P z}{\P y}\right)$,求$\frac{\P u}{\P x}$
    \end{enumerate}

    \item 

\end{enumerate}

\newpage

\section{逐年整理}
\subsection{1987年}
\begin{enumerate}
    \item 由$y=\ln{x}$与两直线$y=(e+1)-x,y=0$围成图像的面积是$\_\_\_\_\_$ 

    \begin{solution}
    \begin{align*}
        S &= \int_{1}^{e}\ln{x}\d x + \int_{e}^{e+1}(e+1-x)\d x \\
        &=\left[(e+1)x-\frac{x^2}{2}\right]\bigg|_{e}^{e+1} \\
        &=1+1-\frac{1}{2}=\frac{3}{2}
    \end{align*}
    \end{solution}
    \item \bl[2] 求微分方程$y'''+6y''+(9+a^2)y'=1$的通解,其中常数$a>0$. 
    \begin{solution}
        本质是对于二阶非齐次方程组的推广.首先求齐次方程的解. 其特征方程为
        $$
        r^3 + 6r^2 + (9+a^2)r = 0
        $$
        其特征方程的解为$r_1=0,r_{2,3}=-3\pm ai$因此可设其齐次通解为
        $$
        y_1=C_1+e^{-3x}\left(C_2\cos{ax}+C_3\sin{ax}\right)
        $$ 
        其中$C_1,C_2,C_3$均为任意常数.
        由于$x=0$是特征方程的一重根,故可设齐次特解为$y^*=Ax$,带入方程可以解出$A=\frac{1}{9+a^2}$.\\
        综上可求出原三阶非齐次方程的通解为
        $$
        y_1+y^*=C_1+C_2e^{-3x}\cos{ax}+C_3e^{-3x}\sin{ax}+\frac{1}{1+9a^2}x 
        $$
        其中$C_1,C_2,C_3$均为任意常数.
    \end{solution}
    \item 求幂级数$\sum_{n=1}^{\infty}\frac{1}{n2^n}x^{n-1}$的收敛域,并求出其和函数 
    \begin{solution}
        由柯西定理容易求出该幂级数的收敛区间为$(-2,2)$当$x=-2$由莱布尼兹判别法知原级数收敛;而当$x=2$的时候原级数为
        调和级数显然发散,故原幂级数的收敛区间为$[-2,2)$ \\
        $S(x)=\frac{1}{x}\sum_{n=1}^{\infty}\frac{1}{n}(\frac{x}{2})^n(x\neq 0)$,当$x=0$的时候$S(0)=\frac{1}{2}$ \\
        记$S_1(x)=\sum_{i=1}^{\infty}\frac{1}{n}\left(\frac{x}{2}\right)^n$ 典型的先导后积模型,故 \\
        $S'_1(x)=\frac{1}{2}\sum_{i=1}^{\infty}\left(\frac{x}{2}\right)^{n-1}=\frac{1}{2}\sum_{n=0}^{\infty}\left(\frac{x}{2}\right)^n=\frac{1}{2}\cdot\frac{1}{1-\frac{1}{2}}=\frac{1}{2-x}$ \\
        利用牛顿-莱布尼兹公式有
        $$
        S_1(x)-S_1(0) = \int_{0}^{x}S'_1(x)\d x = \int_{0}^{x}\frac{\d x}{2 - x} = \ln\frac{2}{2-x}
        $$
        综上可知原幂级数的和函数为
        $$
        S(x) = \begin{cases}
            \frac{1}{2}, &x=0 \\
            \frac{1}{x}\ln{\frac{2}{2-x}}, & -2\leq x\leq 2,\text{且}x\neq 0.
        \end{cases}
        $$
    \end{solution}
    \item 计算曲面积分
    $$
    I = \iint_{\sum}x(8y+1)\d y\d z + 2(1-y^2)\d z\d x - 4yz\d x\d y. 
    $$
    其中曲面$\sum$是由曲线$\begin{cases}
        z = \sqrt{y-1} \\
        x = 0
    \end{cases}(1\leq y\leq 3)$绕$y$轴旋转一周所形成的曲面,其法向量与$y$轴正方向夹角始终大于$\frac{\pi}{2}$. 
    \begin{solution}
        旋转曲面$S$: $y=x^2+z^2+1$. \\
        补面$S_1:y=3(x^2+z^2 \leq 2)$,由题设可知S的法向量是指向外侧的,故$S_1$取右侧. 此时由高斯公式有:
        \begin{align*}
            I &= \iint_{s+s_1} - \iint_{s_1} \\
            &= \iiint_{\Omega}\left(8y+1-4y-4y\right)\d V - \iint_{x^2+z^2\leq 2}2(1-3^2)\d\sigma \\
            &= \iiint_{\Omega}\d V + 16 \iint_{x^2+z^2 \leq 2}  \\
            &= \int_{1}^{3}\d y \iint_{x^2+z^2\leq y - 1}\d x\d z + 32\pi \\
            &= 2\pi + 32\pi = 34\pi
        \end{align*}
    \end{solution}
    \item 已知三维线性空间的一组基$\alpha_1=(1,1,0)^T,\alpha_2=(1,0,1)^T,\alpha_3=(0,1,1)^T$则向量$\alpha=(2,0,0)^T$在
    该基下的坐标是$\_\_\_\_\_$
    \begin{solution}
        关键点寻找一组$k_i$使得$\alpha = \sum_{i=1}^{3}k_i\alpha_i$成立,此时问题转换为求解方程组. 
        $$
        (\alpha_1,\alpha_2,\alpha_3)\begin{pmatrix}
            x_1 \\
            x_2 \\
            x_3
        \end{pmatrix} = \alpha
        $$
        解线性方程有
        $$
        \begin{pmatrix}
            1 & 0 & 0 & 1\\
            0 & 1 & 0 & 1\\
            0 & 0 & 1 & -1
        \end{pmatrix}
        $$
        此时非齐次方程仅有唯一解即$(1,1,-1)^{T}$故$\alpha$在题设基的坐标为$(1,1,-1)^T$
    \end{solution}
    \item 三个箱子,第一个箱子重有4个黑球1个白球,第二个箱子中有3个黑球3个白球,第三个箱子中有3个黑球5个白球,现随机抽取一个箱子,
    再从这个箱子中取出一个球,这个球为白球的概率等于(\qquad).已知取出的是白球,则该球属于第二个箱子的概率为(\qquad).

    \begin{solution}
        
    \end{solution}
    \item 设随机变量X,Y相互独立,概率密度函数分别为
    $$
    f_{X}(x)=\begin{cases}
        1, &0\leq 1\leq 1 \\
        0, &\text{其他}
    \end{cases},f_{Y}(y)=\begin{cases}
        e^{-y}, &y>0 \\
        0, &y\leq 0
    \end{cases}
    $$
    求随机变量$Z=2X+Y$的随机密度函数$f_{Z}(z)$.
\end{enumerate}

\newpage
\section{超越(11-25年)}

\newpage
\section{共创(22,23,24)年}

\newpage
\section{25年模拟卷(百来套)}

\ifx\allfiles\undefined
\end{document}
\fi