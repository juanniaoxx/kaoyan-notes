\documentclass[10pt]{article}
\usepackage[verbose, a4paper, hmargin={2.5cm,3cm}, vmargin=2.5cm]{geometry}

\usepackage{fontspec}
\usepackage{ctex}
% \usepackage{paratype}
\setmainfont{Arial}


\usepackage{amsmath}
\usepackage{amsfonts}
\usepackage{amssymb}
\usepackage{esint}
\usepackage{xcolor}
\usepackage{graphicx}
\usepackage[export]{adjustbox}
\usepackage{mdframed}
\usepackage{booktabs,array,multirow}
\usepackage{adjustbox}
\usepackage{tabularx}
\usepackage{hyperref}
\hypersetup{colorlinks=true, linkcolor=blue, filecolor=magenta, urlcolor=cyan,}
\urlstyle{same}
\usepackage[most]{tcolorbox}
\definecolor{mygray}{RGB}{240,240,240}
\tcbset{
  colback=mygray,
  boxrule=0pt,
}
\graphicspath{ {./images/} }
\newcommand{\HRule}{\begin{center}\rule{0.9\linewidth}{0.2mm}\end{center}}
\newcommand{\customfootnote}[1]{
  \let\thefootnote\relax\footnotetext{#1}
}

\begin{document}
\section*{2025年计算机学科专业基础考试大纲}

\section*{一、数据结构}

\subsection*{【考查目标】}

1. 掌握数据结构的基本概念、基本原理和基本方法。

2. 掌握数据的逻辑结构、存储结构及基本操作的实现,能够对算法进行基本的时间复杂度与空间复杂度的分析。

3. 能够运用数据结构基本原理和方法进行问题的分析与求解,具备采用 \(\mathrm{C}\) 或 \(\mathrm{C} +  +\) 语言设计与实现算法的能力。

\subsection*{一、基本概念}

(一)数据结构的基本概念

(二) 算法的基本概念

\subsection*{二、线性表}

(一)线性表的基本概念 

(二)线性表的实现 

1. 顺序存储 

2. 链式存储 

(三) 线性表的应用

\subsection*{三、栈、队列和数组}

(一)栈和队列的基本概念

(二) 栈和队列的顺序存储结构

(三)栈和队列的链式存储结构

(四) 多维数组的存储

{\color{red} (五) 特殊矩阵的压缩存储}

(六)栈、队列和数组的应用

\subsection*{四、树与二叉树}

(一)树的基本概念

(二)二叉树

1. 二叉树的定义及其主要特征

2. 二叉树的顺序存储结构和链式存储结构

3. 二叉树的遍历

{\color{red} 4. 线索二叉树的基本概念和构造}

{\color{red} (三)树、森林

1. 树的存储结构 

2. 森林与二叉树的转换 

3. 树和森林的遍历 }

(四) 树与二叉树的应用 

1. 哈夫曼(Huffman) 树和哈夫曼编码 

{\color{red}2. 并查集及其应用 

3. 堆及其应用}

\subsection*{五、图}

(一)图的基本概念

(二) 图的存储及基本操作

1. 邻接矩阵

2. 邻接表

{\color{red}3. 邻接多重表、十字链表}

(三)图的遍历

1. 深度优先搜索

2. 广度优先搜索

(四)图的基本应用

1.最小(代价)生成树

2. 最短路径

3. 拓扑排序

4. 关键路径

\subsection*{六、查找}

(一)查找的基本概念

(二)顺序查找法

{\color{red}(三)分块查找法}

(四)折半查找法

{\color{red}(五) 树型查找

1. 二叉树搜索树

2. 平衡二叉树

3. 红黑树}

(六)B 树及其基本操作、 \(B +\) 树的基本概念

(七)散列(Hash)表

{\color{red}(八)字符串模式匹配}

(九)查找算法的分析及应用

\subsection*{七、排序}

(一)排序的基本概念

(二)直接插入排序

(三) 折半插入排序

(四)起泡排序(bubble sort)

(五) 简单选择排序

{\color{red}(六)希尔排序(shell sort)}

(七) 快速排序

(八)堆排序

(九)二路归并排序(merge sort)

{\color{red}(十)基数排序}

(十一)外部排序

(十二)排序算法的分析和应用

\section*{二、计算机组成原理}

\subsection*{【考查目标】}

1. 理解单处理器计算机系统中主要部件的工作原理、组成结构以及相互连接方式。

2.掌握指令集体系结构的基本知识和基本实现方法, 对计算机硬件相关问题进行分析, 并能够对相关部件进行设计。

3. 理解计算机系统的整机概念,能够综合运用计算机组成的基本原理和基本方法,对高级编程语言 (C 语言) 程序中的相关问题进行分析, 具备软硬件协同分析和设计能力。

\subsection*{一、计算机系统概述}

(一)计算机系统层次结构

1. 计算机系统的基本组成 软件加硬件

2. 计算机硬件的基本组成 五大部分

3. 计算机软件和硬件的关系 逻辑等价

4. 计算机系统的工作原理 存储程序

“存储程序” 工作方式, 高级语言程序与机器语言程序之间的转换, 程序和指令的执行过程

(二)计算机性能指标

吞吐量、响应时间;CPU 时钟周期、主频、CPI、CPU 执行时间;{\color{red} MIPS、MFLOPS,GFLOPS,TFLOPS、PFLOPS、EFLOPS、ZFLOPS}。

\subsection*{二、数据的表示和运算}

(一)数制与编码

1. 进位计数制及其数据之间的相互转换

2. 定点数的编码表示

{\color{red} (二)运算方法和运算电路}

1. 基本运算部件

{\color{red} 加法器,算术逻辑部件(ALU)}

2. 加/减运算

{\color{red}补码加/减运算器,标志位的生成。}

3. 乘/除运算

{\color{red}乘/除法运算的基本原理,乘法电路和除法电路的基本结构。}

(三)整数的表示和运算

1. 无符号整数的表示和运算

2. 带符号整数的表示和运算

(四)浮点数的表示和运算

1. 浮点数的表示

{\color{red} IEEE 754 标准}

2. 浮点数的加/减运算

\subsection*{三、存储器层次结构}

(一)存储器的分类

(二)层次化存储器的基本结构

(三)半导体随机存取存储器

1.SRAM 存储器

2.DRAM 存储器

{\color{red} 3. Flash 存储器}

(四)主存储器

1.DRAM 芯片和内存条

2.多模块存储器

3. 主存和CPU 之间的连接

{\color{red} (五)外部存储器

1.磁盘存储器

2. 固态硬盘 (SSD)}

(六)高速缓冲存储器(Cache)

1. Cache 的基本原理

2. Cache 和主存之间的映射方式

3. Cache 中主存块的替换算法

4.Cache 写策略

(七)虚拟存储器

1. 虚拟存储器的基本概念

2. 页式虚拟存储器

基本原理,页表,地址转换, TLB(块表)

3. 段式虚拟存储器

4. 段页式虚拟存储器

\subsection*{四、指令系统}

(一)指令系统的基本概念

(二)指令格式

(三)寻址方式

{\color{red} (四)数据的对齐和大/小端存放方式}

(五) CISC 和 RISC 的基本概念

{\color{red} (六)高级语言程序与机器级代码之间的对应

1.编译器, 汇编器和链路器的基本概念

2. 选择结构语句的机器级表示

3. 循环结构语句的机器级表示

4. 过程 (函数) 调用对应的机器级表示}

\subsection*{五、中央处理器(CPU)}

(一)CPU 的功能和基本结构

(二)指令执行过程

(三)数据通路的功能和基本结构

(四) 控制器的功能和工作原理

(五)异常和中断机制

1. 异常和中断的基本概念

2. 异常和中断的分类

3. 异常和中断的检测与响应

(六)指令流水线

1. 指令流水线的基本概念

2. 指令流水线的基本实现

3. 结构冒险、数据冒险和控制冒险的处理

4. 超标量和动态流水线的基本概念

{\color{red} (七)多处理器基本概念

1.SISD、SIMD、MIMD、 向量处理器的基本概念

2. 硬件多线程的基本概念

3. 多核处理器 (multi-core) 的基本概念

4. 共享内存多处理器 (SMP) 的基本}

\subsection*{六、总线和输入/输出系统}

(一)总线

1. 总线的基本概念 

2. 总线的组成及性能指标 

{\color{red} 3. 总线事务和定时 

(二)I/O 接 口(I/O 控制器) 

1.I/O 接口的功能和基本结构 

2.I/O 端口及其编址 
}
(三) I/O 方式 

1.程序查询方式 

2. 程序中断方式中断的基本概念;中断响应过程;中断处理过程;多重中断和中断屏蔽的概念。 

3.DMA 方式 DMA 控制器的组成, DMA 传送过程

\section*{三、 操作系统}

\subsection*{【考查目标】}

1.掌握操作系统的基本概念、方法和原理,了解操作系统的结构、功能和服务,理解操作系统所采用的的策略、算法和机制。

2. 能够从计算机系统的角度理解并描述应用程序、操作系统内核和计算机硬件协作完成任务的过程。

3. 能够运用操作系统原理,分析并解决计算机系统中与操作系统相关的问题。

\subsection*{一、操作系统概述}

(一)操作系统的基本概念

(二)操作系统的发展历程

(三)程序运行环境

1.CPU 运行模式 内核模式、用户模式

2. 中断和异常的处理

3. 系统调用

4. 程序的链接与装入

5. 程序运行时内存映像与地址空间

{\color{red} (四)操作系统结构

分层, 模块化, 宏内核, 微内核, 外核}

{\color{red} (五)操作系统引导}

{\color{red}(六)虚拟机}
\subsection*{二、进程管理}

(一) 进程与线程

1. 进程与线程的基本概念

2. 进程/线程的状态与转换

3. 线程的实现

内核支持的线程, 线程库支持的线程

4. 进程与线程的组织与控制

5. 进程间通信

{\color{red} 共享内存,消息传递,管道,信号。}

(二) CPU 调度与上下文切换

1. 调度的基本概念 

2. 调度的目标 

3. 调度的实现调度器/调度程序 (scheduler), 调度的时机与调度方式 (抢占式/非抢占式), 闲逛进程, 内核级线程与用户级线程调度 

4.CPU调度算法

{\color{red} 5. 多处理机调度}

6. 上下文及其切换机制

(三) 同步与互斥

1. 同步于互斥的基本概念

2. 基本的实现方法

{\color{red} 软件方法; 硬件方法}。

3. 锁

4. 信号量 

5.条件变量 

6. 经典同步问题生产者-消费者问题;{\color{red} 读者-写者问题;哲学家进餐问题}。

(四)死锁

1. 死锁的基本概念

2. 死锁预防

3. 死锁避免

4. 死锁检测和解除

\subsection*{三、内存管理}

(一)内存管理基础

1. 内存管理的基本概念

逻辑地址空间与物理地址空间, 地址变换, 内存共享, 内存保护, 内存分配与回收

2.连续分配管理方式

3. 页式管理

4. 段式管理

5. 段页式管理

(二)虚拟存储管理

1. 虚拟内存基本概念

2. 请求页式管理

3. 页框分配与回收

4. 页置换算法

5. 内存映射文件 (Memory-Mapped Files)

6. 虚拟存储器性能的影响因素及改进方式

\subsection*{四、文件管理}

(一)文件

1. 文件的基本概念

2. 文件元数据和索引节点 (inode)

3. 文件的操作

建立,删除,打开,关闭,读,写

4. 文件的保护

5. 文件的逻辑结构

6. 文件的物理结构

(二)目录

1. 目录的基本概念

2. 树形目录

3. 目录的操作

4. {\color{red} 硬链接和软链接}

(三)文件系统

1. 文件系统的全局结构(layout)

文件系统在外存中的结构, 文件系统在内存中的结构

2. 外存空闲空间管理办法

3. 虚拟文件系统

{\color{red} 4. 文件系统挂载 (mounting)}

\subsection*{五、输入输出(I/O)管理}

{\color{red}
(一)I/O 管理基础

1. 设备

设备的基本概念,设备的分类, I/O 接口, I/O 端口

2. I/O 控制方式

轮询方式,中断方式, DMA 方式

3.I/O 软件层次结构

中断处理程序, 驱动程序, 设备独立软件, 用户层I/O 软件

4. 输入/输出应用程序接口

字符设备接口,块设备接口,网络设备接口,阻塞/非阻塞I/O

(二)设备独立软件

1. 缓冲区管理

2. 设备分配与回收

3. 假脱机技术(SPOOLing)

4. 设备驱动程序接口
}

(三)外存管理

1.磁盘

磁盘结构,格式化,分区,磁盘调度方法

2. 固态硬盘

读写性能特性, 磨损均衡四

\section*{计算机网络 }

\subsection*{【考查目标】} 

1.掌握计算机网络的基本概念、基本原理和基本方法。 

2. 掌握典型计算机网络的结构、协议、应用以及典型网络设备的工作原理 

3. 能够运用计算机网络的基本概念、基本原理和基本方法进行网络系统的分析、设计和应用。

\subsection*{一、计算机网络概述}

(一)计算机网络基本概念

1.计算机网络的定义、组成与功能

2. 计算机网络的分类

3.计算机网络主要性能指标

(二)计算机网络体系结构

1.计算机网络分层结构

2. 计算机网络协议、接口、服务等概念

3.ISO/OSI 参考模型和 TCP/IP 模型

\subsection*{二、物理层}

(一)通信基础

1. 信道、信号、带宽、码元、波特、速率、信源与信宿等基本概念

2. 奈奎斯特定理与香农定理

3. 编码与调制

4. 电路交换、报文交换与分组交换

5. 数据报与虚电路

(二)传输介质

1. 双绞线、同轴电缆、光纤与无线传输介质 

2. {\color{red} 物理层接口的特性}

(三)物理层设备

1.中继器 

2. 集线器

\subsection*{三、数据链路层}

(一)数据链路层的功能

(二)组帧 

(三)差错控制 1.检错编码 2.{\color{red} 纠错编码-海明码}

(四)流量控制与可靠传输机制 1. 流量控制、可靠传输与滑动窗口机制 2.停止-等待协议 3. 后退 N 帧协议(GBN) 4. 选择重传协议(SR)

(五)介质访问控制 

1. {\color{red}信道划分频分多路复用、时分多路复用、波分多路复用、码分多路复用的概念和基本原理}。 

2. {\color{red} 随即访问 ALOHA 协议;CSMA 协议;} CSMA/CD 协议; CSMA/CA 协议。 

3. 轮询访问令牌传递协议

(六) 局域网

1. 局域网的基本概念与体系结构

2. 以太网与 IEEE 802.3 

3.IEEE802.11 无线局域网 

4.VLAN 基本概念与基本原理

(七)广域网

1. 广域网的基本概念

2.{\color{red} PPP 协议}

(八)数据链路层设备

{\color{red}以太网交换机及其工作原理}

\subsection*{四、网络层}

(一) 网络层的功能

1. 异构网络互联

2. 路由与转发

3.{\color{red}SDN 基本概念}

4. 拥塞控制

(二)路由算法

1. 静态路由与动态路由

2. 距离 - 向量路由算法

3. 链路状态路由算法

4. 层次路由 

(三)IPv4 

1.IPv4 分组 

2.IPv4 地址与 NAT 

3. 子网划分、路由聚集、子网掩码与 CIDR 

4.ARP 协议、DHCP 协议与 ICMP 协议 

(四){\color{red}IPv6} 

1.IPv6 的主要特点 

2.IPv6 地 址

(五)路由协议

1.自治系统 

2. 域内路由与域间路由 

3.RIP 路由协议 

4.OSPF 路由协议 

5.{\color{red}BGP路由协议}


(六){\color{red}IP 组播}


1. 组播的概念


2.IP 组播地址

(七) {\color{red}移动 IP}

1. 移动 IP 的概念

2. 移动 IP 通信过程

(八)网络层设备

1. 路由器的组成和功能

2. 路由表与分组转发

\subsection*{五、传输层}

(一)传输层提供的服务

1. 传输层的功能

2. {\color{red}传输层寻址与端口}

3. 无连接服务与面向连接服务

(二)UDP 协议

1.UDP数据报

2.{\color{red}UDP校验}

(三)TCP协议

1.TCP段(TCP首部固定20字节的内容)

2.TCP连接管理(三次握手建立,四次挥手释放)

3.TCP可靠传输

4.TCP流量控制

5.TCP拥塞控制

\subsection*{六.应用层}

{\color{red} (一) 网络应用模型

1. 客户/服务器(C/S)模型

2. 对等(P2P)模型

(二)DNS 系统

1. 层次域名空间

2. 域名服务器

3. 域名解析过程

(三)FTP

1. FTP 协议的工作原理

2. 控制连接与数据连接

(四)电子邮件

1. 电子邮件系统的组成结构

2. 电子邮件格式与 MIME

3.SMTP 协议与 POP3 协议

(五)WWW

1.WWW 的概念与组成结构

2. HTTP 协议}
\end{document}