\ifx\allfiles\undefined
\documentclass[12pt, a4paper, oneside, UTF8]{ctexbook}
\usepackage{multirow}
\def\path{../../config}
\usepackage{amsthm}
\usepackage{amssymb}
\usepackage{array}
\usepackage{xcolor}
\usepackage{graphicx}
\usepackage{mathrsfs}
\usepackage{enumitem}
\usepackage{geometry}
\usepackage[colorlinks, linkcolor=black]{hyperref}
\usepackage{stackengine}
\usepackage{yhmath}
\usepackage{extarrows}
\usepackage{tikz}
\usepackage{forest}
\usetikzlibrary{decorations.pathreplacing, positioning}
% \usepackage{unicode-math}
\usepackage{esint}
\usepackage{pifont}
\usepackage{tcolorbox}
\tcbuselibrary{skins, breakable}

\usepackage{multicol} 
\usepackage{fontspec} % 使用字体

\setmainfont{Times New Roman}
\setCJKmainfont{LXGWWenKai-Light}[
    SlantedFont=*
]

\usepackage{listings} % 用于插入代码

% 定义代码高亮风格
\lstset{
    basicstyle=\ttfamily\small,        % 基本字体样式(等宽小字体)
    keywordstyle=\color{blue},         % 关键字颜色
    commentstyle=\color{green},        % 注释颜色
    stringstyle=\color{red},           % 字符串颜色
    numbers=none,
    breaklines=true,                   % 自动换行
    frame=single,                      % 代码框边框
    rulecolor=\color{black},           % 边框颜色
    captionpos=b,                      % 标题位置(底部)
    showspaces=false,                  % 不显示空格标记
    showstringspaces=false,            % 不显示字符串中的空格标记
    language=C                         % 设置语言为 C
}

\usepackage{fontawesome5}

\usepackage{amsmath}
\usepackage{booktabs, array}
\usepackage{makecell}
\usepackage{fancyhdr}
\usepackage[dvipsnames, svgnames]{xcolor}
\usepackage{listings}
\usepackage{tasks}[2020/01/11]

\everymath{\displaystyle}

\definecolor{mygreen}{rgb}{0,0.6,0}
\definecolor{mygray}{rgb}{0.5,0.5,0.5}
\definecolor{mymauve}{rgb}{0.58,0,0.82}
\definecolor{NavyBlue}{RGB}{0,0,128}
\definecolor{Rhodamine}{RGB}{255,0,255}
\definecolor{PineGreen}{RGB}{0,128,0}

\graphicspath{ {figures/},{../figures/}, {config/}, {../config/} }

\linespread{1.6}

\geometry{
    top=25.4mm, 
    bottom=25.4mm, 
    left=20mm, 
    right=20mm, 
    headheight=2.17cm, 
    headsep=4mm, 
    footskip=12mm
}

\setenumerate[1]{itemsep=5pt,partopsep=0pt,parsep=\parskip,topsep=5pt}
\setitemize[1]{itemsep=5pt,partopsep=0pt,parsep=\parskip,topsep=5pt}
\setdescription{itemsep=5pt,partopsep=0pt,parsep=\parskip,topsep=5pt}



% \begin{lstlisting}[language=TeX] ... \end{lstlisting}

% 定理环境设置
% ---------- 颜色 ----------
\definecolor{ExBlue}{HTML}{4F81BD}
\definecolor{SolGreen}{HTML}{77933C}
\definecolor{DefRed}{HTML}{C5504B}
\definecolor{ThmOrange}{HTML}{E97132}
\definecolor{RemGray}{HTML}{7F7F7F}
\definecolor{CorPurple}{HTML}{7030A0}
\definecolor{ForGray}{HTML}{595959}

% ---------- 通用“变色”模板 ----------
\tcbset{
    mybox/.style n args={1}{
        enhanced, breakable,
        arc=6pt,
        boxrule=0.6pt,
        left=8pt, right=8pt, top=6pt, bottom=6pt,
        drop shadow={black!25},
        fonttitle=\bfseries,
        coltitle=white,
        colbacktitle=#1!85,
        colback=#1!10,
        colframe=#1,
    }
}

% ---------- 各环境 ----------
% 例题
\newtcolorbox{example}[1][]{mybox={ExBlue}, title={\ifstrempty{#1}{Example}{#1}}}
% 解答
\newtcolorbox{solution}[1][]{mybox={SolGreen}, title={\ifstrempty{#1}{Solution}{#1}}}
% 定义
\newtcolorbox{definition}[1][]{mybox={DefRed}, title={\ifstrempty{#1}{Definition}{#1}}}
% 定理
\newtcolorbox{theorem}[1][]{mybox={ThmOrange}, title={\ifstrempty{#1}{Theorem}{#1}}}
% 标注
\newtcolorbox{remark}[1][]{mybox={RemGray}, title={\ifstrempty{#1}{Remark}{#1}}}
% 推论
\newtcolorbox{corollary}[1][]{mybox={CorPurple}, title={\ifstrempty{#1}{Corollary}{#1}}}
% 公式
\newtcolorbox{formula}[1][]{mybox={ForGray}, title={\ifstrempty{#1}{Formula}{#1}}}


\settasks{
    label-format = \bfseries,
    label        = \Alph*.,
    label-width  = 1.2em,
    label-offset = 0.3em,
    item-indent  = 1.9em,
    column-sep   = 0.5em
}

\newenvironment{choices}[1][4]   % 默认 4 栏
    {\begin{tasks}(#1)}
    {\end{tasks}}

% 自定义命令的文件

\def\d{\mathrm{d}}
\def\R{\mathbb{R}}
\def\P{\partial} 
\newcommand{\bs}[1]{\begin{solution}#1\end{solution}}
\newcommand{\bt}[1][1]{% 默认参数为1
    \ensuremath{% 确保数学模式
        \foreach \n in {1,...,#1} {\blacktriangle}% 循环输出 #1 个黑色三角形
    }%
}

\newcommand{\bl}[1][1]{% 默认参数为1
    \ensuremath{% 确保数学模式
        \foreach \n in {1,...,#1} {\blacklozenge}% 循环输出 #1 个黑色三角形
    }%
}
\newif\ifshowanswers
%\showanswerstrue % 注释掉这行就不显示答案

% 定义答案环境
\newcommand{\answer}[1]{%
    \ifshowanswers
        #1%
    \fi
}




% 修改参数改变封面样式,0 默认原始封面、内置其他1、2、3种封面样式
\def\myIndex{3}


\ifnum\myIndex>0
    \input{\path/cover_package_\myIndex} 
\fi

\def\myTitle{冲刺150笔记}
\def\myAuthor{Weary Bird}
\def\myDateCover{\today}
\def\myDateForeword{\today}
\def\myForeword{行香子}
\def\myForewordText{
树绕村庄,水满陂塘;倚东风、豪兴徜徉。小园几许,收尽春光。有桃花红,李花白,菜花黄。 \\
远远苔墙,隐隐茅堂;飏青旗、流水桥旁。偶然乘兴,步过东冈。正莺儿啼,燕儿舞,蝶儿忙。 \\
}
\def\mySubheading{知错能改善莫大焉}


\begin{document}
% \input{../config/cover}
\else
\fi
\chapter{二次型}

\section{求二次型的标准形}

\begin{enumerate}[label=\arabic*.]
    \item (2016,数二、三) 设二次型 $ f(x_1, x_2, x_3) = a x_1^2 + a x_2^2 + (a-1) x_3^2 + 2 x_1 x_3 - 2 x_2 x_3 $ 的正、负惯性指数分别为 1,2,则
    \begin{enumerate}
        \item  $ a > 1 $
        \item  $ a < -1 $
        \item  $ -1 < a < 1 $
        \item  $ a = 1 $ 或 $ a = -1 $
    \end{enumerate}
    
    \begin{solution}
    【详解】
    \end{solution}
    
    \item (2022,数一) 设二次型 $ f(x_1, x_2, x_3) = \sum_{i=1}^3 \sum_{j=1}^3 i j x_i x_j $。
    \begin{enumerate}
        \item 求 $ f(x_1, x_2, x_3) $ 对应的矩阵;
        \item 求正交变换 $ x = Q y $,将 $ f(x_1, x_2, x_3) $ 化为标准形;
        \item 求 $ f(x_1, x_2, x_3) = 0 $ 的解。
    \end{enumerate}
    
    \begin{solution}
    【详解】
    \end{solution}
    
    \item (2020,数一、三) 设二次型 $ f(x_1, x_2) = 4 x_1^2 + 4 x_2^2 + 4 x_1 x_2 $ 经正交变换 $ \begin{pmatrix} x_1 \\ x_2 \end{pmatrix} = Q \begin{pmatrix} y_1 \\ y_2 \end{pmatrix} $ 化为二次型 $ g(y_1, y_2) = y_1^2 + b y_2^2 $,其中 $ b \geq 0 $。 
    \begin{enumerate}
        \item 求 $ a, b $ 的值;
        \item 求正交矩阵 $ Q $。
    \end{enumerate}
    
    \begin{solution}
    【详解】
    \end{solution}
\end{enumerate}

\section{合同的判定}

\begin{enumerate}[label=\arabic*.,start=4]
    \item (2008,数二、三) 设 $ A = \begin{pmatrix} 1 & 2 \\ 2 & 1 \end{pmatrix} $,与 $ A $ 合同的矩阵是
    \begin{enumerate}
        \item $ \begin{pmatrix} 1 & 1 \\ 1 & 2 \end{pmatrix} $
        \item $ \begin{pmatrix} 2 & 1 \\ 1 & 2 \end{pmatrix} $
        \item $ \begin{pmatrix} 2 & 1 \\ 1 & 1 \end{pmatrix} $
        \item $ \begin{pmatrix} 1 & 2 \\ 1 & 2 \end{pmatrix} $
    \end{enumerate}
    
    \begin{solution}
    【详解】
    \end{solution}
    
    \item 设 $ A, B $ 为 $ n $ 阶实对称可逆矩阵,则存在 $ n $ 阶可逆矩阵 $ P $,使得
    \begin{enumerate}
        \item $ P A P = B $;
        \item $ P^{-1} A B P = B A $;
        \item $ P^{-1} A P = B $;
        \item $ P^T A P = B $。
    \end{enumerate}
    成立的个数是
    \begin{enumerate}
        \item 1
        \item 2
        \item 3
        \item 4
    \end{enumerate}
    
    \begin{solution}
    【详解】
    \end{solution}
\end{enumerate}

\section{二次型正定与正定矩阵的判定}

\begin{enumerate}[label=\arabic*.,start=6]
    \item (2017,数一、二、三) 设 $ A $ 为 $ m \times n $ 阶矩阵,且 $ r(A) = n $,则下列结论
    \begin{enumerate}
        \item $ A^T A $ 与单位矩阵等价;
        \item $ A^T A $ 与对角矩阵相似;
        \item $ A^T A $ 与单位矩阵合同;
        \item $ A^T A $ 正定。
    \end{enumerate}
    正确的个数是
    \begin{enumerate}
        \item 1
        \item 2
        \item 3
        \item 4
    \end{enumerate}
    
    \begin{solution}
    【详解】
    \end{solution}
    
    \item 证明:
    \begin{enumerate}
        \item 设 $ A $ 为 $ n $ 阶正定矩阵,$ B $ 为 $ n $ 阶反对称矩阵,则 $ A - B^2 $ 为正定矩阵;
        \item 设 $ A, B $ 为 $ n $ 阶矩阵,且 $ r(A + B) = n $,则 $ A^T A + B^T B $ 为正定矩阵。
    \end{enumerate}
    
    \begin{solution}
    【详解】
    \end{solution}
\end{enumerate}

\ifx\allfiles\undefined
\end{document}
\fi