\ifx\allfiles\undefined
\documentclass[12pt, a4paper, oneside, UTF8]{ctexbook}
\usepackage{multirow}
\def\path{../../config}
\usepackage{amsthm}
\usepackage{amssymb}
\usepackage{array}
\usepackage{xcolor}
\usepackage{graphicx}
\usepackage{mathrsfs}
\usepackage{enumitem}
\usepackage{geometry}
\usepackage[colorlinks, linkcolor=black]{hyperref}
\usepackage{stackengine}
\usepackage{yhmath}
\usepackage{extarrows}
\usepackage{tikz}
\usepackage{forest}
\usetikzlibrary{decorations.pathreplacing, positioning}
% \usepackage{unicode-math}
\usepackage{esint}
\usepackage{pifont}
\usepackage{tcolorbox}
\tcbuselibrary{skins, breakable}

\usepackage{multicol} 
\usepackage{fontspec} % 使用字体

\setmainfont{Times New Roman}
\setCJKmainfont{LXGWWenKai-Light}[
    SlantedFont=*
]

\usepackage{listings} % 用于插入代码

% 定义代码高亮风格
\lstset{
    basicstyle=\ttfamily\small,        % 基本字体样式(等宽小字体)
    keywordstyle=\color{blue},         % 关键字颜色
    commentstyle=\color{green},        % 注释颜色
    stringstyle=\color{red},           % 字符串颜色
    numbers=none,
    breaklines=true,                   % 自动换行
    frame=single,                      % 代码框边框
    rulecolor=\color{black},           % 边框颜色
    captionpos=b,                      % 标题位置(底部)
    showspaces=false,                  % 不显示空格标记
    showstringspaces=false,            % 不显示字符串中的空格标记
    language=C                         % 设置语言为 C
}

\usepackage{fontawesome5}

\usepackage{amsmath}
\usepackage{booktabs, array}
\usepackage{makecell}
\usepackage{fancyhdr}
\usepackage[dvipsnames, svgnames]{xcolor}
\usepackage{listings}
\usepackage{tasks}[2020/01/11]

\everymath{\displaystyle}

\definecolor{mygreen}{rgb}{0,0.6,0}
\definecolor{mygray}{rgb}{0.5,0.5,0.5}
\definecolor{mymauve}{rgb}{0.58,0,0.82}
\definecolor{NavyBlue}{RGB}{0,0,128}
\definecolor{Rhodamine}{RGB}{255,0,255}
\definecolor{PineGreen}{RGB}{0,128,0}

\graphicspath{ {figures/},{../figures/}, {config/}, {../config/} }

\linespread{1.6}

\geometry{
    top=25.4mm, 
    bottom=25.4mm, 
    left=20mm, 
    right=20mm, 
    headheight=2.17cm, 
    headsep=4mm, 
    footskip=12mm
}

\setenumerate[1]{itemsep=5pt,partopsep=0pt,parsep=\parskip,topsep=5pt}
\setitemize[1]{itemsep=5pt,partopsep=0pt,parsep=\parskip,topsep=5pt}
\setdescription{itemsep=5pt,partopsep=0pt,parsep=\parskip,topsep=5pt}



% \begin{lstlisting}[language=TeX] ... \end{lstlisting}

% 定理环境设置
% ---------- 颜色 ----------
\definecolor{ExBlue}{HTML}{4F81BD}
\definecolor{SolGreen}{HTML}{77933C}
\definecolor{DefRed}{HTML}{C5504B}
\definecolor{ThmOrange}{HTML}{E97132}
\definecolor{RemGray}{HTML}{7F7F7F}
\definecolor{CorPurple}{HTML}{7030A0}
\definecolor{ForGray}{HTML}{595959}

% ---------- 通用“变色”模板 ----------
\tcbset{
    mybox/.style n args={1}{
        enhanced, breakable,
        arc=6pt,
        boxrule=0.6pt,
        left=8pt, right=8pt, top=6pt, bottom=6pt,
        drop shadow={black!25},
        fonttitle=\bfseries,
        coltitle=white,
        colbacktitle=#1!85,
        colback=#1!10,
        colframe=#1,
    }
}

% ---------- 各环境 ----------
% 例题
\newtcolorbox{example}[1][]{mybox={ExBlue}, title={\ifstrempty{#1}{Example}{#1}}}
% 解答
\newtcolorbox{solution}[1][]{mybox={SolGreen}, title={\ifstrempty{#1}{Solution}{#1}}}
% 定义
\newtcolorbox{definition}[1][]{mybox={DefRed}, title={\ifstrempty{#1}{Definition}{#1}}}
% 定理
\newtcolorbox{theorem}[1][]{mybox={ThmOrange}, title={\ifstrempty{#1}{Theorem}{#1}}}
% 标注
\newtcolorbox{remark}[1][]{mybox={RemGray}, title={\ifstrempty{#1}{Remark}{#1}}}
% 推论
\newtcolorbox{corollary}[1][]{mybox={CorPurple}, title={\ifstrempty{#1}{Corollary}{#1}}}
% 公式
\newtcolorbox{formula}[1][]{mybox={ForGray}, title={\ifstrempty{#1}{Formula}{#1}}}


\settasks{
    label-format = \bfseries,
    label        = \Alph*.,
    label-width  = 1.2em,
    label-offset = 0.3em,
    item-indent  = 1.9em,
    column-sep   = 0.5em
}

\newenvironment{choices}[1][4]   % 默认 4 栏
    {\begin{tasks}(#1)}
    {\end{tasks}}

% 自定义命令的文件

\def\d{\mathrm{d}}
\def\R{\mathbb{R}}
\def\P{\partial} 
\newcommand{\bs}[1]{\begin{solution}#1\end{solution}}
\newcommand{\bt}[1][1]{% 默认参数为1
    \ensuremath{% 确保数学模式
        \foreach \n in {1,...,#1} {\blacktriangle}% 循环输出 #1 个黑色三角形
    }%
}

\newcommand{\bl}[1][1]{% 默认参数为1
    \ensuremath{% 确保数学模式
        \foreach \n in {1,...,#1} {\blacklozenge}% 循环输出 #1 个黑色三角形
    }%
}
\newif\ifshowanswers
%\showanswerstrue % 注释掉这行就不显示答案

% 定义答案环境
\newcommand{\answer}[1]{%
    \ifshowanswers
        #1%
    \fi
}




% 修改参数改变封面样式,0 默认原始封面、内置其他1、2、3种封面样式
\def\myIndex{3}


\ifnum\myIndex>0
    \input{\path/cover_package_\myIndex} 
\fi

\def\myTitle{冲刺150笔记}
\def\myAuthor{Weary Bird}
\def\myDateCover{\today}
\def\myDateForeword{\today}
\def\myForeword{行香子}
\def\myForewordText{
树绕村庄,水满陂塘;倚东风、豪兴徜徉。小园几许,收尽春光。有桃花红,李花白,菜花黄。 \\
远远苔墙,隐隐茅堂;飏青旗、流水桥旁。偶然乘兴,步过东冈。正莺儿啼,燕儿舞,蝶儿忙。 \\
}
\def\mySubheading{知错能改善莫大焉}


\begin{document}
% \input{../config/cover}
\else
\fi
\chapter{二次型}
$$
\fbox{
    \begin{tabular}{c}
        二 \\
        次 \\
        型
    \end{tabular}
}\begin{cases}
    \fbox{二次型与标准型} &\begin{cases}
        \fbox{定义} & f = x^TAx \\
        \fbox{标准型的求法} &\begin{cases}
            \fbox{拉格朗日配方法} \\
            \fbox{合同变换法} \\
            \fbox{正交变换法}
        \end{cases} 
    \end{cases} \\
    \fbox{合同矩阵} & \begin{cases}
        \fbox{定义} & B = C^TAC \\
        \fbox{充要条件} & \begin{cases}
            \iff x^TAx\text{与}x^TBx\text{有相同的正负惯性指数} \\
            \iff A,B\text{有相同的正,负特征值个数}
        \end{cases} \\
        \fbox{充分条件} & A\text{与}B\text{相似} \\
        \fbox{必要条件} & A\text{与}B\text{等价}
    \end{cases} \\
    \fbox{正定矩阵} &\begin{cases}
        \fbox{定义} &\forall x\neq 0,\text{有}x^TAx>0 \\
        \fbox{性质} &\\
        \fbox{充要条件}  &\begin{cases}
            \iff f\text{的正惯性指数为} n \\
            \iff A\text{与}E\text{合同} \\
            \iff A\text{的特征值均大于零} \\
            \iff A\text{的顺序主子式均大于零}
        \end{cases} \\
        \fbox{必要条件} &\begin{cases}
            a_{ii} > 0(i=1,2,\ldots,n) \\
            \left|A\right| > 0
        \end{cases}
    \end{cases}
\end{cases}
$$
\section{求二次型的标准形}
\begin{definition}[常规方法]
    (方法一\ 拉格朗日配方法) \\
    $\circ 1$令$f(x_1,x_2,x_3)=d_1(x_1+x_2+x_3)^2+d_2(x_2+cx_3)^2+d_3x_3^2 = d_1y_1^2+d_2y_2^2+d_3y_3^2$ \\
    $\circ 2$换元,令
    $$\begin{cases}
        y_1 =x_1+ax_2+bx_3 \\
        y_2 =x_2+cx_3 \\
        y_3 =x_3
    \end{cases}\implies \begin{cases}
        x_3 = y_3 \\
        x_2 = y_2 -cy_3 \\
        x_3 = y_1 - ay_2 + (ac-b)y_3 
    \end{cases}$$
    从而可以通过\textbf{可逆线性变换}$x=Cy$其中$C=\begin{pmatrix}
        1 & -a & ac-b \\
        0 & 1 & -c \\
        0 & 0 & 1
    \end{pmatrix}$ \\
    (方法二\ 正交变换法) $x=Qy$ 二次型转换为标准型$\lambda_1y_1^2+\lambda_2y_2^2+\lambda_3y_3^2$,系数为特征值 
\end{definition}
\begin{definition}[合同变化法]
    $$
    \begin{pmatrix}
        A \\
        E
    \end{pmatrix} \xrightarrow{\text{行列成对的初等变换}} \begin{pmatrix}
        \Lambda \\
        C
    \end{pmatrix}
    $$此时$C^TAC=\Lambda$,举例说法计算过程
    \begin{align*}
        \begin{pmatrix}
            A \\
            E 
        \end{pmatrix} &= \begin{pmatrix}
            1 & 2 & 3 \\
            2 & 3 & 4 \\
            3 & 4 & 5 \\
            \hline 
            1 & 0 & 0 \\
            0 & 1 & 0 \\
            0 & 0 & 1
        \end{pmatrix}\xrightarrow{}\begin{pmatrix}
            1 & 0 & 0 \\
            0 & -1 & 0 \\
            0 & 0 & 0 \\
            \hline 
            1 & -2 & 1 \\
            0 & 1 & -2 \\
            0 & 0 & 1
        \end{pmatrix}
    \end{align*}
    即经过可逆线性变换$x=Cy$其中$C=\begin{pmatrix}
        1 & -2 & 1 \\
        0 & 1 & -2 \\
        0 & 0 & 1
    \end{pmatrix}$等价于做如下变量代换$\begin{cases}
        x_1 = y_1 - 2y_2 + y_3 \\
        x_2 = y_2 - 2y_3 \\
        x_3 = y_3 
    \end{cases}$此时标准型为$f(y_1,y_2,y_3)=y_1^2-y_2^2$
\end{definition}
\begin{enumerate}[label=\arabic*.]
    \item (2016,数二、三) 设二次型 $ f(x_1, x_2, x_3) = a(x^1_1+x^2_2+x_3^2)+
    2x_1x_2+2x_2x_3+2x_1x_3$ 的正、负惯性指数分别为1,2则 \\
    A. $a > 1$\qquad B. $a < -2$ \qquad C.$-2<a<1$\qquad D.$a=1$或$a=-2$ 
    
    \begin{solution}[直接求特征值]
    由题设可知$A=\begin{pmatrix}
        a & 1 & 1 \\
        1 & a & 1 \\
        1 & 1 & a 
    \end{pmatrix}$
    求解特征值方程$\left|A-\lambda E\right|=0\implies (a-\lambda+2)(a-\lambda-1)^2$ 由题设可知 
    $$
    \begin{cases}
        a+2 > 0 \\
        a-1 < 0 
    \end{cases} \implies -2 < a < 1
    $$
    \end{solution}
    
    \begin{solution}[分解为秩1矩阵]
        $A=(1,1,1)\begin{pmatrix}
            1 \\
            1 \\
            1
        \end{pmatrix}+(a-1)E$ 从而可知其特征值为$\begin{cases}
            \lambda_1 = a + 2 \\
            \lambda_2=\lambda_3 = a - 1
        \end{cases}$
    \end{solution}
    \item (2022,数一) 设二次型 $ f(x_1, x_2, x_3) = 
    \displaystyle \sum_{i=1}^3 \sum_{j=1}^3 i j x_i x_j $。
    \begin{enumerate}
        \item [(1)] 求 $ f(x_1, x_2, x_3) $ 对应的矩阵;
        \item [(2)] 求正交变换 $ x = Q y $,将 $ f(x_1, x_2, x_3) $ 化为标准形;
        \item [(3)] 求 $ f(x_1, x_2, x_3) = 0 $ 的解。
    \end{enumerate}
    
    \begin{solution}
    由题设可知$f=x_1^2+4x_2^2+9x_3^2+4x_1x_2+6x_1x_3+12x_1x_3$ \\
    (1)矩阵A为$\begin{pmatrix}
        1 & 2 & 3 \\
        2 & 4 & 6 \\
        3 & 6 & 9
    \end{pmatrix} = (1,2,3)\begin{pmatrix}
        1 \\
        2 \\
        3
    \end{pmatrix}$,即$r(A)=1$ \\
    (2)由秩一矩阵特性可知$\lambda_1=tr(A)=14,\alpha_1=(1,2,3)^T$ 通过知一求二,设$\alpha_2=(a,b,0)^T,\alpha_3=(-b,a,c)^T$可知
    当$\lambda_2=\lambda_3=0,\alpha_2=(-2,1,0)^T,\alpha_3=(-3,-6,5)$ \\
    单位化后有 
    $$
    \gamma_1=\frac{1}{\sqrt{14}}(1,2,3)^T,\gamma_2=\frac{1}{\sqrt{5}}(-2,1,0)^T,\gamma_3=\frac{1}{\sqrt{70}}(-3,-6,5)^T
    $$
    记$Q=(\gamma_1,\gamma_2,\gamma_3)$,此时经过$x=Qy$二次型化为标准型$f=14y_1^2$ \\
    (3)方法一,解$f=14y_1^2=0\implies y_1=0,y_2=k_1,y_3=k_2$又$x=Qy=(\gamma_1,\gamma_2,\gamma_3)\begin{pmatrix}
        0 \\
        k_1 \\
        k_2
    \end{pmatrix}=k_1\gamma_2+k_2\gamma_3$其中$k_1,k_2$为任意常数 \\
    方法二,配方直接接. 
    $$
    f(x_1,x_2,x_3)=(x_1+2x_2+3x_3)^2 = 0
    $$
    从而可知$x_1+2x_2+3x_3=0$其基础解系为$\xi_1=(-2,1,0)^T,\xi_2=(-3,0,1)^T$,从而可知$f=0$的通解为 
    $k_1\xi_1+k_2\xi_2$其中$k_1,k_2$为任意常数
    \end{solution}
    
    \item (2020,数一、三) 设二次型 $ f(x_1, x_2) = 4 x_1^2 + 4 x_2^2 + 4 x_1 x_2 $
    经正交变换 $ \begin{pmatrix} x_1 \\ x_2 \end{pmatrix} = Q \begin{pmatrix} y_1 \\ y_2 
    \end{pmatrix} $ 化为二次型 $ g(y_1, y_2) = ay_1^2 + 4y_1y_2 + b y_2^2 $,其中 $ b \geq 0 $。 
    \begin{enumerate}
        \item [(1)] 求 $ a, b $ 的值;
        \item [(2)] 求正交矩阵 $ Q $。
    \end{enumerate}
    
    \begin{solution}
    (1)$A=\begin{pmatrix}
        1 & -2 \\
        -2 & 4
    \end{pmatrix}$ 由题设可知$Q_1^TAQ_1=\Lambda=Q_2^TBQ_2$ 从而可知$A\sim B$,又因为$r(A)=1$可知A的特征值与特征向量分别为\\
    当$\lambda_1=tr(A)=5,\alpha_1=(1,-2)^T$ \\
    当$\lambda_2=0,\alpha_2=(2,1)^T$ \\
    单位化后有$\gamma_1=\frac{1}{\sqrt{5}}(1,-2)^T,\gamma_2=\frac{1}{\sqrt{5}}(2,1)^T$从而$Q_1=(\gamma_1,\gamma_2)$ \\
    有题设可知$B=\begin{pmatrix}
        a & 2 \\
        2 & b 
    \end{pmatrix}$通过$B\sim A$ 可知 $\begin{cases}
        ab - 4 = 0 \\
        a + b = 5 \\
        a > b 
    \end{cases}\implies \begin{cases}
        a = 4 \\
        b = 1
    \end{cases}$\\
    当$\lambda_1=tr(B)=5,\beta_1=(2,1)^T;\lambda_2=0,\beta_2=(-1,2)^T$ \\
    单位化后$\gamma_1^{'}=\frac{1}{\sqrt{5}}(2,1)^T;\gamma_2^{'}=\frac{1}{\sqrt{5}}(-1,2)^T$ 从而有$Q_2=(\gamma_1^{'},\gamma_2^{'})$ \\
    因此$B=Q_2Q_1^TAQ_1Q_2^T$进而可知$Q=Q_1Q_2^T=\frac{1}{5}\begin{pmatrix}
        4 & -3 \\
        -3 & -4
    \end{pmatrix}$
    \end{solution}
\end{enumerate}

\section{合同的判定}

\begin{enumerate}[label=\arabic*.,start=4]
    \item (2008,数二、三) 设 $ A = \begin{pmatrix} 1 & 2 \\ 2 & 1 \end{pmatrix} $,与 $ A $ 合同的矩阵是 \\
        A.$ \begin{pmatrix} -2 & 1 \\ 1 & -2 \end{pmatrix} $\qquad
        B.$ \begin{pmatrix} 2 & -1 \\ -1 & 2 \end{pmatrix} $\qquad
        C.$ \begin{pmatrix} 2 & 1 \\ 1 & 2 \end{pmatrix} $\qquad
        D.$\begin{pmatrix} 1 & -2 \\ -2 & 1 \end{pmatrix} $
    
    \begin{solution}
    D 
    \end{solution}
    
    \item 设 $ A, B $ 为 $ n $ 阶实对称可逆矩阵,则存在 $ n $ 阶可逆矩阵 $ P $,使得 \\
    $(I)PA=B\qquad (II)P^{-1}ABP=BA\qquad (III)P^{-1}AP=B\qquad(IV)P^{T}A^2P=B^2$ \\
    成立的个数是 \\
    A.1\qquad B.2\qquad C.3\qquad D.4
    
    \begin{solution}
    (I) 有$BA^{-1}A=B$  \\
    (II) $A^{-1}ABA=BA$  \\
    (III) 令$A=E,B=-E$则$P^{-1}AP=E\neq B$ \\
    (IV) 设$A$的任意特征值为$\lambda$则$A^2$的任意特征值为$\lambda^2$又$A$可逆可知$\lambda^2>0$,同理可知$B^2$的任意特征值为$\lambda^2>0$ 
    从而$A^2,B^2$均只有n个正的特征值从而$A,B$合同. 
    \end{solution}
\end{enumerate}

\section{二次型正定与正定矩阵的判定}
\begin{definition}[方法]
    \begin{enumerate}
        \item [(1)]正定的定义 
            \begin{enumerate}
                \item [$\circ 1$] A为实对称矩阵 
                \item [$\circ 2$] $\forall\alpha\neq 0$有$\alpha^TA\alpha>0$
            \end{enumerate}
            注意着两个条件缺一不可!
        \item [(2)] 充要条件
            \begin{enumerate}
                \item [$\circ 1$] 对于n阶矩阵,其正惯性指数为n 
                \item [$\circ 2$] 与单位矩阵$E$合同 
                \item [$\circ 3$] 对于任意特征值$\lambda_i>0$
                \item [$\circ 4$] 对于任意顺序主子式均大于0
            \end{enumerate}
    \end{enumerate}
\end{definition}
\begin{enumerate}[label=\arabic*.,start=6]
    \item 设 $ A $ 为 $ m \times n $ 阶矩阵,且 $ r(A) = m $,则下列结论
    \begin{enumerate}
        \item[(1)] $ A^T A $ 与单位矩阵等价;
        \item[(2)] $ A^T A $ 与对角矩阵相似;
        \item[(3)] $ A^T A $ 与单位矩阵合同;
        \item[(4)] $ A^T A $ 正定。
    \end{enumerate}
    正确的个数是 \\
    A. 1 \qquad B. 2\qquad C.3\qquad D.4 
    
    \begin{corollary}[$r(A)=m$时有关$AA^T$的结论]
        对于矩阵$A_{m\times n},r(A)=m$此时$AA^T_{m\times m}$ 有如下结论 
        \begin{choices}[2]
        \task $\iff \left|AA^T\right|\neq 0$ \task $\iff AA^T$可逆 
        \task $\iff r(AA^T)=r(A)=m$  \task $\iff AA^T$与$E_{m\times m}$等价 
        \task $\iff AA^T$行(列)向量组线性无关 \task $\iff AA^TX=0$只有零解
        \task $\iff AA^TX=b$有唯一解 
        \task $\iff \lambda_i\neq 0$ \task $\implies AA^T$可相似对角化 
        \task $\implies AA^T$为实对称矩阵 \task $\iff AA^T$与$E$合同 
        \task $\iff AA^T$正定
        \end{choices}
    \end{corollary}
    
    \item 证明:
    \begin{enumerate}
        \item[(1)] 设 $ A $ 为 $ n $ 阶正定矩阵,$ B $ 为 $ n $ 阶反对称矩阵,则 $ A - B^2 $ 为正定矩阵;
        \item[(2)] 设 $ A, B $ 为 $ n $ 阶矩阵,且 $ r(A + B) = n $,则 $ A^T A + B^T B $ 为正定矩阵。
    \end{enumerate}
    
    \begin{solution}
    (1)由$A^T=A,B^T=-B$因此$(A-B^2)^T=A^T-(B^T)^2=A-B^2$故而$A-B^2$为实对称矩阵. \\
    $\forall \alpha\neq 0$ 
    \begin{align*}
        \alpha^T(A-B^2)\alpha &=\alpha^TA\alpha-\alpha^TB^2\alpha\\
        &=\alpha^TA\alpha+\alpha^TB^TB\alpha 
    \end{align*}
    又$\alpha^TA\alpha>0,\alpha^TB^TB\alpha\geq 0$ 从而$\alpha^T(A-B^2)\alpha>0$ 故而$A-B^2$为正定矩阵. \\
    (2)$(A^TA+B^TB)^T=A^TA+B^B$从而其为实对称矩阵 \\
    $\forall \alpha\neq 0$有
    \begin{align*}
        \alpha^T(A^TA+B^TB)\alpha &= \alpha^TA^TA\alpha + \alpha^TB^TB\alpha \\
        &=(A\alpha)^TA\alpha+(B\alpha)^TB\alpha \geq 0
    \end{align*}
    当且仅当$A\alpha=B\alpha=0$时候上式才能取0,此时有$(A+B)\alpha=0$由$\alpha\neq 0$故$(A+B)X=0$由非零解而$r(A+B)=n$矛盾,从而
    不可能$A\alpha=B\alpha=0$故而上式只能大于0. 从而题设得证.
    \end{solution}
\end{enumerate}

\ifx\allfiles\undefined
\end{document}
\fi