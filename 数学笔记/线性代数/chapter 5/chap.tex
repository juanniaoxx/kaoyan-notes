\ifx\allfiles\undefined
\documentclass[12pt, a4paper, oneside, UTF8]{ctexbook}
\usepackage{multirow}
\def\path{../../config}
\usepackage{amsthm}
\usepackage{amssymb}
\usepackage{array}
\usepackage{xcolor}
\usepackage{graphicx}
\usepackage{mathrsfs}
\usepackage{enumitem}
\usepackage{geometry}
\usepackage[colorlinks, linkcolor=black]{hyperref}
\usepackage{stackengine}
\usepackage{yhmath}
\usepackage{extarrows}
\usepackage{tikz}
\usepackage{forest}
\usetikzlibrary{decorations.pathreplacing, positioning}
% \usepackage{unicode-math}
\usepackage{esint}
\usepackage{pifont}
\usepackage{tcolorbox}
\tcbuselibrary{skins, breakable}

\usepackage{multicol} 
\usepackage{fontspec} % 使用字体

\setmainfont{Times New Roman}
\setCJKmainfont{LXGWWenKai-Light}[
    SlantedFont=*
]

\usepackage{listings} % 用于插入代码

% 定义代码高亮风格
\lstset{
    basicstyle=\ttfamily\small,        % 基本字体样式(等宽小字体)
    keywordstyle=\color{blue},         % 关键字颜色
    commentstyle=\color{green},        % 注释颜色
    stringstyle=\color{red},           % 字符串颜色
    numbers=none,
    breaklines=true,                   % 自动换行
    frame=single,                      % 代码框边框
    rulecolor=\color{black},           % 边框颜色
    captionpos=b,                      % 标题位置(底部)
    showspaces=false,                  % 不显示空格标记
    showstringspaces=false,            % 不显示字符串中的空格标记
    language=C                         % 设置语言为 C
}

\usepackage{fontawesome5}

\usepackage{amsmath}
\usepackage{booktabs, array}
\usepackage{makecell}
\usepackage{fancyhdr}
\usepackage[dvipsnames, svgnames]{xcolor}
\usepackage{listings}
\usepackage{tasks}[2020/01/11]

\everymath{\displaystyle}

\definecolor{mygreen}{rgb}{0,0.6,0}
\definecolor{mygray}{rgb}{0.5,0.5,0.5}
\definecolor{mymauve}{rgb}{0.58,0,0.82}
\definecolor{NavyBlue}{RGB}{0,0,128}
\definecolor{Rhodamine}{RGB}{255,0,255}
\definecolor{PineGreen}{RGB}{0,128,0}

\graphicspath{ {figures/},{../figures/}, {config/}, {../config/} }

\linespread{1.6}

\geometry{
    top=25.4mm, 
    bottom=25.4mm, 
    left=20mm, 
    right=20mm, 
    headheight=2.17cm, 
    headsep=4mm, 
    footskip=12mm
}

\setenumerate[1]{itemsep=5pt,partopsep=0pt,parsep=\parskip,topsep=5pt}
\setitemize[1]{itemsep=5pt,partopsep=0pt,parsep=\parskip,topsep=5pt}
\setdescription{itemsep=5pt,partopsep=0pt,parsep=\parskip,topsep=5pt}



% \begin{lstlisting}[language=TeX] ... \end{lstlisting}

% 定理环境设置
% ---------- 颜色 ----------
\definecolor{ExBlue}{HTML}{4F81BD}
\definecolor{SolGreen}{HTML}{77933C}
\definecolor{DefRed}{HTML}{C5504B}
\definecolor{ThmOrange}{HTML}{E97132}
\definecolor{RemGray}{HTML}{7F7F7F}
\definecolor{CorPurple}{HTML}{7030A0}
\definecolor{ForGray}{HTML}{595959}

% ---------- 通用“变色”模板 ----------
\tcbset{
    mybox/.style n args={1}{
        enhanced, breakable,
        arc=6pt,
        boxrule=0.6pt,
        left=8pt, right=8pt, top=6pt, bottom=6pt,
        drop shadow={black!25},
        fonttitle=\bfseries,
        coltitle=white,
        colbacktitle=#1!85,
        colback=#1!10,
        colframe=#1,
    }
}

% ---------- 各环境 ----------
% 例题
\newtcolorbox{example}[1][]{mybox={ExBlue}, title={\ifstrempty{#1}{Example}{#1}}}
% 解答
\newtcolorbox{solution}[1][]{mybox={SolGreen}, title={\ifstrempty{#1}{Solution}{#1}}}
% 定义
\newtcolorbox{definition}[1][]{mybox={DefRed}, title={\ifstrempty{#1}{Definition}{#1}}}
% 定理
\newtcolorbox{theorem}[1][]{mybox={ThmOrange}, title={\ifstrempty{#1}{Theorem}{#1}}}
% 标注
\newtcolorbox{remark}[1][]{mybox={RemGray}, title={\ifstrempty{#1}{Remark}{#1}}}
% 推论
\newtcolorbox{corollary}[1][]{mybox={CorPurple}, title={\ifstrempty{#1}{Corollary}{#1}}}
% 公式
\newtcolorbox{formula}[1][]{mybox={ForGray}, title={\ifstrempty{#1}{Formula}{#1}}}


\settasks{
    label-format = \bfseries,
    label        = \Alph*.,
    label-width  = 1.2em,
    label-offset = 0.3em,
    item-indent  = 1.9em,
    column-sep   = 0.5em
}

\newenvironment{choices}[1][4]   % 默认 4 栏
    {\begin{tasks}(#1)}
    {\end{tasks}}

% 自定义命令的文件

\def\d{\mathrm{d}}
\def\R{\mathbb{R}}
\def\P{\partial} 
\newcommand{\bs}[1]{\begin{solution}#1\end{solution}}
\newcommand{\bt}[1][1]{% 默认参数为1
    \ensuremath{% 确保数学模式
        \foreach \n in {1,...,#1} {\blacktriangle}% 循环输出 #1 个黑色三角形
    }%
}

\newcommand{\bl}[1][1]{% 默认参数为1
    \ensuremath{% 确保数学模式
        \foreach \n in {1,...,#1} {\blacklozenge}% 循环输出 #1 个黑色三角形
    }%
}
\newif\ifshowanswers
%\showanswerstrue % 注释掉这行就不显示答案

% 定义答案环境
\newcommand{\answer}[1]{%
    \ifshowanswers
        #1%
    \fi
}




% 修改参数改变封面样式,0 默认原始封面、内置其他1、2、3种封面样式
\def\myIndex{3}


\ifnum\myIndex>0
    \input{\path/cover_package_\myIndex} 
\fi

\def\myTitle{冲刺150笔记}
\def\myAuthor{Weary Bird}
\def\myDateCover{\today}
\def\myDateForeword{\today}
\def\myForeword{行香子}
\def\myForewordText{
树绕村庄,水满陂塘;倚东风、豪兴徜徉。小园几许,收尽春光。有桃花红,李花白,菜花黄。 \\
远远苔墙,隐隐茅堂;飏青旗、流水桥旁。偶然乘兴,步过东冈。正莺儿啼,燕儿舞,蝶儿忙。 \\
}
\def\mySubheading{知错能改善莫大焉}


\begin{document}
% \input{../config/cover}
\else
\fi

\chapter{特征值与特征向量}

\section{特征值与特征向量的计算}

\begin{enumerate}[label=\arabic*.]
    \item 设
    \begin{align*}
    A = \begin{pmatrix}
    -1 & -1 & -1 \\
    -1 & -1 & -1 \\
    -1 & -1 & -1
    \end{pmatrix}
    \end{align*}
    求 $A$ 的特征值与特征向量。
    
    \begin{solution}
    【详解】
    \end{solution}
    
    \item (2003, 数一) 设
    \begin{align*}
    A = \begin{pmatrix}
    5 & -2 & 3 \\
    3 & 2 & 0 \\
    2 & 0 & 1
    \end{pmatrix}, \quad
    B = P^{-1} A^* P
    \end{align*}
    求 $B + 2E$ 的特征值与特征向量。
    
    \begin{solution}
    【详解】
    \end{solution}
    
    \item 设
    \begin{align*}
    A = \begin{pmatrix}
    1 & 2 & 2 \\
    -1 & 4 & -2 \\
    1 & -2 & a
    \end{pmatrix}
    \end{align*}
    的特征方程有一个二重根,求 $A$ 的特征值与特征向量。
    
    \begin{solution}
    【详解】
    \end{solution}
    
    \item 设 3 阶非零矩阵 $A$ 满足 $A^2 = O$,则 $A$ 的线性无关的特征向量的个数是
    \begin{enumerate}
        \item (A) 0
        \item (B) 1
        \item (C) 2
        \item (D) 3
    \end{enumerate}
    
    \begin{solution}
    【详解】
    \end{solution}
    
    \item 设 $A = \alpha \beta^T + \beta \alpha^T$,其中 $\alpha, \beta$ 为 3 维单位列向量,且 $\alpha^T \beta = \frac{1}{3}$,证明:
    \begin{enumerate}
        \item (I) 0 为 $A$ 的特征值;
        \item (II) $\alpha + \beta, \alpha - \beta$ 为 $A$ 的特征向量;
        \item (III) $A$ 可相似对角化。
    \end{enumerate}
    
    \begin{solution}
    【详解】
    \end{solution}
\end{enumerate}

\section{相似的判定与计算}

\begin{enumerate}[label=\arabic*.,start=6]
    \item (2019, 数一、二、三) 设
    \begin{align*}
    A = \begin{pmatrix}
    2 & 2 & 1 \\
    2 & 0 & 0 \\
    -1 & -1 & -2
    \end{pmatrix}, \quad
    B = \begin{pmatrix}
    -1 & 0 & 0 \\
    0 & -2 & 0 \\
    0 & 0 & -3
    \end{pmatrix}
    \end{align*}
    (I) 求 $x, y$ 的值;
    (II) 求可逆矩阵 $P$,使得 $P^{-1}AP = B$。
    
    \begin{solution}
    【详解】
    \end{solution}
    
    \item 设 $n$ 阶矩阵 $A$ 与 $B$ 相似,满足 $A^2 = 2E$,则 $|AB + A - B - E| = $ \underline{\hspace{3cm}}。
    
    \begin{solution}
    【详解】
    \end{solution}
\end{enumerate}

\section{相似对角化的判定与计算}

\begin{enumerate}[label=\arabic*.,start=8]
    \item (2005, 数一、二) 设 3 阶矩阵 $A$ 的特征值为 1, 3, -2,对应的特征向量分别为 $\alpha_1, \alpha_2, \alpha_3$。若
    \begin{align*}
    P = (\alpha_1, 2\alpha_2, -\alpha_3)
    \end{align*}
    则 $P^{-1}AP = $ \underline{\hspace{3cm}}。
    
    \begin{solution}
    【详解】
    \end{solution}
    
    \item 设 $n$ 阶方阵 $A$ 满足 $A^2 - 3A + 2E = O$,证明 $A$ 可相似对角化。
    
    \begin{solution}
    【详解】
    \end{solution}
    
    \item (2020, 数一、二、三) 设 $A$ 为 2 阶矩阵,$P = (\alpha, A\alpha)$,其中 $\alpha$ 为非零向量且不是 $A$ 的特征向量。
    \begin{enumerate}
        \item (I) 证明 $P$ 为可逆矩阵;
        \item (II) 若 $A^2\alpha + 6A\alpha - 10\alpha = 0$,求 $P^{-1}AP$,并判断 $A$ 是否相似于对角矩阵。
    \end{enumerate}
    
    \begin{solution}
    【详解】
    \end{solution}
\end{enumerate}

\section{实对称矩阵的计算}

\begin{enumerate}[label=\arabic*.,start=11]
    \item (2010, 数二、三) 设
    \begin{align*}
    A = \begin{pmatrix}
    0 & 1 & 4 & 1 \\
    1 & 3 & a & 1 \\
    4 & a & 0 & 1 \\
    1 & 1 & 1 & 0
    \end{pmatrix}
    \end{align*}
    正交矩阵 $Q$ 使得 $Q^T A Q$ 为对角矩阵。若 $Q$ 的第 1 列为 $\frac{1}{\sqrt{6}}(1,2,1,0)^T$,求 $a, Q$。
    
    \begin{solution}
    【详解】
    \end{solution}
    
    \item 设 3 阶实对称矩阵 $A$ 满足 $A^2 + A = O$,$A$ 的各行元素之和均为零,且 $r(A) = 2$。
    \begin{enumerate}
        \item (I) 求 $A$ 的特征值与特征向量;
        \item (II) 求矩阵 $A$。
    \end{enumerate}
    
    \begin{solution}
    【详解】
    \end{solution}
\end{enumerate}

\ifx\allfiles\undefined
\end{document}
\fi