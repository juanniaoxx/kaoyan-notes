\ifx\allfiles\undefined
\documentclass[12pt, a4paper, oneside, UTF8]{ctexbook}
\usepackage{multirow}
\def\path{../../config}
\usepackage{amsmath}
\usepackage{amsthm}
\usepackage{amssymb}
\usepackage{array}
\usepackage{xcolor}
\usepackage{graphicx}
\usepackage{mathrsfs}
\usepackage{enumitem}
\usepackage{geometry}
\usepackage[colorlinks, linkcolor=black]{hyperref}
\usepackage{stackengine}
\usepackage{yhmath}
\usepackage{extarrows}
\usepackage{tikz}
\usepackage{pgfplots}
\usepackage{asymptote}
\usepackage{float}
\usepackage{fontspec} % 使用字体

\setmainfont{Times New Roman}
\setCJKmainfont{LXGWWenKai-Light}[
    SlantedFont=*
]

\everymath{\displaystyle}

\usepgfplotslibrary{polar}
\usepackage{subcaption}
\usetikzlibrary{decorations.pathreplacing, positioning}

\usepgfplotslibrary{fillbetween}
\pgfplotsset{compat=1.18}
% \usepackage{unicode-math}
\usepackage{esint}
\usepackage[most]{tcolorbox}

\usepackage{fancyhdr}
\usepackage[dvipsnames, svgnames]{xcolor}
\usepackage{listings}

\definecolor{mygreen}{rgb}{0,0.6,0}
\definecolor{mygray}{rgb}{0.5,0.5,0.5}
\definecolor{mymauve}{rgb}{0.58,0,0.82}
\definecolor{NavyBlue}{RGB}{0,0,128}
\definecolor{Rhodamine}{RGB}{255,0,255}
\definecolor{PineGreen}{RGB}{0,128,0}

\graphicspath{ {figures/},{../figures/}, {config/}, {../config/} }

\linespread{1.6}

\geometry{
    top=25.4mm, 
    bottom=25.4mm, 
    left=20mm, 
    right=20mm, 
    headheight=2.17cm, 
    headsep=4mm, 
    footskip=12mm
}

\setenumerate[1]{itemsep=5pt,partopsep=0pt,parsep=\parskip,topsep=5pt}
\setitemize[1]{itemsep=5pt,partopsep=0pt,parsep=\parskip,topsep=5pt}
\setdescription{itemsep=5pt,partopsep=0pt,parsep=\parskip,topsep=5pt}

\lstset{
    language=Mathematica,
    basicstyle=\tt,
    breaklines=true,
    keywordstyle=\bfseries\color{NavyBlue}, 
    emphstyle=\bfseries\color{Rhodamine},
    commentstyle=\itshape\color{black!50!white}, 
    stringstyle=\bfseries\color{PineGreen!90!black},
    columns=flexible,
    numbers=left,
    numberstyle=\footnotesize,
    frame=tb,
    breakatwhitespace=false,
} 

\lstset{
    language=TeX, % 设置语言为 TeX
    basicstyle=\ttfamily, % 使用等宽字体
    breaklines=true, % 自动换行
    keywordstyle=\bfseries\color{NavyBlue}, % 关键字样式
    emphstyle=\bfseries\color{Rhodamine}, % 强调样式
    commentstyle=\itshape\color{black!50!white}, % 注释样式
    stringstyle=\bfseries\color{PineGreen!90!black}, % 字符串样式
    columns=flexible, % 列的灵活性
    numbers=left, % 行号在左侧
    numberstyle=\footnotesize, % 行号字体大小
    frame=tb, % 顶部和底部边框
    breakatwhitespace=false % 不在空白处断行
}

% \begin{lstlisting}[language=TeX] ... \end{lstlisting}

% 定理环境设置
\usepackage[strict]{changepage} 
\usepackage{framed}

\definecolor{greenshade}{rgb}{0.90,1,0.92}
\definecolor{redshade}{rgb}{1.00,0.88,0.88}
\definecolor{brownshade}{rgb}{0.99,0.95,0.9}
\definecolor{lilacshade}{rgb}{0.95,0.93,0.98}
\definecolor{orangeshade}{rgb}{1.00,0.88,0.82}
\definecolor{lightblueshade}{rgb}{0.8,0.92,1}
\definecolor{purple}{rgb}{0.81,0.85,1}

\theoremstyle{definition}
\newtheorem{myDefn}{\indent Definition}[section]
\newtheorem{myLemma}{\indent Lemma}[section]
\newtheorem{myThm}[myLemma]{\indent Theorem}
\newtheorem{myCorollary}[myLemma]{\indent Corollary}
\newtheorem{myCriterion}[myLemma]{\indent Criterion}
\newtheorem*{myRemark}{\indent Remark}
\newtheorem{myProposition}{\indent Proposition}[section]

\newenvironment{formal}[2][]{%
	\def\FrameCommand{%
		\hspace{1pt}%
		{\color{#1}\vrule width 2pt}%
		{\color{#2}\vrule width 4pt}%
		\colorbox{#2}%
	}%
	\MakeFramed{\advance\hsize-\width\FrameRestore}%
	\noindent\hspace{-4.55pt}%
	\begin{adjustwidth}{}{7pt}\vspace{2pt}\vspace{2pt}}{%
		\vspace{2pt}\end{adjustwidth}\endMakeFramed%
}

\newenvironment{definition}{\vspace{-\baselineskip * 2 / 3}%
	\begin{formal}[Green]{greenshade}\vspace{-\baselineskip * 4 / 5}\begin{myDefn}}
	{\end{myDefn}\end{formal}\vspace{-\baselineskip * 2 / 3}}

\newenvironment{theorem}{\vspace{-\baselineskip * 2 / 3}%
	\begin{formal}[LightSkyBlue]{lightblueshade}\vspace{-\baselineskip * 4 / 5}\begin{myThm}}%
	{\end{myThm}\end{formal}\vspace{-\baselineskip * 2 / 3}}

\newenvironment{lemma}{\vspace{-\baselineskip * 2 / 3}%
	\begin{formal}[Plum]{lilacshade}\vspace{-\baselineskip * 4 / 5}\begin{myLemma}}%
	{\end{myLemma}\end{formal}\vspace{-\baselineskip * 2 / 3}}

\newenvironment{corollary}{\vspace{-\baselineskip * 2 / 3}%
	\begin{formal}[BurlyWood]{brownshade}\vspace{-\baselineskip * 4 / 5}\begin{myCorollary}}%
	{\end{myCorollary}\end{formal}\vspace{-\baselineskip * 2 / 3}}

\newenvironment{criterion}{\vspace{-\baselineskip * 2 / 3}%
	\begin{formal}[DarkOrange]{orangeshade}\vspace{-\baselineskip * 4 / 5}\begin{myCriterion}}%
	{\end{myCriterion}\end{formal}\vspace{-\baselineskip * 2 / 3}}
	

\newenvironment{remark}{\vspace{-\baselineskip * 2 / 3}%
	\begin{formal}[LightCoral]{redshade}\vspace{-\baselineskip * 4 / 5}\begin{myRemark}}%
	{\end{myRemark}\end{formal}\vspace{-\baselineskip * 2 / 3}}

\newenvironment{proposition}{\vspace{-\baselineskip * 2 / 3}%
	\begin{formal}[RoyalPurple]{purple}\vspace{-\baselineskip * 4 / 5}\begin{myProposition}}%
	{\end{myProposition}\end{formal}\vspace{-\baselineskip * 2 / 3}}


\newtheorem{example}{\indent \color{SeaGreen}{Example}}[section]
\renewcommand{\proofname}{\indent\textbf{\textcolor{TealBlue}{Proof}}}
\NewEnviron{solution}{%
	\begin{proof}[\indent\textbf{\textcolor{TealBlue}{Solution}}]%
		\color{blue}% 设置内容为蓝色
		\BODY% 插入环境内容
		\color{black}% 恢复默认颜色(可选,避免影响后续文字)
	\end{proof}%
}

% 自定义命令的文件

\def\d{\mathrm{d}}
\def\R{\mathbb{R}}
%\newcommand{\bs}[1]{\boldsymbol{#1}}
%\newcommand{\ora}[1]{\overrightarrow{#1}}
\newcommand{\myspace}[1]{\par\vspace{#1\baselineskip}}
\newcommand{\xrowht}[2][0]{\addstackgap[.5\dimexpr#2\relax]{\vphantom{#1}}}
\newenvironment{mycases}[1][1]{\linespread{#1} \selectfont \begin{cases}}{\end{cases}}
\newenvironment{myvmatrix}[1][1]{\linespread{#1} \selectfont \begin{vmatrix}}{\end{vmatrix}}
\newcommand{\tabincell}[2]{\begin{tabular}{@{}#1@{}}#2\end{tabular}}
\newcommand{\pll}{\kern 0.56em/\kern -0.8em /\kern 0.56em}
\newcommand{\dive}[1][F]{\mathrm{div}\;\boldsymbol{#1}}
\newcommand{\rotn}[1][A]{\mathrm{rot}\;\boldsymbol{#1}}

\newif\ifshowanswers
\showanswerstrue % 注释掉这行就不显示答案

% 定义答案环境
\newcommand{\answer}[1]{%
    \ifshowanswers
        #1%
    \fi
}

% 修改参数改变封面样式,0 默认原始封面、内置其他1、2、3种封面样式
\def\myIndex{0}


\ifnum\myIndex>0
    \input{\path/cover_package_\myIndex} 
\fi

\def\myTitle{考研数学笔记}
\def\myAuthor{Weary Bird}
\def\myDateCover{\today}
\def\myDateForeword{\today}
\def\myForeword{相见欢·林花谢了春红}
\def\myForewordText{
    林花谢了春红,太匆匆。
    无奈朝来寒雨晚来风。
    胭脂泪,相留醉,几时重。
    自是人生长恨水长东。
}
\def\mySubheading{以姜晓千强化课讲义为底本}


\begin{document}
% \input{\path/cover_text_\myIndex.tex}

\newpage
\thispagestyle{empty}
\begin{center}
    \Huge\textbf{\myForeword}
\end{center}
\myForewordText
\begin{flushright}
    \begin{tabular}{c}
        \myDateForeword
    \end{tabular}
\end{flushright}

\newpage
\pagestyle{plain}
\setcounter{page}{1}
\pagenumbering{Roman}
\tableofcontents

\newpage
\pagenumbering{arabic}
% \setcounter{chapter}{-1}
\setcounter{page}{1}

\pagestyle{fancy}
\fancyfoot[C]{\thepage}
\renewcommand{\headrulewidth}{0.4pt}
\renewcommand{\footrulewidth}{0pt}








\else
\fi
\chapter{线性方程组}

\section{解的判定}

\begin{enumerate}[label=\arabic*.]
    \item (2001,数三) 设 $A$ 为 $n$ 阶矩阵, $\alpha$ 为 $n$ 维列向量, 且 $\begin{pmatrix} A & \alpha \\ \alpha^T & 0 \end{pmatrix} = r(A)$,则线性方程组
    \begin{enumerate}
        \item (A) $Ax = \alpha$ 有无穷多解
        \item (B) $Ax = \alpha$ 有唯一解
        \item (C) $\begin{pmatrix} A & \alpha \\ \alpha^T & 0 \end{pmatrix} \begin{pmatrix} x \\ y \end{pmatrix} = 0$ 只有零解
        \item (D) $\begin{pmatrix} A & \alpha \\ \alpha^T & 0 \end{pmatrix} \begin{pmatrix} x \\ y \end{pmatrix} = 0$ 有非零解
    \end{enumerate}
    
    \begin{solution}
    【详解】
    \end{solution}
    
    \item 设 $A$ 为 $m \times n$ 阶矩阵, 且 $r(A) = m < n$,则下列结论不正确的是
    \begin{enumerate}
        \item (A) 线性方程组 $A^T x = 0$ 只有零解
        \item (B) 线性方程组 $A^T A x = 0$ 有非零解
        \item (C) $\forall b$,线性方程组 $A^T x = b$ 有唯一解
        \item (D) $\forall b$,线性方程组 $A x = b$ 有无穷多解
    \end{enumerate}
    
    \begin{solution}
    【详解】
    \end{solution}
\end{enumerate}

\section{求齐次线性方程组的基础解系与通解}

\begin{enumerate}[label=\arabic*.,start=2]
    \item (2011, 数一,二) 设 $A = (\alpha_1, \alpha_2, \alpha_3, \alpha_4)$ 为 4 阶矩阵, $(1,0,1,0)^T$ 为线性方程组 $Ax = 0$ 的基础解系,则 $A^* x = 0$ 的基础解系可为
    \begin{enumerate}
        \item (A) $\alpha_1, \alpha_2$
        \item (B) $\alpha_1, \alpha_3$
        \item (C) $\alpha_1, \alpha_2, \alpha_3$
        \item (D) $\alpha_2, \alpha_3, \alpha_4$
    \end{enumerate}
    
    \begin{solution}
    【详解】
    \end{solution}
    
    \item (2005, 数一、二) 设 3 阶矩阵 $A$ 的第 1 行为 $(a, b, c)$, $a, b, c$ 不全为零, $B = \begin{pmatrix} 2 & 4 & 6 \\ 3 & 6 & k \end{pmatrix}$ 满足 $AB = O$,求线性方程组 $Ax = 0$ 的通解。
    
    \begin{solution}
    【详解】
    \end{solution}
    
    \item (2002, 数三) 设线性方程组
    \begin{align*}
    a x_1 + b x_2 + b x_3 + \cdots + b x_n &= 0 \\
    b x_1 + a x_2 + b x_3 + \cdots + b x_n &= 0 \\
    \vdots \\
    b x_1 + b x_2 + b x_3 + \cdots + a x_n &= 0
    \end{align*}
    其中 $a \neq 0, b \neq 0, n \geq 2$。当 $a, b$ 为何值时,方程组只有零解、有非零解,当方程组有非零解时,求其通解。
    
    \begin{solution}
    【详解】
    \end{solution}
\end{enumerate}

\section{求非齐次线性方程组的通解}

\begin{enumerate}[label=\arabic*.,start=5]
    \item 设 $A$ 为 4 阶矩阵, $k$ 为任意常数, $\eta_1, \eta_2, \eta_3$ 为非齐次线性方程组 $Ax = b$ 的三个解, 满足
    \begin{align*}
    \eta_1 + \eta_2 &= \begin{pmatrix} 1 \\ 2 \\ 3 \\ 4 \end{pmatrix}, \quad \eta_2 + 2\eta_3 = \begin{pmatrix} 2 \\ 3 \\ 4 \\ 5 \end{pmatrix}.
    \end{align*}
    
    \begin{solution}
    【详解】
    \end{solution}
    
    \item (2017, 数一、三、三) 设 3 阶矩阵 $A = (\alpha_1', \alpha_2', \alpha_3')$ 有三个不同的特征值, 其中 $\alpha_3 = \alpha_1 + 2\alpha_2$。
    \begin{enumerate}
        \item (I) 证明 $r(A) = 2$;
        \item (II) 若 $\beta = \alpha_1 + \alpha_2 + \alpha_3$,求线性方程组 $Ax = \beta$ 的通解。
    \end{enumerate}
    
    \begin{solution}
    【详解】
    \end{solution}
    
    \item (I) 求 $\lambda, a$ 的值;
    \item (II) 求方程组 $Ax = b$ 的通解。
    
    \begin{solution}
    【详解】
    \end{solution}
    
    \item (I) $\eta$ 为非齐次线性方程组 $Ax = b$ 的特解, 证明:
    \begin{enumerate}
        \item (II) $\eta, \eta + \xi_1, \eta + \xi_2, \cdots, \eta + \xi_{n-r}$ 线性无关;
        \item (III) $\eta, \eta + \xi_1, \eta + \xi_2, \cdots, \eta + \xi_{n-r}$ 为 $Ax = b$ 所有解的极大线性无关组。
    \end{enumerate}
    
    \begin{solution}
    【详解】
    \end{solution}
\end{enumerate}

\section{解矩阵方程}

\begin{enumerate}[label=\arabic*.,start=9]
    \item 矩阵方程解的判定
    \begin{align*}
    AX = B \text{ 无解 } \Leftrightarrow r(A) < r(A|B) \\
    AX = B \text{ 有唯一解 } \Leftrightarrow r(A) = r(A|B) = n \\
    AX = B \text{ 有无穷多解 } \Leftrightarrow r(A) = r(A|B) < n
    \end{align*}
    
    \item 矩阵方程的求法
    对 $(A|B)$ 作初等行变换,化为行最简形矩阵,得矩阵 $X$。
    
    \item (例 4.10) 设 
    \begin{align*}
    A = \begin{pmatrix}
    1 & -2 & 3 & -4 \\
    0 & 1 & -1 & 1 \\
    1 & 2 & 0 & -3
    \end{pmatrix}
    \end{align*}
    矩阵 $X$ 满足 $AX + E = A^{2022} + 2X$,求矩阵 $X$。
    
    \begin{solution}
    【详解】
    \end{solution}
    
    \item (例 4.11) (2014, 数一、二、三) 设 
    \begin{align*}
    A = \begin{pmatrix}
    1 & -2 & 3 & -4 \\
    0 & 1 & -1 & 1 \\
    1 & 2 & 0 & -3
    \end{pmatrix}
    \end{align*}
    \begin{enumerate}
        \item (I) 求线性方程组 $Ax = 0$ 的一个基础解系;
        \item (II) 求满足 $AB = E$ 的所有矩阵 $B$。
    \end{enumerate}
    
    \begin{solution}
    【详解】
    \end{solution}
\end{enumerate}

\section{公共解的判定与计算}

\begin{enumerate}[label=\arabic*.,start=12]
    \item (2007, 数三) 设线性方程组
    \begin{align*}
    (I) \begin{cases}
    x_1 + x_2 + x_3 = 0 \\
    x_1 + 2x_2 + a x_3 = 0 \\
    x_1 + 4x_2 + a^2 x_3 = 0
    \end{cases}
    \end{align*}
    与方程
    \begin{align*}
    (II) x_1 + 2x_2 + x_3 = a - 1
    \end{align*}
    有公共解,求 $a$ 的值及所有公共解。
    
    \begin{solution}
    【详解】
    \end{solution}
    
    \item 设齐次线性方程组
    \begin{align*}
    (I) \begin{cases}
    2x_1 + 3x_2 - x_3 = 0 \\
    x_1 + 2x_2 + x_3 - x_4 = 0
    \end{cases}
    \end{align*}
    齐次线性方程组 (II) 的一个基础解系为 $\alpha_1 = (2, -1, a+2, 1)^T$, $\alpha_2 = (-1, 2, 4, a+8)^T$ 
    \begin{enumerate}
        \item (1) 求方程组 (I) 的一个基础解系;
        \item (2) 当 $a$ 为何值时,方程组 (I) 与 (II) 有非零公共解,并求所有非零公共解。
    \end{enumerate}
    
    \begin{solution}
    【详解】
    \end{solution}
    
    \item (2005,数三) 设线性方程组
    \begin{align*}
    (I) \begin{cases}
    x_1 + 2x_2 + 3x_3 = 0 \\
    2x_1 + 3x_2 + 5x_3 = 0 \\
    x_1 + x_2 + a x_3 = 0
    \end{cases}
    \end{align*}
    与 (II) 
    \begin{align*}
    \begin{cases}
    x_1 + b x_2 + c x_3 = 0 \\
    2x_1 + b^2 x_2 + (c+1) x_3 = 0
    \end{cases}
    \end{align*}
    同解,求 $a, b, c$ 的值。
    
    \begin{solution}
    【详解】
    \end{solution}
\end{enumerate}


\ifx\allfiles\undefined
\end{document}
\fi