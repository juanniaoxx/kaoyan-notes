\ifx\allfiles\undefined
\documentclass[12pt, a4paper, oneside, UTF8]{ctexbook}
\usepackage{multirow}
\def\path{../../config}
\usepackage{amsmath}
\usepackage{amsthm}
\usepackage{amssymb}
\usepackage{array}
\usepackage{xcolor}
\usepackage{graphicx}
\usepackage{mathrsfs}
\usepackage{enumitem}
\usepackage{geometry}
\usepackage[colorlinks, linkcolor=black]{hyperref}
\usepackage{stackengine}
\usepackage{yhmath}
\usepackage{extarrows}
\usepackage{tikz}
\usepackage{pgfplots}
\usepackage{asymptote}
\usepackage{float}
\usepackage{fontspec} % 使用字体

\setmainfont{Times New Roman}
\setCJKmainfont{LXGWWenKai-Light}[
    SlantedFont=*
]

\everymath{\displaystyle}

\usepgfplotslibrary{polar}
\usepackage{subcaption}
\usetikzlibrary{decorations.pathreplacing, positioning}

\usepgfplotslibrary{fillbetween}
\pgfplotsset{compat=1.18}
% \usepackage{unicode-math}
\usepackage{esint}
\usepackage[most]{tcolorbox}

\usepackage{fancyhdr}
\usepackage[dvipsnames, svgnames]{xcolor}
\usepackage{listings}

\definecolor{mygreen}{rgb}{0,0.6,0}
\definecolor{mygray}{rgb}{0.5,0.5,0.5}
\definecolor{mymauve}{rgb}{0.58,0,0.82}
\definecolor{NavyBlue}{RGB}{0,0,128}
\definecolor{Rhodamine}{RGB}{255,0,255}
\definecolor{PineGreen}{RGB}{0,128,0}

\graphicspath{ {figures/},{../figures/}, {config/}, {../config/} }

\linespread{1.6}

\geometry{
    top=25.4mm, 
    bottom=25.4mm, 
    left=20mm, 
    right=20mm, 
    headheight=2.17cm, 
    headsep=4mm, 
    footskip=12mm
}

\setenumerate[1]{itemsep=5pt,partopsep=0pt,parsep=\parskip,topsep=5pt}
\setitemize[1]{itemsep=5pt,partopsep=0pt,parsep=\parskip,topsep=5pt}
\setdescription{itemsep=5pt,partopsep=0pt,parsep=\parskip,topsep=5pt}

\lstset{
    language=Mathematica,
    basicstyle=\tt,
    breaklines=true,
    keywordstyle=\bfseries\color{NavyBlue}, 
    emphstyle=\bfseries\color{Rhodamine},
    commentstyle=\itshape\color{black!50!white}, 
    stringstyle=\bfseries\color{PineGreen!90!black},
    columns=flexible,
    numbers=left,
    numberstyle=\footnotesize,
    frame=tb,
    breakatwhitespace=false,
} 

\lstset{
    language=TeX, % 设置语言为 TeX
    basicstyle=\ttfamily, % 使用等宽字体
    breaklines=true, % 自动换行
    keywordstyle=\bfseries\color{NavyBlue}, % 关键字样式
    emphstyle=\bfseries\color{Rhodamine}, % 强调样式
    commentstyle=\itshape\color{black!50!white}, % 注释样式
    stringstyle=\bfseries\color{PineGreen!90!black}, % 字符串样式
    columns=flexible, % 列的灵活性
    numbers=left, % 行号在左侧
    numberstyle=\footnotesize, % 行号字体大小
    frame=tb, % 顶部和底部边框
    breakatwhitespace=false % 不在空白处断行
}

% \begin{lstlisting}[language=TeX] ... \end{lstlisting}

% 定理环境设置
\usepackage[strict]{changepage} 
\usepackage{framed}

\definecolor{greenshade}{rgb}{0.90,1,0.92}
\definecolor{redshade}{rgb}{1.00,0.88,0.88}
\definecolor{brownshade}{rgb}{0.99,0.95,0.9}
\definecolor{lilacshade}{rgb}{0.95,0.93,0.98}
\definecolor{orangeshade}{rgb}{1.00,0.88,0.82}
\definecolor{lightblueshade}{rgb}{0.8,0.92,1}
\definecolor{purple}{rgb}{0.81,0.85,1}

\theoremstyle{definition}
\newtheorem{myDefn}{\indent Definition}[section]
\newtheorem{myLemma}{\indent Lemma}[section]
\newtheorem{myThm}[myLemma]{\indent Theorem}
\newtheorem{myCorollary}[myLemma]{\indent Corollary}
\newtheorem{myCriterion}[myLemma]{\indent Criterion}
\newtheorem*{myRemark}{\indent Remark}
\newtheorem{myProposition}{\indent Proposition}[section]

\newenvironment{formal}[2][]{%
	\def\FrameCommand{%
		\hspace{1pt}%
		{\color{#1}\vrule width 2pt}%
		{\color{#2}\vrule width 4pt}%
		\colorbox{#2}%
	}%
	\MakeFramed{\advance\hsize-\width\FrameRestore}%
	\noindent\hspace{-4.55pt}%
	\begin{adjustwidth}{}{7pt}\vspace{2pt}\vspace{2pt}}{%
		\vspace{2pt}\end{adjustwidth}\endMakeFramed%
}

\newenvironment{definition}{\vspace{-\baselineskip * 2 / 3}%
	\begin{formal}[Green]{greenshade}\vspace{-\baselineskip * 4 / 5}\begin{myDefn}}
	{\end{myDefn}\end{formal}\vspace{-\baselineskip * 2 / 3}}

\newenvironment{theorem}{\vspace{-\baselineskip * 2 / 3}%
	\begin{formal}[LightSkyBlue]{lightblueshade}\vspace{-\baselineskip * 4 / 5}\begin{myThm}}%
	{\end{myThm}\end{formal}\vspace{-\baselineskip * 2 / 3}}

\newenvironment{lemma}{\vspace{-\baselineskip * 2 / 3}%
	\begin{formal}[Plum]{lilacshade}\vspace{-\baselineskip * 4 / 5}\begin{myLemma}}%
	{\end{myLemma}\end{formal}\vspace{-\baselineskip * 2 / 3}}

\newenvironment{corollary}{\vspace{-\baselineskip * 2 / 3}%
	\begin{formal}[BurlyWood]{brownshade}\vspace{-\baselineskip * 4 / 5}\begin{myCorollary}}%
	{\end{myCorollary}\end{formal}\vspace{-\baselineskip * 2 / 3}}

\newenvironment{criterion}{\vspace{-\baselineskip * 2 / 3}%
	\begin{formal}[DarkOrange]{orangeshade}\vspace{-\baselineskip * 4 / 5}\begin{myCriterion}}%
	{\end{myCriterion}\end{formal}\vspace{-\baselineskip * 2 / 3}}
	

\newenvironment{remark}{\vspace{-\baselineskip * 2 / 3}%
	\begin{formal}[LightCoral]{redshade}\vspace{-\baselineskip * 4 / 5}\begin{myRemark}}%
	{\end{myRemark}\end{formal}\vspace{-\baselineskip * 2 / 3}}

\newenvironment{proposition}{\vspace{-\baselineskip * 2 / 3}%
	\begin{formal}[RoyalPurple]{purple}\vspace{-\baselineskip * 4 / 5}\begin{myProposition}}%
	{\end{myProposition}\end{formal}\vspace{-\baselineskip * 2 / 3}}


\newtheorem{example}{\indent \color{SeaGreen}{Example}}[section]
\renewcommand{\proofname}{\indent\textbf{\textcolor{TealBlue}{Proof}}}
\NewEnviron{solution}{%
	\begin{proof}[\indent\textbf{\textcolor{TealBlue}{Solution}}]%
		\color{blue}% 设置内容为蓝色
		\BODY% 插入环境内容
		\color{black}% 恢复默认颜色(可选,避免影响后续文字)
	\end{proof}%
}

% 自定义命令的文件

\def\d{\mathrm{d}}
\def\R{\mathbb{R}}
%\newcommand{\bs}[1]{\boldsymbol{#1}}
%\newcommand{\ora}[1]{\overrightarrow{#1}}
\newcommand{\myspace}[1]{\par\vspace{#1\baselineskip}}
\newcommand{\xrowht}[2][0]{\addstackgap[.5\dimexpr#2\relax]{\vphantom{#1}}}
\newenvironment{mycases}[1][1]{\linespread{#1} \selectfont \begin{cases}}{\end{cases}}
\newenvironment{myvmatrix}[1][1]{\linespread{#1} \selectfont \begin{vmatrix}}{\end{vmatrix}}
\newcommand{\tabincell}[2]{\begin{tabular}{@{}#1@{}}#2\end{tabular}}
\newcommand{\pll}{\kern 0.56em/\kern -0.8em /\kern 0.56em}
\newcommand{\dive}[1][F]{\mathrm{div}\;\boldsymbol{#1}}
\newcommand{\rotn}[1][A]{\mathrm{rot}\;\boldsymbol{#1}}

\newif\ifshowanswers
\showanswerstrue % 注释掉这行就不显示答案

% 定义答案环境
\newcommand{\answer}[1]{%
    \ifshowanswers
        #1%
    \fi
}

% 修改参数改变封面样式,0 默认原始封面、内置其他1、2、3种封面样式
\def\myIndex{0}


\ifnum\myIndex>0
    \input{\path/cover_package_\myIndex} 
\fi

\def\myTitle{考研数学笔记}
\def\myAuthor{Weary Bird}
\def\myDateCover{\today}
\def\myDateForeword{\today}
\def\myForeword{相见欢·林花谢了春红}
\def\myForewordText{
    林花谢了春红,太匆匆。
    无奈朝来寒雨晚来风。
    胭脂泪,相留醉,几时重。
    自是人生长恨水长东。
}
\def\mySubheading{以姜晓千强化课讲义为底本}


\begin{document}
% \input{\path/cover_text_\myIndex.tex}

\newpage
\thispagestyle{empty}
\begin{center}
    \Huge\textbf{\myForeword}
\end{center}
\myForewordText
\begin{flushright}
    \begin{tabular}{c}
        \myDateForeword
    \end{tabular}
\end{flushright}

\newpage
\pagestyle{plain}
\setcounter{page}{1}
\pagenumbering{Roman}
\tableofcontents

\newpage
\pagenumbering{arabic}
% \setcounter{chapter}{-1}
\setcounter{page}{1}

\pagestyle{fancy}
\fancyfoot[C]{\thepage}
\renewcommand{\headrulewidth}{0.4pt}
\renewcommand{\footrulewidth}{0pt}








\else
\fi

\chapter{矩阵}
\section{求高次幂}
\begin{remark}
    基本方法 \\
    (1) 若$r(A)=1$,则$A^n=tr(A)^{n-1}A$, 关键点在于$r(A)=1\implies A=\alpha\beta^T$ \\
    (2) 若$A$可以分解为$E+B$,且$B$是类似于如下形式(非零元素仅在对角线的上方或下方)的矩阵则有如下结论.
    \[
    B=\left(
    \begin{array}{ccc}
        0& a& b \\
        0& 0& c \\
        0& 0& 0 \\
    \end{array}\right),
    \text{则}B^2=\left(\begin{array}{ccc}
        0& 0& ac \\
        0& 0& 0 \\
        0& 0& 0 \\
    \end{array}\right), B^3 = \mathbf{0}
    \]
    $A^n=C_{n}^{n}E+C_{n}^{1}B+C_{n}^{2}B^2$ \\
    (3) 分块矩阵
    \[
        A=\left(\begin{array}{cc}
        \mathbf{B} & \mathbf{0} \\
        \mathbf{0} & \mathbf{C}
    \end{array}\right), 
    A^n=\left(\begin{array}{cc}
        \mathbf{B}^n & \mathbf{0} \\
        \mathbf{0} & \mathbf{C}^n
    \end{array}\right)
    \] 
    (4) 相似对角化 \\
    $P^{-1}AP=\Lambda$则$A=P\Lambda P^{-1}$,
    \[
    A^n=P\Lambda^n P^{-1}=P diag(\lambda_1^n,\ldots,\lambda_n^n)P^{-1}
    \]
\end{remark}
\begin{enumerate}[label=\arabic*.]
    % 例题2.1
    \item 设 A=$\left(\begin{array}{rrr}
        2&-1&3\\
        a&1&b\\
        4&c&6
    \end{array}\right)$,
        $ B $ 为3阶矩阵,满足 $ BA = O $,且 $ r(B) > 1 $,则 $ A^n =\_\_\_\_$.
    \begin{solution}
    由$BA=0$知$r(A)+r(B)\leq n$,又$r(B)>1,r(A)\geq 1$所以$1\leq r(A)\leq 1,\implies r(A)=1$,
    \begin{align*}
        A=\left(\begin{array}{rrr}
        2&-1&3\\
        a&1&b\\
        4&c&6
    \end{array}\right)=\begin{pmatrix}
        2\\
        -1\\
        3
    \end{pmatrix}\begin{pmatrix}
        1,&-1,&2
    \end{pmatrix} \\
    A^n=tr(A)^{n-1}\alpha\beta^{T}=9^{n-1}\begin{pmatrix}
        2& -1& 3\\
        -2& 1& -3\\
        4& -2& 6
    \end{pmatrix}
    \end{align*}
    \end{solution}
    % 例题2.2
    \item 设$
    A = \begin{pmatrix}
    2 & 0 & 0 \\
    0 & 2 & 0 \\
    4 & 1&  2
    \end{pmatrix}$
    则 $ A^n = $ \underline{\hspace{3cm}}.
    
    \begin{solution}
    $A=2E+B$,$B=\begin{pmatrix}
        0& 0& 0\\
        -3& 0& 0\\
        4& 1& 0
    \end{pmatrix}$, $B^2=\begin{pmatrix}
        0& 0& 0\\
        0& 0& 0\\
        -3& 0& 0
    \end{pmatrix}$, $B^3=\mathbf{0}$,则
    \[
    A^n=2^nE+2^{n-1}nB+2^{n-3}n(n-1)B^2
    \]
    \end{solution},
    
    % 例题2.3
    \item 设
    $A = \begin{pmatrix}
    -1 & 2 & -1 \\
    -1 & 2 & -1 \\
    -3 & 6 & -3
    \end{pmatrix}$
    $ P $ 为3阶可逆矩阵,$ B = P^{-1}AP $,则 $ (B + E)^{100} =\_\_\_\_ $ 
    
    \begin{solution}
    $r(A)=1, A^2=tr(A)\cdot A=-2A$即$A^2+2A=\mathbf{0}, (A+E)^2=E$,由题 \\
    $(B+E)^{100}=(P^{-1}AP+E)^{100}=(P^{-1}AP+P^{-1}EP)^{100}=(P^{-1}(A+E)P)^{100}=E$
    \end{solution}
\end{enumerate}

\section{逆的判定与计算}

\begin{enumerate}[label=\arabic*.,start=4]
    % 例题2.4
    \item 设 $ n $ 阶矩阵 $ A $ 满足 $ A^2 = 2A $,则下列结论不正确的是: \\
    (A) $A$可逆 \qquad (B) $A-E$可逆\qquad (C)$A+E$可逆\qquad (D)$A-3E$可逆 
    \begin{solution}
    \newpage
    \end{solution}
    
    % 例题2.5
    \item 设 $ A, B $ 为 $ n $ 阶矩阵,$ a, b $ 为非零常数.证明:
    \begin{enumerate}
        \item 若 $ AB = aA + bB $,则 $ AB = BA $;
        \item 若 $ A^2 + aAB = E $,则 $ AB = BA $.
    \end{enumerate}
    
    \begin{solution}
    \newpage
    \end{solution}
    
    \begin{tcolorbox}[title=总结]
    (1)$A_{n\times n}B_{n\times n}=E\implies\begin{cases}
        \text{可逆}\\
        \text{求逆}, B=A^{-1}, A=B^{-1} \\
        \text{满足交换律},AB=BA
    \end{cases}$ \\
    (2)$AB$可交换的充分条件$\begin{cases}
        B=f(A),A^{-1},A^* \\
        AB=aA+bB(a,b\neq 0) \\
        A^2+aAB=E,(a\neq 0)
    \end{cases}$
    \end{tcolorbox}
    % 例题2.6
    \item 设 
    $A = \begin{pmatrix}
    a & 1 & 0 \\
    1 & a & -1 \\
    0 & 1 & a
    \end{pmatrix}$
    满足 $ A^3 = O $.
    \begin{enumerate}
        \item 求 $ a $ 的值;
        \item 若矩阵 $ X $ 满足 $ X - XA^2 - AX + AXA^2 = E $,求 $ X $.
    \end{enumerate}
    
    \begin{solution}
    \newpage
    \end{solution}
\end{enumerate}

\section{秩的计算与证明}
\begin{remark} 秩 \\
    秩的定义:$\exists r$阶子式非零且$\forall r+1$阶子式均为零 \\
    秩的性质
    \item[(1)] 设$A$为$m\times n$阶矩阵,则$r(A)<\min\{m,n\}$
    \item[(2)] $r(A+B)\leq r(A)+r(B)$
    \item[(3)] $r(AB)\leq\min\{r(A),r(B)\}$
    \item[(4)] $\max\{r(A),r(B)\}\leq r(A\mid B)\leq r(A) + r(B)$
    \item[(5)] $r(A)=r(kA)(k\neq 0)$
    \item[(6)] 设$A$为$m\times n$阶矩阵,$P$为$m$阶可逆矩阵,$Q$为$n$阶可逆矩阵,则
    $r(A)=r(PA)=r(AQ)=r(PAQ)$
    \item[(7)] 设$A$为$m\times n$阶矩阵,若$r(A)=n$则$r(AB)=r(B)$,若$r(A)=m$则$r(CA)=r(C)$ \\
    左乘列满秩,右乘行满秩,秩不变
    \item[(8)] $r(A)=r(A^T)=r(A^TA)=r(AA^T)$
    \item[(9)] 设$A$为$m\times n$阶矩阵, $B$为$n\times s$阶矩阵,$AB=0$,则$r(A)+r(B)\leq n$
\end{remark}
\begin{enumerate}[label=\arabic*.,start=7]
    \item (2018, 数一、二、三) 设 $ A, B $ 为 $ n $ 阶矩阵,$ (XY) $ 表示分块矩阵,则:
    \begin{enumerate}
        \item $ r(A\ AB) = r(A) $
        \item $ r(A\ BA) = r(A) $
        \item $ r(A\ B) = \max\{r(A), r(B)\} $
        \item $ r(A\ B) = r(A^T B^T) $
    \end{enumerate}
    
    \begin{solution}
    \newpage
    \end{solution}
    \item 设$A$为$n$阶矩阵,证明:
    \begin{enumerate}
        \item [(1)] 若 $ A^2 = A $,则 $ r(A) + r(A - E) = n $.
        \item [(II)] 若 $ A^2 = E $,则 $ r(A + E) + r(A - E) = n $.
    \end{enumerate}
    
    \begin{solution}
    \newpage
    \end{solution}
\end{enumerate}

\section{关于伴随矩阵}
\begin{remark}
    伴随矩阵的性质
    \item[(1)] $AA^*=A^*A=\left|A\right|\xrightarrow{|A|\neq 0}A^{-1}=\frac{1}{|A|}A^*,A^*=|A|A^{-1}$
    \item[(2)] $(kA)^*=k^{n-1}A^*$
    \item[(3)] $(AB)^*=B^*A^*$
    \item[(4)] $|A^*|=\left|A\right|^{n-1}$
    \item[(5)] $(A^T)^*=(A^*)^T$
    \item[(6)] $(A^{-1})^*=(A^*)^{-1}=\frac{A}{\left|A\right|}$
    \item[(7)] $(A^*)^*=\left|A\right|^{n-2}A$
    \item[(8)] 
    $r(A)=
    \begin{cases}
        n,& r(A)=n\\
        1,& r(A)=n-1\\
        0,& r(A)<n-1
    \end{cases}
    $       
\end{remark}
\begin{enumerate}[label=\arabic*.,start=9]
    \item 设 $ n $ 阶矩阵 $ A $ 的各列元素之和均为 2,且 $ |A| = 6 $,则 $ A^* $ 的各列元素之和均为:\\
    (A) 2\qquad (B) $\frac{1}{3}$\qquad (C) 3\qquad (D)6
    
    \begin{solution}
    \newpage
    \end{solution}
    
    \item 设 $ A = (a_{ij}) $ 为 $ n(n \geq 3) $ 阶非零矩阵,$ A_{ij} $ 为 $ a_{ij} $ 的代数余子式,证明:
    \begin{enumerate}
        \item $ a_{ij} = A_{ij}(i, j = 1, 2, \cdots, n) \Leftrightarrow A^* = A^T \Leftrightarrow AA^T = E $ 且 $ |A| = 1 $;
        \item $ a_{ij} = -A_{ij}(i, j = 1, 2, \cdots, n) \Leftrightarrow A^* = -A^T \Leftrightarrow AA^T = E $ 且 $ |A| = -1 $.
    \end{enumerate}
    
    \begin{solution}
    \newpage
    \end{solution}
\end{enumerate}

\section{初等变换与初等矩阵}
\begin{remark}
    初等变换与初等矩阵的性质
    \item [(1)] $\left|E(i,j)\right|=-1,\left|E(i(k))\right|=k,\left|E(ij(k))\right|=1$
    \item [(2)] $E(i, j)^T=E(i,j),E(i(k))^T=E(i(k)),E(ij(k))^T=E(ji(k))$
    \item [(3)] $E(i,j)^{-1}=E(i,j),E(i(k))^{-1}=E(i(\frac{1}{k})),E(ij(k)^{-1})=E(ij(-k))$
    \item [(4)] 初等行(列)变换相当于左(乘)对应的初等矩阵
    \item [(5)] 可逆矩阵可以写成有限个初等矩阵的乘积
\end{remark}
\begin{enumerate}[label=\arabic*.,start=11]
    \item (2005, 数一、二) 设 $ A $ 为 $ n(n \geq 2) $ 阶可逆矩阵,交换 $ A $ 的第 1 行与第 2 行得到矩阵 $ B $,则:
    \begin{enumerate}
        \item [(A)] 交换 $ A^* $ 的第 1 列与第 2 列,得 $ B^* $
        \item [(B)] 交换 $ A^* $ 的第1行与第2行,的$B^*$
        \item [(C)] 交换 $ A^* $ 的第 1 列与第 2 列,得 $ -B^* $
        \item [(D)] 交换 $ A $ 的第 1 行与第 2 行,得 $ -B^* $
    \end{enumerate}
    
    \begin{solution}
    \newpage
    \end{solution}
    
    \item 设 
    \begin{align*}
    A = \begin{pmatrix}
    1 & 2 & 3 \\
    0 & 1 & 2 \\
    0 & 0 & 1  
    \end{pmatrix}, \quad
    P = \begin{pmatrix}
    0 & 0 & 1 \\
    0 & 1 & 0 \\
    1 & 0 & 0 
    \end{pmatrix}, \quad
    Q = \begin{pmatrix}
    1 & 1 & 0 \\
    0 & 1 & 0 \\
    1 & 0 & 0
    \end{pmatrix}
    \end{align*}
    则 $ (P^{-1})^{2023} A (Q^T)^{2022} = $ \underline{\hspace{3cm}}.
    
    \begin{solution}
    \newpage
    \end{solution}
\end{enumerate}

\ifx\allfiles\undefined
\end{document}
\fi