\ifx\allfiles\undefined
\documentclass[12pt, a4paper, oneside, UTF8]{ctexbook}
\def\path{../../config}
\usepackage{amsmath}
\usepackage{amsthm}
\usepackage{amssymb}
\usepackage{array}
\usepackage{xcolor}
\usepackage{graphicx}
\usepackage{mathrsfs}
\usepackage{enumitem}
\usepackage{geometry}
\usepackage[colorlinks, linkcolor=black]{hyperref}
\usepackage{stackengine}
\usepackage{yhmath}
\usepackage{extarrows}
\usepackage{tikz}
\usepackage{pgfplots}
\usepackage{asymptote}
\usepackage{float}
\usepackage{fontspec} % 使用字体

\setmainfont{Times New Roman}
\setCJKmainfont{LXGWWenKai-Light}[
    SlantedFont=*
]

\everymath{\displaystyle}

\usepgfplotslibrary{polar}
\usepackage{subcaption}
\usetikzlibrary{decorations.pathreplacing, positioning}

\usepgfplotslibrary{fillbetween}
\pgfplotsset{compat=1.18}
% \usepackage{unicode-math}
\usepackage{esint}
\usepackage[most]{tcolorbox}

\usepackage{fancyhdr}
\usepackage[dvipsnames, svgnames]{xcolor}
\usepackage{listings}

\definecolor{mygreen}{rgb}{0,0.6,0}
\definecolor{mygray}{rgb}{0.5,0.5,0.5}
\definecolor{mymauve}{rgb}{0.58,0,0.82}
\definecolor{NavyBlue}{RGB}{0,0,128}
\definecolor{Rhodamine}{RGB}{255,0,255}
\definecolor{PineGreen}{RGB}{0,128,0}

\graphicspath{ {figures/},{../figures/}, {config/}, {../config/} }

\linespread{1.6}

\geometry{
    top=25.4mm, 
    bottom=25.4mm, 
    left=20mm, 
    right=20mm, 
    headheight=2.17cm, 
    headsep=4mm, 
    footskip=12mm
}

\setenumerate[1]{itemsep=5pt,partopsep=0pt,parsep=\parskip,topsep=5pt}
\setitemize[1]{itemsep=5pt,partopsep=0pt,parsep=\parskip,topsep=5pt}
\setdescription{itemsep=5pt,partopsep=0pt,parsep=\parskip,topsep=5pt}

\lstset{
    language=Mathematica,
    basicstyle=\tt,
    breaklines=true,
    keywordstyle=\bfseries\color{NavyBlue}, 
    emphstyle=\bfseries\color{Rhodamine},
    commentstyle=\itshape\color{black!50!white}, 
    stringstyle=\bfseries\color{PineGreen!90!black},
    columns=flexible,
    numbers=left,
    numberstyle=\footnotesize,
    frame=tb,
    breakatwhitespace=false,
} 

\lstset{
    language=TeX, % 设置语言为 TeX
    basicstyle=\ttfamily, % 使用等宽字体
    breaklines=true, % 自动换行
    keywordstyle=\bfseries\color{NavyBlue}, % 关键字样式
    emphstyle=\bfseries\color{Rhodamine}, % 强调样式
    commentstyle=\itshape\color{black!50!white}, % 注释样式
    stringstyle=\bfseries\color{PineGreen!90!black}, % 字符串样式
    columns=flexible, % 列的灵活性
    numbers=left, % 行号在左侧
    numberstyle=\footnotesize, % 行号字体大小
    frame=tb, % 顶部和底部边框
    breakatwhitespace=false % 不在空白处断行
}

% \begin{lstlisting}[language=TeX] ... \end{lstlisting}

% 定理环境设置
\usepackage[strict]{changepage} 
\usepackage{framed}

\definecolor{greenshade}{rgb}{0.90,1,0.92}
\definecolor{redshade}{rgb}{1.00,0.88,0.88}
\definecolor{brownshade}{rgb}{0.99,0.95,0.9}
\definecolor{lilacshade}{rgb}{0.95,0.93,0.98}
\definecolor{orangeshade}{rgb}{1.00,0.88,0.82}
\definecolor{lightblueshade}{rgb}{0.8,0.92,1}
\definecolor{purple}{rgb}{0.81,0.85,1}

\theoremstyle{definition}
\newtheorem{myDefn}{\indent Definition}[section]
\newtheorem{myLemma}{\indent Lemma}[section]
\newtheorem{myThm}[myLemma]{\indent Theorem}
\newtheorem{myCorollary}[myLemma]{\indent Corollary}
\newtheorem{myCriterion}[myLemma]{\indent Criterion}
\newtheorem*{myRemark}{\indent Remark}
\newtheorem{myProposition}{\indent Proposition}[section]

\newenvironment{formal}[2][]{%
	\def\FrameCommand{%
		\hspace{1pt}%
		{\color{#1}\vrule width 2pt}%
		{\color{#2}\vrule width 4pt}%
		\colorbox{#2}%
	}%
	\MakeFramed{\advance\hsize-\width\FrameRestore}%
	\noindent\hspace{-4.55pt}%
	\begin{adjustwidth}{}{7pt}\vspace{2pt}\vspace{2pt}}{%
		\vspace{2pt}\end{adjustwidth}\endMakeFramed%
}

\newenvironment{definition}{\vspace{-\baselineskip * 2 / 3}%
	\begin{formal}[Green]{greenshade}\vspace{-\baselineskip * 4 / 5}\begin{myDefn}}
	{\end{myDefn}\end{formal}\vspace{-\baselineskip * 2 / 3}}

\newenvironment{theorem}{\vspace{-\baselineskip * 2 / 3}%
	\begin{formal}[LightSkyBlue]{lightblueshade}\vspace{-\baselineskip * 4 / 5}\begin{myThm}}%
	{\end{myThm}\end{formal}\vspace{-\baselineskip * 2 / 3}}

\newenvironment{lemma}{\vspace{-\baselineskip * 2 / 3}%
	\begin{formal}[Plum]{lilacshade}\vspace{-\baselineskip * 4 / 5}\begin{myLemma}}%
	{\end{myLemma}\end{formal}\vspace{-\baselineskip * 2 / 3}}

\newenvironment{corollary}{\vspace{-\baselineskip * 2 / 3}%
	\begin{formal}[BurlyWood]{brownshade}\vspace{-\baselineskip * 4 / 5}\begin{myCorollary}}%
	{\end{myCorollary}\end{formal}\vspace{-\baselineskip * 2 / 3}}

\newenvironment{criterion}{\vspace{-\baselineskip * 2 / 3}%
	\begin{formal}[DarkOrange]{orangeshade}\vspace{-\baselineskip * 4 / 5}\begin{myCriterion}}%
	{\end{myCriterion}\end{formal}\vspace{-\baselineskip * 2 / 3}}
	

\newenvironment{remark}{\vspace{-\baselineskip * 2 / 3}%
	\begin{formal}[LightCoral]{redshade}\vspace{-\baselineskip * 4 / 5}\begin{myRemark}}%
	{\end{myRemark}\end{formal}\vspace{-\baselineskip * 2 / 3}}

\newenvironment{proposition}{\vspace{-\baselineskip * 2 / 3}%
	\begin{formal}[RoyalPurple]{purple}\vspace{-\baselineskip * 4 / 5}\begin{myProposition}}%
	{\end{myProposition}\end{formal}\vspace{-\baselineskip * 2 / 3}}


\newtheorem{example}{\indent \color{SeaGreen}{Example}}[section]
\renewcommand{\proofname}{\indent\textbf{\textcolor{TealBlue}{Proof}}}
\NewEnviron{solution}{%
	\begin{proof}[\indent\textbf{\textcolor{TealBlue}{Solution}}]%
		\color{blue}% 设置内容为蓝色
		\BODY% 插入环境内容
		\color{black}% 恢复默认颜色(可选,避免影响后续文字)
	\end{proof}%
}

% 自定义命令的文件

\def\d{\mathrm{d}}
\def\R{\mathbb{R}}
%\newcommand{\bs}[1]{\boldsymbol{#1}}
%\newcommand{\ora}[1]{\overrightarrow{#1}}
\newcommand{\myspace}[1]{\par\vspace{#1\baselineskip}}
\newcommand{\xrowht}[2][0]{\addstackgap[.5\dimexpr#2\relax]{\vphantom{#1}}}
\newenvironment{mycases}[1][1]{\linespread{#1} \selectfont \begin{cases}}{\end{cases}}
\newenvironment{myvmatrix}[1][1]{\linespread{#1} \selectfont \begin{vmatrix}}{\end{vmatrix}}
\newcommand{\tabincell}[2]{\begin{tabular}{@{}#1@{}}#2\end{tabular}}
\newcommand{\pll}{\kern 0.56em/\kern -0.8em /\kern 0.56em}
\newcommand{\dive}[1][F]{\mathrm{div}\;\boldsymbol{#1}}
\newcommand{\rotn}[1][A]{\mathrm{rot}\;\boldsymbol{#1}}

\newif\ifshowanswers
\showanswerstrue % 注释掉这行就不显示答案

% 定义答案环境
\newcommand{\answer}[1]{%
    \ifshowanswers
        #1%
    \fi
}

% 修改参数改变封面样式,0 默认原始封面、内置其他1、2、3种封面样式
\def\myIndex{0}


\ifnum\myIndex>0
    \input{\path/cover_package_\myIndex} 
\fi

\def\myTitle{考研数学笔记}
\def\myAuthor{Weary Bird}
\def\myDateCover{\today}
\def\myDateForeword{\today}
\def\myForeword{相见欢·林花谢了春红}
\def\myForewordText{
    林花谢了春红,太匆匆。
    无奈朝来寒雨晚来风。
    胭脂泪,相留醉,几时重。
    自是人生长恨水长东。
}
\def\mySubheading{以姜晓千强化课讲义为底本}


\begin{document}
\input{\path/cover_text_\myIndex.tex}

\newpage
\thispagestyle{empty}
\begin{center}
    \Huge\textbf{\myForeword}
\end{center}
\myForewordText
\begin{flushright}
    \begin{tabular}{c}
        \myDateForeword
    \end{tabular}
\end{flushright}

\newpage
\pagestyle{plain}
\setcounter{page}{1}
\pagenumbering{Roman}
\tableofcontents

\newpage
\pagenumbering{arabic}
% \setcounter{chapter}{-1}
\setcounter{page}{1}

\pagestyle{fancy}
\fancyfoot[C]{\thepage}
\renewcommand{\headrulewidth}{0.4pt}
\renewcommand{\footrulewidth}{0pt}








\else
\fi

\chapter{一元函数微分学}

\section{导数与微分的概念}

\begin{enumerate}[label=\arabic*.]
    \item (2000,数三)设函数$f(x)$在点$x=a$处可导,则函数$|f(x)|$在点$x=a$处不可导的充分条件是 \\
    A\quad $f(a)=0$且$f'(a)=0$ \qquad B\quad $f(a)=0$且$f'(a)\neq 0$ \\
    C\quad $f(a)>0$且$f'(a)>0$ \qquad D\quad $f(a)<0$且$f'(a)<0$
    
    \begin{solution}
    \newpage
    \end{solution}
    
    \item (2001,数一)设$f(0)=0$,则$f(x)$在$x=0$处可导的充要条件为 \\
        (A)\ $\lim_{h\to0}\frac{1}{h^2}f(1-\cos h)\ \text{存在}$ \qquad
        (B)\ $\lim_{h\to0}\frac{1}{h}f(1-e^h)\ \text{存在}$ \\
        (C)\ $\lim_{h\to0}\frac{1}{h^2}f(h-\sin h)\ \text{存在}$ \qquad
        (D)\ $\lim_{h\to0}\frac{1}{h}[f(2h)-f(h)]\ \text{存在}$
    
    \begin{solution}
    \newpage
    \end{solution}
    
    \item (2016,数一)已知函数$f(x)=\begin{cases}
        x, & x\leq0 \\
        \frac{1}{n}, & \frac{1}{n+1}<x\leq\frac{1}{n},n=1,2,\cdots
    \end{cases}$,则 \\
        (A)\ $x=0\ \text{是}\ f(x)\ \text{的第一类间断点}$ \qquad
        (B)\ $x=0\ \text{是}\ f(x)\ \text{的第二类间断点}$ \\
        (C)\ $f(x)\ \text{在}\ x=0\ \text{处连续但不可导}$ \qquad
        (D)\ $f(x)\ \text{在}\ x=0\ \text{处可导}$
    
    \begin{solution}
    \newpage
    \end{solution}
\end{enumerate}

\section{导数与微分的计算}

\begin{remark}[类型一 分段函数求导]
\end{remark}

\begin{enumerate}[label=\arabic*.,start=4]
    \item (1997,数一、数二)设函数$f(x)$连续,$\varphi(x)=\int_0^1 f(xt)dt$,且$\lim_{x\to0}\frac{f(x)}{x}=A$($A$为常数),求$\varphi'(x)$,并讨论$\varphi'(x)$在$x=0$处的连续性。
    
    \begin{solution}
    \newpage
    \end{solution}
\end{enumerate}

\begin{remark}[类型二 复合函数求导]
\end{remark}

\begin{enumerate}[label=\arabic*.,start=5]
    \item (2012,数三)设函数$f(x)=\begin{cases}
        \ln\sqrt{x}, & x\geq1 \\
        2x-1, & x<1
    \end{cases}$,$y=f(f(x))$,求$\left.\frac{dy}{dx}\right|_{x=e}$
    
    \begin{solution}
    \newpage
    \end{solution}
\end{enumerate}

\begin{remark}[类型三 隐函数求导]
\end{remark}

\begin{enumerate}[label=\arabic*.,start=6]
    \item (2007,数二)已知函数$f(u)$具有二阶导数,且$f'(0)=1$,函数$y=y(x)$由方程$y-xe^{y-1}=1$所确定。设$z=f(\ln y-\sin x)$,求$\left.\frac{dz}{dx}\right|_{x=0}$和$\left.\frac{d^2z}{dx^2}\right|_{x=0}$
    
    \begin{solution}
    \newpage
    \end{solution}
\end{enumerate}

\begin{remark}[类型四 反函数求导]
\end{remark}

\begin{enumerate}[label=\arabic*.,start=7]
    \item (2003,数一、数二)设函数$y=y(x)$在$(-\infty,+\infty)$内具有二阶导数,且$y'\neq0$,$x=x(y)$是$y=y(x)$的反函数。
    \begin{enumerate}[label=(\roman*)]
        \item 将$x=x(y)$所满足的微分方程$\frac{d^2x}{dy^2}+(y+\sin x)\left(\frac{dx}{dy}\right)^3=0$变换为$y=y(x)$满足的微分方程
        \item 求变换后的微分方程满足初始条件$y(0)=0$,$y'(0)=\frac{3}{2}$的解
    \end{enumerate}
    
    \begin{solution}
    \newpage
    \end{solution}
\end{enumerate}

\begin{remark}[类型五 参数方程求导]
\end{remark}

\begin{enumerate}[label=\arabic*.,start=8]
    \item (2008,数二)设函数$y=y(x)$由参数方程$\begin{cases}
        x=x(t) \\
        y=\int_0^{t^2}\ln(1+u)du
    \end{cases}$确定,其中$x(t)$是初值问题$\begin{cases}
        \frac{dx}{dt}-2te^{-x}=0 \\
        x|_{t=0}=0
    \end{cases}$的解,求$\frac{d^2y}{dx^2}$
    
    \begin{solution}
    \newpage
    \end{solution}
\end{enumerate}

\begin{remark}[类型六 高阶导数]
\end{remark}

\begin{enumerate}[label=\arabic*.,start=9]
    \item (2015,数二)函数$f(x)=x^2\cdot2^x$在$x=0$处的$n$阶导数$f^{(n)}(0)=$\underline{\quad}
    
    \begin{solution}
    \newpage
    \end{solution}
\end{enumerate}

\section{导数应用-切线与法线}

\begin{remark}[类型一 直角坐标表示的曲线]
\end{remark}

\begin{enumerate}[label=\arabic*.,start=10]
    \item  (2000,数二)已知$f(x)$是周期为5的连续函数,它在$x=0$的某个邻域内满足关系式$f(1+\sin x)-3f(1-\sin x)=8x+\alpha(x)$,其中$\alpha(x)$是当$x\to0$时比$x$高阶的无穷小,且$f(x)$在$x=1$处可导,求曲线$y=f(x)$在点$(6,f(6))$处的切线方程。
    
    \begin{solution}
    \newpage
    \end{solution}
\end{enumerate}

\begin{remark}[类型二 参数方程表示的曲线]
\end{remark}

\begin{enumerate}[label=\arabic*.,start=11]
    \item  曲线$\begin{cases}
        x=\int_0^{1-t}e^{-u^2}du \\
        y=t^2\ln(2-t^2)
    \end{cases}$在$(0,0)$处的切线方程为\underline{\quad}
    
    \begin{solution}
    【详解
    \end{solution}
\end{enumerate}

\begin{remark}[类型三 极坐标表示的曲线]
\end{remark}

\begin{enumerate}[label=\arabic*.,start=12]
    \item  (1997,数一)对数螺线$r=e^\theta$在点$(e^\frac{\pi}{2},\frac{\pi}{2})$处切线的直角坐标方程为\underline{\quad}
    
    \begin{solution}
    \newpage
    \end{solution}
\end{enumerate}

\section{导数应用-渐近线}

\begin{enumerate}[label=\arabic*.,start=13]
    \item  (2014,数一、数二、数三)下列曲线中有渐近线的是 \\
        (A)\ $y=x+\sin x$ \qquad (B)\ $y=x^2+\sin x$ \\
        (C)\ $y=x+\sin\frac{1}{x}$ \qquad (D)\ $y=x^2+\sin\frac{1}{x}$
    
    \begin{solution}
    \newpage
    \end{solution}
    
    \item  (2007,数一、数二、数三)曲线$y=\frac{1}{x}+\ln(1+e^x)$渐近线的条数为 \\
        (A)\ 0 \qquad (B)\ 1 \qquad (C)\ 2 \qquad (D)\ 3
    
    \begin{solution}
    \newpage
    \end{solution}
\end{enumerate}

\section{导数应用-曲率}

\begin{enumerate}[label=\arabic*.,start=15]
    \item  (2014,数二)曲线$\begin{cases}
        x=t^2+7 \\
        y=t^2+4t+1
    \end{cases}$对应于$t=1$的点处的曲率半径是 \\
        $(A)\ \frac{\sqrt{10}}{50} \quad (B)\ 
        \frac{\sqrt{10}}{100} \quad (C)\ 10\sqrt{10} \quad (D)\ 5\sqrt{10}$
    
    \begin{solution}
    \newpage
    \end{solution}
\end{enumerate}

\section{导数应用-极值与最值}

\begin{enumerate}[label=\arabic*.,start=17]
    \item  (2000,数二)设函数$f(x)$满足关系式$f''(x)+[f'(x)]^2=x$,且$f'(0)=0$,则 \\
    $(A)\ f(0)\ \text{是}\ f(x)\ \text{的极大值}$ \\
    $(B)\ f(0)\ \text{是}\ f(x)\ \text{的极小值}$ \\
    $(C)\ \text{点}(0,f(0))\ \text{是曲线}\ y=f(x)\ \text{的拐点}$ \\
    $(D)\ f(0)\ \text{不是}\ f(x)\ \text{的极值,点}(0,f(0))\ \text{也不是曲线}\ y=f(x)\ \text{的拐点}$
    
    \begin{solution}
    \newpage
    \end{solution}
    
    \item  (2010,数一、数二)求函数$f(x)=\int_1^{x^2}(x^2-t)e^{-t^2}dt$的单调区间与极值
    
    \begin{solution}
    \newpage
    \end{solution}
    
    \item  (2014,数二)已知函数$y=y(x)$满足微分方程$x^2+y^2y'=1-y'$,且$y(2)=0$,求$y(x)$的极大值与极小值
    
    \begin{solution}
    \newpage
    \end{solution}
\end{enumerate}

\section{导数应用-凹凸性与拐点}

\begin{enumerate}[label=\arabic*.,start=20]
    \item  (2011,数一)曲线$y=(x-1)(x-2)^2(x-3)^3(x-4)^4$的拐点是 \\
    $(A)\ (1,0) \qquad (B)\ (2,0) \qquad (C)\ (3,0) \qquad (D)\ (4,0)$
    
    \begin{solution}
    \newpage
    \end{solution}
\end{enumerate}

\section{导数应用-证明不等式}

\begin{enumerate}[label=\arabic*.,start=21]
    \item  (2017,数一、数三)设函数$f(x)$可导,且$f(x)f'(x)>0$,则 \\
        $(A)\ f(1)>f(-1) \quad (B)\ f(1)<f(-1) \qquad
        (C)\ |f(1)| >|f(-1)| \quad (D)\ |f(1)|<|f(-1)| $
    
    \begin{solution}
    \newpage
    \end{solution}
    
    \item  (2015,数二)已知函数$f(x)$在区间$[a,+\infty)$上具有二阶导数,$f(a)=0$,$f'(x)>0$,$f''(x)>0$。设$b>a$,曲线$y=f(x)$在点$(b,f(b))$处的切线与$x$轴的交点是$(x_0,0)$,证明$a<x_0<b$。
    
    \begin{solution}
    \newpage
    \end{solution}
\end{enumerate}

\section{导数应用-求方程的根}

\begin{enumerate}[label=\arabic*.,start=23]
    \item  (2003,数二)讨论曲线$y=4\ln x+k$与$y=4x+\ln^4 x$的交点个数。
    
    \begin{solution}
    \newpage
    \end{solution}
    
    \item  (2015,数二)已知函数$f(x)=\int_x^1\sqrt{1+t^2}dt+\int_1^{x^2}\sqrt{1+t}dt$,求$f(x)$零点的个数。
    
    \begin{solution}
    \newpage
    \end{solution}
\end{enumerate}

\section{微分中值定理证明题}

\begin{remark}[类型一 证明含有一个$\xi$的等式]
\end{remark}

\begin{enumerate}[label=\arabic*.,start=25]
    \item  (2013,数一、数二)设奇函数$f(x)$在$[-1,1]$上具有二阶导数,且$f(1)=1$。证明:
    \begin{enumerate}[label=(\roman*)]
        \item 存在$\xi\in(0,1)$,使得$f'(\xi)=1$;
        \item 存在$\eta\in(-1,1)$,使得$f''(\eta)+f'(\eta)=1$。
    \end{enumerate}
    
    \begin{solution}
    \newpage
    \end{solution}
    
    \item  设函数$f(x)$在$[0,1]$上连续,在$(0,1)$内可导,$f(1)=0$,证明:存在$\xi\in(0,1)$,使得$(2\xi+1)f(\xi)+\xi f'(\xi)=0$。
    
    \begin{solution}
    \newpage
    \end{solution}
\end{enumerate}

\begin{remark}[类型二 证明含有两个点的等式]
\end{remark}

\begin{enumerate}[label=\arabic*.,start=27]
    \item  设$f(x)$在$[0,1]$上连续,在$(0,1)$内可导,且$f(0)=0$,$f(1)=1$。证明:
    \begin{enumerate}[label=(\roman*)]
        \item 存在两个不同的点$\xi_1,\xi_2\in(0,1)$,使得$f'(\xi_1)+f'(\xi_2)=2$;
        \item 存在$\xi,\eta\in(0,1)$,使得$\eta f'(\xi)=f(\eta)f'(\eta)$。
    \end{enumerate}
    
    \begin{solution}
    \newpage
    \end{solution}
\end{enumerate}

\begin{remark}[类型三 证明含有高阶导数的等式或不等式]
\end{remark}

\begin{enumerate}[label=\arabic*.,start=28]
    \item  (2019,数二)已知函数$f(x)$在$[0,1]$上具有二阶导数,且$f(0)=0$,$f(1)=1$,$\int_0^1 f(x)dx=1$。证明:
    \begin{enumerate}[label=(\roman*)]
        \item 存在$\xi\in(0,1)$,使得$f'(\xi)=0$;
        \item 存在$\eta\in(0,1)$,使得$f''(\eta)<-2$。
    \end{enumerate}
    
    \begin{solution}
    \newpage
    \end{solution}
\end{enumerate}

\ifx\allfiles\undefined
\end{document}
\fi