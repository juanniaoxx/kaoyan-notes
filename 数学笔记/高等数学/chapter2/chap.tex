\ifx\allfiles\undefined
\documentclass[12pt, a4paper, oneside, UTF8]{ctexbook}
\def\path{../../config}
\usepackage{amsthm}
\usepackage{amssymb}
\usepackage{array}
\usepackage{xcolor}
\usepackage{graphicx}
\usepackage{mathrsfs}
\usepackage{enumitem}
\usepackage{geometry}
\usepackage[colorlinks, linkcolor=black]{hyperref}
\usepackage{stackengine}
\usepackage{yhmath}
\usepackage{extarrows}
\usepackage{tikz}
\usepackage{forest}
\usetikzlibrary{decorations.pathreplacing, positioning}
% \usepackage{unicode-math}
\usepackage{esint}
\usepackage{pifont}
\usepackage{tcolorbox}
\tcbuselibrary{skins, breakable}

\usepackage{multicol} 
\usepackage{fontspec} % 使用字体

\setmainfont{Times New Roman}
\setCJKmainfont{LXGWWenKai-Light}[
    SlantedFont=*
]

\usepackage{listings} % 用于插入代码

% 定义代码高亮风格
\lstset{
    basicstyle=\ttfamily\small,        % 基本字体样式(等宽小字体)
    keywordstyle=\color{blue},         % 关键字颜色
    commentstyle=\color{green},        % 注释颜色
    stringstyle=\color{red},           % 字符串颜色
    numbers=none,
    breaklines=true,                   % 自动换行
    frame=single,                      % 代码框边框
    rulecolor=\color{black},           % 边框颜色
    captionpos=b,                      % 标题位置(底部)
    showspaces=false,                  % 不显示空格标记
    showstringspaces=false,            % 不显示字符串中的空格标记
    language=C                         % 设置语言为 C
}

\usepackage{fontawesome5}

\usepackage{amsmath}
\usepackage{booktabs, array}
\usepackage{makecell}
\usepackage{fancyhdr}
\usepackage[dvipsnames, svgnames]{xcolor}
\usepackage{listings}
\usepackage{tasks}[2020/01/11]

\everymath{\displaystyle}

\definecolor{mygreen}{rgb}{0,0.6,0}
\definecolor{mygray}{rgb}{0.5,0.5,0.5}
\definecolor{mymauve}{rgb}{0.58,0,0.82}
\definecolor{NavyBlue}{RGB}{0,0,128}
\definecolor{Rhodamine}{RGB}{255,0,255}
\definecolor{PineGreen}{RGB}{0,128,0}

\graphicspath{ {figures/},{../figures/}, {config/}, {../config/} }

\linespread{1.6}

\geometry{
    top=25.4mm, 
    bottom=25.4mm, 
    left=20mm, 
    right=20mm, 
    headheight=2.17cm, 
    headsep=4mm, 
    footskip=12mm
}

\setenumerate[1]{itemsep=5pt,partopsep=0pt,parsep=\parskip,topsep=5pt}
\setitemize[1]{itemsep=5pt,partopsep=0pt,parsep=\parskip,topsep=5pt}
\setdescription{itemsep=5pt,partopsep=0pt,parsep=\parskip,topsep=5pt}



% \begin{lstlisting}[language=TeX] ... \end{lstlisting}

% 定理环境设置
% ---------- 颜色 ----------
\definecolor{ExBlue}{HTML}{4F81BD}
\definecolor{SolGreen}{HTML}{77933C}
\definecolor{DefRed}{HTML}{C5504B}
\definecolor{ThmOrange}{HTML}{E97132}
\definecolor{RemGray}{HTML}{7F7F7F}
\definecolor{CorPurple}{HTML}{7030A0}
\definecolor{ForGray}{HTML}{595959}

% ---------- 通用“变色”模板 ----------
\tcbset{
    mybox/.style n args={1}{
        enhanced, breakable,
        arc=6pt,
        boxrule=0.6pt,
        left=8pt, right=8pt, top=6pt, bottom=6pt,
        drop shadow={black!25},
        fonttitle=\bfseries,
        coltitle=white,
        colbacktitle=#1!85,
        colback=#1!10,
        colframe=#1,
    }
}

% ---------- 各环境 ----------
% 例题
\newtcolorbox{example}[1][]{mybox={ExBlue}, title={\ifstrempty{#1}{Example}{#1}}}
% 解答
\newtcolorbox{solution}[1][]{mybox={SolGreen}, title={\ifstrempty{#1}{Solution}{#1}}}
% 定义
\newtcolorbox{definition}[1][]{mybox={DefRed}, title={\ifstrempty{#1}{Definition}{#1}}}
% 定理
\newtcolorbox{theorem}[1][]{mybox={ThmOrange}, title={\ifstrempty{#1}{Theorem}{#1}}}
% 标注
\newtcolorbox{remark}[1][]{mybox={RemGray}, title={\ifstrempty{#1}{Remark}{#1}}}
% 推论
\newtcolorbox{corollary}[1][]{mybox={CorPurple}, title={\ifstrempty{#1}{Corollary}{#1}}}
% 公式
\newtcolorbox{formula}[1][]{mybox={ForGray}, title={\ifstrempty{#1}{Formula}{#1}}}


\settasks{
    label-format = \bfseries,
    label        = \Alph*.,
    label-width  = 1.2em,
    label-offset = 0.3em,
    item-indent  = 1.9em,
    column-sep   = 0.5em
}

\newenvironment{choices}[1][4]   % 默认 4 栏
    {\begin{tasks}(#1)}
    {\end{tasks}}

% 自定义命令的文件

\def\d{\mathrm{d}}
\def\R{\mathbb{R}}
\def\P{\partial} 
\newcommand{\bs}[1]{\begin{solution}#1\end{solution}}
\newcommand{\bt}[1][1]{% 默认参数为1
    \ensuremath{% 确保数学模式
        \foreach \n in {1,...,#1} {\blacktriangle}% 循环输出 #1 个黑色三角形
    }%
}

\newcommand{\bl}[1][1]{% 默认参数为1
    \ensuremath{% 确保数学模式
        \foreach \n in {1,...,#1} {\blacklozenge}% 循环输出 #1 个黑色三角形
    }%
}
\newif\ifshowanswers
%\showanswerstrue % 注释掉这行就不显示答案

% 定义答案环境
\newcommand{\answer}[1]{%
    \ifshowanswers
        #1%
    \fi
}




% 修改参数改变封面样式,0 默认原始封面、内置其他1、2、3种封面样式
\def\myIndex{3}


\ifnum\myIndex>0
    \input{\path/cover_package_\myIndex} 
\fi

\def\myTitle{冲刺150笔记}
\def\myAuthor{Weary Bird}
\def\myDateCover{\today}
\def\myDateForeword{\today}
\def\myForeword{行香子}
\def\myForewordText{
树绕村庄,水满陂塘;倚东风、豪兴徜徉。小园几许,收尽春光。有桃花红,李花白,菜花黄。 \\
远远苔墙,隐隐茅堂;飏青旗、流水桥旁。偶然乘兴,步过东冈。正莺儿啼,燕儿舞,蝶儿忙。 \\
}
\def\mySubheading{知错能改善莫大焉}


\begin{document}
\input{../../config/cover}
\else
\fi

\chapter{一元函数微分学}

\section{导数与微分的概念}

\begin{enumerate}[label=\arabic*.]
    \item (2000,数三)设函数$f(x)$在点$x=a$处可导,则函数$|f(x)|$在点$x=a$处不可导的充分条件是
    \begin{align*}
        (A)&\ f(a)=0\ \text{且}\ f'(a)=0 \\
        (B)&\ f(a)=0\ \text{且}\ f'(a)\neq0 \\
        (C)&\ f(a)>0\ \text{且}\ f'(a)>0 \\
        (D)&\ f(a)<0\ \text{且}\ f'(a)<0
    \end{align*}
    
    \begin{solution}
    【详解】
    \end{solution}
    
    \item (2001,数一)设$f(0)=0$,则$f(x)$在$x=0$处可导的充要条件为
    \begin{align*}
        (A)&\ \lim_{h\to0}\frac{1}{h^2}f(1-\cos h)\ \text{存在} \\
        (B)&\ \lim_{h\to0}\frac{1}{h}f(1-e^h)\ \text{存在} \\
        (C)&\ \lim_{h\to0}\frac{1}{h^2}f(h-\sin h)\ \text{存在} \\
        (D)&\ \lim_{h\to0}\frac{1}{h}[f(2h)-f(h)]\ \text{存在}
    \end{align*}
    
    \begin{solution}
    【详解】
    \end{solution}
    
    \item (2016,数一)已知函数$f(x)=\begin{cases}
        x, & x\leq0 \\
        \frac{1}{n}, & \frac{1}{n+1}<x\leq\frac{1}{n},n=1,2,\cdots
    \end{cases}$,则
    \begin{align*}
        (A)&\ x=0\ \text{是}\ f(x)\ \text{的第一类间断点} \\
        (B)&\ x=0\ \text{是}\ f(x)\ \text{的第二类间断点} \\
        (C)&\ f(x)\ \text{在}\ x=0\ \text{处连续但不可导} \\
        (D)&\ f(x)\ \text{在}\ x=0\ \text{处可导}
    \end{align*}
    
    \begin{solution}
    【详解】
    \end{solution}
\end{enumerate}

\section{导数与微分的计算}

\begin{remark}[类型一 分段函数求导]
\end{remark}

\begin{enumerate}[label=\arabic*.,start=4]
    \item (1997,数一、数二)设函数$f(x)$连续,$\varphi(x)=\int_0^1 f(xt)dt$,且$\lim_{x\to0}\frac{f(x)}{x}=A$($A$为常数),求$\varphi'(x)$,并讨论$\varphi'(x)$在$x=0$处的连续性。
    
    \begin{solution}
    【详解】
    \end{solution}
\end{enumerate}

\begin{remark}[类型二 复合函数求导]
\end{remark}

\begin{enumerate}[label=\arabic*.,start=5]
    \item (2012,数三)设函数$f(x)=\begin{cases}
        \ln\sqrt{x}, & x\geq1 \\
        2x-1, & x<1
    \end{cases}$,$y=f(f(x))$,求$\left.\frac{dy}{dx}\right|_{x=e}$
    
    \begin{solution}
    【详解】
    \end{solution}
\end{enumerate}

\begin{remark}[类型三 隐函数求导]
\end{remark}

\begin{enumerate}[label=\arabic*.,start=6]
    \item (2007,数二)已知函数$f(u)$具有二阶导数,且$f'(0)=1$,函数$y=y(x)$由方程$y-xe^{y-1}=1$所确定。设$z=f(\ln y-\sin x)$,求$\left.\frac{dz}{dx}\right|_{x=0}$和$\left.\frac{d^2z}{dx^2}\right|_{x=0}$
    
    \begin{solution}
    【详解】
    \end{solution}
\end{enumerate}

\begin{remark}[类型四 反函数求导]
\end{remark}

\begin{enumerate}[label=\arabic*.,start=7]
    \item (2003,数一、数二)设函数$y=y(x)$在$(-\infty,+\infty)$内具有二阶导数,且$y'\neq0$,$x=x(y)$是$y=y(x)$的反函数。
    \begin{enumerate}[label=(\roman*)]
        \item 将$x=x(y)$所满足的微分方程$\frac{d^2x}{dy^2}+(y+\sin x)\left(\frac{dx}{dy}\right)^3=0$变换为$y=y(x)$满足的微分方程
        \item 求变换后的微分方程满足初始条件$y(0)=0$,$y'(0)=\frac{3}{2}$的解
    \end{enumerate}
    
    \begin{solution}
    【详解】
    \end{solution}
\end{enumerate}

\begin{remark}[类型五 参数方程求导]
\end{remark}

\begin{enumerate}[label=\arabic*.,start=8]
    \item (2008,数二)设函数$y=y(x)$由参数方程$\begin{cases}
        x=x(t) \\
        y=\int_0^{t^2}\ln(1+u)du
    \end{cases}$确定,其中$x(t)$是初值问题$\begin{cases}
        \frac{dx}{dt}-2te^{-x}=0 \\
        x|_{t=0}=0
    \end{cases}$的解,求$\frac{d^2y}{dx^2}$
    
    \begin{solution}
    【详解】
    \end{solution}
\end{enumerate}

\begin{remark}[类型六 高阶导数]
\end{remark}

\begin{enumerate}[label=\arabic*.,start=9]
    \item (2015,数二)函数$f(x)=x^2\cdot2^x$在$x=0$处的$n$阶导数$f^{(n)}(0)=$\underline{\quad}
    
    \begin{solution}
    【详解】
    \end{solution}
\end{enumerate}

\section{导数应用-切线与法线}

\begin{remark}[类型一 直角坐标表示的曲线]
\end{remark}

\begin{enumerate}[label=\arabic*.,start=10]
    \item  (2000,数二)已知$f(x)$是周期为5的连续函数,它在$x=0$的某个邻域内满足关系式$f(1+\sin x)-3f(1-\sin x)=8x+\alpha(x)$,其中$\alpha(x)$是当$x\to0$时比$x$高阶的无穷小,且$f(x)$在$x=1$处可导,求曲线$y=f(x)$在点$(6,f(6))$处的切线方程。
    
    \begin{solution}
    【详解】
    \end{solution}
\end{enumerate}

\begin{remark}[类型二 参数方程表示的曲线]
\end{remark}

\begin{enumerate}[label=\arabic*.,start=11]
    \item  曲线$\begin{cases}
        x=\int_0^{1-t}e^{-u^2}du \\
        y=t^2\ln(2-t^2)
    \end{cases}$在$(0,0)$处的切线方程为\underline{\quad}
    
    \begin{solution}
    【详解】
    \end{solution}
\end{enumerate}

\begin{remark}[类型三 极坐标表示的曲线]
\end{remark}

\begin{enumerate}[label=\arabic*.,start=12]
    \item  (1997,数一)对数螺线$r=e^\theta$在点$(\frac{\pi}{2},\frac{\pi}{2})$处切线的直角坐标方程为\underline{\quad}
    
    \begin{solution}
    【详解】
    \end{solution}
\end{enumerate}

\section{导数应用-渐近线}

\begin{enumerate}[label=\arabic*.,start=13]
    \item  (2014,数一、数二、数三)下列曲线中有渐近线的是
    \begin{align*}
        (A)&\ y=x+\sin x \quad (B)\ y=x^2+\sin x \\
        (C)&\ y=x+\sin\frac{1}{x} \quad (D)\ y=x^2+\sin\frac{1}{x}
    \end{align*}
    
    \begin{solution}
    【详解】
    \end{solution}
    
    \item  (2007,数一、数二、数三)曲线$y=\frac{1}{x}+\ln(1+e^x)$渐近线的条数为
    \begin{align*}
        (A)\ 0 \quad (B)\ 1 \quad (C)\ 2 \quad (D)\ 3
    \end{align*}
    
    \begin{solution}
    【详解】
    \end{solution}
\end{enumerate}

\section{导数应用-曲率}

\begin{enumerate}[label=\arabic*.,start=15]
    \item  (2014,数二)曲线$\begin{cases}
        x=t^2+7 \\
        y=t^2+4t+1
    \end{cases}$对应于$t=1$的点处的曲率半径是
    \begin{align*}
        (A)\ \frac{\sqrt{10}}{50} \quad (B)\ \frac{\sqrt{10}}{100} \quad (C)\ 10\sqrt{10} \quad (D)\ 5\sqrt{10}
    \end{align*}
    
    \begin{solution}
    【详解】
    \end{solution}
\end{enumerate}

\section{导数应用-极值与最值}

\begin{enumerate}[label=\arabic*.,start=17]
    \item  (2000,数二)设函数$f(x)$满足关系式$f''(x)+[f'(x)]^2=x$,且$f'(0)=0$,则
    \begin{align*}
        (A)&\ f(0)\ \text{是}\ f(x)\ \text{的极大值} \\
        (B)&\ f(0)\ \text{是}\ f(x)\ \text{的极小值} \\
        (C)&\ \text{点}(0,f(0))\ \text{是曲线}\ y=f(x)\ \text{的拐点} \\
        (D)&\ f(0)\ \text{不是}\ f(x)\ \text{的极值,点}(0,f(0))\ \text{也不是曲线}\ y=f(x)\ \text{的拐点}
    \end{align*}
    
    \begin{solution}
    【详解】
    \end{solution}
    
    \item  (2010,数一、数二)求函数$f(x)=\int_1^{x^2}(x^2-t)e^{-t^2}dt$的单调区间与极值
    
    \begin{solution}
    【详解】
    \end{solution}
    
    \item  (2014,数二)已知函数$y=y(x)$满足微分方程$x^2+y^2y'=1-y'$,且$y(2)=0$,求$y(x)$的极大值与极小值
    
    \begin{solution}
    【详解】
    \end{solution}
\end{enumerate}

\section{导数应用-凹凸性与拐点}

\begin{enumerate}[label=\arabic*.,start=20]
    \item  (2011,数一)曲线$y=(x-1)(x-2)^2(x-3)^3(x-4)^4$的拐点是
    \begin{align*}
        (A)\ (1,0) \quad (B)\ (2,0) \quad (C)\ (3,0) \quad (D)\ (4,0)
    \end{align*}
    
    \begin{solution}
    【详解】
    \end{solution}
\end{enumerate}

\section{导数应用-证明不等式}

\begin{enumerate}[label=\arabic*.,start=21]
    \item  (2017,数一、数三)设函数$f(x)$可导,且$f(x)f'(x)>0$,则
    \begin{align*}
        (A)\ f(1)>f(-1) \quad (B)\ f(1)<f(-1) \\
        (C)\ |f(1)|>|f(-1)| \quad (D)\ |f(1)|<|f(-1)|
    \end{align*}
    
    \begin{solution}
    【详解】
    \end{solution}
    
    \item  (2015,数二)已知函数$f(x)$在区间$[a,+\infty)$上具有二阶导数,$f(a)=0$,$f'(x)>0$,$f''(x)>0$。设$b>a$,曲线$y=f(x)$在点$(b,f(b))$处的切线与$x$轴的交点是$(x_0,0)$,证明$a<x_0<b$。
    
    \begin{solution}
    【详解】
    \end{solution}
\end{enumerate}

\section{导数应用-求方程的根}

\begin{enumerate}[label=\arabic*.,start=23]
    \item  (2003,数二)讨论曲线$y=4\ln x+k$与$y=4x+\ln^4 x$的交点个数。
    
    \begin{solution}
    【详解】
    \end{solution}
    
    \item  (2015,数二)已知函数$f(x)=\int_x^1\sqrt{1+t^2}dt+\int_1^{x^2}\sqrt{1+t}dt$,求$f(x)$零点的个数。
    
    \begin{solution}
    【详解】
    \end{solution}
\end{enumerate}

\section{微分中值定理证明题}

\begin{remark}[类型一 证明含有一个点的等式]
\end{remark}

\begin{enumerate}[label=\arabic*.,start=25]
    \item  (2013,数一、数二)设奇函数$f(x)$在$[-1,1]$上具有二阶导数,且$f(1)=1$。证明:
    \begin{enumerate}[label=(\roman*)]
        \item 存在$\xi\in(0,1)$,使得$f'(\xi)=1$;
        \item 存在$\eta\in(-1,1)$,使得$f''(\eta)+f'(\eta)=1$。
    \end{enumerate}
    
    \begin{solution}
    【详解】
    \end{solution}
    
    \item  设函数$f(x)$在$[0,1]$上连续,在$(0,1)$内可导,$f(1)=0$,证明:存在$\xi\in(0,1)$,使得$(2\xi+1)f(\xi)+\xi f'(\xi)=0$。
    
    \begin{solution}
    【详解】
    \end{solution}
\end{enumerate}

\begin{remark}[类型二 证明含有两个点的等式]
\end{remark}

\begin{enumerate}[label=\arabic*.,start=27]
    \item  设$f(x)$在$[0,1]$上连续,在$(0,1)$内可导,且$f(0)=0$,$f(1)=1$。证明:
    \begin{enumerate}[label=(\roman*)]
        \item 存在两个不同的点$\xi_1,\xi_2\in(0,1)$,使得$f'(\xi_1)+f'(\xi_2)=2$;
        \item 存在$\xi,\eta\in(0,1)$,使得$\eta f'(\xi)=f(\eta)f'(\eta)$。
    \end{enumerate}
    
    \begin{solution}
    【详解】
    \end{solution}
\end{enumerate}

\begin{remark}[类型三 证明含有高阶导数的等式或不等式]
\end{remark}

\begin{enumerate}[label=\arabic*.,start=28]
    \item  (2019,数二)已知函数$f(x)$在$[0,1]$上具有二阶导数,且$f(0)=0$,$f(1)=1$,$\int_0^1 f(x)dx=1$。证明:
    \begin{enumerate}[label=(\roman*)]
        \item 存在$\xi\in(0,1)$,使得$f'(\xi)=0$;
        \item 存在$\eta\in(0,1)$,使得$f''(\eta)<-2$。
    \end{enumerate}
    
    \begin{solution}
    【详解】
    \end{solution}
\end{enumerate}

\ifx\allfiles\undefined
\end{document}
\fi