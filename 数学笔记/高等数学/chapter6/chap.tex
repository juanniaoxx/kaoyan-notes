\ifx\allfiles\undefined
\documentclass[12pt, a4paper, oneside, UTF8]{ctexbook}
\def\path{../../config}
\usepackage{amsmath}
\usepackage{amsthm}
\usepackage{amssymb}
\usepackage{array}
\usepackage{xcolor}
\usepackage{graphicx}
\usepackage{mathrsfs}
\usepackage{enumitem}
\usepackage{geometry}
\usepackage[colorlinks, linkcolor=black]{hyperref}
\usepackage{stackengine}
\usepackage{yhmath}
\usepackage{extarrows}
\usepackage{tikz}
\usepackage{pgfplots}
\usepackage{asymptote}
\usepackage{float}
\usepackage{fontspec} % 使用字体

\setmainfont{Times New Roman}
\setCJKmainfont{LXGWWenKai-Light}[
    SlantedFont=*
]

\everymath{\displaystyle}

\usepgfplotslibrary{polar}
\usepackage{subcaption}
\usetikzlibrary{decorations.pathreplacing, positioning}

\usepgfplotslibrary{fillbetween}
\pgfplotsset{compat=1.18}
% \usepackage{unicode-math}
\usepackage{esint}
\usepackage[most]{tcolorbox}

\usepackage{fancyhdr}
\usepackage[dvipsnames, svgnames]{xcolor}
\usepackage{listings}

\definecolor{mygreen}{rgb}{0,0.6,0}
\definecolor{mygray}{rgb}{0.5,0.5,0.5}
\definecolor{mymauve}{rgb}{0.58,0,0.82}
\definecolor{NavyBlue}{RGB}{0,0,128}
\definecolor{Rhodamine}{RGB}{255,0,255}
\definecolor{PineGreen}{RGB}{0,128,0}

\graphicspath{ {figures/},{../figures/}, {config/}, {../config/} }

\linespread{1.6}

\geometry{
    top=25.4mm, 
    bottom=25.4mm, 
    left=20mm, 
    right=20mm, 
    headheight=2.17cm, 
    headsep=4mm, 
    footskip=12mm
}

\setenumerate[1]{itemsep=5pt,partopsep=0pt,parsep=\parskip,topsep=5pt}
\setitemize[1]{itemsep=5pt,partopsep=0pt,parsep=\parskip,topsep=5pt}
\setdescription{itemsep=5pt,partopsep=0pt,parsep=\parskip,topsep=5pt}

\lstset{
    language=Mathematica,
    basicstyle=\tt,
    breaklines=true,
    keywordstyle=\bfseries\color{NavyBlue}, 
    emphstyle=\bfseries\color{Rhodamine},
    commentstyle=\itshape\color{black!50!white}, 
    stringstyle=\bfseries\color{PineGreen!90!black},
    columns=flexible,
    numbers=left,
    numberstyle=\footnotesize,
    frame=tb,
    breakatwhitespace=false,
} 

\lstset{
    language=TeX, % 设置语言为 TeX
    basicstyle=\ttfamily, % 使用等宽字体
    breaklines=true, % 自动换行
    keywordstyle=\bfseries\color{NavyBlue}, % 关键字样式
    emphstyle=\bfseries\color{Rhodamine}, % 强调样式
    commentstyle=\itshape\color{black!50!white}, % 注释样式
    stringstyle=\bfseries\color{PineGreen!90!black}, % 字符串样式
    columns=flexible, % 列的灵活性
    numbers=left, % 行号在左侧
    numberstyle=\footnotesize, % 行号字体大小
    frame=tb, % 顶部和底部边框
    breakatwhitespace=false % 不在空白处断行
}

% \begin{lstlisting}[language=TeX] ... \end{lstlisting}

% 定理环境设置
\usepackage[strict]{changepage} 
\usepackage{framed}

\definecolor{greenshade}{rgb}{0.90,1,0.92}
\definecolor{redshade}{rgb}{1.00,0.88,0.88}
\definecolor{brownshade}{rgb}{0.99,0.95,0.9}
\definecolor{lilacshade}{rgb}{0.95,0.93,0.98}
\definecolor{orangeshade}{rgb}{1.00,0.88,0.82}
\definecolor{lightblueshade}{rgb}{0.8,0.92,1}
\definecolor{purple}{rgb}{0.81,0.85,1}

\theoremstyle{definition}
\newtheorem{myDefn}{\indent Definition}[section]
\newtheorem{myLemma}{\indent Lemma}[section]
\newtheorem{myThm}[myLemma]{\indent Theorem}
\newtheorem{myCorollary}[myLemma]{\indent Corollary}
\newtheorem{myCriterion}[myLemma]{\indent Criterion}
\newtheorem*{myRemark}{\indent Remark}
\newtheorem{myProposition}{\indent Proposition}[section]

\newenvironment{formal}[2][]{%
	\def\FrameCommand{%
		\hspace{1pt}%
		{\color{#1}\vrule width 2pt}%
		{\color{#2}\vrule width 4pt}%
		\colorbox{#2}%
	}%
	\MakeFramed{\advance\hsize-\width\FrameRestore}%
	\noindent\hspace{-4.55pt}%
	\begin{adjustwidth}{}{7pt}\vspace{2pt}\vspace{2pt}}{%
		\vspace{2pt}\end{adjustwidth}\endMakeFramed%
}

\newenvironment{definition}{\vspace{-\baselineskip * 2 / 3}%
	\begin{formal}[Green]{greenshade}\vspace{-\baselineskip * 4 / 5}\begin{myDefn}}
	{\end{myDefn}\end{formal}\vspace{-\baselineskip * 2 / 3}}

\newenvironment{theorem}{\vspace{-\baselineskip * 2 / 3}%
	\begin{formal}[LightSkyBlue]{lightblueshade}\vspace{-\baselineskip * 4 / 5}\begin{myThm}}%
	{\end{myThm}\end{formal}\vspace{-\baselineskip * 2 / 3}}

\newenvironment{lemma}{\vspace{-\baselineskip * 2 / 3}%
	\begin{formal}[Plum]{lilacshade}\vspace{-\baselineskip * 4 / 5}\begin{myLemma}}%
	{\end{myLemma}\end{formal}\vspace{-\baselineskip * 2 / 3}}

\newenvironment{corollary}{\vspace{-\baselineskip * 2 / 3}%
	\begin{formal}[BurlyWood]{brownshade}\vspace{-\baselineskip * 4 / 5}\begin{myCorollary}}%
	{\end{myCorollary}\end{formal}\vspace{-\baselineskip * 2 / 3}}

\newenvironment{criterion}{\vspace{-\baselineskip * 2 / 3}%
	\begin{formal}[DarkOrange]{orangeshade}\vspace{-\baselineskip * 4 / 5}\begin{myCriterion}}%
	{\end{myCriterion}\end{formal}\vspace{-\baselineskip * 2 / 3}}
	

\newenvironment{remark}{\vspace{-\baselineskip * 2 / 3}%
	\begin{formal}[LightCoral]{redshade}\vspace{-\baselineskip * 4 / 5}\begin{myRemark}}%
	{\end{myRemark}\end{formal}\vspace{-\baselineskip * 2 / 3}}

\newenvironment{proposition}{\vspace{-\baselineskip * 2 / 3}%
	\begin{formal}[RoyalPurple]{purple}\vspace{-\baselineskip * 4 / 5}\begin{myProposition}}%
	{\end{myProposition}\end{formal}\vspace{-\baselineskip * 2 / 3}}


\newtheorem{example}{\indent \color{SeaGreen}{Example}}[section]
\renewcommand{\proofname}{\indent\textbf{\textcolor{TealBlue}{Proof}}}
\NewEnviron{solution}{%
	\begin{proof}[\indent\textbf{\textcolor{TealBlue}{Solution}}]%
		\color{blue}% 设置内容为蓝色
		\BODY% 插入环境内容
		\color{black}% 恢复默认颜色(可选,避免影响后续文字)
	\end{proof}%
}

% 自定义命令的文件

\def\d{\mathrm{d}}
\def\R{\mathbb{R}}
%\newcommand{\bs}[1]{\boldsymbol{#1}}
%\newcommand{\ora}[1]{\overrightarrow{#1}}
\newcommand{\myspace}[1]{\par\vspace{#1\baselineskip}}
\newcommand{\xrowht}[2][0]{\addstackgap[.5\dimexpr#2\relax]{\vphantom{#1}}}
\newenvironment{mycases}[1][1]{\linespread{#1} \selectfont \begin{cases}}{\end{cases}}
\newenvironment{myvmatrix}[1][1]{\linespread{#1} \selectfont \begin{vmatrix}}{\end{vmatrix}}
\newcommand{\tabincell}[2]{\begin{tabular}{@{}#1@{}}#2\end{tabular}}
\newcommand{\pll}{\kern 0.56em/\kern -0.8em /\kern 0.56em}
\newcommand{\dive}[1][F]{\mathrm{div}\;\boldsymbol{#1}}
\newcommand{\rotn}[1][A]{\mathrm{rot}\;\boldsymbol{#1}}

\newif\ifshowanswers
\showanswerstrue % 注释掉这行就不显示答案

% 定义答案环境
\newcommand{\answer}[1]{%
    \ifshowanswers
        #1%
    \fi
}

% 修改参数改变封面样式,0 默认原始封面、内置其他1、2、3种封面样式
\def\myIndex{0}


\ifnum\myIndex>0
    \input{\path/cover_package_\myIndex} 
\fi

\def\myTitle{考研数学笔记}
\def\myAuthor{Weary Bird}
\def\myDateCover{\today}
\def\myDateForeword{\today}
\def\myForeword{相见欢·林花谢了春红}
\def\myForewordText{
    林花谢了春红,太匆匆。
    无奈朝来寒雨晚来风。
    胭脂泪,相留醉,几时重。
    自是人生长恨水长东。
}
\def\mySubheading{以姜晓千强化课讲义为底本}


\begin{document}
\input{\path/cover_text_\myIndex.tex}

\newpage
\thispagestyle{empty}
\begin{center}
    \Huge\textbf{\myForeword}
\end{center}
\myForewordText
\begin{flushright}
    \begin{tabular}{c}
        \myDateForeword
    \end{tabular}
\end{flushright}

\newpage
\pagestyle{plain}
\setcounter{page}{1}
\pagenumbering{Roman}
\tableofcontents

\newpage
\pagenumbering{arabic}
% \setcounter{chapter}{-1}
\setcounter{page}{1}

\pagestyle{fancy}
\fancyfoot[C]{\thepage}
\renewcommand{\headrulewidth}{0.4pt}
\renewcommand{\footrulewidth}{0pt}








\else
\fi

\chapter{二重积分}
\section{二重积分的概念}

\begin{enumerate}[label=\arabic*.]
    \item 例1 (2010,数一、数二) 
    \begin{align*}
        \lim_{n\rightarrow\infty}\sum_{i=1}^n\sum_{j=1}^n\frac{n}{(n+i)(n^2+j^2)}=
    \end{align*}
    \begin{align*}
        (A)\int_0^1 dx\int_0^x\frac{1}{(1+x)(1+y^2)}dy \quad (B)\int_0^1 dx\int_0^x\frac{1}{(1+x)(1+y)}dy \\
        (C)\int_0^1 dx\int_0^1\frac{1}{(1+x)(1+y)}dy \quad (D)\int_0^1 dx\int_0^1\frac{1}{(1+x)(1+y^2)}dy
    \end{align*}
    
    \begin{solution}
    【详解】
    \end{solution}
    
    \item 例2 (2016,数三)设$J_i=\iint_{D_i}\sqrt[3]{x-y}dxdy(i=1,2,3)$,其中
    \begin{align*}
        D_1=\{(x,y)|0\leq x\leq 1,0\leq y\leq 1\}, \\
        D_2=\{(x,y)|0\leq x\leq 1,0\leq y\leq \sqrt{x}\}, \\
        D_3=\{(x,y)|0\leq x\leq 1,x^2\leq y\leq 1\},
    \end{align*}
    则
    \begin{align*}
        (A)\ J_1<J_2<J_3 \quad (B)\ J_3<J_1<J_2 \\
        (C)\ J_2<J_3<J_1 \quad (D)\ J_2<J_1<J_3
    \end{align*}
    
    \begin{solution}
    【详解】
    \end{solution}
\end{enumerate}

\section{交换积分次序}

\begin{enumerate}[label=\arabic*.,start=3]
    \item 例3 (2001,数一)交换二次积分的积分次序:
    \begin{align*}
        \int_{-1}^0 dy\int_2^{1-y} f(x,y)dx=
    \end{align*}
    
    \begin{solution}
    【详解】
    \end{solution}
    
    \item 例5 交换$I=\int_{-\frac{\pi}{4}}^{\frac{\pi}{2}}d\theta\int_0^{a\cos\theta}f(r,\theta)dr$的积分次序。
    
    \begin{solution}
    【详解】
    \end{solution}
\end{enumerate}

\section{二重积分的计算}

\begin{enumerate}[label=\arabic*.,start=6]
    \item 例6 (2011,数一、数二)已知函数$f(x,y)$具有二阶连续偏导数,且$f(1,y)=0$,$f(x,1)=0$,$\iint_D f(x,y)dxdy=a$,其中$D=\{(x,y)|0\leq x\leq 1,0\leq y\leq 1\}$,计算二重积分
    \begin{align*}
        I=\iint_D xyf_{xy}''(x,y)dxdy.
    \end{align*}
    
    \begin{solution}
    【详解】
    \end{solution}
    
    \item 例7 计算$\iint_D\sqrt{|y-x^2|}dxdy$,其中$D=\{(x,y)|-1\leq x\leq 1,0\leq y\leq 2\}$。
    
    \begin{solution}
    【详解】
    \end{solution}
    
    \item 例8 (2018,数二)设平面区域$D$由曲线$\begin{cases}x=t-\sin t \\ y=1-\cos t\end{cases}(0\leq t\leq 2\pi)$与$x$轴围成,计算二重积分$\iint_D(x+2y)dxdy$。
    
    \begin{solution}
    【详解】
    \end{solution}
    
    \item 例9 (2007,数二、数三)设二元函数
    \begin{align*}
        f(x,y)=\begin{cases}
            x^2, & |x|+|y|\leq 1 \\
            \frac{1}{\sqrt{x^2+y^2}}, & 1<|x|+|y|\leq 2
        \end{cases}
    \end{align*}
    计算二重积分$\iint_D f(x,y)dxdy$,其中$D=\{(x,y)||x|+|y|\leq 2\}$。
    
    \begin{solution}
    【详解】
    \end{solution}
    
    \item 例10 (2014,数二、数三)设平面区域$D=\{(x,y)|1\leq x^2+y^2\leq 4,x\geq 0,y\geq 0\}$,计算
    \begin{align*}
        \iint_D\frac{x\sin(\pi\sqrt{x^2+y^2})}{x+y}dxdy.
    \end{align*}
    
    \begin{solution}
    【详解】
    \end{solution}
    
    \item 例11 (2019,数二)已知平面区域$D=\{(x,y)||x|\leq y,(x^2+y^2)^3\leq y^4\}$,计算二重积分
    \begin{align*}
        \iint_D\frac{x+y}{\sqrt{x^2+y^2}}dxdy.
    \end{align*}
    
    \begin{solution}
    【详解】
    \end{solution}
\end{enumerate}

\section{其他题型}

\begin{enumerate}[label=\arabic*.,start=13]
    \item 例12 (2010,数二)计算二重积分$I=\iint_D r^2\sin\theta\sqrt{1-r^2\cos 2\theta} drd\theta$,其中(题目描述不完整)
    
    \begin{solution}
    【详解】
    \end{solution}
    
    \item 例13 (2009,数二、数三)计算二重积分$\iint_D(x-y)dxdy$,其中
    \begin{align*}
        D=\{(x,y)|(x-1)^2+(y-1)^2\leq 2,y\geq x\}.
    \end{align*}
    
    \begin{solution}
    【详解】
    \end{solution}
\end{enumerate}

\ifx\allfiles\undefined
\end{document}
\fi