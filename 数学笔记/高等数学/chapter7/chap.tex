\ifx\allfiles\undefined
\documentclass[12pt, a4paper, oneside, UTF8]{ctexbook}
\def\path{../../config}
\usepackage{amsmath}
\usepackage{amsthm}
\usepackage{amssymb}
\usepackage{array}
\usepackage{xcolor}
\usepackage{graphicx}
\usepackage{mathrsfs}
\usepackage{enumitem}
\usepackage{geometry}
\usepackage[colorlinks, linkcolor=black]{hyperref}
\usepackage{stackengine}
\usepackage{yhmath}
\usepackage{extarrows}
\usepackage{tikz}
\usepackage{pgfplots}
\usepackage{asymptote}
\usepackage{float}
\usepackage{fontspec} % 使用字体

\setmainfont{Times New Roman}
\setCJKmainfont{LXGWWenKai-Light}[
    SlantedFont=*
]

\everymath{\displaystyle}

\usepgfplotslibrary{polar}
\usepackage{subcaption}
\usetikzlibrary{decorations.pathreplacing, positioning}

\usepgfplotslibrary{fillbetween}
\pgfplotsset{compat=1.18}
% \usepackage{unicode-math}
\usepackage{esint}
\usepackage[most]{tcolorbox}

\usepackage{fancyhdr}
\usepackage[dvipsnames, svgnames]{xcolor}
\usepackage{listings}

\definecolor{mygreen}{rgb}{0,0.6,0}
\definecolor{mygray}{rgb}{0.5,0.5,0.5}
\definecolor{mymauve}{rgb}{0.58,0,0.82}
\definecolor{NavyBlue}{RGB}{0,0,128}
\definecolor{Rhodamine}{RGB}{255,0,255}
\definecolor{PineGreen}{RGB}{0,128,0}

\graphicspath{ {figures/},{../figures/}, {config/}, {../config/} }

\linespread{1.6}

\geometry{
    top=25.4mm, 
    bottom=25.4mm, 
    left=20mm, 
    right=20mm, 
    headheight=2.17cm, 
    headsep=4mm, 
    footskip=12mm
}

\setenumerate[1]{itemsep=5pt,partopsep=0pt,parsep=\parskip,topsep=5pt}
\setitemize[1]{itemsep=5pt,partopsep=0pt,parsep=\parskip,topsep=5pt}
\setdescription{itemsep=5pt,partopsep=0pt,parsep=\parskip,topsep=5pt}

\lstset{
    language=Mathematica,
    basicstyle=\tt,
    breaklines=true,
    keywordstyle=\bfseries\color{NavyBlue}, 
    emphstyle=\bfseries\color{Rhodamine},
    commentstyle=\itshape\color{black!50!white}, 
    stringstyle=\bfseries\color{PineGreen!90!black},
    columns=flexible,
    numbers=left,
    numberstyle=\footnotesize,
    frame=tb,
    breakatwhitespace=false,
} 

\lstset{
    language=TeX, % 设置语言为 TeX
    basicstyle=\ttfamily, % 使用等宽字体
    breaklines=true, % 自动换行
    keywordstyle=\bfseries\color{NavyBlue}, % 关键字样式
    emphstyle=\bfseries\color{Rhodamine}, % 强调样式
    commentstyle=\itshape\color{black!50!white}, % 注释样式
    stringstyle=\bfseries\color{PineGreen!90!black}, % 字符串样式
    columns=flexible, % 列的灵活性
    numbers=left, % 行号在左侧
    numberstyle=\footnotesize, % 行号字体大小
    frame=tb, % 顶部和底部边框
    breakatwhitespace=false % 不在空白处断行
}

% \begin{lstlisting}[language=TeX] ... \end{lstlisting}

% 定理环境设置
\usepackage[strict]{changepage} 
\usepackage{framed}

\definecolor{greenshade}{rgb}{0.90,1,0.92}
\definecolor{redshade}{rgb}{1.00,0.88,0.88}
\definecolor{brownshade}{rgb}{0.99,0.95,0.9}
\definecolor{lilacshade}{rgb}{0.95,0.93,0.98}
\definecolor{orangeshade}{rgb}{1.00,0.88,0.82}
\definecolor{lightblueshade}{rgb}{0.8,0.92,1}
\definecolor{purple}{rgb}{0.81,0.85,1}

\theoremstyle{definition}
\newtheorem{myDefn}{\indent Definition}[section]
\newtheorem{myLemma}{\indent Lemma}[section]
\newtheorem{myThm}[myLemma]{\indent Theorem}
\newtheorem{myCorollary}[myLemma]{\indent Corollary}
\newtheorem{myCriterion}[myLemma]{\indent Criterion}
\newtheorem*{myRemark}{\indent Remark}
\newtheorem{myProposition}{\indent Proposition}[section]

\newenvironment{formal}[2][]{%
	\def\FrameCommand{%
		\hspace{1pt}%
		{\color{#1}\vrule width 2pt}%
		{\color{#2}\vrule width 4pt}%
		\colorbox{#2}%
	}%
	\MakeFramed{\advance\hsize-\width\FrameRestore}%
	\noindent\hspace{-4.55pt}%
	\begin{adjustwidth}{}{7pt}\vspace{2pt}\vspace{2pt}}{%
		\vspace{2pt}\end{adjustwidth}\endMakeFramed%
}

\newenvironment{definition}{\vspace{-\baselineskip * 2 / 3}%
	\begin{formal}[Green]{greenshade}\vspace{-\baselineskip * 4 / 5}\begin{myDefn}}
	{\end{myDefn}\end{formal}\vspace{-\baselineskip * 2 / 3}}

\newenvironment{theorem}{\vspace{-\baselineskip * 2 / 3}%
	\begin{formal}[LightSkyBlue]{lightblueshade}\vspace{-\baselineskip * 4 / 5}\begin{myThm}}%
	{\end{myThm}\end{formal}\vspace{-\baselineskip * 2 / 3}}

\newenvironment{lemma}{\vspace{-\baselineskip * 2 / 3}%
	\begin{formal}[Plum]{lilacshade}\vspace{-\baselineskip * 4 / 5}\begin{myLemma}}%
	{\end{myLemma}\end{formal}\vspace{-\baselineskip * 2 / 3}}

\newenvironment{corollary}{\vspace{-\baselineskip * 2 / 3}%
	\begin{formal}[BurlyWood]{brownshade}\vspace{-\baselineskip * 4 / 5}\begin{myCorollary}}%
	{\end{myCorollary}\end{formal}\vspace{-\baselineskip * 2 / 3}}

\newenvironment{criterion}{\vspace{-\baselineskip * 2 / 3}%
	\begin{formal}[DarkOrange]{orangeshade}\vspace{-\baselineskip * 4 / 5}\begin{myCriterion}}%
	{\end{myCriterion}\end{formal}\vspace{-\baselineskip * 2 / 3}}
	

\newenvironment{remark}{\vspace{-\baselineskip * 2 / 3}%
	\begin{formal}[LightCoral]{redshade}\vspace{-\baselineskip * 4 / 5}\begin{myRemark}}%
	{\end{myRemark}\end{formal}\vspace{-\baselineskip * 2 / 3}}

\newenvironment{proposition}{\vspace{-\baselineskip * 2 / 3}%
	\begin{formal}[RoyalPurple]{purple}\vspace{-\baselineskip * 4 / 5}\begin{myProposition}}%
	{\end{myProposition}\end{formal}\vspace{-\baselineskip * 2 / 3}}


\newtheorem{example}{\indent \color{SeaGreen}{Example}}[section]
\renewcommand{\proofname}{\indent\textbf{\textcolor{TealBlue}{Proof}}}
\NewEnviron{solution}{%
	\begin{proof}[\indent\textbf{\textcolor{TealBlue}{Solution}}]%
		\color{blue}% 设置内容为蓝色
		\BODY% 插入环境内容
		\color{black}% 恢复默认颜色(可选,避免影响后续文字)
	\end{proof}%
}

% 自定义命令的文件

\def\d{\mathrm{d}}
\def\R{\mathbb{R}}
%\newcommand{\bs}[1]{\boldsymbol{#1}}
%\newcommand{\ora}[1]{\overrightarrow{#1}}
\newcommand{\myspace}[1]{\par\vspace{#1\baselineskip}}
\newcommand{\xrowht}[2][0]{\addstackgap[.5\dimexpr#2\relax]{\vphantom{#1}}}
\newenvironment{mycases}[1][1]{\linespread{#1} \selectfont \begin{cases}}{\end{cases}}
\newenvironment{myvmatrix}[1][1]{\linespread{#1} \selectfont \begin{vmatrix}}{\end{vmatrix}}
\newcommand{\tabincell}[2]{\begin{tabular}{@{}#1@{}}#2\end{tabular}}
\newcommand{\pll}{\kern 0.56em/\kern -0.8em /\kern 0.56em}
\newcommand{\dive}[1][F]{\mathrm{div}\;\boldsymbol{#1}}
\newcommand{\rotn}[1][A]{\mathrm{rot}\;\boldsymbol{#1}}

\newif\ifshowanswers
\showanswerstrue % 注释掉这行就不显示答案

% 定义答案环境
\newcommand{\answer}[1]{%
    \ifshowanswers
        #1%
    \fi
}

% 修改参数改变封面样式,0 默认原始封面、内置其他1、2、3种封面样式
\def\myIndex{0}


\ifnum\myIndex>0
    \input{\path/cover_package_\myIndex} 
\fi

\def\myTitle{考研数学笔记}
\def\myAuthor{Weary Bird}
\def\myDateCover{\today}
\def\myDateForeword{\today}
\def\myForeword{相见欢·林花谢了春红}
\def\myForewordText{
    林花谢了春红,太匆匆。
    无奈朝来寒雨晚来风。
    胭脂泪,相留醉,几时重。
    自是人生长恨水长东。
}
\def\mySubheading{以姜晓千强化课讲义为底本}


\begin{document}
\input{\path/cover_text_\myIndex.tex}

\newpage
\thispagestyle{empty}
\begin{center}
    \Huge\textbf{\myForeword}
\end{center}
\myForewordText
\begin{flushright}
    \begin{tabular}{c}
        \myDateForeword
    \end{tabular}
\end{flushright}

\newpage
\pagestyle{plain}
\setcounter{page}{1}
\pagenumbering{Roman}
\tableofcontents

\newpage
\pagenumbering{arabic}
% \setcounter{chapter}{-1}
\setcounter{page}{1}

\pagestyle{fancy}
\fancyfoot[C]{\thepage}
\renewcommand{\headrulewidth}{0.4pt}
\renewcommand{\footrulewidth}{0pt}








\else
\fi

\chapter{无穷级数}
\section{数项级数敛散性的判定}

\begin{enumerate}[label=\arabic*.]
    \item (2015,数三)下列级数中发散的是 \\
        $\displaystyle (A)\sum_{n=1}^{\infty}\frac{n}{3^n} \qquad(B)\ \sum_{n=1}^{\infty}\frac{1}{\sqrt{n}}\ln{\left(1+\frac{1}{n}\right)} $ \\
        $\displaystyle (C)\sum_{n=2}^{\infty}\frac{(-1)^n+1}{\ln n} \qquad (D)\sum_{n=1}^{\infty}\frac{n!}{n^n}$
    
    \begin{solution}
    \newpage
    \end{solution}
    
    \item (2017,数三)若级数$\displaystyle \sum_{n=2}^{\infty}\left[\sin\frac{1}{n}-k\ln\left(1-\frac{1}{n}\right)\right]$收敛,则$k=$ \\
    $(A)\ 1 \qquad (B)\ 2 \qquad (C)\ -1 \qquad (D)\ -2$
    
    \begin{solution}
    \newpage
    \end{solution}
\end{enumerate}

\section{交错级数}

\begin{enumerate}[label=\arabic*.,start=3]
    \item 判定下列级数的敛散性:
    \begin{align*}
        (1)\sum_{n=1}^{\infty}\frac{(-1)^{n-1}}{n-\ln n} \quad (2)\sum_{n=2}^{\infty}\frac{(-1)^n}{\sqrt{n}+(-1)^n}.
    \end{align*}
    
    \begin{solution}
    \newpage
    \end{solution}
\end{enumerate}

\section{任意项级数}

\begin{enumerate}[label=\arabic*.,start=4]
    \item (2002,数一)设$u_n\neq 0(n=1,2,3,\cdots)$,且$\displaystyle\lim_{n\rightarrow\infty}\frac{n}{u_n}=1$ \\
    则级数$\displaystyle\sum_{n=1}^{\infty}(-1)^{n+1}\left(\frac{1}{u_n}+\frac{1}{u_{n+1}}\right)$ \\
    (A)\ 发散 \quad (B)\ 绝对收敛 \quad (C)\ 条件收敛 \quad (D)\ 敛散性根据所给条件不能判定
    
    \begin{solution}
    \newpage
    \end{solution}
    
    \item (2019,数三)若级数$\displaystyle\sum_{n=1}^{\infty}nu_n$绝对收敛,$\displaystyle\sum_{n=1}^{\infty}\frac{v_n}{n}$条件收敛,则 \\
    $\displaystyle(A)\sum_{n=1}^{\infty} u_n v_n\text{条件收敛} \quad (B)\sum_{n=1}^{\infty} u_n v_n\text{绝对收敛}$ \\
    $\displaystyle(C)\sum_{n=1}^{\infty}\left(u_n+v_n\right)\text{收敛} \quad (D)\sum_{n=1}^{\infty}\left(u_n+v_n\right)\text{发散}$
    
    
    \begin{solution}
    \newpage
    \end{solution}
\end{enumerate}

\section{幂级数求收敛半径与收敛域}

\begin{enumerate}[label=\arabic*.,start=6]
    \item (2015,数一)若级数$\displaystyle\sum_{n=1}^{\infty} a_n$条件收敛则$x=\sqrt{3}$与$x=3$依次为幂级数 \\
    $\displaystyle\sum_{n=1}^{\infty} n a_n(x-1)^n$的 \\
    (A)\ 收敛点,收敛点 \qquad (B)\ 收敛点,发散点\\ 
    (C)\ 发散点,收敛点 \qquad (D)\ 发散点,发散点
    
    \begin{solution}
    \newpage
    \end{solution}
    
    \item 求幂级数$\displaystyle\sum_{n=1}^{\infty}(-1)^n\frac{x^{2n+1}}{3^n(2n+1)}$的收敛域.
    
    \begin{solution}
    \newpage
    \end{solution}
\end{enumerate}

\section{幂级数求和}

\begin{enumerate}[label=\arabic*.,start=8]
    \item (2005,数一)求幂级数$\displaystyle\sum_{n=1}^{\infty}(-1)^{n-1}\left[1+\frac{1}{n(2n-1)}\right] x^{2n}$的收敛区间与和函数$f(x)$.
    
    \begin{solution}
    \newpage
    \end{solution}
    
    \item (2012,数一)求幂级数$\displaystyle\sum_{n=0}^{\infty}\frac{4n^2+4n+3}{2n+1} x^{2n}$的收敛域及和函数.
    
    \begin{solution}
    \newpage
    \end{solution}
    
    \item (2004,数三)设级数$\frac{x^4}{2\cdot 4}+\frac{x^6}{2\cdot 4\cdot 6}+\frac{x^8}{2\cdot 4\cdot 6\cdot 8}+\cdots\quad(-\infty<x<+\infty)$的和函数为$S(x)$。求:
    \begin{enumerate}[label=(\roman*)]
        \item[(1)] $S(x)$所满足的一阶微分方程;
        \item[(2)] $S(x)$的表达式.
    \end{enumerate}
    
    \begin{solution}
    \newpage
    \end{solution}
\end{enumerate}

\section{幂级数展开}

\begin{enumerate}[label=\arabic*.,start=11]
    \item 例11 (2007,数三)将函数$f(x)=\frac{1}{x^2-3x-4}$展开成$x-1$的幂级数,并指出其收敛区间.
    
    \begin{solution}
    \newpage
    \end{solution}
    
    \item 将函数$f(x)=\ln\frac{x}{x+1}$在$x=1$处展开成幂级数.
    
    \begin{solution}
    \newpage
    \end{solution}
\end{enumerate}

\section{无穷级数证明题}

\begin{enumerate}[label=\arabic*.,start=13]
    \item (2016,数一)已知函数$f(x)$可导,且$f(0)=1$,$0<f'(x)<\frac{1}{2}$。设数列$\{x_n\}$满足$x_{n+1}=f(x_n)(n=1,2,\cdots)$。证明:
    \begin{enumerate}[label=(\roman*)]
        \item 级数$\displaystyle\sum_{n=1}^{\infty}(x_{n+1}-x_n)$绝对收敛;
        \item $\displaystyle\lim_{n\rightarrow\infty} x_n$存在,且$\displaystyle 0<\lim_{n\rightarrow\infty} x_n<2$.
    \end{enumerate}
    
    \begin{solution}
    \newpage
    \end{solution}
    
    \item (2014,数一)设数列$\{a_n\}$,$\{b_n\}$满足$0<a_n<\frac{\pi}{2}$,$0<b_n<\frac{\pi}{2}$,$\cos a_n-a_n=\cos b_n$,且级数$\sum_{n=1}^{\infty} b_n$收敛。
    \begin{enumerate}[label=(\roman*)]
        \item[(1)] 证明$\displaystyle\lim_{n\rightarrow\infty} a_n=0$;
        \item[(2)] 证明级数$\displaystyle\sum_{n=1}^{\infty}\frac{a_n}{b_n}$收敛.
    \end{enumerate}
    
    \begin{solution}
    \newpage
    \end{solution}
\end{enumerate}

\section{傅里叶级数}

\begin{enumerate}[label=\arabic*.,start=15]
    \item 设函数
    \begin{align*}
    f(x)=\begin{cases}
    e^x, & -\pi\leq x<0 \\
    1, & 0\leq x<\pi
    \end{cases}
    \end{align*}
    则其以$2\pi$为周期的傅里叶级数在$x=\pi$收敛于?,在$x=2\pi$收敛于?.
    \begin{solution}
    由狄利克雷收敛定理知,$f(x)$以$2\pi$为周期的傅里叶级数在$x=\pi$收敛于
    \begin{align*}
    S(\pi)=\frac{f(\pi-0)+f(-\pi+0)}{2}=\frac{1+e^{-\pi}}{2}
    \end{align*}
    在$x=2\pi$收敛于
    \begin{align*}
    S(2\pi)=S(0)=\frac{f(0-0)+f(0+0)}{2}=\frac{1+1}{2}=1
    \end{align*}
    \end{solution}
    
    \item 将$f(x)=1-x^2,0\leq x\leq\pi$,展开成余弦级数,并求级数$\sum_{n=1}^{\infty}\frac{(-1)^{n-1}}{n^2}$的和.
    
    \begin{solution}
    对$f(x)=1-x^2$进行偶延拓,由$f(x)=1-x^2$为偶函数,知$b_n=0$。
    \begin{align*}
    a_0&=\frac{2}{\pi}\int_0^\pi(1-x^2)dx=2\left(1-\frac{\pi^2}{3}\right) \\
    a_n&=\frac{2}{\pi}\int_0^\pi(1-x^2)\cos nx dx=\frac{4(-1)^{n+1}}{n^2} \quad (n=1,2,\cdots)
    \end{align*}
    \begin{align*}
    f(x)=1-x^2=\frac{a_0}{2}+\sum_{n=1}^{\infty}a_n\cos nx=1-\frac{\pi^2}{3}+\sum_{n=1}^{\infty}\frac{4(-1)^{n+1}}{n^2}\cos nx
    \end{align*}
    令$x=0$,代入上式,得
    \begin{align*}
    \sum_{n=1}^{\infty}\frac{(-1)^{n-1}}{n^2}=\frac{\pi^2}{12}
    \end{align*}
    \end{solution}
\end{enumerate}

\ifx\allfiles\undefined
\end{document}
\fi