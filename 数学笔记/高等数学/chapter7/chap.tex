\ifx\allfiles\undefined
\documentclass[12pt, a4paper, oneside, UTF8]{ctexbook}
\def\path{../../config}
\usepackage{amsmath}
\usepackage{amsthm}
\usepackage{amssymb}
\usepackage{array}
\usepackage{xcolor}
\usepackage{graphicx}
\usepackage{mathrsfs}
\usepackage{enumitem}
\usepackage{geometry}
\usepackage[colorlinks, linkcolor=black]{hyperref}
\usepackage{stackengine}
\usepackage{yhmath}
\usepackage{extarrows}
\usepackage{tikz}
\usepackage{pgfplots}
\usepackage{asymptote}
\usepackage{float}
\usepackage{fontspec} % 使用字体

\setmainfont{Times New Roman}
\setCJKmainfont{LXGWWenKai-Light}[
    SlantedFont=*
]

\everymath{\displaystyle}

\usepgfplotslibrary{polar}
\usepackage{subcaption}
\usetikzlibrary{decorations.pathreplacing, positioning}

\usepgfplotslibrary{fillbetween}
\pgfplotsset{compat=1.18}
% \usepackage{unicode-math}
\usepackage{esint}
\usepackage[most]{tcolorbox}

\usepackage{fancyhdr}
\usepackage[dvipsnames, svgnames]{xcolor}
\usepackage{listings}

\definecolor{mygreen}{rgb}{0,0.6,0}
\definecolor{mygray}{rgb}{0.5,0.5,0.5}
\definecolor{mymauve}{rgb}{0.58,0,0.82}
\definecolor{NavyBlue}{RGB}{0,0,128}
\definecolor{Rhodamine}{RGB}{255,0,255}
\definecolor{PineGreen}{RGB}{0,128,0}

\graphicspath{ {figures/},{../figures/}, {config/}, {../config/} }

\linespread{1.6}

\geometry{
    top=25.4mm, 
    bottom=25.4mm, 
    left=20mm, 
    right=20mm, 
    headheight=2.17cm, 
    headsep=4mm, 
    footskip=12mm
}

\setenumerate[1]{itemsep=5pt,partopsep=0pt,parsep=\parskip,topsep=5pt}
\setitemize[1]{itemsep=5pt,partopsep=0pt,parsep=\parskip,topsep=5pt}
\setdescription{itemsep=5pt,partopsep=0pt,parsep=\parskip,topsep=5pt}

\lstset{
    language=Mathematica,
    basicstyle=\tt,
    breaklines=true,
    keywordstyle=\bfseries\color{NavyBlue}, 
    emphstyle=\bfseries\color{Rhodamine},
    commentstyle=\itshape\color{black!50!white}, 
    stringstyle=\bfseries\color{PineGreen!90!black},
    columns=flexible,
    numbers=left,
    numberstyle=\footnotesize,
    frame=tb,
    breakatwhitespace=false,
} 

\lstset{
    language=TeX, % 设置语言为 TeX
    basicstyle=\ttfamily, % 使用等宽字体
    breaklines=true, % 自动换行
    keywordstyle=\bfseries\color{NavyBlue}, % 关键字样式
    emphstyle=\bfseries\color{Rhodamine}, % 强调样式
    commentstyle=\itshape\color{black!50!white}, % 注释样式
    stringstyle=\bfseries\color{PineGreen!90!black}, % 字符串样式
    columns=flexible, % 列的灵活性
    numbers=left, % 行号在左侧
    numberstyle=\footnotesize, % 行号字体大小
    frame=tb, % 顶部和底部边框
    breakatwhitespace=false % 不在空白处断行
}

% \begin{lstlisting}[language=TeX] ... \end{lstlisting}

% 定理环境设置
\usepackage[strict]{changepage} 
\usepackage{framed}

\definecolor{greenshade}{rgb}{0.90,1,0.92}
\definecolor{redshade}{rgb}{1.00,0.88,0.88}
\definecolor{brownshade}{rgb}{0.99,0.95,0.9}
\definecolor{lilacshade}{rgb}{0.95,0.93,0.98}
\definecolor{orangeshade}{rgb}{1.00,0.88,0.82}
\definecolor{lightblueshade}{rgb}{0.8,0.92,1}
\definecolor{purple}{rgb}{0.81,0.85,1}

\theoremstyle{definition}
\newtheorem{myDefn}{\indent Definition}[section]
\newtheorem{myLemma}{\indent Lemma}[section]
\newtheorem{myThm}[myLemma]{\indent Theorem}
\newtheorem{myCorollary}[myLemma]{\indent Corollary}
\newtheorem{myCriterion}[myLemma]{\indent Criterion}
\newtheorem*{myRemark}{\indent Remark}
\newtheorem{myProposition}{\indent Proposition}[section]

\newenvironment{formal}[2][]{%
	\def\FrameCommand{%
		\hspace{1pt}%
		{\color{#1}\vrule width 2pt}%
		{\color{#2}\vrule width 4pt}%
		\colorbox{#2}%
	}%
	\MakeFramed{\advance\hsize-\width\FrameRestore}%
	\noindent\hspace{-4.55pt}%
	\begin{adjustwidth}{}{7pt}\vspace{2pt}\vspace{2pt}}{%
		\vspace{2pt}\end{adjustwidth}\endMakeFramed%
}

\newenvironment{definition}{\vspace{-\baselineskip * 2 / 3}%
	\begin{formal}[Green]{greenshade}\vspace{-\baselineskip * 4 / 5}\begin{myDefn}}
	{\end{myDefn}\end{formal}\vspace{-\baselineskip * 2 / 3}}

\newenvironment{theorem}{\vspace{-\baselineskip * 2 / 3}%
	\begin{formal}[LightSkyBlue]{lightblueshade}\vspace{-\baselineskip * 4 / 5}\begin{myThm}}%
	{\end{myThm}\end{formal}\vspace{-\baselineskip * 2 / 3}}

\newenvironment{lemma}{\vspace{-\baselineskip * 2 / 3}%
	\begin{formal}[Plum]{lilacshade}\vspace{-\baselineskip * 4 / 5}\begin{myLemma}}%
	{\end{myLemma}\end{formal}\vspace{-\baselineskip * 2 / 3}}

\newenvironment{corollary}{\vspace{-\baselineskip * 2 / 3}%
	\begin{formal}[BurlyWood]{brownshade}\vspace{-\baselineskip * 4 / 5}\begin{myCorollary}}%
	{\end{myCorollary}\end{formal}\vspace{-\baselineskip * 2 / 3}}

\newenvironment{criterion}{\vspace{-\baselineskip * 2 / 3}%
	\begin{formal}[DarkOrange]{orangeshade}\vspace{-\baselineskip * 4 / 5}\begin{myCriterion}}%
	{\end{myCriterion}\end{formal}\vspace{-\baselineskip * 2 / 3}}
	

\newenvironment{remark}{\vspace{-\baselineskip * 2 / 3}%
	\begin{formal}[LightCoral]{redshade}\vspace{-\baselineskip * 4 / 5}\begin{myRemark}}%
	{\end{myRemark}\end{formal}\vspace{-\baselineskip * 2 / 3}}

\newenvironment{proposition}{\vspace{-\baselineskip * 2 / 3}%
	\begin{formal}[RoyalPurple]{purple}\vspace{-\baselineskip * 4 / 5}\begin{myProposition}}%
	{\end{myProposition}\end{formal}\vspace{-\baselineskip * 2 / 3}}


\newtheorem{example}{\indent \color{SeaGreen}{Example}}[section]
\renewcommand{\proofname}{\indent\textbf{\textcolor{TealBlue}{Proof}}}
\NewEnviron{solution}{%
	\begin{proof}[\indent\textbf{\textcolor{TealBlue}{Solution}}]%
		\color{blue}% 设置内容为蓝色
		\BODY% 插入环境内容
		\color{black}% 恢复默认颜色(可选,避免影响后续文字)
	\end{proof}%
}

% 自定义命令的文件

\def\d{\mathrm{d}}
\def\R{\mathbb{R}}
%\newcommand{\bs}[1]{\boldsymbol{#1}}
%\newcommand{\ora}[1]{\overrightarrow{#1}}
\newcommand{\myspace}[1]{\par\vspace{#1\baselineskip}}
\newcommand{\xrowht}[2][0]{\addstackgap[.5\dimexpr#2\relax]{\vphantom{#1}}}
\newenvironment{mycases}[1][1]{\linespread{#1} \selectfont \begin{cases}}{\end{cases}}
\newenvironment{myvmatrix}[1][1]{\linespread{#1} \selectfont \begin{vmatrix}}{\end{vmatrix}}
\newcommand{\tabincell}[2]{\begin{tabular}{@{}#1@{}}#2\end{tabular}}
\newcommand{\pll}{\kern 0.56em/\kern -0.8em /\kern 0.56em}
\newcommand{\dive}[1][F]{\mathrm{div}\;\boldsymbol{#1}}
\newcommand{\rotn}[1][A]{\mathrm{rot}\;\boldsymbol{#1}}

\newif\ifshowanswers
\showanswerstrue % 注释掉这行就不显示答案

% 定义答案环境
\newcommand{\answer}[1]{%
    \ifshowanswers
        #1%
    \fi
}

% 修改参数改变封面样式,0 默认原始封面、内置其他1、2、3种封面样式
\def\myIndex{0}


\ifnum\myIndex>0
    \input{\path/cover_package_\myIndex} 
\fi

\def\myTitle{考研数学笔记}
\def\myAuthor{Weary Bird}
\def\myDateCover{\today}
\def\myDateForeword{\today}
\def\myForeword{相见欢·林花谢了春红}
\def\myForewordText{
    林花谢了春红,太匆匆。
    无奈朝来寒雨晚来风。
    胭脂泪,相留醉,几时重。
    自是人生长恨水长东。
}
\def\mySubheading{以姜晓千强化课讲义为底本}


\begin{document}
\else
\fi

\chapter{无穷级数}
\section{数项级数敛散性的判定}
\begin{remark}
    正项级数敛散性的判断 \\
    比较判别法\ (放缩/等价/Taylor展开) \\
    比值判别法\ (当出现$n!$) \\
    根值判别法\ (当出现$n^n$) \\
    积分判别法\ (P级数/对数P级数) 
    $$
    \sum_{n=1}^{\infty}\frac{1}{n^p} \begin{cases}
        \text{收敛}, & p > 1 \\
        \text{发散}, & p \leq 1 
    \end{cases}
    $$
    推广 
    $$
    \sum_{n=1}^{\infty}\frac{\ln^{\alpha}{n}}{n^p} \sim \sum_{n=1}^{\infty}\frac{1}{n^p}
    $$
    对数P级数
    $$
    \sum_{n=2}^{\infty}\frac{1}{n\ln^{p}{n}}\begin{cases}
        \text{收敛}, & p > 1 \\
        \text{发散}, & p \leq 1 
    \end{cases} \leftarrow \int\frac{\d x}{x\ln^{p}{n}} = \int\frac{d \ln x}{\ln^{p}{x}}
    $$
    故其与$P$级数的敛散性与P的关系一致,推广
    $$
    \sum_{n=2}^{\infty}\frac{1}{n^{\alpha}\ln^{p}{n}} \sim \sum_{n=1}^{\infty}\frac{1}{n^{\alpha}}\begin{cases}
        \alpha > 1, &\text{收敛} \\
        \aleph \leq 1, &\text{发散}
    \end{cases}
    $$
\end{remark}

\begin{enumerate}[label=\arabic*.]
    \item (2015,数三)下列级数中发散的是 \\
        $\displaystyle (A)\sum_{n=1}^{\infty}\frac{n}{3^n} \qquad(B)\ \sum_{n=1}^{\infty}\frac{1}{\sqrt{n}}\ln{\left(1+\frac{1}{n}\right)} $ \\
        $\displaystyle (C)\sum_{n=2}^{\infty}\frac{(-1)^n+1}{\ln n} \qquad (D)\sum_{n=1}^{\infty}\frac{n!}{n^n}$
    
    \begin{solution}
    (A) 由根值判别法 $\displaystyle \lim_{n\to\infty}\sqrt[n]{\frac{n}{3^n}}=\frac{1}{3}<1$收敛 \\
    (B) 由于$\displaystyle \frac{1}{\sqrt{n}}\ln{(1+\frac{1}{n})} \sim \frac{1}{n^{\frac{3}{2}}}$,而$\frac{3}{2} > 1$故原级数收敛 \\
    (C) 原级数等于$\displaystyle \sum_{n=2}^{\infty}\frac{(-1)^n}{\ln{n}}+\sum_{n=2}^{\infty}\frac{1}{\ln{n}}$ 
    前一个级数由莱布尼兹判别法知收敛,第二个级数由P级数的推广容易得知其发散,故原级数发散 \\
    (D) 由比值判别法有$\displaystyle \lim_{n\to\infty}\frac{u_{n+1}}{u_n}=\lim_{n\to\infty}\left(\frac{n}{n+1}\right)^2=e^{-1}<1$故原级数收敛
    \end{solution}

    
    \item (2017,数三)若级数$\displaystyle \sum_{n=2}^{\infty}\left[\sin\frac{1}{n}-k\ln\left(1-\frac{1}{n}\right)\right]$收敛,则$k=$ \\
    $(A)\ 1 \qquad (B)\ 2 \qquad (C)\ -1 \qquad (D)\ -2$
    
    \begin{solution}
    \begin{align*}
        \text{原式} &\xlongequal{Taylor} \frac{1}{n}-\frac{1}{6n^3}+o(\frac{1}{n^3}) - k\left[\frac{1}{n}-\frac{1}{2n^2}+O(\frac{1}{n^2})\right] \\
        &= \frac{1+k}{n}+\frac{k}{2}\cdot\frac{1}{n^2}+o(\frac{1}{n^2}) 
    \end{align*}
    由P级数判别法可知,$1+k=0\implies k = -1$
    \end{solution}
\end{enumerate}

\section{交错级数}
\begin{remark}
    交错级数敛散性的判断 \\
    莱布尼兹判别,\underline{通项单调递减趋于0}可以判断原级数收敛. \\
    取绝对值,若其绝对收敛则原级数也收敛 
\end{remark}
\begin{enumerate}[label=\arabic*.,start=3]
    \item 判定下列级数的敛散性: \\
    $\displaystyle (1)\sum_{n=1}^{\infty}\frac{(-1)^{n-1}}{n-\ln n}$ \\
    $\displaystyle (2)\sum_{n=2}^{\infty}\frac{(-1)^n}{\sqrt{n}+(-1)^n}.$
    
    \begin{solution}
    (1) 记$f(x)=\frac{1}{x-\ln{x}},f'(x)=-\frac{1-\frac{1}{x}}{(x-\ln{x})^2} < 0$ 从而$u_n$单调递减,又$\lim_{n\to\infty}u_n=0$
    故由莱布尼兹判别法可知$\sum_{i=1}^{n}\frac{(-1)^{n-1}}{n-\ln{n}}$收敛 \\
    (2) 在一起不好判断的时候,把它们拆开了分别做 
    \begin{align*}
        \text{原式} = \sum_{n=2}^{\infty}\frac{(-1)^n\sqrt{n}}{n-1} - \sum_{n=1}^{\infty}\frac{1}{n-1}
    \end{align*}
    由莱布尼兹判别法易知第一个级数收敛,第二个级数由P级数可知其发散.故原级数发散
    \end{solution}
\end{enumerate}

\section{任意项级数}
\begin{remark}
    任意项级数 \\
    收敛级数的定义(部分和极限存在) 
    $$
    S_n=u_1+u_2+\ldots u_n = \sum_{i=1}^{n}u_i, \text{若级数收敛} \iff \lim_{n\to\infty}S_n \exists
    $$
    级数的性质 - 线性组合 
    $
    \begin{cases}
        \text{收敛} + \text{收敛} = \text{收敛} \\
        \text{收敛} + \text{发散} = \text{发散} \\
        \text{发散} + \text{发散} = ? 
    \end{cases}
    $  \\
    级数的性质\ 改变有限项级数的敛散性不变 \\
    级数的性质\ 结合律,若级数$\displaystyle \sum_{n=1}^{\infty}u_n$收敛,则\underline{不改变其项的次序间任意添加符号},并把每个括号
    内的数作为一项,这样得到的新奇数\underline{仍然收敛},且其和不变. 反之不然. \\
    结合律的推论1\ 若加括号后的级数发散,则原级数必然发散 \\
    结合律的推论2\ 若$\displaystyle \lim_{n\to\infty} = 0$又其相继两项加括号后的级数收敛,则原级数也收敛,且和相等 \\
    收敛级数的必要条件\ 若$\displaystyle \sum_{n=1}^{\infty}u_n$收敛,则$\displaystyle \lim_{n\to\infty} u_n = 0$
\end{remark}

\begin{enumerate}[label=\arabic*.,start=4]
    \item (2002,数一)设$u_n\neq 0(n=1,2,3,\cdots)$,且$\displaystyle\lim_{n\rightarrow\infty}\frac{n}{u_n}=1$ \\
    则级数$\displaystyle\sum_{n=1}^{\infty}(-1)^{n+1}\left(\frac{1}{u_n}+\frac{1}{u_{n+1}}\right)$ \\
    (A)\ 发散 \quad (B)\ 绝对收敛 \quad (C)\ 条件收敛 \quad (D)\ 敛散性根据所给条件不能判定
    
    \begin{solution}
    这种题首先判断是否绝对收敛,由$\displaystyle \lim_{n\to\infty}\frac{n}{u_n}=1$可知其一定不可能绝对收敛 \\
    让后判断级数本身是否收敛,这种形式的题目大概率就是要使用定义,求其部分和
    $$
        \lim_{n\to\infty}S_n=\lim_{n\to\infty}\left(\frac{1}{u_1}+\frac{1}{u_2}-\frac{1}{u_2}-\frac{1}{u_3}+\ldots+(-1)^{n+1}(\frac{1}{u_n}+\frac{1}{u_{n+1}})\right) 
    $$
    故
    $$
        \lim_{n\to\infty}S_n = \frac{1}{n_1} 
    $$
    因此原级数条件收敛
    \end{solution}
    
    \item (2019,数三)若级数$\displaystyle\sum_{n=1}^{\infty}nu_n$绝对收敛,$\displaystyle\sum_{n=1}^{\infty}\frac{v_n}{n}$条件收敛,则 \\
    $\displaystyle(A)\sum_{n=1}^{\infty} u_n v_n\text{条件收敛} \quad (B)\sum_{n=1}^{\infty} u_n v_n\text{绝对收敛}$ \\
    $\displaystyle(C)\sum_{n=1}^{\infty}\left(u_n+v_n\right)\text{收敛} \quad (D)\sum_{n=1}^{\infty}\left(u_n+v_n\right)\text{发散}$
    
    
    \begin{solution}
    这种题目比较好的解法是用特殊值筛选掉错误答案. 如令$u_n=0$则$A$错误,$v_n=(-1)^n$则$B$错误,$v_n=\frac{(-1)^n}{\ln{n}}$则$D$错误 \\
    证明B选项正确,关键点考虑\underline{极限的有界性} 由$\sum\frac{v_n}{n}$收敛可知 
    $$
        \lim_{n\to\infty}\frac{v_n}{n}= 0
    $$
    由极限的有界性,可知 
    $$
    \exists M, \forall n, \left|\frac{v_n}{n}\right|\leq M 
    $$
    从而
    $$
    \left|u_nv_n\right| = \left|nu_n\cdot\frac{v_n}{n}\right| \leq M\left|n\u_n\right|
    $$
    故B选项正确
    \end{solution}
\end{enumerate}

\section{幂级数求收敛半径与收敛域}
\begin{remark}
    方法一:阿贝尔定理.收敛的幂级数在收敛区间内\underline{绝对收敛},在收敛域外\underline{发散},在边界点上\underline{可能收敛也可能发散,可能绝对收敛也可能条件收敛} \\
    方法二:比值定理/根值定理 \\
    方法三:\underline{柯西判别法} 最常用 \\
    逐项求导/逐项积分,{\color{red}收敛区间不变},需要注意边界点,其敛散性可能发生改变.
\end{remark}

\newpage

\begin{enumerate}[label=\arabic*.,start=6]
    \item (2015,数一)若级数$\displaystyle\sum_{n=1}^{\infty} a_n$条件收敛则$x=\sqrt{3}$与$x=3$依次为幂级数 \\
    $\displaystyle\sum_{n=1}^{\infty} n a_n(x-1)^n$的 \\
    (A)\ 收敛点,收敛点 \qquad (B)\ 收敛点,发散点\\ 
    (C)\ 发散点,收敛点 \qquad (D)\ 发散点,发散点
    
    \begin{solution}
    由题设条件可知级数$\displaystyle\sum_{n=1}^{\infty} a_nx^n$的收敛区间为$(-1,1)$ 
    \begin{align*}
        \sum_{n=1}^{\infty} n a_n(x-1)^n &= (x-1)\sum_{n=1}^{\infty} n a_n(x-1)^{n-1} \\
        &=(x-1)\left[\sum_{n=1}^{\infty} n a_n(x-1)^n\right]' 
    \end{align*}
    故其收敛区间为$-1<x-1<1\implies x\in(0,2)$ 由阿贝尔定理可知$x=\sqrt{3}$为绝对收敛点,$x=3$为发散点  
    \end{solution}
    
    \item 求幂级数$\displaystyle\sum_{n=1}^{\infty}(-1)^n\frac{x^{2n+1}}{3^n(2n+1)}$的收敛域.
    
    \begin{solution}
    这种题目优先考虑柯西定理,即
    $$
    \lim_{n\to\infty}\sqrt[n]{\left|u_n(x)\right|} = \frac{x^2}{3} < 1
    $$
    即$x\in(-\sqrt{3},\sqrt{3})$(收敛区间).接着判断边界点的敛散性. 当$x=\pm\sqrt{3}$有
    $$
    \sum_{n=1}^{\infty}(-1)^n\frac{\pm\sqrt{3}}{2n+1}
    $$
    由莱布尼兹判别法可知其条件收敛,故原级数的收敛域为$\left[-\sqrt{3},\sqrt{3}\right]$
    \end{solution}
\end{enumerate}

\newpage

\section{幂级数求和}
\begin{remark}
    关键就是六组公式
    \begin{align*}
        e^x &=\sum_{n=0}^{\infty}\frac{x^n}{n!}, x\in(-\infty,+\infty) \\
        \sin{x} &=\sum_{n=0}^{\infty}\frac{(-1)^nx^{2n+1}}{(2n+1)!},, x\in(-\infty,+\infty) \\
        \cos{x} &=\sum_{n=0}^{\infty}\frac{(-1)^nx^{2n}}{(2n)!},, x\in(-\infty,+\infty) \\
        \arctan{x} &=\sum_{n=0}^{\infty}\frac{(-1)^nx^{2n+1}}{2n+1},, x\in(-\infty,+\infty) \\
        \frac{1}{1-x} &=\sum_{n=0}^{\infty}x^n,x\in(-1,1) \\
        \frac{1}{1+x} &=\sum_{n=0}^{\infty}(-1)^nx^n,x\in(-1,1) \\
        \ln{(1+x)} &=\sum_{n=1}^{\infty}\frac{(-1)^{\color{red}n-1}}{n}x^n, x\in(-1,1] \\
        \ln{(1-x)} &=-\sum_{n=1}^{\infty}\frac{x^n}{n}, x\in[-1,1) 
    \end{align*}
\end{remark}
\begin{enumerate}[label=\arabic*.,start=8]
    \item (2005,数一)求幂级数$\displaystyle\sum_{n=1}^{\infty}(-1)^{n-1}\left[1+\frac{1}{n(2n-1)}\right] x^{2n}$的收敛区间与和函数$f(x)$.
    
    \begin{solution}
    这种题都可以说是套路题,第一步先求收敛域. 由柯西定理有
    $$
        \lim_{n\to\infty}\sqrt[n]{\left|(-1)^{n-1}\left[1+\frac{1}{n(2n-1)}\right]x^{2n}\right|} = x^2 < 1
    $$
    故收敛区间为$(-1,1)$ 
    $$
        S(x)=\sum_{n=1}^{\infty}(-1)^{n-1}x^{2n}+\sum_{n=1}^{\infty}\frac{(-1)^{n-1}}{n(2n-1)}x^{2n} 
    $$
    其中
    $$
        S_1(x) = \sum_{n=1}^{\infty}(-1)^{n-1}x^{2n} = \frac{x^2}{1+x^2}, x\in(-1,1)
    $$
    \begin{align*}
        S_2(x) & = \sum_{n=1}^{\infty}\frac{(-1)^{n-1}}{n(2n-1)}x^{2n} \\
        &= 2x\sum_{n=1}^{\infty}\frac{(-1)^{n-1}}{2n-1}\cdot x^{2n-1}-\sum_{n=1}^{\infty}\frac{(-1)^{n-1}}{n}x^{2n} \\
        &=2x\sum_{n=0}^{\infty}\frac{(-1)^{n}}{2n+1}\cdot x^{2n+1} - \ln{(1+x^2)} \\
        &=2x\arctan{x}-\ln{(1+x^2)}
    \end{align*}
    综上,和函数为$f(x)=\frac{x^2}{1+x^2}+2x\arctan{x}-\ln{(1+x^2)}$
    \end{solution}
    
    \item (2012,数一)求幂级数$\displaystyle\sum_{n=0}^{\infty}\frac{4n^2+4n+3}{2n+1} x^{2n}$的收敛域及和函数.
    
    \begin{solution}
    由柯西定理有
    $$
        \lim_{n\to\infty}\sqrt[n]{\left|\frac{4n^2+4n+3}{2n+1}\cdot x^{2n}\right|} = x^2 < 1 
    $$
    从而收敛区间为$x\in(-1,1)$当$x=\pm 1$时级数为
    $$
        \sum_{n=0}^{\infty}\frac{4n^2+4n+3}{2n+1}
    $$
    显然发散.故收敛域为$(-1,1)$,接下来求和函数. 
    \begin{align*}
        S(x)=\sum_{n=0}^{\infty}(2n+1)x^{2n} + \frac{2}{x}\sum_{n=0}^{\infty}\frac{x^{2n+1}}{2n+1}{\color{red},x\neq 0}
    \end{align*}
    其中
    $$
    S_1(x) = \sum_{n=0}^{\infty}(2n+1)x^{2n} = (x\sum_{n=0}^{\infty}x^{2n})' = \left(\frac{x}{1-x^2}\right)' = \frac{1+x^2}{(1-x^2)^2}
    $$
    $$
    S'_2(x) = \sum_{n=0}^{\infty}x^{2n} = \frac{1}{1-x^2}
    $$
    故
    $$
    S_2(x)={\color{red}S_2(0)}+\int_{0}^{x}S'(t)\d t = \frac{1}{2}\ln{\frac{1+x}{1-x}}
    $$ 
    需要单独计算$S(0)=3$ \\
    综上和函数为$S(x)\begin{cases}
        \displaystyle \frac{1+x^2}{(1-x^2)^2}+\frac{1}{x}\ln{\frac{1+x}{1-x}}, &x\in(-1,0)\cup(0,1) \\
        3, &x=0
    \end{cases}$
    \end{solution}
    
    \item (2004,数三)设级数$\frac{x^4}{2\cdot 4}+\frac{x^6}{2\cdot 4\cdot 6}+\frac{x^8}{2\cdot 4\cdot 6\cdot 8}+\cdots\quad(-\infty<x<+\infty)$的和函数为$S(x)$。求:
    \begin{enumerate}[label=(\roman*)]
        \item[(1)] $S(x)$所满足的一阶微分方程;
        \item[(2)] $S(x)$的表达式.
    \end{enumerate}
    
    \begin{solution}
    (1)求上述级数求导
    \begin{align*}
        S'(x) &= \frac{x^3}{2} + \frac{x^5}{2\cdot 4} + \ldots \\
        &= x\left(\frac{x^2}{2}+\frac{x^4}{2\cdot 4}+\ldots\right) \\
        &=x\left[\frac{x^2}{2}+S(x)\right] 
    \end{align*}
    且有初值$S(0)=0$.
    (2) 上述问题转换为如下初值问题 
    $$
    \begin{cases}
        y' - xy = \frac{x^3}{2} \\
        y(0) = 0
    \end{cases}
    $$
    可以解出$S(x)=e^{\frac{x^2}{2}}-\frac{x^2}{2}-1$
    \end{solution}
\end{enumerate}

\section{幂级数展开}

\begin{enumerate}[label=\arabic*.,start=11]
    \item (2007,数三)将函数$f(x)=\frac{1}{x^2-3x-4}$展开成$x-1$的幂级数,并指出其收敛区间.
    
    \begin{solution}
    $$
        f(x) =\frac{1}{(x-4)(x+1)} = \frac{1}{5}\left(\frac{1}{x-4}-\frac{1}{x+1}\right)
    $$
    其中
    \begin{align*}
        \frac{1}{x-4} &=\frac{1}{-3+x-1} \\ 
        &= -\frac{1}{3}\frac{1}{1-\frac{x-1}{3}} \\ 
        &=-\frac{1}{3}\sum_{n=0}^{\infty}\left(\frac{x-1}{3}\right)^n, x\in(-2,4) \\
        &=\sum_{n=0}^{\infty}-\frac{1}{3^{n+1}}(x-1)^n
    \end{align*}
    同理另一部分为
    $$
    \frac{1}{x+1} = \sum_{n=0}^{\infty}\frac{(-1)^n}{2^{n+1}}(x-1)^n, x\in(-1,3)
    $$
    故
    $$
    f(x)=-\frac{1}{5}\sum_{n=0}^{\infty}\left[\frac{1}{3^{n+1}}+\frac{(-1)^n}{2^{n+1}}\right](x-1)^n, x\in(-1,3)
    $$

    \end{solution}
    
\end{enumerate}

\section{无穷级数证明题}

\begin{enumerate}[label=\arabic*.,start=12]
    \item  设$\displaystyle a_n=\int_{0}^{\frac{\pi}{4}}\tan^{n}x\d x$ 
    \begin{enumerate}
        \item [(I)] 求$\displaystyle \sum_{n=1}^{\infty}\frac{1}{n}(a_n+a_{n+2})$的值 
        \item [(II)] 证明任意常数$\lambda>0$,级数$\displaystyle\sum_{n=1}^{\infty}\frac{a_n}{n^{\lambda}}$收敛
    \end{enumerate}

    \begin{solution}
        (1)
        $$
        a_{n+2}=\int_{0}^{\frac{\pi}{4}}\tan^{n+2}x\d x 
        $$
        \begin{align*}
            a_{n}+a_{n+2} &= \int_{0}^{\frac{\pi}{4}}\tan^{n}x(1+\tan^2{x})\d x \\
            &=\frac{\tan^{n+1}x}{n+1}\bigg|^{\frac{\pi}{4}}_{0} \\
            &=\frac{1}{n+1}
        \end{align*}
        故原级数等于
        $$
        \sum_{n=1}^{\infty}\frac{1}{n(n+1)} = \lim_{n\to\infty}\left(1-\frac{1}{2}+\frac{1}{2}-\ldots+\frac{1}{n}-\frac{1}{n+1}\right) = 1
        $$
        (2) 由一可知 
        $$
        a_n = \frac{1}{n+1} - a_{n+2} \implies a_n < \frac{1}{n+1}
        $$
        故要证级数的通项满足
        $$
        \frac{a_n}{n^{\lambda}} < \frac{1}{n^{\lambda}(n+1)} < \frac{1}{n^{(\lambda+1)}}
        $$
        当$\lambda>0$级数$\displaystyle \sum_{n=1}^{\infty}\frac{1}{n^{(\lambda+1)}}$收敛,由比较判别法可知原级数收敛
    \end{solution}
    \newpage
    \item (2016,数一)已知函数$f(x)$可导,且$f(0)=1$,$0<f'(x)<\frac{1}{2}$。设数列$\{x_n\}$满足$x_{n+1}=f(x_n)(n=1,2,\cdots)$。证明:
    \begin{enumerate}[label=(\roman*)]
        \item[(I)] 级数$\displaystyle\sum_{n=1}^{\infty}(x_{n+1}-x_n)$绝对收敛;
        \item[(II)] $\displaystyle\lim_{n\rightarrow\infty} x_n$存在,且$\displaystyle 0<\lim_{n\rightarrow\infty} x_n<2$.
    \end{enumerate}
    
    \begin{solution}
    (1)本质考察的为压缩映射的证明 
    \begin{align*}
        \left|x_{n+1}-x_n\right| &= \left|f(x_n)-f(x_{n-1})\right| \\
        &=\left|f'(\xi)\right|\cdot\left|x_n-x_{n-1}\right| \\
        &<\frac{1}{2}\left|x_n-x_{n-1}\right| \\
        &\ldots \\
        &<\frac{1}{2^{n-1}}\left|x_2-x_1\right|
    \end{align*}
    由级数$\displaystyle \sum_{n=1}^{\infty}\frac{1}{2^{n-1}}$收敛,故原级数收敛 \\
    (2)由(1)级数的收敛有 
    $$
    \lim_{n\to\infty}S_n\exists \implies \lim_{n\to\infty}x_{n+1}=A+x_1=a
    $$
    故极限存在,有题设有$f(a) = a$记$g(x) = x - f(x)$ 有$g'(x)=1-f'(x) > 0$ 故$g(x)$单调递增,又$g(0)=-1<0$ 
    $$
    g(2) = 2 - f(2) = 1 - \left[f(2) - f(0)\right] = 1 - 2f'(\xi) > 0, \xi\in(0,2)
    $$
    由零点存在定理可知有且仅有唯一零点且$0<a<2$
    \end{solution}
    
    \item (2014,数一)设数列$\{a_n\}$,$\{b_n\}$满足$0<a_n<\frac{\pi}{2}$,$0<b_n<\frac{\pi}{2}$,$\cos a_n-a_n=\cos b_n$,且级数$\displaystyle \sum_{n=1}^{\infty} b_n$收敛。
    \begin{enumerate}[label=(\roman*)]
        \item[(1)] 证明$\displaystyle\lim_{n\rightarrow\infty} a_n=0$;
        \item[(2)] 证明级数$\displaystyle\sum_{n=1}^{\infty}\frac{a_n}{b_n}$收敛.
    \end{enumerate}
    
    \begin{solution}
    (1)由题设条件有
    $$
    \cos{b_n} > \cos{a_n} \implies 0 < a_n < b_n
    $$ 
    由于级数$\displaystyle\sum_{n=1}^{\infty}b_n$收敛,故$\displaystyle \lim_{n\to\infty}b_n=0$再由夹逼定理有 
    $$
    \lim_{n\to\infty}a_n=0
    $$
    (2)方法一:拉格朗日中值定理 
    \begin{align*}
        \frac{a_n}{b_n} &= \frac{\cos{a_n}-\cos{b_n}}{b_n}  \\
        &= \frac{-\sin\xi(a_n-b_n)}{b_n}, \xi\in(a_n,b_n) \\
        &= \frac{(b_n-a_n)\cdot\sin\xi}{a_n} < b_n - a_n < b_n
    \end{align*}
    方法二:等价代换 
    $$
    \frac{a_n}{b_n}=\frac{\cos{a_n}-\cos{b_n}}{b_n} < \frac{1-\cos{b_n}}{b_n} \sim \frac{1}{2}b_n
    $$
    级数$\displaystyle\sum_{n=1}^{\infty}b_n$收敛,故原级数收敛
    \end{solution}
\end{enumerate}

\section{傅里叶级数}
\begin{remark}
    傅里叶级数就两个考点 \\
    (一)求傅里叶级数的展开式(以$2l$为周期)
    $$
    f(x)\sim\frac{a_0}{2}+\sum_{n=1}^{\infty}\left(a_n\cos\frac{n\pi}{l}x+b_n\sin\frac{n\pi}{l}x\right)
    $$
    其中系数为
    $$
    \begin{cases}
        \displaystyle a_n = \frac{1}{l}\int_{-l}^{l}f(x)\cos\frac{n\pi}{l}x\d x, n = 0, 1, 2, \ldots \\
        \displaystyle b_n = \frac{1}{l}\int_{-l}^{l}f(x)\sin\frac{n\pi}{l}x\d x, n = 1, 2, 3, \ldots \\
    \end{cases}
    $$
    (二)狄利克雷收敛定理(充分条件)若函数在区间$[-1,1]$上满足 
    \begin{enumerate}
        \item [(1)] 连续,或只有有限个间断点,且都是第一类间断点 
        \item [(2)] 只有有限个极值点
    \end{enumerate}
    则$f(x)$在区间$(-l,l)$上的傅里叶级数收敛,且满足 
    $$
    f(x)\text{对应傅里叶级数}=
    \begin{cases}
        f(x), &x\text{为连续点} \\
        \frac{1}{2}\left[f(x+0)+f(x-0)\right], &x\text{为第一类间断点} \\
        \frac{1}{2}\left[f(-l+0)+f(l-0)\right], &x\text{为区间端点}
    \end{cases}
    $$
\end{remark}
\begin{enumerate}[label=\arabic*.,start=15]
    \item 设函数
    \begin{align*}
    f(x)=\begin{cases}
    e^x, & -\pi\leq x<0 \\
    1, & 0\leq x<\pi
    \end{cases}
    \end{align*}
    则其以$2\pi$为周期的傅里叶级数在$x=\pi$收敛于?,在$x=2\pi$收敛于?.
    \begin{solution}
    由狄利克雷收敛定理知,$f(x)$以$2\pi$为周期的傅里叶级数在$x=\pi$收敛于
    \begin{align*}
    S(\pi)=\frac{f(\pi-0)+f(-\pi+0)}{2}=\frac{1+e^{-\pi}}{2}
    \end{align*}
    在$x=2\pi$收敛于
    \begin{align*}
    S(2\pi)=S(0)=\frac{f(0-0)+f(0+0)}{2}=\frac{1+1}{2}=1
    \end{align*}
    \end{solution}
    
    \item 将$f(x)=1-x^2,0\leq x\leq\pi$,展开成余弦级数,并求级数$\displaystyle \sum_{n=1}^{\infty}\frac{(-1)^{n-1}}{n^2}$的和.
    
    \begin{solution}
    对$f(x)=1-x^2$进行偶延拓,由$f(x)=1-x^2$为偶函数,知$b_n=0$。
    \begin{align*}
    a_0&=\frac{2}{\pi}\int_0^\pi(1-x^2)dx=2\left(1-\frac{\pi^2}{3}\right) \\
    a_n&=\frac{2}{\pi}\int_0^\pi(1-x^2)\cos nx dx=\frac{4(-1)^{n+1}}{n^2} \quad (n=1,2,\cdots)
    \end{align*}
    \begin{align*}
    f(x)=1-x^2=\frac{a_0}{2}+\sum_{n=1}^{\infty}a_n\cos nx=1-\frac{\pi^2}{3}+\sum_{n=1}^{\infty}\frac{4(-1)^{n+1}}{n^2}\cos nx
    \end{align*}
    令$x=0$,代入上式,得
    \begin{align*}
    \sum_{n=1}^{\infty}\frac{(-1)^{n-1}}{n^2}=\frac{\pi^2}{12}
    \end{align*}
    \end{solution}
\end{enumerate}

\ifx\allfiles\undefined
\end{document}
\fi