\ifx\allfiles\undefined
\documentclass[12pt, a4paper, oneside, UTF8]{ctexbook}
\def\path{../../config}
\usepackage{amsthm}
\usepackage{amssymb}
\usepackage{array}
\usepackage{xcolor}
\usepackage{graphicx}
\usepackage{mathrsfs}
\usepackage{enumitem}
\usepackage{geometry}
\usepackage[colorlinks, linkcolor=black]{hyperref}
\usepackage{stackengine}
\usepackage{yhmath}
\usepackage{extarrows}
\usepackage{tikz}
\usepackage{forest}
\usetikzlibrary{decorations.pathreplacing, positioning}
% \usepackage{unicode-math}
\usepackage{esint}
\usepackage{pifont}
\usepackage{tcolorbox}
\tcbuselibrary{skins, breakable}

\usepackage{multicol} 
\usepackage{fontspec} % 使用字体

\setmainfont{Times New Roman}
\setCJKmainfont{LXGWWenKai-Light}[
    SlantedFont=*
]

\usepackage{listings} % 用于插入代码

% 定义代码高亮风格
\lstset{
    basicstyle=\ttfamily\small,        % 基本字体样式(等宽小字体)
    keywordstyle=\color{blue},         % 关键字颜色
    commentstyle=\color{green},        % 注释颜色
    stringstyle=\color{red},           % 字符串颜色
    numbers=none,
    breaklines=true,                   % 自动换行
    frame=single,                      % 代码框边框
    rulecolor=\color{black},           % 边框颜色
    captionpos=b,                      % 标题位置(底部)
    showspaces=false,                  % 不显示空格标记
    showstringspaces=false,            % 不显示字符串中的空格标记
    language=C                         % 设置语言为 C
}

\usepackage{fontawesome5}

\usepackage{amsmath}
\usepackage{booktabs, array}
\usepackage{makecell}
\usepackage{fancyhdr}
\usepackage[dvipsnames, svgnames]{xcolor}
\usepackage{listings}
\usepackage{tasks}[2020/01/11]

\everymath{\displaystyle}

\definecolor{mygreen}{rgb}{0,0.6,0}
\definecolor{mygray}{rgb}{0.5,0.5,0.5}
\definecolor{mymauve}{rgb}{0.58,0,0.82}
\definecolor{NavyBlue}{RGB}{0,0,128}
\definecolor{Rhodamine}{RGB}{255,0,255}
\definecolor{PineGreen}{RGB}{0,128,0}

\graphicspath{ {figures/},{../figures/}, {config/}, {../config/} }

\linespread{1.6}

\geometry{
    top=25.4mm, 
    bottom=25.4mm, 
    left=20mm, 
    right=20mm, 
    headheight=2.17cm, 
    headsep=4mm, 
    footskip=12mm
}

\setenumerate[1]{itemsep=5pt,partopsep=0pt,parsep=\parskip,topsep=5pt}
\setitemize[1]{itemsep=5pt,partopsep=0pt,parsep=\parskip,topsep=5pt}
\setdescription{itemsep=5pt,partopsep=0pt,parsep=\parskip,topsep=5pt}



% \begin{lstlisting}[language=TeX] ... \end{lstlisting}

% 定理环境设置
% ---------- 颜色 ----------
\definecolor{ExBlue}{HTML}{4F81BD}
\definecolor{SolGreen}{HTML}{77933C}
\definecolor{DefRed}{HTML}{C5504B}
\definecolor{ThmOrange}{HTML}{E97132}
\definecolor{RemGray}{HTML}{7F7F7F}
\definecolor{CorPurple}{HTML}{7030A0}
\definecolor{ForGray}{HTML}{595959}

% ---------- 通用“变色”模板 ----------
\tcbset{
    mybox/.style n args={1}{
        enhanced, breakable,
        arc=6pt,
        boxrule=0.6pt,
        left=8pt, right=8pt, top=6pt, bottom=6pt,
        drop shadow={black!25},
        fonttitle=\bfseries,
        coltitle=white,
        colbacktitle=#1!85,
        colback=#1!10,
        colframe=#1,
    }
}

% ---------- 各环境 ----------
% 例题
\newtcolorbox{example}[1][]{mybox={ExBlue}, title={\ifstrempty{#1}{Example}{#1}}}
% 解答
\newtcolorbox{solution}[1][]{mybox={SolGreen}, title={\ifstrempty{#1}{Solution}{#1}}}
% 定义
\newtcolorbox{definition}[1][]{mybox={DefRed}, title={\ifstrempty{#1}{Definition}{#1}}}
% 定理
\newtcolorbox{theorem}[1][]{mybox={ThmOrange}, title={\ifstrempty{#1}{Theorem}{#1}}}
% 标注
\newtcolorbox{remark}[1][]{mybox={RemGray}, title={\ifstrempty{#1}{Remark}{#1}}}
% 推论
\newtcolorbox{corollary}[1][]{mybox={CorPurple}, title={\ifstrempty{#1}{Corollary}{#1}}}
% 公式
\newtcolorbox{formula}[1][]{mybox={ForGray}, title={\ifstrempty{#1}{Formula}{#1}}}


\settasks{
    label-format = \bfseries,
    label        = \Alph*.,
    label-width  = 1.2em,
    label-offset = 0.3em,
    item-indent  = 1.9em,
    column-sep   = 0.5em
}

\newenvironment{choices}[1][4]   % 默认 4 栏
    {\begin{tasks}(#1)}
    {\end{tasks}}

% 自定义命令的文件

\def\d{\mathrm{d}}
\def\R{\mathbb{R}}
\def\P{\partial} 
\newcommand{\bs}[1]{\begin{solution}#1\end{solution}}
\newcommand{\bt}[1][1]{% 默认参数为1
    \ensuremath{% 确保数学模式
        \foreach \n in {1,...,#1} {\blacktriangle}% 循环输出 #1 个黑色三角形
    }%
}

\newcommand{\bl}[1][1]{% 默认参数为1
    \ensuremath{% 确保数学模式
        \foreach \n in {1,...,#1} {\blacklozenge}% 循环输出 #1 个黑色三角形
    }%
}
\newif\ifshowanswers
%\showanswerstrue % 注释掉这行就不显示答案

% 定义答案环境
\newcommand{\answer}[1]{%
    \ifshowanswers
        #1%
    \fi
}




% 修改参数改变封面样式,0 默认原始封面、内置其他1、2、3种封面样式
\def\myIndex{3}


\ifnum\myIndex>0
    \input{\path/cover_package_\myIndex} 
\fi

\def\myTitle{冲刺150笔记}
\def\myAuthor{Weary Bird}
\def\myDateCover{\today}
\def\myDateForeword{\today}
\def\myForeword{行香子}
\def\myForewordText{
树绕村庄,水满陂塘;倚东风、豪兴徜徉。小园几许,收尽春光。有桃花红,李花白,菜花黄。 \\
远远苔墙,隐隐茅堂;飏青旗、流水桥旁。偶然乘兴,步过东冈。正莺儿啼,燕儿舞,蝶儿忙。 \\
}
\def\mySubheading{知错能改善莫大焉}


\begin{document}
\input{../../config/cover}
\else
\fi

\chapter{多元函数积分学}
\section{三重积分的计算}

\begin{enumerate}[label=\arabic*.]
    \item 例1 (2013,数一)设直线$L$过$A(1,0,0)$,$B(0,1,1)$两点,将$L$绕$z$轴旋转一周得到曲面$\Sigma$,$\Sigma$与平面$z=0$,$z=2$所围成的立体为$\Omega$.
    \begin{enumerate}
        \item[(I)] 求曲面$\Sigma$的方程;
        \item[(II)] 求$\Omega$的形心坐标.
    \end{enumerate}
    
    \begin{solution}
    【详解】
    \end{solution}
    
    \item 例2 (2019,数一)设$\Omega$是由锥面$x^{2}+(y-z)^{2}=(1-z)^{2}(0\leq z\leq 1)$与平面$z=0$围成的锥体,求$\Omega$的形心坐标.
    
    \begin{solution}
    【详解】
    \end{solution}
\end{enumerate}

\section{第一类曲线积分的计算}

\begin{enumerate}[label=\arabic*.,start=3]
    \item 例3 (2018,数一)设$L$为球面$x^2+y^2+z^2=1$与平面$x+y+z=0$的交线,则$\oint_L xy ds=$
    
    \begin{solution}
    【详解】
    \end{solution}
    
    \item 例4 设连续函数$f(x,y)$满足$f(x,y)=(x+3y)^2+\int_L f(x,y) ds$,其中$L$为曲线$y=\sqrt{1-x^2}$,求曲线积分$\int_L f(x,y) ds$.
    
    \begin{solution}
    【详解】
    \end{solution}
\end{enumerate}

\section{第二类曲线积分的计算}

\begin{remark}[类型一 平面第二类曲线积分]
\end{remark}

\begin{enumerate}[label=\arabic*.,start=5]
    \item 例5 (2021,数一)设$D\subset \mathbb{R}^2$是有界单连通闭区域,$I(D)=\iint_D(4-x^2-y^2)dxdy$取得最大值的积分域记为$D_1$.
    \begin{enumerate}
        \item[(I)] 求$I(D_1)$的值;
        \item[(II)] 计算$\oint_{\partial D_1}\frac{(xe^{x^2+4y^2}+y)dx+(4ye^{x^2+4y^2}-x)dy}{x^2+4y^2}$,其中$\partial D_1$是$D_1$的正向边界.
    \end{enumerate}
    
    \begin{solution}
    【详解】
    \end{solution}
\end{enumerate}

\begin{remark}[类型二 空间第二类曲线积分]
\end{remark}

\begin{enumerate}[label=\arabic*.,start=6]
    \item 例6 (2011,数一)设$L$是柱面$x^2+y^2=1$与平面$z=x+y$的交线,从$z$轴正向往$z$轴负向看去为逆时针方向,则曲线积分$\oint_L xz dx+xdy+\frac{y^2}{2}dz=$
    
    \begin{solution}
    【详解】
    \end{solution}
\end{enumerate}

\section{第一类曲面积分的计算}

\begin{remark}[方法]
\end{remark}

\begin{enumerate}[label=\arabic*.,start=7]
    \item 例7 (2010,数一)设$P$为椭球面$S:x^2+y^2+z^2-yz=1$上的动点,若$S$在点$P$的切平面与$xOy$面垂直,求$P$点的轨迹$C$,并计算曲面积分
    \begin{align*}
    I=\iint_{\Sigma}\frac{(x+\sqrt{3})|y-2z|}{\sqrt{4+y^2+z^2-4yz}}dS,
    \end{align*}
    其中$\Sigma$是椭球面$S$位于曲线$C$上方的部分.
    
    \begin{solution}
    【详解】
    \end{solution}
\end{enumerate}

\section{第二类曲面积分的计算}

\begin{remark}[方法]
\end{remark}

\begin{enumerate}[label=\arabic*.,start=8]
    \item 例8 (2009,数一)计算曲面积分
    \begin{align*}
    I=\oint_{\Sigma}\frac{xdydz+ydzdx+zdxdy}{(x^2+y^2+z^2)^{\frac{3}{2}}},
    \end{align*}
    其中$\Sigma$是曲面$2x^2+2y^2+z^2=4$的外侧.
    
    \begin{solution}
    【详解】
    \end{solution}
    
    \item 例9 计算
    \begin{align*}
    \iint_{\Sigma}\frac{axdydz+(z+a)^2dxdy}{(x^2+y^2+z^2)^2},
    \end{align*}
    其中$\Sigma$为下半球面$z=-\sqrt{a^2-x^2-y^2}$的上侧,$a$为大于零的常数.
    
    \begin{solution}
    【详解】
    \end{solution}
    
    \item 例10 (2020,数一)设$\Sigma$为曲面$z=\sqrt{x^2+y^2}(1\leq x^2+y^2\leq 4)$的下侧,$f(x)$为连续函数,计算
    \begin{align*}
    I=\iint_{\Sigma}[xf(xy)+2x-y]dydz+[yf(xy)+2y+x]dzdx+[zf(xy)+z]dxdy.
    \end{align*}
    
    \begin{solution}
    【详解】
    \end{solution}
\end{enumerate}

\ifx\allfiles\undefined
\end{document}
\fi