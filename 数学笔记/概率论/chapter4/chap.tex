\ifx\allfiles\undefined
\documentclass[12pt, a4paper, oneside, UTF8]{ctexbook}
\def\path{../../config}
\usepackage{amsmath}
\usepackage{amsthm}
\usepackage{amssymb}
\usepackage{array}
\usepackage{xcolor}
\usepackage{graphicx}
\usepackage{mathrsfs}
\usepackage{enumitem}
\usepackage{geometry}
\usepackage[colorlinks, linkcolor=black]{hyperref}
\usepackage{stackengine}
\usepackage{yhmath}
\usepackage{extarrows}
\usepackage{tikz}
\usepackage{pgfplots}
\usepackage{asymptote}
\usepackage{float}
\usepackage{fontspec} % 使用字体

\setmainfont{Times New Roman}
\setCJKmainfont{LXGWWenKai-Light}[
    SlantedFont=*
]

\everymath{\displaystyle}

\usepgfplotslibrary{polar}
\usepackage{subcaption}
\usetikzlibrary{decorations.pathreplacing, positioning}

\usepgfplotslibrary{fillbetween}
\pgfplotsset{compat=1.18}
% \usepackage{unicode-math}
\usepackage{esint}
\usepackage[most]{tcolorbox}

\usepackage{fancyhdr}
\usepackage[dvipsnames, svgnames]{xcolor}
\usepackage{listings}

\definecolor{mygreen}{rgb}{0,0.6,0}
\definecolor{mygray}{rgb}{0.5,0.5,0.5}
\definecolor{mymauve}{rgb}{0.58,0,0.82}
\definecolor{NavyBlue}{RGB}{0,0,128}
\definecolor{Rhodamine}{RGB}{255,0,255}
\definecolor{PineGreen}{RGB}{0,128,0}

\graphicspath{ {figures/},{../figures/}, {config/}, {../config/} }

\linespread{1.6}

\geometry{
    top=25.4mm, 
    bottom=25.4mm, 
    left=20mm, 
    right=20mm, 
    headheight=2.17cm, 
    headsep=4mm, 
    footskip=12mm
}

\setenumerate[1]{itemsep=5pt,partopsep=0pt,parsep=\parskip,topsep=5pt}
\setitemize[1]{itemsep=5pt,partopsep=0pt,parsep=\parskip,topsep=5pt}
\setdescription{itemsep=5pt,partopsep=0pt,parsep=\parskip,topsep=5pt}

\lstset{
    language=Mathematica,
    basicstyle=\tt,
    breaklines=true,
    keywordstyle=\bfseries\color{NavyBlue}, 
    emphstyle=\bfseries\color{Rhodamine},
    commentstyle=\itshape\color{black!50!white}, 
    stringstyle=\bfseries\color{PineGreen!90!black},
    columns=flexible,
    numbers=left,
    numberstyle=\footnotesize,
    frame=tb,
    breakatwhitespace=false,
} 

\lstset{
    language=TeX, % 设置语言为 TeX
    basicstyle=\ttfamily, % 使用等宽字体
    breaklines=true, % 自动换行
    keywordstyle=\bfseries\color{NavyBlue}, % 关键字样式
    emphstyle=\bfseries\color{Rhodamine}, % 强调样式
    commentstyle=\itshape\color{black!50!white}, % 注释样式
    stringstyle=\bfseries\color{PineGreen!90!black}, % 字符串样式
    columns=flexible, % 列的灵活性
    numbers=left, % 行号在左侧
    numberstyle=\footnotesize, % 行号字体大小
    frame=tb, % 顶部和底部边框
    breakatwhitespace=false % 不在空白处断行
}

% \begin{lstlisting}[language=TeX] ... \end{lstlisting}

% 定理环境设置
\usepackage[strict]{changepage} 
\usepackage{framed}

\definecolor{greenshade}{rgb}{0.90,1,0.92}
\definecolor{redshade}{rgb}{1.00,0.88,0.88}
\definecolor{brownshade}{rgb}{0.99,0.95,0.9}
\definecolor{lilacshade}{rgb}{0.95,0.93,0.98}
\definecolor{orangeshade}{rgb}{1.00,0.88,0.82}
\definecolor{lightblueshade}{rgb}{0.8,0.92,1}
\definecolor{purple}{rgb}{0.81,0.85,1}

\theoremstyle{definition}
\newtheorem{myDefn}{\indent Definition}[section]
\newtheorem{myLemma}{\indent Lemma}[section]
\newtheorem{myThm}[myLemma]{\indent Theorem}
\newtheorem{myCorollary}[myLemma]{\indent Corollary}
\newtheorem{myCriterion}[myLemma]{\indent Criterion}
\newtheorem*{myRemark}{\indent Remark}
\newtheorem{myProposition}{\indent Proposition}[section]

\newenvironment{formal}[2][]{%
	\def\FrameCommand{%
		\hspace{1pt}%
		{\color{#1}\vrule width 2pt}%
		{\color{#2}\vrule width 4pt}%
		\colorbox{#2}%
	}%
	\MakeFramed{\advance\hsize-\width\FrameRestore}%
	\noindent\hspace{-4.55pt}%
	\begin{adjustwidth}{}{7pt}\vspace{2pt}\vspace{2pt}}{%
		\vspace{2pt}\end{adjustwidth}\endMakeFramed%
}

\newenvironment{definition}{\vspace{-\baselineskip * 2 / 3}%
	\begin{formal}[Green]{greenshade}\vspace{-\baselineskip * 4 / 5}\begin{myDefn}}
	{\end{myDefn}\end{formal}\vspace{-\baselineskip * 2 / 3}}

\newenvironment{theorem}{\vspace{-\baselineskip * 2 / 3}%
	\begin{formal}[LightSkyBlue]{lightblueshade}\vspace{-\baselineskip * 4 / 5}\begin{myThm}}%
	{\end{myThm}\end{formal}\vspace{-\baselineskip * 2 / 3}}

\newenvironment{lemma}{\vspace{-\baselineskip * 2 / 3}%
	\begin{formal}[Plum]{lilacshade}\vspace{-\baselineskip * 4 / 5}\begin{myLemma}}%
	{\end{myLemma}\end{formal}\vspace{-\baselineskip * 2 / 3}}

\newenvironment{corollary}{\vspace{-\baselineskip * 2 / 3}%
	\begin{formal}[BurlyWood]{brownshade}\vspace{-\baselineskip * 4 / 5}\begin{myCorollary}}%
	{\end{myCorollary}\end{formal}\vspace{-\baselineskip * 2 / 3}}

\newenvironment{criterion}{\vspace{-\baselineskip * 2 / 3}%
	\begin{formal}[DarkOrange]{orangeshade}\vspace{-\baselineskip * 4 / 5}\begin{myCriterion}}%
	{\end{myCriterion}\end{formal}\vspace{-\baselineskip * 2 / 3}}
	

\newenvironment{remark}{\vspace{-\baselineskip * 2 / 3}%
	\begin{formal}[LightCoral]{redshade}\vspace{-\baselineskip * 4 / 5}\begin{myRemark}}%
	{\end{myRemark}\end{formal}\vspace{-\baselineskip * 2 / 3}}

\newenvironment{proposition}{\vspace{-\baselineskip * 2 / 3}%
	\begin{formal}[RoyalPurple]{purple}\vspace{-\baselineskip * 4 / 5}\begin{myProposition}}%
	{\end{myProposition}\end{formal}\vspace{-\baselineskip * 2 / 3}}


\newtheorem{example}{\indent \color{SeaGreen}{Example}}[section]
\renewcommand{\proofname}{\indent\textbf{\textcolor{TealBlue}{Proof}}}
\NewEnviron{solution}{%
	\begin{proof}[\indent\textbf{\textcolor{TealBlue}{Solution}}]%
		\color{blue}% 设置内容为蓝色
		\BODY% 插入环境内容
		\color{black}% 恢复默认颜色(可选,避免影响后续文字)
	\end{proof}%
}

% 自定义命令的文件

\def\d{\mathrm{d}}
\def\R{\mathbb{R}}
%\newcommand{\bs}[1]{\boldsymbol{#1}}
%\newcommand{\ora}[1]{\overrightarrow{#1}}
\newcommand{\myspace}[1]{\par\vspace{#1\baselineskip}}
\newcommand{\xrowht}[2][0]{\addstackgap[.5\dimexpr#2\relax]{\vphantom{#1}}}
\newenvironment{mycases}[1][1]{\linespread{#1} \selectfont \begin{cases}}{\end{cases}}
\newenvironment{myvmatrix}[1][1]{\linespread{#1} \selectfont \begin{vmatrix}}{\end{vmatrix}}
\newcommand{\tabincell}[2]{\begin{tabular}{@{}#1@{}}#2\end{tabular}}
\newcommand{\pll}{\kern 0.56em/\kern -0.8em /\kern 0.56em}
\newcommand{\dive}[1][F]{\mathrm{div}\;\boldsymbol{#1}}
\newcommand{\rotn}[1][A]{\mathrm{rot}\;\boldsymbol{#1}}

\newif\ifshowanswers
\showanswerstrue % 注释掉这行就不显示答案

% 定义答案环境
\newcommand{\answer}[1]{%
    \ifshowanswers
        #1%
    \fi
}

% 修改参数改变封面样式,0 默认原始封面、内置其他1、2、3种封面样式
\def\myIndex{0}


\ifnum\myIndex>0
    \input{\path/cover_package_\myIndex} 
\fi

\def\myTitle{考研数学笔记}
\def\myAuthor{Weary Bird}
\def\myDateCover{\today}
\def\myDateForeword{\today}
\def\myForeword{相见欢·林花谢了春红}
\def\myForewordText{
    林花谢了春红,太匆匆。
    无奈朝来寒雨晚来风。
    胭脂泪,相留醉,几时重。
    自是人生长恨水长东。
}
\def\mySubheading{以姜晓千强化课讲义为底本}


\begin{document}
% \input{\path/cover_text_\myIndex.tex}

\newpage
\thispagestyle{empty}
\begin{center}
    \Huge\textbf{\myForeword}
\end{center}
\myForewordText
\begin{flushright}
    \begin{tabular}{c}
        \myDateForeword
    \end{tabular}
\end{flushright}

\newpage
\pagestyle{plain}
\setcounter{page}{1}
\pagenumbering{Roman}
\tableofcontents

\newpage
\pagenumbering{arabic}
% \setcounter{chapter}{-1}
\setcounter{page}{1}

\pagestyle{fancy}
\fancyfoot[C]{\thepage}
\renewcommand{\headrulewidth}{0.4pt}
\renewcommand{\footrulewidth}{0pt}








\else
\fi

\chapter{数字特征}

\section{期望与方差的计算}
\begin{remark}
    期望与方差 \\
    期望的定义
    \item[(1)]设随机变量X的概率分布为$P\{X=x_i\}=p_i,i=1,2,\ldots,$则$EX=\sum_{i}x_ip_i$ \\
    推广:若$Y=g(X)$则$EY=\sum_{i}g(x_i)p_i$ 
    \item[(2)]设随机变量X的概率密度为$f(x)$则$EX=\int_{-\infty}^{+\infty}f(x)\d x$ \\
    推广:若$Y=g(X)$则$EY=\int_{-\infty}^{+\infty}g(x)f(x)\d x$
    \item[(3)]设二维随机变量$(X,Y)$的联合概率分布为$P\{X=x_i,Y=y_j\}=p_{ij},i,j=1,2,\ldots$
    则$EZ=\sum_{i}\sum_{j}g(x_i,y_j)p_{ij}$
    \item[(4)]设二维随机变量$(X,Y)$的联合概率密度为$f(x,y),Z=g(X,Y)$则 \\
    $EZ=\int_{-\infty}^{+\infty}\int_{-\infty}^{+\infty}g(x,y)f(x,y)\d x\d y$ \\
    特别的 $EX=\int_{-\infty}^{+\infty}\int_{-\infty}^{+\infty}xf(x,y)\d x \d y$, 
    $EY=\int_{-\infty}^{+\infty}\int_{-\infty}^{+\infty}yf(x,y)\d x\d y$ \\
    期望的性质
    \item[(1)] $E(aX+bY+c)=aE(X)+bE(Y)+c$ 
    \item[(2)] $EXY=EX\cdot EY \iff $ X与Y不相关 \\
    特别的若X与Y相互独立,由$EXY=EXEY$ \\
    方差的定义
    \item[(1)] $DX=E(X-EX)^2=EX^2-(EX)^2$ \\
    方差的性质
    \item[(1)] $D(aX+c)=a^2DX$ 
    \item[(2)] $D(X\pm Y)=DX+DY\pm 2Cov(X,Y)$ \\
    推论 $D(X\pm Y)=D(X)+D(Y)\iff$ X与Y不相关 \\
    特别的,若X与Y独立,则有$D(X\pm Y)=D(X)+D(Y)$
    \item[(3)]若$X$与$Y$独立,则$DXY=DXDY+(EX)^2DY+(EY)^2DX$
\end{remark}
\begin{enumerate}[label=\arabic*.]
    \item 设随机变量$X$的概率密度为$f(x)=\frac{1}{\pi(1+x^2)}$,$-\infty<x<\infty$,则$E[\min\{|X|,1\}]=$\_\_\_\_.
    
    \begin{solution}
    \begin{align*}
        E\left[\min{(|X|, 1)}\right] &= \int_{-\infty}^{+\infty}\min{(|x|,1)}f(x)\d x \\
        &=2\int_{0}^{+\infty}\min{(|x|,1)}f(x)\d x\\
        &=2(\int_{0}^{1}xf(x)\d x + \int_{1}^{+\infty}f(x)\d x) \\
        &=\frac{1}{\pi}\ln{(1+x^2)}\mid^{1}_0+\frac{2}{\pi}\arctan{x}\mid^{+\infty}_{1} \\
        &=\frac{1}{\pi}\ln{2}+\frac{1}{2}
    \end{align*}
    \end{solution}
    
    \item (2016,数三)设随机变量$X$与$Y$相互独立,$X\sim N(1,2)$,$Y\sim N(1,4)$,则$D(XY)=$
    \begin{align*}
        (A)\ 6 \qquad (B)\ 8 \qquad (C)\ 14 \qquad (D)\ 15
    \end{align*}
    
    \begin{solution}
    \item[(方法一)通过计算方法做]
    \begin{align*}
        DXY &=E(XY)^2 - (EXY)^2 \\
        &=EX^2\cdot EY^2 - (EXEY)^2 \\
        &=[DX+(EX)^2][DY+(EY)^2]-(EXEY)^2 \\
        &=3\times 5 - 1 = 14
    \end{align*}
    \item[(方法二)用结论] 
    \begin{align*}
        DXY &= DXDY + (EX)^2DY + (EY)^2DX \\
        &= 8 + 4 + 2 = 14
    \end{align*}
    \end{solution}
    
    \item 设随机变量$X$与$Y$同分布,则$E\left(\frac{X^2}{X^2+Y^2}\right)=$\_\_\_\_
    
    \begin{solution}
    由轮换对称性有
    $$E\left(\frac{X^2}{X^2+Y^2}\right)=E\left(\frac{Y^2}{X^2+Y^2}\right)
    =\frac{1}{2}E\left(\frac{X^2+Y^2}{X^2+Y^2}\right)=\frac{1}{2}$$
    \end{solution}
    \begin{tcolorbox}[title=总结]
        若$X,Y$同分布,则$X,Y$具有相同的$F,f,E,D$,上题的推广结论 
        \[
        \text{若}X_1,X_2\ldots,X_n\text{同分布,则}E\left(\frac{X_1^2}{X_1^2+\ldots+X_n^2}\right)=\frac{1}{n}
        \]
    \end{tcolorbox}
    \item 设随机变量$X$与$Y$相互独立,$X\sim P(\lambda_1)$,$Y\sim P(\lambda_2)$,
    且$P\{X+Y>0\}=1-e^{-1}$,则$E(X+Y)^2=$\_\_\_\_.
    
    \begin{solution}
    利用参数可加性可知,$X+Y\sim P(\lambda_1+\lambda_2)$,由$P\{X+Y>0\}=1-e^{-1}=1-P\{X=0\}\implies \lambda_1+\lambda_2=1$,
    则$E(X+Y)^2=D(X+Y)+(E(X+Y))^2=1+1=2$
    \end{solution}
    
    \item 设随机变量$X$与$Y$相互独立,$X\sim E(\frac{1}{3})$,
    $Y\sim E\left(\frac{1}{6}\right)$, 若$U=\max\{X,Y\}$,
    \\$V=\min\{X,Y\}$,则$EU=$\_\_\_\_,$EV=$\_\_\_\_.
    
    \begin{solution}
    $EV$是比较好求的,由参数可加性有$V\sim E(\frac{1}{2})$ \\
    方法一 利用二维概率密度计算:\\
    由$X,Y$独立,知$f(x,y)=f_{X}(x)f_{Y}(y)$,则
    \[
    EU=\int_{-\infty}^{+\infty}\int_{-\infty}^{+\infty}\max{(x,y)}f(x,y)\d x\d y = \ldots = 7
    \]
    方法二 求$U$的概率密度:\\
    由$U=\max{(X,Y)}$知$F_{U}(u)=F_1F_2\implies f_{u}=f_1F_2+F_1f_2$ 
    \[
    EU=\int_{-\infty}^{+\infty}uf_{u}\d u=\ldots=7
    \]
    方法三 利用性质 \\
    \[
    E(U+V)=E(X+Y)=EX+EY=3+6=9
    \]
    $EV=2\implies EU=7$
    \end{solution}
    
    \begin{tcolorbox}[title=总结]
        若$U=\max\{X,Y\}$,$V=\min\{X,Y\}$,则$E(U+V)=E(X+Y),E(UV)=E(XY)$ \\
        独立同分布随机变量的最大值与最小值的分布函数,由如下结果\\
        令$Z=\max{(X_1,X_2,\ldots,X_n)}$,则
        \[
        F_Z{z}=F_{X_1}F_{X_2}\ldots F_{X_n}
        \]
        令$Z=\min{(X_1,X_2,\ldots,X_n)}$,则
        \[
        F_Z{z}=1-[(1-F_{(X_2)})][(1-F_{(X_2)})]\ldots[(1-F_{(X_n)})]
        \]
    \end{tcolorbox}
    \item (2017,数一)设随机变量$X$的分布函数为
    $F(x)=0.5\Phi(x)+0.5\Phi\left(\frac{x-4}{2}\right)$,
    其中$\Phi(x)$为标准正态分布函数,则$EX=$\_\_\_\_
    
    \begin{solution}
    \item [(方法一)]
    $f(x)=\frac{1}{2}\phi(x)+\frac{1}{2}\phi(\frac{x-4}{2})$,则$EX=\int_{-\infty}^{+\infty}f(x)\d x = 2$
    \item [(方法二)]
    考虑$F(X_1)=0.5\Phi(x),F(X_2)=0.5\Phi(\frac{x-4}{2})$,则由第二章的结论$aF_1+bF_2,(a,b > 0, a+ b = 1)$的时候
    也是分布函数,故$EX=\frac{1}{2}EX_1+\frac{1}{2}EX_2=0+\frac{4}{2}=2$
    \end{solution}
    
    \item 设随机变量$X\sim N(0,1)$,则$E|X|=$\_\_\_\_,$D|X|=$\_\_\_\_.
    
    \begin{solution}
    \begin{align*}
        E|X| &=\int_{-\infty}^{+\infty}|x|\phi(x)\d x \\
        &=2\int_{0}^{+\infty}x\phi(x)\d x \\
        &=\frac{-2}{\sqrt{2\pi}}\int_{0}^{+\infty}e^{-\frac{x^2}{2}}\d (-\frac{x^2}{2}) \\
        &=\sqrt{\frac{2}{\pi}}
    \end{align*}
    \begin{align*}
        D|X| &=E(|X|)^2-(E|X|)^2 \\
        &=EX^2-(E|X|)^2 \\
        &=DX+(EX)^2-(E|X|)^2 \\
        &=1-\frac{2}{\pi}
    \end{align*}
    \end{solution}
    
    \begin{tcolorbox}[title=总结]
        \begin{enumerate}
            \item [(1)] 若$X\sim N(0,1)$,则$E|X|=\sqrt{\frac{2}{\pi}},D|X|=1-\frac{2}{\pi}$
            \item [(2)] 若$X\sim N(0,\sigma^2)$,则$E|X|=\sqrt{\frac{2}{\pi}}\cdot\sigma,D|X|=(1-\frac{2}{\pi})\cdot\sigma^2$
            \item [(3)] 若$X\sim N(\mu,\sigma^2)$,则$E|X-\mu|=\sqrt{\frac{2}{\pi}}\cdot\sigma,D|X|=(1-\frac{2}{\pi})\cdot\sigma^2$
        \end{enumerate}
    \end{tcolorbox}
    \item 设随机变量$X$与$Y$相互独立,均服从$N(\mu,\sigma^2)$,求$E[\max\{X,Y\}]$,$E[\min\{X,Y\}]$.
    
    \begin{solution}
    由$X,Y$独立,有$X-Y\sim N(0,2\sigma^2)$,$E\left|X-Y\right|=\frac{2\sigma}{\sqrt{\pi}}$ \\
    由下述总结,可知所求期望为
    \[
    E\left[max\{X,Y\}\right]=\frac{1}{2}\left[E(X)+E(Y)+E\left|X-Y\right|\right] = \mu + \frac{\sigma}{\sqrt{\pi}} \\
    \]
    \[
    E\left[min\{X,Y\}\right]=\frac{1}{2}\left[E(X)+E(Y)-E\left|X-Y\right|\right] = \mu - \frac{\sigma}{\sqrt{\pi}} \\
    \]
    \end{solution}
    
    \begin{tcolorbox}[title=总结]
        关于最大值最小值函数的拆法 \\
        $max\{X,Y\}=\frac{X+Y+|X-Y|}{2}$ \\
        $min\{X,Y\}=\frac{X+Y-|X-Y|}{2}$
    \end{tcolorbox}
    \item 设独立重复的射击每次命中的概率为$p$,$X$表示第$n$次命中时的射击次数,求$EX$,$DX$.
    
    \begin{solution}
    Pascal分布(负二项分布),关键在于{\color{red}分解随机变量},设$X_i$表示第$i-1$次命中到$i$命中所需要的射击次数,则有
    $X_1,X_2,\ldots$之间相互独立,且$X_i\sim G(p)$,对于$X=X_1+X_2\ldots X_n$,故
    \[
    EX=EX_1+EX_2+\ldots+EX_n=\frac{n}{p}
    \]
    \[
    DX=DX_1+DX_2+\ldots+DX_n=\frac{n(1-p)}{p^2}
    \]
    \end{solution}
    
    \item  (2015,数一、三)设随机变量$X$的概率密度为$f(x)=\begin{cases}2^{-x}\ln2, & x>0 \\ 0, & x\leq0\end{cases}$,对$X$进行独立的观测,直到第2个大于3的观测值出现时停止,记$Y$为观测次数。
    \begin{enumerate}
        \item 求$Y$的概率分布;
        \item 求$EY$.
    \end{enumerate}

    \begin{solution}
    不妨令$p=P\{X>3\}=\int_{3}^{+\infty}2^{-x}\ln{2}\d x=\frac{1}{8}$ 
    \item [(1)]
    \begin{align*}
        P\{Y=k\} &= C_{k-1}^{1}p^2(1-p)^{k-2} \\
        &=(k-1)(\frac{1}{8})^2(\frac{7}{8})^{k-2},{\color{red}k=2,3,\ldots}
    \end{align*}
    \item [(2)]
    \begin{align*}
        EY &= \sum_{k=2}^{\infty}kP\{Y=k\} \\
        &=p^2\sum_{k=2}^{\infty}k(k-1)(1-p)^{k-2} \\
        &\xlongequal{\text{幂级数求和}}\ldots \\
        &=16
    \end{align*}
    也可以用Pascal分布的结论直接得出$EX=\frac{2}{\frac{1}{8}}=16$
    \end{solution}
\end{enumerate}

\section{协方差的计算}
\begin{remark}
    协方差 \\
    协方差的定义 $Cov(X,Y)=E\left[(X-EX)(Y-EY)\right]=E(XY)-EX\cdot EY$ \\
    协方差的性质
    \item[(1)] $Cov(X,Y)=Cov(Y,X),Cov(X,X)=DX$
    \item[(2)] $Cov(aX+bY+c,Z)=aCov(X,Z)+bCov(Y,Z)$ 
\end{remark}
\begin{enumerate}[label=\arabic*.,start=11]
    \item  设$X_1,X_2,\cdots,X_n$为来自总体$X$的简单随机样本。若$DX=4$,正整数$s\leq n$,$t\leq n$,则
    $$
    \text{Cov}\left(\frac{1}{s}\sum_{i=1}^s X_i,\frac{1}{t}\sum_{j=1}^t X_j\right)=
    $$
    \begin{align*}
        (A)\ 4\max\{s,t\} \qquad\quad (B)\ 4\min\{s,t\} \quad\qquad (C)\ \frac{4}{\max\{s,t\}} \quad\qquad (D)\ \frac{4}{\min\{s,t\}}
    \end{align*}

    \begin{solution}
    \begin{align*}
        \text{Cov}\left(\frac{1}{s}\sum_{i=1}^s X_i,\frac{1}{t}\sum_{j=1}^t X_j\right) 
        &= \frac{1}{st}[Cov(X_1,X_1)+Cov(X_1,X_2)+\ldots \\
        &+Cov(X_2,X_1)+\ldots+Cov(X_s,X_t)] \\
        &\xlongequal{Cov(X_i,X_i)=DX_i,Cov(X_i,X_j)=0}=\frac{\min{(s,t)}}{st}\cdot DX \\
        &=\frac{4}{\max{(s,t)}}
    \end{align*}
    来自总体$X$的简单随机样本必然是独立同分布的.
    \end{solution}
    
    \item  (2005,数三)设$X_1,X_2,\cdots,X_n(n>2)$为来自总体$N(0,\sigma^2)$的简单随机样本,样本均值为
    $\bar{X}$。记$Y_i=X_i-\bar{X}$,$i=1,2,\cdots,n$。
    \begin{enumerate}
        \item[(1)] 求$Y_i$的方差$DY_i$,$i=1,2,\cdots,n$;
        \item[(2)] 求$Y_1$与$Y_n$的协方差$Cov(Y_1,Y_n)$; 
        \item[(3)] 若$c(Y_1+Y_n)^2$为$\sigma^2$的无偏估计量,求常数$c$.
    \end{enumerate}
    
    \begin{solution}
    \item [(1)]
    方法一:
    \begin{align*}
        DY_i &=D(X_i-\bar{X}) \\
        &=DX_i+D\bar{X}-2Cov(X_i,\bar{X}) \\
        &\xlongequal{E\bar{X}=\mu,D\bar{X}=\sigma^2/n}\sigma^2+\frac{\sigma^2}{n} 
        -2Cov(X_i,\frac{1}{n}\sum_{i=1}^{n}X_i) \\
        &=\frac{n-1}{n}\sigma^2
    \end{align*}
    方法二:
    \begin{align*}
        DY_i &=D(\frac{n-1}{n}X_i-\frac{1}{n}\sum_{i=j}^{n}X_j(j\neq i)) \\
        &=(\frac{n-1}{n})^2\sigma^2-\frac{n-1}{n^2}\sigma^2 \\
        &=\frac{n-1}{n}\sigma^2
    \end{align*}
    \item [(2)]
    \begin{align*}
        Cov(Y_1,Y_n) &= Cov(X_1,\bar{X},X_n-\bar{X}) \\
        &=Cov(X_1,X_n)-Cov(X_1,\bar{X})-Cov(X_n-\bar{X})+D\bar{X} \\
        &=\frac{-\sigma^2}{n}
    \end{align*}
    \item [(3)]
    由无偏性有$cE(Y_1+Y_n)^2=\sigma^2\implies c=\frac{\sigma^2}{E(Y_1+Y_n)^2}$
    \begin{align*}
        E(Y_1+Y_n)^2 &= D(Y_1+Y_n)+(EY_1EY_n)^2 \\
        &=DY_1+DY_n+2Cov(Y_1,Y_n)+0 \\
        &=\frac{2(n-2)}{n}\sigma^2
    \end{align*}
    故$c=\frac{n}{2(n-2)}$
    \end{solution}
\end{enumerate}

\section{相关系数的计算}
\begin{remark}
    相关系数 \\
    相关系数的定义 $\rho_{XY}=\frac{Cov(X,Y)}{\sqrt{DX}\sqrt{DY}}=\frac{EXY-EXEY}{\sqrt{DX}\sqrt{DY}}$ \\
    相关系数的性质 
    \item[(1)] $\left|\rho_{XY}\right|\leq 1$
    \item[(2)] $\rho_{XY}=0\iff Cov(X,Y)=0 \iff EXY=EXEY \iff D(X+Y)=DX+DY$
    \item[(3)] $\rho_{XY}=1\iff P\{Y=aX+b\}=1(a>0);\rho_{XY}=-1\iff P\{Y=aX+b\}=1(a<0)$  
\end{remark}
\begin{enumerate}[label=\arabic*.,start=13]
    \item  (2016,数一)设试验有三个两两互不相容的结果$A_1,A_2,A_3$,且三个结果发生的概率均为$\frac{1}{3}$。将试验独立重复地做两次,$X$表示两次试验中$A_1$发生的次数,$Y$表示两次试验中$A_2$发生的次数,则$X$与$Y$的相关系数为
    \begin{align*}
        (A)\ -\frac{1}{2} \quad\qquad (B)\ -\frac{1}{3} \quad\qquad (C)\ \frac{1}{3} \quad\qquad (D)\ \frac{1}{2}
    \end{align*}
    
    \begin{solution}
    \item[(方法一)]
    由题意有$X,Y$均服从$B(2,\frac{1}{3})$,而$P\{XY=1\}=P{X=1,Y=1}=C_{2}^{1}(\frac{1}{3})^2$,且$P\{XY=0\}=\frac{7}{9}$,故
    XY的概率分布如下所示
    \[
    \begin{array}{c|cc}
        XY & 0 & 1\\
        \hline
        P & \frac{7}{9} & \frac{2}{9}
    \end{array}
    \]
    故$EXY=\frac{2}{9}$,进而可以求出$\rho_{XY}=\frac{-\frac{2}{9}}{\frac{4}{9}}=-\frac{1}{2}$ 
    \item [(方法二)] 设$Z$为"$A_3$在两次试验中发生的次数" 
    \\由题意有$Z\sim B(2, \frac{1}{3}),X+Y+Z=2$ 
    而$D(X+Y)=DX+DY+2Cov(X,Y)=\frac{8}{9}+2Cov(X,Y)$,其中$D(X+Y)=D(2-Z)=DZ=\frac{4}{9}$,故$Cov(X,Y)=\frac{-2}{9}$
    \item [(方法三)] 
    \begin{align*}
        Cov(X,X+Y+Z) &=DX+Cov(X,Y)+Cov(X,Z) \\
        &\xlongequal{\text{轮换对称性}}\frac{4}{9}+2Cov(X,Y) \\
        &=Cov(X,2) = 0 \implies Cov(X,Y)=-\frac{2}{9}
    \end{align*}
    \end{solution}
    
    \item  设随机变量$X\sim B\left(1,\frac{3}{4}\right)$,$Y\sim B\left(1,\frac{1}{2}\right)$,且$\rho_{XY}=\frac{\sqrt{3}}{3}$。
    \begin{enumerate}
        \item 求$(X,Y)$的联合概率分布;
        \item 求$P\{Y=1|X=1\}$.
    \end{enumerate}
    
    \begin{solution}
    这道题比较简单,直接给答案
    \[
    \begin{array}{c|c|c|c}
        X/Y & 0 & 1 & P_i \\
        \hline
        0 & \frac{1}{4} & 0 &\frac{1}{4} \\
        \hline
        1 & \frac{1}{4} & \frac{1}{2} & \frac{3}{4} \\
        \hline
        P_j& \frac{1}{2} & \frac{1}{2} & 1
    \end{array}
    \]
    $P\{Y=1|X=1\}=\frac{2}{3}$
    \end{solution}
\end{enumerate}

\section{相关与独立的判定}
\begin{remark}
    相关与独立性
    \item[(1)] 一般来说独立是强于不相关的条件,即$\text{独立}\implies\text{不相关}$ 
    \item[(2)] 对于二维正态分布有$\text{独立}\iff\text{不相关}$ 
    \item[(3)] 对于0-1分布有 $\text{独立}\iff\text{不相关}$
\end{remark}
\begin{remark}
    判断是否独立的基本方法
    \item [(1)] $P(AB)=P(A)P(B)$,对于离散型选点,对于连续型选区间 
    \item [(2)] 三个充要条件$\forall (x,y)\text{或}(i,j) F(x,y)=F_XF_Y,f(x,y)=f_Xf_Y,P(ij)=P_{\cdot i}P_{j\cdot}$
    \item [(3)] $\rho_{XY}\neq 0\implies X,Y$不独立
\end{remark}
\begin{enumerate}[label=\arabic*.,start=15]
    \item  设二维随机变量$(X,Y)$服从区域$D=\{(x,y)|x^2+y^2\leq a^2\}$上的均匀分布,则 \\
        (A)\ X与Y不相关,也不相互独立 \qquad
        (B)\ X与Y相互独立 \\
        (C)\ X与Y相关 \qquad\qquad
        (D)\ X与Y均服从$U(-a,a)$
    \begin{solution}
    这道题可以记结论,对于均匀分布若其区域不为$(a,b)\times(c,d)$的矩形,则必然不独立,其中$X\in(a,b),Y\in(c,d)$ \\
    正常来做的话,步骤如下
    \[f(x,y)=
    \begin{cases}
        \frac{1}{\pi a^2}, &(x,y)\in D \\
        0, & (x,y)\notin D 
    \end{cases}
    \]
    \[
    EX=\int_{-\infty}^{+\infty}\int_{-\infty}^{+\infty}xf(x,y)\d x\d y\xlongequal{\text{对称性}}0
    \]
    同理根据对称性可知$EXY=EX=EY=0$,故$X,Y$一定不相关,现在求$X,Y$的边缘分布概率密度,有
    \[
    f_{X}(x)=\int_{-\infty}^{+\infty}f(x,y)\d y=\begin{cases}
        \frac{2}{\pi a^2}\sqrt{a^2-x^2}, &x\in (-a, a) \\
        0, & x\notin (-a, a)
    \end{cases}
    \]
    同理可以求出
    \[
    f_{Y}(y)=\int_{-\infty}^{+\infty}f(x,y)\d x=\begin{cases}
        \frac{2}{\pi a^2}\sqrt{a^2-y^2}, &y\in (-a, a) \\
        0, & y\notin (-a, a)
    \end{cases}
    \]
    显然$f_Yf_X\neq f(x,y)$故$X,Y$不独立.
    \end{solution}
    
    \item  设随机变量$X$的概率密度为$f(x)=\frac{1}{2}e^{-|x|}$,$-\infty<x<+\infty$。
    \begin{enumerate}
        \item 求$X$的期望与方差;
        \item 求$X$与$|X|$的协方差,问$X$与$|X|$是否不相关?
        \item 问$X$与$|X|$是否相互独立?并说明理由.
    \end{enumerate}
    
    \begin{solution}
    \item [(1)]
    \begin{align*}
        &EX=\int_{-\infty}^{+\infty}xf(x)\d x = 0 \\
        &EX^2=\int_{-\infty}^{+\infty}x^2f(x)\d x=\int_{0}^{+\infty}x^2e^{-x}\d x = 2 \\
        &DX=EX^2-(EX)^2=2
    \end{align*} 
    \item [(2)]
    \begin{align*}
        E(X|X|)=\int_{-\infty}^{+\infty}|X|Xf(x)\d x = 0 = EXE|X| \implies \rho_{X|X|}=0, Cov(X,|X|)=0
    \end{align*}
    \item [(3)]
    设$A=\{0<X<1\},B=\{|X|<1\}$,故
    \[
    P(AB) = P\{0<X<1,|X|<1\} = P\{0<X<1\}=P(A)
    \]
    而$P(B)<1$是显然的,故$P(AB)\neq P(A)P(B)$,即$X|X|$不独立
    \end{solution}
\end{enumerate}
\ifx\allfiles\undefined
\end{document}
\fi