\ifx\allfiles\undefined
\documentclass[12pt, a4paper, oneside, UTF8]{ctexbook}
\def\path{../../config}
\usepackage{amsmath}
\usepackage{amsthm}
\usepackage{amssymb}
\usepackage{array}
\usepackage{xcolor}
\usepackage{graphicx}
\usepackage{mathrsfs}
\usepackage{enumitem}
\usepackage{geometry}
\usepackage[colorlinks, linkcolor=black]{hyperref}
\usepackage{stackengine}
\usepackage{yhmath}
\usepackage{extarrows}
\usepackage{tikz}
\usepackage{pgfplots}
\usepackage{asymptote}
\usepackage{float}
\usepackage{fontspec} % 使用字体

\setmainfont{Times New Roman}
\setCJKmainfont{LXGWWenKai-Light}[
    SlantedFont=*
]

\everymath{\displaystyle}

\usepgfplotslibrary{polar}
\usepackage{subcaption}
\usetikzlibrary{decorations.pathreplacing, positioning}

\usepgfplotslibrary{fillbetween}
\pgfplotsset{compat=1.18}
% \usepackage{unicode-math}
\usepackage{esint}
\usepackage[most]{tcolorbox}

\usepackage{fancyhdr}
\usepackage[dvipsnames, svgnames]{xcolor}
\usepackage{listings}

\definecolor{mygreen}{rgb}{0,0.6,0}
\definecolor{mygray}{rgb}{0.5,0.5,0.5}
\definecolor{mymauve}{rgb}{0.58,0,0.82}
\definecolor{NavyBlue}{RGB}{0,0,128}
\definecolor{Rhodamine}{RGB}{255,0,255}
\definecolor{PineGreen}{RGB}{0,128,0}

\graphicspath{ {figures/},{../figures/}, {config/}, {../config/} }

\linespread{1.6}

\geometry{
    top=25.4mm, 
    bottom=25.4mm, 
    left=20mm, 
    right=20mm, 
    headheight=2.17cm, 
    headsep=4mm, 
    footskip=12mm
}

\setenumerate[1]{itemsep=5pt,partopsep=0pt,parsep=\parskip,topsep=5pt}
\setitemize[1]{itemsep=5pt,partopsep=0pt,parsep=\parskip,topsep=5pt}
\setdescription{itemsep=5pt,partopsep=0pt,parsep=\parskip,topsep=5pt}

\lstset{
    language=Mathematica,
    basicstyle=\tt,
    breaklines=true,
    keywordstyle=\bfseries\color{NavyBlue}, 
    emphstyle=\bfseries\color{Rhodamine},
    commentstyle=\itshape\color{black!50!white}, 
    stringstyle=\bfseries\color{PineGreen!90!black},
    columns=flexible,
    numbers=left,
    numberstyle=\footnotesize,
    frame=tb,
    breakatwhitespace=false,
} 

\lstset{
    language=TeX, % 设置语言为 TeX
    basicstyle=\ttfamily, % 使用等宽字体
    breaklines=true, % 自动换行
    keywordstyle=\bfseries\color{NavyBlue}, % 关键字样式
    emphstyle=\bfseries\color{Rhodamine}, % 强调样式
    commentstyle=\itshape\color{black!50!white}, % 注释样式
    stringstyle=\bfseries\color{PineGreen!90!black}, % 字符串样式
    columns=flexible, % 列的灵活性
    numbers=left, % 行号在左侧
    numberstyle=\footnotesize, % 行号字体大小
    frame=tb, % 顶部和底部边框
    breakatwhitespace=false % 不在空白处断行
}

% \begin{lstlisting}[language=TeX] ... \end{lstlisting}

% 定理环境设置
\usepackage[strict]{changepage} 
\usepackage{framed}

\definecolor{greenshade}{rgb}{0.90,1,0.92}
\definecolor{redshade}{rgb}{1.00,0.88,0.88}
\definecolor{brownshade}{rgb}{0.99,0.95,0.9}
\definecolor{lilacshade}{rgb}{0.95,0.93,0.98}
\definecolor{orangeshade}{rgb}{1.00,0.88,0.82}
\definecolor{lightblueshade}{rgb}{0.8,0.92,1}
\definecolor{purple}{rgb}{0.81,0.85,1}

\theoremstyle{definition}
\newtheorem{myDefn}{\indent Definition}[section]
\newtheorem{myLemma}{\indent Lemma}[section]
\newtheorem{myThm}[myLemma]{\indent Theorem}
\newtheorem{myCorollary}[myLemma]{\indent Corollary}
\newtheorem{myCriterion}[myLemma]{\indent Criterion}
\newtheorem*{myRemark}{\indent Remark}
\newtheorem{myProposition}{\indent Proposition}[section]

\newenvironment{formal}[2][]{%
	\def\FrameCommand{%
		\hspace{1pt}%
		{\color{#1}\vrule width 2pt}%
		{\color{#2}\vrule width 4pt}%
		\colorbox{#2}%
	}%
	\MakeFramed{\advance\hsize-\width\FrameRestore}%
	\noindent\hspace{-4.55pt}%
	\begin{adjustwidth}{}{7pt}\vspace{2pt}\vspace{2pt}}{%
		\vspace{2pt}\end{adjustwidth}\endMakeFramed%
}

\newenvironment{definition}{\vspace{-\baselineskip * 2 / 3}%
	\begin{formal}[Green]{greenshade}\vspace{-\baselineskip * 4 / 5}\begin{myDefn}}
	{\end{myDefn}\end{formal}\vspace{-\baselineskip * 2 / 3}}

\newenvironment{theorem}{\vspace{-\baselineskip * 2 / 3}%
	\begin{formal}[LightSkyBlue]{lightblueshade}\vspace{-\baselineskip * 4 / 5}\begin{myThm}}%
	{\end{myThm}\end{formal}\vspace{-\baselineskip * 2 / 3}}

\newenvironment{lemma}{\vspace{-\baselineskip * 2 / 3}%
	\begin{formal}[Plum]{lilacshade}\vspace{-\baselineskip * 4 / 5}\begin{myLemma}}%
	{\end{myLemma}\end{formal}\vspace{-\baselineskip * 2 / 3}}

\newenvironment{corollary}{\vspace{-\baselineskip * 2 / 3}%
	\begin{formal}[BurlyWood]{brownshade}\vspace{-\baselineskip * 4 / 5}\begin{myCorollary}}%
	{\end{myCorollary}\end{formal}\vspace{-\baselineskip * 2 / 3}}

\newenvironment{criterion}{\vspace{-\baselineskip * 2 / 3}%
	\begin{formal}[DarkOrange]{orangeshade}\vspace{-\baselineskip * 4 / 5}\begin{myCriterion}}%
	{\end{myCriterion}\end{formal}\vspace{-\baselineskip * 2 / 3}}
	

\newenvironment{remark}{\vspace{-\baselineskip * 2 / 3}%
	\begin{formal}[LightCoral]{redshade}\vspace{-\baselineskip * 4 / 5}\begin{myRemark}}%
	{\end{myRemark}\end{formal}\vspace{-\baselineskip * 2 / 3}}

\newenvironment{proposition}{\vspace{-\baselineskip * 2 / 3}%
	\begin{formal}[RoyalPurple]{purple}\vspace{-\baselineskip * 4 / 5}\begin{myProposition}}%
	{\end{myProposition}\end{formal}\vspace{-\baselineskip * 2 / 3}}


\newtheorem{example}{\indent \color{SeaGreen}{Example}}[section]
\renewcommand{\proofname}{\indent\textbf{\textcolor{TealBlue}{Proof}}}
\NewEnviron{solution}{%
	\begin{proof}[\indent\textbf{\textcolor{TealBlue}{Solution}}]%
		\color{blue}% 设置内容为蓝色
		\BODY% 插入环境内容
		\color{black}% 恢复默认颜色(可选,避免影响后续文字)
	\end{proof}%
}

% 自定义命令的文件

\def\d{\mathrm{d}}
\def\R{\mathbb{R}}
%\newcommand{\bs}[1]{\boldsymbol{#1}}
%\newcommand{\ora}[1]{\overrightarrow{#1}}
\newcommand{\myspace}[1]{\par\vspace{#1\baselineskip}}
\newcommand{\xrowht}[2][0]{\addstackgap[.5\dimexpr#2\relax]{\vphantom{#1}}}
\newenvironment{mycases}[1][1]{\linespread{#1} \selectfont \begin{cases}}{\end{cases}}
\newenvironment{myvmatrix}[1][1]{\linespread{#1} \selectfont \begin{vmatrix}}{\end{vmatrix}}
\newcommand{\tabincell}[2]{\begin{tabular}{@{}#1@{}}#2\end{tabular}}
\newcommand{\pll}{\kern 0.56em/\kern -0.8em /\kern 0.56em}
\newcommand{\dive}[1][F]{\mathrm{div}\;\boldsymbol{#1}}
\newcommand{\rotn}[1][A]{\mathrm{rot}\;\boldsymbol{#1}}

\newif\ifshowanswers
\showanswerstrue % 注释掉这行就不显示答案

% 定义答案环境
\newcommand{\answer}[1]{%
    \ifshowanswers
        #1%
    \fi
}

% 修改参数改变封面样式,0 默认原始封面、内置其他1、2、3种封面样式
\def\myIndex{0}


\ifnum\myIndex>0
    \input{\path/cover_package_\myIndex} 
\fi

\def\myTitle{考研数学笔记}
\def\myAuthor{Weary Bird}
\def\myDateCover{\today}
\def\myDateForeword{\today}
\def\myForeword{相见欢·林花谢了春红}
\def\myForewordText{
    林花谢了春红,太匆匆。
    无奈朝来寒雨晚来风。
    胭脂泪,相留醉,几时重。
    自是人生长恨水长东。
}
\def\mySubheading{以姜晓千强化课讲义为底本}


\begin{document}
% \input{\path/cover_text_\myIndex.tex}

\newpage
\thispagestyle{empty}
\begin{center}
    \Huge\textbf{\myForeword}
\end{center}
\myForewordText
\begin{flushright}
    \begin{tabular}{c}
        \myDateForeword
    \end{tabular}
\end{flushright}

\newpage
\pagestyle{plain}
\setcounter{page}{1}
\pagenumbering{Roman}
\tableofcontents

\newpage
\pagenumbering{arabic}
% \setcounter{chapter}{-1}
\setcounter{page}{1}

\pagestyle{fancy}
\fancyfoot[C]{\thepage}
\renewcommand{\headrulewidth}{0.4pt}
\renewcommand{\footrulewidth}{0pt}








\else
\fi

\chapter{数字特征}

\section{期望与方差的计算}

\begin{enumerate}[label=\arabic*.]
    \item 设随机变量$X$的概率密度为$f(x)=\frac{1}{\pi(1+x^2)}$,$-\infty<x<\infty$,则$E[\min\{|X|,1\}]=$?.
    
    \begin{solution}
    【详解】
    \end{solution}
    
    \item (2016,数三)设随机变量$X$与$Y$相互独立,$X\sim N(1,2)$,$Y\sim N(1,4)$,则$D(XY)=$
    \begin{align*}
        (A)\ 6 \quad (B)\ 8 \quad (C)\ 14 \quad (D)\ 15
    \end{align*}
    
    \begin{solution}
    【详解】
    \end{solution}
    
    \item 设随机变量$X$与$Y$同分布,则$E\left(\frac{X+Y}{2}\right)=$?.
    
    \begin{solution}
    【详解】
    \end{solution}
    
    \item 设随机变量$X$与$Y$相互独立,$X\sim P(\lambda_1)$,$Y\sim P(\lambda_2)$,且$P\{X+Y>0\}=1-e^{-1}$,则$E(X+Y)^2=$?.
    
    \begin{solution}
    【详解】
    \end{solution}
    
    \item 设随机变量$X$与$Y$相互独立,$X\sim E(\lambda)$,$Y\sim E\left(\frac{1}{6}\right)$,若$U=\max\{X,Y\}$,$V=\min\{X,Y\}$,则$EU=$?,$EV=$?.
    
    \begin{solution}
    【详解】
    \end{solution}
    
    \item (2017,数一)设随机变量$X$的分布函数为$F(x)=0.5\Phi(x)+0.5\Phi\left(\frac{x-4}{2}\right)$,其中$\Phi(x)$为标准正态分布函数,则$EX=$?.
    
    \begin{solution}
    【详解】
    \end{solution}
    
    \item 设随机变量$X\sim N(0,1)$,则$E|X|=$?,$D|X|=$?.
    
    \begin{solution}
    【详解】
    \end{solution}
    
    \item 设随机变量$X$与$Y$相互独立,均服从$N(\mu,\sigma^2)$,求$E[\max\{X,Y\}]$,$E[\min\{X,Y\}]$.
    
    \begin{solution}
    【详解】
    \end{solution}
    
    \item 设独立重复的射击每次命中的概率为$p$,$X$表示第$n$次命中时的射击次数,求$EX$,$DX$.
    
    \begin{solution}
    【详解】
    \end{solution}
    
    \item  (2015,数一、三)设随机变量$X$的概率密度为$f(x)=\begin{cases}2^{-x}\ln2, & x>0 \\ 0, & x\leq0\end{cases}$,对$X$进行独立的观测,直到第2个大于3的观测值出现时停止,记$Y$为观测次数。
    \begin{enumerate}
        \item 求$Y$的概率分布;
        \item 求$EY$.
    \end{enumerate}
    
    \begin{solution}
    【详解】
    \end{solution}
\end{enumerate}

\section{协方差的计算}

\begin{enumerate}[label=\arabic*.,start=11]
    \item  设$X_1,X_2,\cdots,X_n$为来自总体$X$的简单随机样本。若$DX=4$,正整数$s\leq n$,$t\leq n$,则
    \begin{align*}
        \text{Cov}\left(\frac{1}{s}\sum_{i=1}^s X_i,\frac{1}{t}\sum_{j=1}^t X_j\right)=
    \end{align*}
    \begin{align*}
        (A)\ 4\max\{s,t\} \quad (B)\ 4\min\{s,t\} \quad (C)\ \frac{4}{\max\{s,t\}} \quad (D)\ \frac{4}{\min\{s,t\}}
    \end{align*}
    
    \begin{solution}
    【详解】
    \end{solution}
    
    \item  (2005,数三)设$X_1,X_2,\cdots,X_n(n>2)$为来自总体$N(0,\sigma^2)$的简单随机样本,样本均值为$\bar{X}$。记$Y_i=X_i-\bar{X}$,$i=1,2,\cdots,n$。
    \begin{enumerate}
        \item 求$Y_i$的方差$DY_i$,$i=1,2,\cdots,n$;
        \item 若$c(Y_1+Y_n)^2$为$\sigma^2$的无偏估计量,求常数$c$.
    \end{enumerate}
    
    \begin{solution}
    【详解】
    \end{solution}
\end{enumerate}

\section{相关系数的计算}

\begin{enumerate}[label=\arabic*.,start=13]
    \item  (2016,数一)设试验有三个两两互不相容的结果$A_1,A_2,A_3$,且三个结果发生的概率均为$\frac{1}{3}$。将试验独立重复地做两次,$X$表示两次试验中$A_1$发生的次数,$Y$表示两次试验中$A_2$发生的次数,则$X$与$Y$的相关系数为
    \begin{align*}
        (A)\ -\frac{1}{2} \quad (B)\ -\frac{1}{3} \quad (C)\ \frac{1}{3} \quad (D)\ \frac{1}{2}
    \end{align*}
    
    \begin{solution}
    【详解】
    \end{solution}
    
    \item  设随机变量$X\sim B\left(1,\frac{3}{4}\right)$,$Y\sim B\left(1,\frac{1}{2}\right)$,且$\rho_{XY}=\frac{\sqrt{3}}{3}$。
    \begin{enumerate}
        \item 求$(X,Y)$的联合概率分布;
        \item 求$P\{Y=1|X=1\}$.
    \end{enumerate}
    
    \begin{solution}
    【详解】
    \end{solution}
\end{enumerate}

\section{相关与独立的判定}

\begin{enumerate}[label=\arabic*.,start=15]
    \item  设二维随机变量$(X,Y)$服从区域$D=\{(x,y)|x^2+y^2\leq a^2\}$上的均匀分布,则
    \begin{align*}
        (A)\ X与Y不相关,也不相互独立 \\
        (B)\ X与Y相互独立 \\
        (C)\ X与Y相关 \\
        (D)\ X与Y均服从U(-a,a)
    \end{align*}
    
    \begin{solution}
    【详解】
    \end{solution}
    
    \item  设随机变量$X$的概率密度为$f(x)=\frac{1}{2}e^{-|x|}$,$-\infty<x<+\infty$。
    \begin{enumerate}
        \item 求$X$的期望与方差;
        \item 求$X$与$|X|$的协方差,问$X$与$|X|$是否不相关?
        \item 问$X$与$|X|$是否相互独立?并说明理由.
    \end{enumerate}
    
    \begin{solution}
    【详解】
    \end{solution}
\end{enumerate}
\ifx\allfiles\undefined
\end{document}
\fi