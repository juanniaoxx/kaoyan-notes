\ifx\allfiles\undefined
\documentclass[12pt, a4paper, oneside, UTF8]{ctexbook}
\def\path{../../config}
\usepackage{amsmath}
\usepackage{amsthm}
\usepackage{amssymb}
\usepackage{array}
\usepackage{xcolor}
\usepackage{graphicx}
\usepackage{mathrsfs}
\usepackage{enumitem}
\usepackage{geometry}
\usepackage[colorlinks, linkcolor=black]{hyperref}
\usepackage{stackengine}
\usepackage{yhmath}
\usepackage{extarrows}
\usepackage{tikz}
\usepackage{pgfplots}
\usepackage{asymptote}
\usepackage{float}
\usepackage{fontspec} % 使用字体

\setmainfont{Times New Roman}
\setCJKmainfont{LXGWWenKai-Light}[
    SlantedFont=*
]

\everymath{\displaystyle}

\usepgfplotslibrary{polar}
\usepackage{subcaption}
\usetikzlibrary{decorations.pathreplacing, positioning}

\usepgfplotslibrary{fillbetween}
\pgfplotsset{compat=1.18}
% \usepackage{unicode-math}
\usepackage{esint}
\usepackage[most]{tcolorbox}

\usepackage{fancyhdr}
\usepackage[dvipsnames, svgnames]{xcolor}
\usepackage{listings}

\definecolor{mygreen}{rgb}{0,0.6,0}
\definecolor{mygray}{rgb}{0.5,0.5,0.5}
\definecolor{mymauve}{rgb}{0.58,0,0.82}
\definecolor{NavyBlue}{RGB}{0,0,128}
\definecolor{Rhodamine}{RGB}{255,0,255}
\definecolor{PineGreen}{RGB}{0,128,0}

\graphicspath{ {figures/},{../figures/}, {config/}, {../config/} }

\linespread{1.6}

\geometry{
    top=25.4mm, 
    bottom=25.4mm, 
    left=20mm, 
    right=20mm, 
    headheight=2.17cm, 
    headsep=4mm, 
    footskip=12mm
}

\setenumerate[1]{itemsep=5pt,partopsep=0pt,parsep=\parskip,topsep=5pt}
\setitemize[1]{itemsep=5pt,partopsep=0pt,parsep=\parskip,topsep=5pt}
\setdescription{itemsep=5pt,partopsep=0pt,parsep=\parskip,topsep=5pt}

\lstset{
    language=Mathematica,
    basicstyle=\tt,
    breaklines=true,
    keywordstyle=\bfseries\color{NavyBlue}, 
    emphstyle=\bfseries\color{Rhodamine},
    commentstyle=\itshape\color{black!50!white}, 
    stringstyle=\bfseries\color{PineGreen!90!black},
    columns=flexible,
    numbers=left,
    numberstyle=\footnotesize,
    frame=tb,
    breakatwhitespace=false,
} 

\lstset{
    language=TeX, % 设置语言为 TeX
    basicstyle=\ttfamily, % 使用等宽字体
    breaklines=true, % 自动换行
    keywordstyle=\bfseries\color{NavyBlue}, % 关键字样式
    emphstyle=\bfseries\color{Rhodamine}, % 强调样式
    commentstyle=\itshape\color{black!50!white}, % 注释样式
    stringstyle=\bfseries\color{PineGreen!90!black}, % 字符串样式
    columns=flexible, % 列的灵活性
    numbers=left, % 行号在左侧
    numberstyle=\footnotesize, % 行号字体大小
    frame=tb, % 顶部和底部边框
    breakatwhitespace=false % 不在空白处断行
}

% \begin{lstlisting}[language=TeX] ... \end{lstlisting}

% 定理环境设置
\usepackage[strict]{changepage} 
\usepackage{framed}

\definecolor{greenshade}{rgb}{0.90,1,0.92}
\definecolor{redshade}{rgb}{1.00,0.88,0.88}
\definecolor{brownshade}{rgb}{0.99,0.95,0.9}
\definecolor{lilacshade}{rgb}{0.95,0.93,0.98}
\definecolor{orangeshade}{rgb}{1.00,0.88,0.82}
\definecolor{lightblueshade}{rgb}{0.8,0.92,1}
\definecolor{purple}{rgb}{0.81,0.85,1}

\theoremstyle{definition}
\newtheorem{myDefn}{\indent Definition}[section]
\newtheorem{myLemma}{\indent Lemma}[section]
\newtheorem{myThm}[myLemma]{\indent Theorem}
\newtheorem{myCorollary}[myLemma]{\indent Corollary}
\newtheorem{myCriterion}[myLemma]{\indent Criterion}
\newtheorem*{myRemark}{\indent Remark}
\newtheorem{myProposition}{\indent Proposition}[section]

\newenvironment{formal}[2][]{%
	\def\FrameCommand{%
		\hspace{1pt}%
		{\color{#1}\vrule width 2pt}%
		{\color{#2}\vrule width 4pt}%
		\colorbox{#2}%
	}%
	\MakeFramed{\advance\hsize-\width\FrameRestore}%
	\noindent\hspace{-4.55pt}%
	\begin{adjustwidth}{}{7pt}\vspace{2pt}\vspace{2pt}}{%
		\vspace{2pt}\end{adjustwidth}\endMakeFramed%
}

\newenvironment{definition}{\vspace{-\baselineskip * 2 / 3}%
	\begin{formal}[Green]{greenshade}\vspace{-\baselineskip * 4 / 5}\begin{myDefn}}
	{\end{myDefn}\end{formal}\vspace{-\baselineskip * 2 / 3}}

\newenvironment{theorem}{\vspace{-\baselineskip * 2 / 3}%
	\begin{formal}[LightSkyBlue]{lightblueshade}\vspace{-\baselineskip * 4 / 5}\begin{myThm}}%
	{\end{myThm}\end{formal}\vspace{-\baselineskip * 2 / 3}}

\newenvironment{lemma}{\vspace{-\baselineskip * 2 / 3}%
	\begin{formal}[Plum]{lilacshade}\vspace{-\baselineskip * 4 / 5}\begin{myLemma}}%
	{\end{myLemma}\end{formal}\vspace{-\baselineskip * 2 / 3}}

\newenvironment{corollary}{\vspace{-\baselineskip * 2 / 3}%
	\begin{formal}[BurlyWood]{brownshade}\vspace{-\baselineskip * 4 / 5}\begin{myCorollary}}%
	{\end{myCorollary}\end{formal}\vspace{-\baselineskip * 2 / 3}}

\newenvironment{criterion}{\vspace{-\baselineskip * 2 / 3}%
	\begin{formal}[DarkOrange]{orangeshade}\vspace{-\baselineskip * 4 / 5}\begin{myCriterion}}%
	{\end{myCriterion}\end{formal}\vspace{-\baselineskip * 2 / 3}}
	

\newenvironment{remark}{\vspace{-\baselineskip * 2 / 3}%
	\begin{formal}[LightCoral]{redshade}\vspace{-\baselineskip * 4 / 5}\begin{myRemark}}%
	{\end{myRemark}\end{formal}\vspace{-\baselineskip * 2 / 3}}

\newenvironment{proposition}{\vspace{-\baselineskip * 2 / 3}%
	\begin{formal}[RoyalPurple]{purple}\vspace{-\baselineskip * 4 / 5}\begin{myProposition}}%
	{\end{myProposition}\end{formal}\vspace{-\baselineskip * 2 / 3}}


\newtheorem{example}{\indent \color{SeaGreen}{Example}}[section]
\renewcommand{\proofname}{\indent\textbf{\textcolor{TealBlue}{Proof}}}
\NewEnviron{solution}{%
	\begin{proof}[\indent\textbf{\textcolor{TealBlue}{Solution}}]%
		\color{blue}% 设置内容为蓝色
		\BODY% 插入环境内容
		\color{black}% 恢复默认颜色(可选,避免影响后续文字)
	\end{proof}%
}

% 自定义命令的文件

\def\d{\mathrm{d}}
\def\R{\mathbb{R}}
%\newcommand{\bs}[1]{\boldsymbol{#1}}
%\newcommand{\ora}[1]{\overrightarrow{#1}}
\newcommand{\myspace}[1]{\par\vspace{#1\baselineskip}}
\newcommand{\xrowht}[2][0]{\addstackgap[.5\dimexpr#2\relax]{\vphantom{#1}}}
\newenvironment{mycases}[1][1]{\linespread{#1} \selectfont \begin{cases}}{\end{cases}}
\newenvironment{myvmatrix}[1][1]{\linespread{#1} \selectfont \begin{vmatrix}}{\end{vmatrix}}
\newcommand{\tabincell}[2]{\begin{tabular}{@{}#1@{}}#2\end{tabular}}
\newcommand{\pll}{\kern 0.56em/\kern -0.8em /\kern 0.56em}
\newcommand{\dive}[1][F]{\mathrm{div}\;\boldsymbol{#1}}
\newcommand{\rotn}[1][A]{\mathrm{rot}\;\boldsymbol{#1}}

\newif\ifshowanswers
\showanswerstrue % 注释掉这行就不显示答案

% 定义答案环境
\newcommand{\answer}[1]{%
    \ifshowanswers
        #1%
    \fi
}

% 修改参数改变封面样式,0 默认原始封面、内置其他1、2、3种封面样式
\def\myIndex{0}


\ifnum\myIndex>0
    \input{\path/cover_package_\myIndex} 
\fi

\def\myTitle{考研数学笔记}
\def\myAuthor{Weary Bird}
\def\myDateCover{\today}
\def\myDateForeword{\today}
\def\myForeword{相见欢·林花谢了春红}
\def\myForewordText{
    林花谢了春红,太匆匆。
    无奈朝来寒雨晚来风。
    胭脂泪,相留醉,几时重。
    自是人生长恨水长东。
}
\def\mySubheading{以姜晓千强化课讲义为底本}


\begin{document}
% \input{\path/cover_text_\myIndex.tex}

\newpage
\thispagestyle{empty}
\begin{center}
    \Huge\textbf{\myForeword}
\end{center}
\myForewordText
\begin{flushright}
    \begin{tabular}{c}
        \myDateForeword
    \end{tabular}
\end{flushright}

\newpage
\pagestyle{plain}
\setcounter{page}{1}
\pagenumbering{Roman}
\tableofcontents

\newpage
\pagenumbering{arabic}
% \setcounter{chapter}{-1}
\setcounter{page}{1}

\pagestyle{fancy}
\fancyfoot[C]{\thepage}
\renewcommand{\headrulewidth}{0.4pt}
\renewcommand{\footrulewidth}{0pt}








\else
\fi
\chapter{二维随机变量}

\section{联合分布函数的计算}
\begin{remark}
    (联合分布函数的性质)
    \item[(1)] $0\leq F(x,y)\leq 1,-\infty<x<+\infty, F(-\infty,y)=F(x,\infty)=F(-\infty,-\infty)=0,F(+\infty,+\infty)=1$
    \item[(2)] $F(x,y)$关于$x$和$y$均单调不减
    \item[(2)] $F(x,y)$关于$x$和$y$均右连续
    \item[(4)] $P\{a<X\leq b,c<Y\leq b\}=F(b,d)-F(b,c)-F(a,d)+F(a,c)$  
\end{remark}

\begin{enumerate}[label=\arabic*.]
    % 例题3.1
    \item 设随机变量$X$与$Y$相互独立,$X\sim B(1,p)$,$Y\sim E(\lambda)$,则$(X,Y)$的联合分布函数$F(x,y)=$\_\_\_.
    
    \begin{solution}
    \newpage
    \end{solution}
\end{enumerate}

\section{二维离散型随机变量分布的计算}

\begin{enumerate}[label=\arabic*.,start=2]
    % 例题3.2
    \item 设随机变量$X$与$Y$相互独立,均服从参数为$p$的几何分布。
    \begin{enumerate}
        \item 求在$X+Y=n(n\geq 2)$的条件下,$X$的条件概率分布;
        \item 求$P\{X+Y\geq n\}(n\geq 2)$.
    \end{enumerate}
    
    \begin{solution}
    \newpage
    \end{solution}
\end{enumerate}

\section{二维连续型随机变量分布的计算}
\begin{remark}
\begin{enumerate} 主要内容 \\
联合概率密度的性质
    \item [(1)] $f\left( {x,y}\right)  \geq  0, - \infty  < x <  + \infty , - \infty  < y <  + \infty$ ;
    \item [(2)] ${\int }_{-\infty }^{+\infty }{\int }_{-\infty }^{+\infty }f\left( {x,y}\right) {dxdy} = 1$ ;
    \item [(3)] $P\{ \left( {X,Y}\right)  \in  D\}  = {\iint }_{D}f\left( {x,y}\right) {dxdy}$ ;
    \item [(4)]在 $f\left( {x,y}\right)$ 的连续点处有 $\frac{{\partial }^{2}F\left( {x,y}\right) }{\partial x\partial y} = f\left( {x,y}\right)$ .\\
边缘概率密度
    \item[(1)](X,Y)关于 $X$ 的边缘概率密度 ${f}_{X}\left( x\right)  = {\int }_{-\infty }^{+\infty }f\left( {x,y}\right) {dy}$
    \item[(2)](X,Y)关于 $Y$ 的边缘概率密度 ${f}_{Y}\left( y\right)  = {\int }_{-\infty }^{+\infty }f\left( {x,y}\right) {dx}$\\
条件概率密度
    \item[(1)]在 $Y = y$ 的条件下, $X$ 的条件概率密度 ${f}_{X \mid  Y}\left( {x \mid  y}\right)  = \frac{f\left( {x,y}\right) }{{f}_{Y}\left( y\right) }$
    \item[(2)]在 $X = x$ 的条件下, $Y$ 的条件概率密度 ${f}_{Y \mid  X}\left( {y \mid  x}\right)  = \frac{f\left( {x,y}\right) }{{f}_{X}\left( x\right) }$
\end{enumerate}
\end{remark}
\begin{enumerate}[label=\arabic*.,start=3]
    % 例题2.3
    \item (2010,数一、三)设二维随机变量$(X,Y)$的概率密度为
    \begin{align*}
        f(x,y)=Ae^{-2x^2+2xy-y^2}, \quad -\infty<x<+\infty, -\infty<y<+\infty
    \end{align*}
    求常数$A$及条件概率密度$f_{Y|X}(y|x)$.
    
    \begin{solution}
    \newpage
    \end{solution}
    
    % 例题2.4
    \item 设随机变量$X\sim U(0,1)$,在$X=x(0<x<1)$的条件下,随机变量$Y\sim U(x,1)$。
    \begin{enumerate}
        \item 求$(X,Y)$的联合概率密度;
        \item 求$(X,Y)$关于$Y$的边缘概率密度$f_Y(y)$;
        \item 求$P\{X+Y>1\}$.
    \end{enumerate}
    
    \begin{solution}
    \newpage
    \end{solution}
    \section {关于二维正态分布}
    % 例题2.5
    \item 设二维随机变量$(X,Y)\sim N(1,2;1,4;-\frac{1}{2})$,且$P\{aX+bY\leq 1\}=\frac{1}{2}$,则$(a,b)$可以为
    \begin{align*}
        (A)\ \left(\frac{1}{2},-\frac{1}{4}\right) \quad (B)\ \left(\frac{1}{4},-\frac{1}{2}\right)\ 
        (C)\ \left(-\frac{1}{4},\frac{1}{2}\right) \quad (D)\ \left(\frac{1}{2},\frac{1}{4}\right)
    \end{align*}
    
    \begin{solution}
    \newpage
    \end{solution}
    
    % 例题2.6
    \item (2020,数三)设二维随机变量$(X,Y)\sim N(0,0;1,4;-\frac{1}{2})$,则下列随机变量服从标准正态分布且与$X$相互独立的是
    \begin{align*}
        (A)\ \frac{\sqrt{5}}{5}(X+Y) \quad (B)\ \frac{\sqrt{5}}{5}(X-Y) \ 
        (C)\ \frac{\sqrt{3}}{3}(X+Y) \quad (D)\ \frac{\sqrt{3}}{3}(X-Y)
    \end{align*}
    
    \begin{solution}
    \newpage
    \end{solution}
    
    \item (2022,数一)设随机变量$X\sim N(0,1)$,在$X=x$的条件下,随机变量$Y\sim N(x,1)$,则$X$与$Y$的相关系数为
    \begin{align*}
        (A)\ \frac{1}{4} \quad (B)\ \frac{1}{2} \quad (C)\ \frac{\sqrt{3}}{3} \quad (D)\ \frac{\sqrt{2}}{2}
    \end{align*}
    
    \begin{solution}
    \newpage
    \end{solution}
\end{enumerate}

\section{求二维离散型随机变量函数的分布}

\begin{enumerate}[label=\arabic*.,start=8]
    \item 设随机变量$X$与$Y$相互独立,$X\sim P(\lambda_1)$,$Y\sim P(\lambda_2)$,求$Z=X+Y$的概率分布.
    
    \begin{solution}
    \newpage
    \end{solution}
\end{enumerate}

\section{求二维连续型随机变量函数的分布}
\begin{remark}
\textbf{问题描述}\\
设二维随机变量(X,Y)的联合概率密度为 $f\left( {x,y}\right)$ ,
求 $Z = g\left( {X,Y}\right)$ 的概率密度 ${f}_{Z}\left( z\right)$ .\\
\textbf{分布函数法} \\
(1)设 $Z$ 的分布函数为 ${F}_{Z}\left( z\right)$ ,则 ${F}_{Z}\left( z\right)  = P\{ Z \leq  z\}  = P\{ g\left( {X,Y}\right)  \leq  z\}$ .\\
(2)求 $Z = g\left( {X,Y}\right)$ 在(X,Y)的正概率密度区域的值域 $\left( {\alpha ,\beta }\right)$ ,讨论 $z$ .\\
$z < \alpha$ 时, ${F}_{Z}\left( z\right)  = 0$ ; \\
当 $\alpha  \leq  z < \beta$ 时, ${F}_{Z}\left( z\right)  = {\iint }_{g\left( {x,y}\right)  \leq  z}f\left( {x,y}\right) {dxdy}$; \\
当 $z \geq  \beta$ 时, ${F}_{Z}\left( z\right)  = 1$ .\\
(3) $Z$ 的概率密度为 ${f}_{Z}\left( z\right)  = {F}_{Z}^{\prime }\left( z\right)$ . \\
\textbf{卷积公式}\\
(1)设 $Z = {aX} + {bY}$ ,则 
${f}_{Z}\left( z\right)  = {\int }_{-\infty }^{+\infty }\frac{1}{\left| b\right| }f\left( {x,\frac{z - {ax}}{b}}\right) {dx} = {\int }_{-\infty }^{+\infty }\frac{1}{\left| a\right| }f\left( {\frac{z - {by}}{a},y}\right) {dy}$; \\
(2)设 $Z = {XY}$ ,则 
${f}_{Z}\left( z\right)  = {\int }_{-\infty }^{+\infty }\frac{1}{\left| x\right| }f\left( {x,\frac{z}{x}}\right) {dx} = {\int }_{-\infty }^{+\infty }\frac{1}{\left| y\right| }f\left( {\frac{z}{y},y}\right) {dy}$ ; \\
(3)设 $Z = \frac{Y}{X}$ ,
则 ${f}_{Z}\left( z\right)={\int}_{-\infty}^{+\infty}\left|x\right|f\left({x,{xz}}\right){dx}$; \\
(4)设$Z=\frac{X}{Y}$,则${f}_{Z}\left(z\right)={\int}_{-\infty}^{+\infty}\left|y\right|f\left({{yz},y}\right){dy}$ 
\end{remark}
\begin{enumerate}[label=\arabic*.,start=9]
    % 例题3.9
    \item 设二维随机变量$(X,Y)$的联合概率密度为
        $f(x,y)=\begin{cases}
            1, & 0<x<1,0<y<2x \\
            0, & \text{其他}
        \end{cases}$
    求:
    \begin{enumerate}
        \item $(X,Y)$的联合分布函数$F(x,y)$;
        \item $(X,Y)$的边缘概率密度$f_X(x),f_Y(y)$;
        \item 条件概率密度$f_{X|Y}(x|y),f_{Y|X}(y|x)$;
        \item $P\left\{Y\leq \frac{1}{2}|X\leq \frac{1}{2}\right\}$,$P\left\{Y\leq \frac{1}{2}|X=\frac{1}{2}\right\}$;
        \item $Z=2X-Y$的概率密度$f_Z(z)$.
    \end{enumerate}
    
    \begin{solution}
    \newpage
    \end{solution}
\end{enumerate}

\section{求一离散一连续随机变量函数的分布}

\begin{enumerate}[label=\arabic*.,start=10]
    % 例题3.10
    \item  (2020,数一)设随机变量$X_1,X_2,X_3$相互独立,$X_1$与$X_2$均服从标准正态分布,$X_3$的概率分布为$P\{X_3=0\}=P\{X_3=1\}=\frac{1}{2}$,$Y=X_3X_1+(1-X_3)X_2$。
    \begin{enumerate}
        \item 求$(X_1,Y)$的联合分布函数(结果用标准正态分布函数$\Phi(x)$表示);
        \item 证明$Y$服从标准正态分布.
    \end{enumerate}
    
    \begin{solution}
    \newpage
    \end{solution}
\end{enumerate}

\ifx\allfiles\undefined
\end{document}
\fi