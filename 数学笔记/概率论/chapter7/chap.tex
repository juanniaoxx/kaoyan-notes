\ifx\allfiles\undefined
\documentclass[12pt, a4paper, oneside, UTF8]{ctexbook}
\def\path{../../config}
\usepackage{amsthm}
\usepackage{amssymb}
\usepackage{array}
\usepackage{xcolor}
\usepackage{graphicx}
\usepackage{mathrsfs}
\usepackage{enumitem}
\usepackage{geometry}
\usepackage[colorlinks, linkcolor=black]{hyperref}
\usepackage{stackengine}
\usepackage{yhmath}
\usepackage{extarrows}
\usepackage{tikz}
\usepackage{forest}
\usetikzlibrary{decorations.pathreplacing, positioning}
% \usepackage{unicode-math}
\usepackage{esint}
\usepackage{pifont}
\usepackage{tcolorbox}
\tcbuselibrary{skins, breakable}

\usepackage{multicol} 
\usepackage{fontspec} % 使用字体

\setmainfont{Times New Roman}
\setCJKmainfont{LXGWWenKai-Light}[
    SlantedFont=*
]

\usepackage{listings} % 用于插入代码

% 定义代码高亮风格
\lstset{
    basicstyle=\ttfamily\small,        % 基本字体样式(等宽小字体)
    keywordstyle=\color{blue},         % 关键字颜色
    commentstyle=\color{green},        % 注释颜色
    stringstyle=\color{red},           % 字符串颜色
    numbers=none,
    breaklines=true,                   % 自动换行
    frame=single,                      % 代码框边框
    rulecolor=\color{black},           % 边框颜色
    captionpos=b,                      % 标题位置(底部)
    showspaces=false,                  % 不显示空格标记
    showstringspaces=false,            % 不显示字符串中的空格标记
    language=C                         % 设置语言为 C
}

\usepackage{fontawesome5}

\usepackage{amsmath}
\usepackage{booktabs, array}
\usepackage{makecell}
\usepackage{fancyhdr}
\usepackage[dvipsnames, svgnames]{xcolor}
\usepackage{listings}
\usepackage{tasks}[2020/01/11]

\everymath{\displaystyle}

\definecolor{mygreen}{rgb}{0,0.6,0}
\definecolor{mygray}{rgb}{0.5,0.5,0.5}
\definecolor{mymauve}{rgb}{0.58,0,0.82}
\definecolor{NavyBlue}{RGB}{0,0,128}
\definecolor{Rhodamine}{RGB}{255,0,255}
\definecolor{PineGreen}{RGB}{0,128,0}

\graphicspath{ {figures/},{../figures/}, {config/}, {../config/} }

\linespread{1.6}

\geometry{
    top=25.4mm, 
    bottom=25.4mm, 
    left=20mm, 
    right=20mm, 
    headheight=2.17cm, 
    headsep=4mm, 
    footskip=12mm
}

\setenumerate[1]{itemsep=5pt,partopsep=0pt,parsep=\parskip,topsep=5pt}
\setitemize[1]{itemsep=5pt,partopsep=0pt,parsep=\parskip,topsep=5pt}
\setdescription{itemsep=5pt,partopsep=0pt,parsep=\parskip,topsep=5pt}



% \begin{lstlisting}[language=TeX] ... \end{lstlisting}

% 定理环境设置
% ---------- 颜色 ----------
\definecolor{ExBlue}{HTML}{4F81BD}
\definecolor{SolGreen}{HTML}{77933C}
\definecolor{DefRed}{HTML}{C5504B}
\definecolor{ThmOrange}{HTML}{E97132}
\definecolor{RemGray}{HTML}{7F7F7F}
\definecolor{CorPurple}{HTML}{7030A0}
\definecolor{ForGray}{HTML}{595959}

% ---------- 通用“变色”模板 ----------
\tcbset{
    mybox/.style n args={1}{
        enhanced, breakable,
        arc=6pt,
        boxrule=0.6pt,
        left=8pt, right=8pt, top=6pt, bottom=6pt,
        drop shadow={black!25},
        fonttitle=\bfseries,
        coltitle=white,
        colbacktitle=#1!85,
        colback=#1!10,
        colframe=#1,
    }
}

% ---------- 各环境 ----------
% 例题
\newtcolorbox{example}[1][]{mybox={ExBlue}, title={\ifstrempty{#1}{Example}{#1}}}
% 解答
\newtcolorbox{solution}[1][]{mybox={SolGreen}, title={\ifstrempty{#1}{Solution}{#1}}}
% 定义
\newtcolorbox{definition}[1][]{mybox={DefRed}, title={\ifstrempty{#1}{Definition}{#1}}}
% 定理
\newtcolorbox{theorem}[1][]{mybox={ThmOrange}, title={\ifstrempty{#1}{Theorem}{#1}}}
% 标注
\newtcolorbox{remark}[1][]{mybox={RemGray}, title={\ifstrempty{#1}{Remark}{#1}}}
% 推论
\newtcolorbox{corollary}[1][]{mybox={CorPurple}, title={\ifstrempty{#1}{Corollary}{#1}}}
% 公式
\newtcolorbox{formula}[1][]{mybox={ForGray}, title={\ifstrempty{#1}{Formula}{#1}}}


\settasks{
    label-format = \bfseries,
    label        = \Alph*.,
    label-width  = 1.2em,
    label-offset = 0.3em,
    item-indent  = 1.9em,
    column-sep   = 0.5em
}

\newenvironment{choices}[1][4]   % 默认 4 栏
    {\begin{tasks}(#1)}
    {\end{tasks}}

% 自定义命令的文件

\def\d{\mathrm{d}}
\def\R{\mathbb{R}}
\def\P{\partial} 
\newcommand{\bs}[1]{\begin{solution}#1\end{solution}}
\newcommand{\bt}[1][1]{% 默认参数为1
    \ensuremath{% 确保数学模式
        \foreach \n in {1,...,#1} {\blacktriangle}% 循环输出 #1 个黑色三角形
    }%
}

\newcommand{\bl}[1][1]{% 默认参数为1
    \ensuremath{% 确保数学模式
        \foreach \n in {1,...,#1} {\blacklozenge}% 循环输出 #1 个黑色三角形
    }%
}
\newif\ifshowanswers
%\showanswerstrue % 注释掉这行就不显示答案

% 定义答案环境
\newcommand{\answer}[1]{%
    \ifshowanswers
        #1%
    \fi
}




% 修改参数改变封面样式,0 默认原始封面、内置其他1、2、3种封面样式
\def\myIndex{3}


\ifnum\myIndex>0
    \input{\path/cover_package_\myIndex} 
\fi

\def\myTitle{冲刺150笔记}
\def\myAuthor{Weary Bird}
\def\myDateCover{\today}
\def\myDateForeword{\today}
\def\myForeword{行香子}
\def\myForewordText{
树绕村庄,水满陂塘;倚东风、豪兴徜徉。小园几许,收尽春光。有桃花红,李花白,菜花黄。 \\
远远苔墙,隐隐茅堂;飏青旗、流水桥旁。偶然乘兴,步过东冈。正莺儿啼,燕儿舞,蝶儿忙。 \\
}
\def\mySubheading{知错能改善莫大焉}


\begin{document}
% \input{../config/cover}
\else
\fi

\chapter{参数估计}

\section{求矩估计与最大似然估计}
\begin{remark}
    矩估计与最大似然估计 \\
    矩估计\\
    令$EX^k=\frac{1}{n}\sum_{i=1}^{n}X_i^k$或者$E(X-EX)^k=\frac{1}{n}\sum_{i=1}^{n}(X_i-\bar{X})^k,k=1,2,\ldots$
    得到$\theta_1,\theta_2\ldots$的矩估计量 \\
    \[
    \begin{cases}
        EX=\bar{X}, &\text{一个参数} \\
        EX^2=\frac{1}{n}\sum_{i=1}^{n}X_i^2 & \text{两个参数}
    \end{cases}
    \]
    最大似然估计
    \begin{enumerate}
        \item[(1)]对样本点$x_1,x_2\ldots,x_n$,似然函数为$L(\theta)\begin{cases}
        \prod_{i=1}^{n}p(x_i;\theta) \\
        \prod_{i=1}^{n}f(x_i;\theta) 
    \end{cases}$
    \item[(2)]似然函数两端取对数求导 
    \item[(3)]令$\frac{\d\ln{L(\theta)}}{\d\theta}=0$就可以得到$\theta$的最大似然估计值
    \end{enumerate}
    一个关于规范的小提示,如果问估计值用小写字母(样本值),问估计量用大写字母(随机变量)
\end{remark}
\begin{enumerate}[label=\arabic*.]
    \item (2002,数一)设总体$X$的概率分布为
    \begin{align*}
        \begin{array}{c|cccc}
        X & 0 & 1 & 2 &3\\
        \hline
        P & \theta^2 & 2\theta(1-\theta) &\theta^2& 1-2\theta
        \end{array}
    \end{align*}
    其中$0<\theta<\frac{1}{2}$为未知参数,利用总体$X$的如下样本值$3,1,3,0,3,1,2,3$,求$\theta$的矩估计值与最大似然估计值。
    
    \begin{solution}
    \item [(矩估计)] 这道题只有一个参数,只需要用一阶矩估计$EX=2\theta(1-\theta)+2\theta^2+3-6\theta=\bar{X}$,其中
    $\bar{X}=\frac{16}{8}=2$,故$\theta$的矩估计值$\hat{\theta}=\frac{1}{4}$
    \item[(最大似然估计)]对于样本$3,1,3,0,3,1,2,3$,似然估计函数为
    \[
    L(\theta)=4\theta^{6}(1-\theta)^2(1-2\theta)^4
    \]
    令$\frac{\d\ln{\theta}}{\d\theta}=0$有$\theta=\frac{7+\sqrt{13}}{12}$又$0<\theta<\frac{1}{2}$,故最终$\theta=\frac{7-\sqrt{13}}{12}$
    \end{solution}
    
    \item (2011,数一)设$X_1,X_2,\cdots,X_n$为来自正态总体$N(\mu,\sigma^2)$的简单随机样本,其中$\mu$已知,$\sigma^2>0$未知,样本均值为$\bar{X}$,样本方差为$S^2$。
    \begin{enumerate}
        \item[(1)] 求$\sigma^2$的最大似然估计量$\hat{\sigma}^2$;
        \item[(2)] 求$E(\hat{\sigma}^2)$与$D(\hat{\sigma}^2)$。
    \end{enumerate}
    
    \begin{solution}
    \item[(1)] 对于样本$X_1,\ldots,X_n$其最大似然函数为 
    \[
    L(\sigma^2)=\prod_{i=1}^{n}\frac{1}{\sqrt{2\pi}\sigma}e^{-\frac{(x_i-\mu)^2}{2\sigma^2}} 
    \]
    注意参数为$\sigma^2$,令$\frac{\d\ln{\sigma^2}}{\d\sigma^2}=0$,有$\hat{\sigma^2}=\frac{1}{n}\sum_{i=1}{n}(X_i-\mu)^2$
    \item[(2)] 这种题优先考虑$\chi^2$分布的期望与方差结论,有题(1)有
    \[
    \frac{X_i-\mu}{\sigma}\sim N(0,1) \implies \sum_{i=1}^{n}\left(\frac{X_i-\mu}{\sigma}\right)^2\sim \chi^2(n)
    \]
    故$E(\hat{\sigma}^2)=\sigma^2,D(\hat{\sigma}^2)=\frac{2\sigma^4}{n}$
    \end{solution}
    
    \item (2022,数一、三)设$X_1,X_2,\cdots,X_n$为来自期望为$\theta$的指数分布总体的简单随机样本,$Y_1,Y_2,\cdots,Y_m$为来自期望为$2\theta$的指数分布总体的简单随机样本,两个样本相互独立。利用$X_1,X_2,\cdots,X_n$与$Y_1,Y_2,\cdots,Y_m$,
    \begin{enumerate}
        \item[(1)] 求$\theta$的最大似然估计量$\hat{\theta}$;
        \item[(2)] 求$D(\hat{\theta})$。
    \end{enumerate}
    
    \begin{solution}
    这是双总体,但基本上和单总体一致,不要被唬住了哦!
    \item [(1)]
    由题有$X\sim E(\frac{1}{\theta}),Y\sim E(\frac{1}{2\theta})$,故其概率密度分别为
    \[
    f_{X}(x)=\begin{cases}
        \frac{1}{\theta}e^{-\frac{x}{\theta}}, &x > 0\\
        0, & x\leq 0
    \end{cases}\qquad f_{Y}(y)=\begin{cases}
        \frac{1}{2\theta}e^{-\frac{y}{2\theta}}, &y > 0\\
        0, & x\leq 0
    \end{cases}
    \]
    则对于样本$X_1,X_2,\ldots,X_n$与$Y_1,Y_2,\ldots,Y_n$,最大似然估计函数为
    \[
        L(\theta)= (\frac{1}{2})^m\theta^{-(m+n)}e^{-\frac{1}{\theta}(\sum_{i=1}^{n}X_i+\frac{1}{2}\sum_{j=1}^{m}Y_j)}
    \]
    则令$\frac{\d\ln{\theta}}{\d\theta}=0$,有$\hat{\theta}=\frac{1}{n+m}(\sum_{i=1}^{n}X_i+\frac{1}{2}\sum_{j=1}^{m}Y_j)$
    \item [(2)]
    \begin{align*}
        D(\hat{\theta}) &=(\frac{1}{m+n})^2D(\sum_{i=1}^{n}X_i+\frac{1}{2}\sum_{j=1}^{m}Y_j) \\
        &=\frac{\theta^2}{m+n}
    \end{align*}
    \end{solution}
\end{enumerate}

\section{估计量的评价标准}
\begin{remark}
    估计量的评价标准
    \begin{enumerate}
    \item[(1)] (无偏性)设$\hat{\theta}$为$\theta$的估计量,若$E\hat{\theta}=\theta$则称其为$\theta$无偏估计量
    \item[(2)] (有效性)设$\hat{\theta_1},\hat{\theta_2}$为$\theta$的无偏估计,若$D(\hat{\theta_1})<D(\hat{\theta_2})$则称$\hat{\theta_1}$
    比$\hat{\theta_2}$更有效 
    \item[(3)] 设$\hat{\theta}$为$\theta$的估计量,若$\hat{\theta}$依概率收敛于$\theta$,则称$\hat{\theta}$为$\theta$
    \end{enumerate}
    一致(相合)估计量 \\
    一致性的考点在于---$\frac{1}{n}\sum\fbox{}\xrightarrow{P}E\fbox{}$
\end{remark}
\begin{enumerate}[label=\arabic*.,start=4]
    \item 设总体$X$的概率密度为
    \begin{align*}
        f(x)=\begin{cases}
            2e^{-2(x-\theta)}, & x>\theta \\
            0, & x\leq\theta
        \end{cases}
    \end{align*}
    其中$\theta>0$为未知参数,$X_1,X_2,\cdots,X_n$为来自总体$X$的简单随机样本。
    \begin{enumerate}
        \item[(1)] 求$\theta$的最大似然估计量$\hat{\theta}$;
        \item[(2)] 问$\hat{\theta}$是否为$\theta$的无偏估计量?并说明理由。
    \end{enumerate}
    
    \begin{solution}
    \item[(1)] 对于样本$X_1,X_2,\ldots,X_n$的最大似然估计函数为 
    \[
        L(\theta)=\prod_{i=1}^{n}2e^{-2(x_i-\theta)}=2^ne^{-\sum_{i=1}^{n}(x_i-\theta)}
    \]
    显然$L(\theta)$关于$\theta$是单调递增的,则根据最大似然的定义,应该取使得$L(\theta)$最大的值,而由题目有$X_1>\theta,X_2>\theta,
    \ldots$,故$\hat{\theta}=\min{\{X_1,X_2\ldots,X_n\}}$
    \item [(2)] 由概率密度函数有$F_X(x)=\int_{-\infty}^{x}f(t)dt$,故
    \[
    F_X(x)=\int_{-\infty}^{x}f(t)dt=\begin{cases}
        1-e^{-2(x-\theta)}, & x > \theta \\
        0, & x\leq\theta
    \end{cases}
    \]
    故$F_{min}=1-\left[1-F_X(x)\right]^{n}$即 
    \[
    F_{min}=\begin{cases}
        1-e^{-2n(x-\theta)}, & x > \theta \\
        0, x\leq\theta
    \end{cases}
    \]
    故
    \[
    f_{min}=\begin{cases}
        2ne^{-2n(x-\theta)}, &x > \theta\\
        0, & x\leq\theta
    \end{cases}
    \]
    由期望的定义有
    \[
    E\hat{\theta}=\int_{\theta}^{+\infty}2nxe^{-2n(x-\theta)}=\theta+\frac{1}{2n}
    \]
    \end{solution}

    \item (2010,数一)设总体$X$的概率分布为 
    \[
    \begin{array}{|c|c|c|c|}
        \hline
        X & 1 & 2 & 3 \\
        \hline
        P & 1-\theta & \theta - \theta^2 & \theta^2 \\
        \hline
    \end{array}
    \]
    其中参数$\theta\in(0,1)$未知,$N_i$表示来自总体$X$的简单随机样本(样本容量为n)中等于$i$的个数$(i=1,2,3)$
    求常数$a_1,a_2,a_3$使得$T=\sum_{i=1}^{3}a_iN_i$为$\theta$的无偏估计量,并求T的方差.
    \begin{solution}
        由题可知$N_i\sim B(n,p)$,具体来说有
        \[
        \begin{cases}
            N_1\sim B(n, 1-\theta) \\
            N_2\sim B(n, \theta-\theta^2)\\
            N_3\sim B(n, \theta^2)
        \end{cases}
        \]且有$N_1+N_2+N_3=n$ \\
        故$ET=\sum_{i=1}^{3}a_iEN_i=n\left[a_1+(a_2-a_1)\theta+(a_3-a_2)\theta^2\right]=\theta$,只需要令
        \[
        \begin{cases}
            a_1 = 0\\
            a_2 = \frac{1}{n} \\
            a_3 = \frac{1}{n}
        \end{cases}
        \]
        求$DT=\frac{1}{n^2}D(n-N_1)=\frac{1}{n^2}DN_1=\frac{\theta(1-\theta)}{n}$
    \end{solution}
\end{enumerate}
\section{区间估计与假设检验}
\begin{remark}[区间估计与假设检验]
    这一节内容很少,只需要掌握置信度的概念,假设检验的基本过程与第一类错误/第二类错误的概念即可
    \begin{enumerate}
        \item 置信度与置信区间 \\
        设总体X的分布函数$F(x,\theta)$含有一个\textbf{未知参数}$\theta,\theta\in\Theta$其中$\Theta$是其所有
        可能取值的集合,对于给定值$0<\alpha<1$,若由来自总体X的样本$X_1,X_2,\ldots,X_n$确定了两个统计量$\theta_1,\theta_2,\theta_1\leq\theta_2$
        对于$\forall\theta\in\Theta$都有
        $$
        P\{\hat{\theta_1} < \theta < \hat{\theta_2}\} \geq 1 - \alpha
        $$
        则称区间$(\theta_1,\theta_2)$为$\theta$置信水平为$1-\alpha$的置信区间,$\hat{\theta_1},\hat{\theta_2}$分别称
        置信水平为$1-\alpha$的双侧置信区间的置信下限和置信上限,$1-\alpha$称为置信水平或置信度
        \item 原假设$H_0$与备择假设$H_1$
        \begin{center}
        \begin{tabular}{|c|c|c|}
            \hline
            类型 & $H_0$ & $H_1$ \\
            \hline
            双边检验& $\theta=\theta_0$ & $\theta\neq\theta_0$ \\
            \hline
            单边检验-左边& $\theta\geq\theta_0$ & $\theta < \theta_0$ \\
            \hline
            单边检验-右边& $\theta\leq\theta_0$ & $\theta > \theta_0$ \\
            \hline
        \end{tabular}
        \end{center}

        \vspace{\baselineskip}

        \item 假设检验的过程 
        \begin{enumerate}
            \item [(1)] 根据题意写出原假设$H_0$和备择假设$H_1$
            \item [(2)] 选择检验方式,写出检验统计量及其分布
            \item [(3)] 根据给定的显著性水平确定拒绝域
            \item [(4)] 统计检验统计量的值,做出推断
        \end{enumerate}

        \item 第一类错误/第二类错误
        \begin{center}
            \begin{tabular}{|p{2.5cm}|p{6cm}|p{5cm}|}
                \hline
                类型 & 含义 & 犯错的概率  \\
                \hline
                第一类错误 & 原假设$H_0$为真,但却拒绝$H_0$,即弃真概率 & 
                $\alpha=P\{\text{拒绝}H_0\mid H_0\text{为真}\}$ \\
                \hline
                第二类错误 & 原假设$H_0$为假,但却接受$H_0$,即取伪概率 &
                $\beta=p\{\text{接受}H_0\mid H_0\text{不真}\}$  \\
                \hline
            \end{tabular}
        \end{center}
        \begin{enumerate}
            \item [(1)]仅控制犯第一类错误的检验称为显著检验,$\alpha$为显著性水平
            \item [(2)] 当样本容量固定时,$\alpha$和$\beta$中任意一个减少,另一个必然增大;
            如果要使$\alpha$和$\beta$同时减少,只能增大样本容量
        \end{enumerate}
    \end{enumerate}
\end{remark}
\ifx\allfiles\undefined
\end{document}
\fi